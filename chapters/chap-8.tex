% !TEX root = thesis.tex

\startcomponent chap-8
\environment common
\product thesis

\startchapter[reference=chap:tool_support,title={Tool support}]

  The techniques presented in the previous chapters involve a significant
  amount of computation. For example, computing the satisfaction rate of
  high-level goals together with their uncertainty margins typically requires a
  large number of propagations to achieve reasonable precision. Manual
  propagation or the use of standard spreadsheets appears not feasible in
  practice. Other examples include the computation of most appropriate
  countermeasures, which requires computing the satisfaction rate of every
  countermeasure combination.
  
  Dedicated tools are thereby needed to apply our techniques in a realistic
  setting. This chapter discusses the toolset, called KAOSTools, supporting our
  techniques. These tools were used in the presented running examples troughout
  the preceding chapters and evaluated on larger models in
  \in{Chapter}[chap:evaluation].
  
  Unlike \cite[Dar97], the proposed toolset is freely available together with
  its source code enabling other researcher to easily reuse or extend the
  tools; the toolset accepts a textual declaration of goal and obstacle models
  that can be easily shared and integrated with existing versionning tools; the
  proposed toolset was designed to run on remote servers through command-line
  tools.
  
  This chapter is organized as follows. \in{Section}[sec:kt_lang] describes the
  declaration language allowing analysts to provide a textual declaration of
  the goal and obstacle models. \in{Section}[sec:kt_arch] describes the
  architecture of the KAOSTools toolset. \in{Section}[sec:kt_tools] details the
  command-line tools available for applying the techniques.
  
  All tools described in this chapter are written in C\# and run on Windows,
  Linux, and MacOS. The source code is freely available under MIT Licence
  \cite[Cai17b,Cai17d,Cai17f]. Packaged versions are also available in NuGet
  \cite[NuGet].

  \startsection[reference=sec:kt_lang,title={KAOSTools: the declaration language}]
  
  The techniques presented in the previous chapters rely on a goal model and
  its corresponding obstacle model. Such models are declared in a textual
  format. This section presents the declaration language used for input to the
  tools.
  
  In this language, element declaration start with the keyword
  \typeKAOSTool{declare} and end with the keyword \typeKAOSTool{end}. All
  elements have a unique identifier used to reference the element.
  Attributes precisely declare each element. They start with a keyword,
  such as \typeKAOSTool{name} or \typeKAOSTool{definition}, and end with a
  value, such as a string or an integer. The accepted values depend on the
  attribute.
  
  The following subsections describe the attributes for each element, such as
  goals and obstacles, and provide examples of such declarations.
  
  \subsubject{Goals, soft goals, domain properties, and domain hypotheses}
  
  To declare a goal, the keyword \typeKAOSTool{declare} must be followed by the
  keyword \typeKAOSTool{goal}. \in{Code}[code:goal_1] provides an example of
  declaring the goal \goal{Achieve [Make Up Water Provided When Loss Of
  Cooling]}. The unique identifier \typeKAOSTool{make_up_water_provided}
  immediately follows the keyword \typeKAOSTool{goal} between square brackets.
  Comments start with a dash (\typeKAOSTool{#}) and end at the end of the
  line. \in{Table}[tab:goal_attributes] describes the attributes for declaring
  goals.
  
    \placecode[][code:goal_1]
    {Declaration of the goal \goal{Achieve [Make Up Water Provided When Loss Of Cooling]}.}
    {\startKAOSTool
      declare goal [ make_up_water_provided ]
        name "Achieve [Make Up Water Provided When Loss Of Cooling]"
        definition 
          "Make up water shall be provided when loss of cooling occurs."
        rsr .99 # or rsr 99%
        refinedby response_to_loss_of_cooling, pump_on_when_activated
      end
    \stopKAOSTool}
  
  \placetable[][tab:goal_attributes]
  {Attributes for goal declarations}
  {
      \setupTABLE[c][each][align={right,lohi},frame=off,offset=0pt]
      \setupTABLE[r][1][style=bold,bottomframe=on,boffset=4pt]
      \setupTABLE[c][1,2,3][roffset=4pt,loffset=4pt]
      \setupTABLE[r][2][toffset=4pt]
      \setupTABLE[c][1][width=3cm]
      \setupTABLE[c][2][width=3cm]
      \setupTABLE[c][3][width=6cm]
      \switchtobodyfont[small]
      \bTABLE
  \bTR \bTD Attribute                   \eTD \bTD Used for                  \eTD \bTD Accepted value \eTD \eTR
  \bTR \bTD \typeKAOSTool{name}         \eTD \bTD Name                        \eTD \bTD String \eTD \eTR
  \bTR \bTD \typeKAOSTool{definition}   \eTD \bTD Used for                  \eTD \bTD String \eTD \eTR
  \bTR \bTD \typeKAOSTool{refinedby}    \eTD \bTD Goal AND-Refinement          \eTD \bTD List of identifiers \eTD \eTR
  \bTR \bTD \typeKAOSTool{rsr}          \eTD \bTD Required Satisfaction Rate  \eTD \bTD Number or percentage \eTD \eTR
  \bTR \bTD \typeKAOSTool{obstructedby} \eTD \bTD Goal obstruction            \eTD \bTD Identifier \eTD \eTR
  \bTR \bTD \typeKAOSTool{assignedto}   \eTD \bTD Goal assignment             \eTD \bTD Identifier \eTD \eTR
  \bTR \bTD \typeKAOSTool{formalspec}   \eTD \bTD Formal specification        \eTD \bTD Formal specification \eTD \eTR
  \eTABLE
  }
  
  {\it Goal refinements} are declared using the
  keyword \typeKAOSTool{refinedby}. The list of identifiers refers to other
  goals, domain properties or domain hypotheses in the corresponding AND-Refinement.
  If an identifier is not explicitly defined, a goal with corresponding
  identifier is added to the model. For example, the \in{Code}[code:goal_1]
  specifies a single AND-Refinement with two subgoals,
  \typeKAOSTool{response_to_loss_of_cooling} and
  \typeKAOSTool{pump_on_when_activated}.
  
  Multiple \typeKAOSTool{refinedby} attributes are used to declare alternative
  refinements. Goal refinements can be decorated with their refinement
  pattern using the name of the pattern between square brackets, as seen in
  \in{Code}[code:goal_2]. In addition, a partial entailment can be declared
  for the subgoals; the probability to satisfy the parent goal alone is
  declared between brackets after the subgoal identifier.
  \in{Code}[code:goal_2] declare that subgoal
  \typeKAOSTool{amb_mobilized_when_amb_allocated_on_ road} satisfy its parent
  goal with a probability $.8$.
  
  \placecode[][code:goal_2]
      {Declaration of the refinement pattern and partial entailment.}
      {\startKAOSTool
      refinedby [case]
        amb_mobilized_when_amb_allocated_on_road [.8], 
        amb_mobilized_when_amb_allocated_at_station [.2]
      \stopKAOSTool}
      
  Similarly to goals, soft goals can be declared using the keywork
  \typeKAOSTool{softgoal} with attributes \typeKAOSTool{name} and
  \typeKAOSTool{definition}. Domain properties and domain hypotheses are
  declared using the keyword \typeKAOSTool{domprop} and \typeKAOSTool{domhyp},
  respectively. The available attributes are summarized in
  \in{Table}[tab:dom_attributes].
  
  \placetable[here][tab:dom_attributes]
  {Attributes for declarations of domain properties and hypotheses}
  {
      \setupTABLE[c][each][align={right,lohi},frame=off,offset=0pt]
      \setupTABLE[r][1][style=bold,bottomframe=on,boffset=4pt]
      \setupTABLE[c][1,2,3][roffset=4pt,loffset=4pt]
      \setupTABLE[r][2][toffset=4pt]
      \setupTABLE[c][1][width=3cm]
      \setupTABLE[c][2][width=3cm]
      \setupTABLE[c][3][width=6cm]
      \switchtobodyfont[small]
      \bTABLE
  \bTR \bTD Attribute                   \eTD \bTD Used for                  \eTD \bTD Accepted value \eTD \eTR
  \bTR \bTD \typeKAOSTool{name}         \eTD \bTD Name                        \eTD \bTD String \eTD \eTR
  \bTR \bTD \typeKAOSTool{definition}   \eTD \bTD Used for                  \eTD \bTD String \eTD \eTR
  \bTR \bTD \typeKAOSTool{formalspec}   \eTD \bTD Formal specification        \eTD \bTD Formal specification \eTD \eTR
  \eTABLE
  }
  
  \subsubject{Obstacles} 
  
  Obstacles are declared using the keyword \typeKAOSTool{obstacle} followed by
  their unique identifier between square brackets. \in{Code}[code:obstacle_1]
  shows the declaration for the non-leaf obstacle \obstacle{Valve Not Opened
  And Requested}. \in{Table}[tab:obstacle_attributes] shows the attributes
  available for declaring obstacles.
  
  \placetable[here][tab:obstacle_attributes]
  {Attributes for obstacle declaration}
  {
      \setupTABLE[c][each][align={right,lohi},frame=off,offset=0pt]
      \setupTABLE[r][1][style=bold,bottomframe=on,boffset=4pt]
      \setupTABLE[c][1,2,3][roffset=4pt,loffset=4pt]
      \setupTABLE[r][2][toffset=4pt]
      \setupTABLE[c][1][width=3cm]
      \setupTABLE[c][2][width=3cm]
      \setupTABLE[c][3][width=6cm]
      \switchtobodyfont[small]
      \bTABLE
  \bTR \bTD Attribute                   \eTD \bTD Used for                  \eTD \bTD Accepted value \eTD \eTR
  \bTR \bTD \typeKAOSTool{name}         \eTD \bTD Name                        \eTD \bTD String \eTD \eTR
  \bTR \bTD \typeKAOSTool{definition}   \eTD \bTD Used for                  \eTD \bTD String \eTD \eTR
  \bTR \bTD \typeKAOSTool{refinedby}    \eTD \bTD Obstacle AND-Refinement     \eTD \bTD List of identifiers \eTD \eTR
  \bTR \bTD \typeKAOSTool{resolvedby}   \eTD \bTD Obstacle resolution         \eTD \bTD Identifier \eTD \eTR
  \bTR \bTD \typeKAOSTool{formalspec}   \eTD \bTD Formal specification        \eTD \bTD Formal specification \eTD \eTR
  \bTR \bTD \typeKAOSTool{esr}          \eTD \bTD Estimated Satisfaction Rate \eTD \bTD Number, probability distribution or percentage \eTD \eTR
  \eTABLE
  }
      
  \noindent {\bf Obstacle refinements.} The declaration of obstacle
  refinements is similar to the declaration of goal refinements. The
  identifiers must refer to obstacles, domain properties or domain hypotheses.
  If an identifier is not explicitly defined, a new obstacle with 
  corresponding identifier is added to the model. \in{Code}[code:obstacle_1]
  declares four alternative refinements each with a single obstacle.
  
  \placecode[][code:obstacle_1]
      {Declaration of the obstacles \obstacle{Valve Not Opened And Requested}.}
      {\startKAOSTool
      declare obstacle [ valve_not_open ]
        name "Valve Not Opened And Requested"
        refinedby valve_failure
        refinedby no_power_available
        refinedby unavailable_due_to_maintenance
        refinedby valve_electronic_failure
      end
      \stopKAOSTool}
  
  \noindent {\bf Goal obstructions.} Obstructions to goals are declared using
  the keyword \typeKAOSTool{obstructedby}. The value of this attribute is a
  single identifier referring to an obstacle\emdash{}if not explicitly
  defined, a new obstacle will be added. \in{Code}[code:goal_3] shows how the
  obstruction of \goal{Achieve [Make Up Pump Motor On When Water Requested]} is
  declared.
  
  \placecode[][code:goal_3]
      {Declaration of the goal \goal{Achieve [Make Up Pump Motor On When Water Requested]}.}
      {\startKAOSTool  
        declare goal [ pump_motor_on ]
          ...
          obstructedby motor_not_on
        end
      \stopKAOSTool}
  
  \noindent {\bf Obstacle satisfaction rates.} Leaf obstacles are estimated by
  experts. The declaration language supports both single-value and
  multi-value satisfaction rates. \in{Code}[code:obstacle_2] shows the
  declaration of the obstacle \obstacle{Valve Mechanical Failure} with a
  single-value  satisfaction rate of $.002$, whereas \in{Code}[code:obstacle_3]
  shows the declaration of the obstacle \obstacle{Pump Mechanical Failure}
  with a multi-value satisfaction rate. The supported multi-value satisfaction
  rates are summarized in \in{Table}[tab:multisatrate].
  
  \placecode[][code:obstacle_2]
      {Declaration of the obstacle \obstacle{Valve Mechanical Failure}.}
      {\startKAOSTool
      declare obstacle [ valve_failure ]
        name "Valve Mechanical Failure"
        resolvedby [restoration:make_up_water_provided] 
                   cooling_system_repaired
        esr 0.002
      end
      \stopKAOSTool}
      
    \placecode[][code:obstacle_3]
        {Declaration of the obstacle \obstacle{Valve Not Opened When Valve Control Not Unavailable}.}
        {\startKAOSTool
@experts.quantiles "(10%, 50%, 90%)"
declare obstacle [ manual_opening_valve_failure ]
  ...
  esr quantile[0.001,0.002,0.003]
end
        \stopKAOSTool}
  
  As seen in \in{Chapter}[chap:knowledge-uncertainty], a quantile is a single
  probability value attached to a cumulative probability. To declare the
  cumulative probabilities, the model attribute \typeKAOSTool{@experts.quantiles}
  must be provided. For example, \in{Code}[code:obstacle_3] specify that
  $10\%$ of the values are below $0.001$, $50\%$ are below $0.002$ and $90\%$
  of the values are below $0.003$. The model attribute
  \typeKAOSTool{@experts.quantiles} need only be declared once for the model.
  
  \placetable[here][tab:multisatrate]
  {Multi-value satisfaction rates}
  {
  \setupTABLE[c][each][align={right,lohi},frame=off,offset=0pt]
  \setupTABLE[r][1][style=bold,bottomframe=on,boffset=4pt]
  \setupTABLE[c][1,2,3][roffset=4pt,loffset=4pt]
  \setupTABLE[r][2][toffset=4pt]
  \switchtobodyfont[small]
  \bTABLE[option=stretch]
  \bTR \bTD Value                       \eTD \bTD Used for               \eTD \bTD Example \eTD \eTR
  \bTR \bTD \typeKAOSTool{beta}         \eTD \bTD Beta Distribution        \eTD \bTD \typeKAOSTool{beta[2,4]}. A beta distribution with $\alpha = 2$ and $\beta = 4$. See \cite[Vos08] for details on parameters $\alpha$ and $\beta$. \eTD \eTR
  \bTR \bTD \typeKAOSTool{uniform}      \eTD \bTD Uniform Distribution     \eTD \bTD \typeKAOSTool{uniform[.1,.3]}. A uniform distribution between $.1$ and $.3$. \eTD \eTR
  \bTR \bTD \typeKAOSTool{triangular}   \eTD \bTD Triangular Distribution  \eTD \bTD \typeKAOSTool{triangular[.1,.4,.6]}. A triangular distribution with a $.1$ min-value, $.6$ max-value and a $.4$ mode. \eTD \eTR
  \bTR \bTD \typeKAOSTool{pert}         \eTD \bTD PERT Distribution        \eTD \bTD \typeKAOSTool{pert[.1,.4,.6]}. A PERT distribution with a $.1$ min-value, $.6$ max-value and a $.4$ mode. See \cite[Vos08] for details about PERT distributions. \eTD \eTR
  \bTR \bTD \typeKAOSTool{quantiles}    \eTD \bTD Quantiles                \eTD \bTD \typeKAOSTool{quantiles[.1,.4,.6]}. A list of quantiles. \eTD \eTR
  \eTABLE
  }

  \noindent {\bf Obstacle resolutions.} Obstacles resolutions are declared
  using the keyword \typeKAOSTool{resolvedby} followed by a goal
  identifier\emdash{}if not explicitely defined, a new goal will be added.
  \in{Code}[code:obstacle_2] shows the declaration of the resolution of
  obstacle \obstacle{Valve Mechanical Failure} by the goal
  \typeKAOSTool{cooling_system_repaired}.
  
  Resolution tactics can be declared between square brackets (see
  \typeKAOSTool{restoration} in \in{Code}[code:obstacle_2]), followed by the
  anchor goal (\typeKAOSTool{make_up_water_provided} in the example).
  \in{Table}[tab:kt_rtact] summarizes the supported resolution tactics.

  \placetable[here][tab:kt_rtact]
  {Resolution tactics}
  {
  \setupTABLE[c][each][align={right,lohi},frame=off,offset=0pt]
  \setupTABLE[r][1][style=bold,bottomframe=on,boffset=4pt]
  \setupTABLE[c][1,2,3][roffset=4pt,loffset=4pt]
  \setupTABLE[r][2][toffset=4pt]
  \switchtobodyfont[small]
  \bTABLE[option=stretch]
  \bTR \bTD Resolution tactic                 \eTD \bTD Used for               \eTD \eTR
  \bTR \bTD \typeKAOSTool{substitution}       \eTD \bTD Goal substitution        \eTD \eTR
  \bTR \bTD \typeKAOSTool{prevention}         \eTD \bTD Obstacle prevention      \eTD \eTR
  \bTR \bTD \typeKAOSTool{obstacle_reduction} \eTD \bTD Obstacle reduction       \eTD \eTR
  \bTR \bTD \typeKAOSTool{restoration}        \eTD \bTD Goal restoration         \eTD \eTR
  \bTR \bTD \typeKAOSTool{weakening}          \eTD \bTD Weakening                \eTD \eTR
%  \bTR \bTD \typeKAOSTool{mitigation}         \eTD \bTD Mitigation (strong or weak) \eTD \eTR
  \bTR \bTD \typeKAOSTool{weak_mitigation}    \eTD \bTD Weak mitigation          \eTD \eTR
  \bTR \bTD \typeKAOSTool{strong_mitigation}  \eTD \bTD Strong mitigation        \eTD \eTR
  \eTABLE
  }
  
  %\page[yes]
  
  \subsubject{Agents}
  
  Agents are declared using the keyword \typeKAOSTool{agent} followed by their
  unique identifier between square brackets. \in{Table}[tab:agent_attributes]
  summarizes the attributes for agents. \in{Code}[code:agent_1] provides
  an example of agent declaration.
        
  \placecode[here][code:agent_1]
    {Declaration of the agent \agent{Controlling Software}.}
    {\startKAOSTool
    declare agent [ operator ]
      name "Operator"
    end
    \stopKAOSTool}
  
  \noindent {\bf Agent assignments.} An agent assignment to a goal is declared
  using the keyword \typeKAOSTool{assignedto}. \in{Code}[code:agent_2] shows
  how the agent \agent{Operator} is assigned to the goal \goal{Achieve [Valve
  Opened When Unavailable Due To Maintenance]}. If not explicitly defined, an
  agent with corresponding identifier is added to the model.
  
  \placecode[here][code:agent_2]
    {Declaration of the goal \goal{Achieve [Valve Opened When Unavailable Due To Maintenance]}.}
    {\startKAOSTool
    declare goal [ manual_opening_of_valve ]
      name "Achieve [Valve Opened When Unavailable Due To Maintenance]"
      obstructedby manual_opening_valve_failure
      assignedto operator
    end
    \stopKAOSTool}
  
  \placetable[here][tab:agent_attributes]
  {Attributes for agent declaration}
  {
      \setupTABLE[c][each][align={right,lohi},frame=off,offset=0pt]
      \setupTABLE[r][1][style=bold,bottomframe=on,boffset=4pt]
      \setupTABLE[c][1,2,3][roffset=4pt,loffset=4pt]
      \setupTABLE[r][2][toffset=4pt]
      \setupTABLE[c][1][width=3cm]
      \setupTABLE[c][2][width=3cm]
      \setupTABLE[c][3][width=6cm]
      \switchtobodyfont[small]
      \bTABLE
  \bTR \bTD Attribute                   \eTD \bTD Used for                  \eTD \bTD Accepted value \eTD \eTR
  \bTR \bTD \typeKAOSTool{name}         \eTD \bTD Name                        \eTD \bTD String \eTD \eTR
  \bTR \bTD \typeKAOSTool{definition}   \eTD \bTD Used for                  \eTD \bTD String \eTD \eTR
  \bTR \bTD \typeKAOSTool{type}         \eTD \bTD Whether the agent is a {\it software} or a {\it environment} agent.
     \eTD \bTD \typeKAOSTool{software} or \typeKAOSTool{environment} \eTD \eTR
  \eTABLE
  }
  
  \subsubject{Predicates and formal declarations}
  
  Formal specifications of goals and obstacles can be provided using the keyword
  \typeKAOSTool{formalspec}. The language supports first-order, linear temporal
  logic assertions. A complete BNF grammar is available in
  \in{Appendix}[chap:bnf]. \in{Code}[code:formalspec_1] provides an example of
  a formal specification\emdash{}here, the LTL formula 
  
  \startformula\startalign[n=1,align={left}]
  
    \NC \ltlF (dieselGeneratorRequested \NR\noalign{\vskip-.5em}
    \NC \hskip1em \wedge \ltlG_{\geq 5 min} \neg (\exists g: DieselGenerator \cdot generatorStarted(g))) \NR
  
  \stopalign\stopformula

  \placecode[here][code:formalspec_1]
    {Formal specification of the obstacle \predicate{Diesel Generator Not Started}.}
    {\startKAOSTool
    declare obstacle [ diesel_generator_not_started ]
      name "Diesel Generator Not Started"
      formalspec 
        sooner-or-later ( dieselGeneratorRequested and
          always, for more than 5 minutes, 
          not (exists g: DieselGenerator . generatorStarted(g))
        )
      probability 0.002
    end
    \stopKAOSTool} 
  
  The identifier \typeKAOSTool{generatorStarted} refers to a predicate with a
  single argument. Predicates can be explicitly declared using the keyword
  \typeKAOSTool{predicate} followed by a unique identifier.
  \in{Code}[code:pred_1] shows the declaration of the predicate
  \predicate{GeneratorStarted}. \in{Table}[tab:predicate_attributes] provides
  the attributes for declaring predicates.
  
  \placecode[here][code:pred_1]
    {Declaration of the predicate \predicate{ValveOpened}.}
    {\startKAOSTool
    declare predicate [ generatorStarted ]
      name "GeneratorStarted"
      argument g: dieselGenerator
      formalspec g.Started
    end
    \stopKAOSTool}  
  
  \noindent {\bf Arguments.} As seen in \in{Code}[code:pred_1], predicates
  may have arguments in a formal specification. In the example, the formal specification states that
  the predicate is true if the attribute \typeKAOSTool{Started} is true for the
  instance of \typeKAOSTool{DieselGenerator} provided as an argument. When
  referring to a predicate in a formal specification, the actual arguments must
  match the order and the respective entity types declared in the 
  argument list.
  
  \subsubject{Entities, associations and types}
  
  Formal specifications refer to entities and associations to
  express their truth value. All referenced entities, entity attributes,
  associations, association attributes and types from the object model
  associated with the considered goal model are automatically inferred
  from the formal specification of goals, domain properties, domain hypotheses,
  obstacles, and predicates. This appears  particularly useful for checking
  whether formal specifications refer to the appropriate elements and to avoid
  similar names (e.g. \typeKAOSTool{dieselGenerator} and
  \typeKAOSTool{diesel_generator}).
  
  The analyst might enrich the object model by explicitly defining 
  elements. \in{Tables}[tab:entity_attributes],
  \in{}[tab:association_attributes] and \in{}[tab:type_attributes] provide the
  attributes for declaring entities, associations and given types. 
  
  \startplacetable[location={page,none}]
    \startfloatcombination[nx=1,ny=4,after={\blank[10mm]}]
   
      \placetable[here][tab:predicate_attributes]
      {Attributes for predicate declaration}
      {
      \setupTABLE[c][each][align={right,lohi},frame=off,offset=0pt]
      \setupTABLE[r][1][style=bold,bottomframe=on,boffset=4pt]
      \setupTABLE[c][1,2,3][roffset=4pt,loffset=4pt]
      \setupTABLE[r][2][toffset=4pt]
      \setupTABLE[c][1][width=3cm]
      \setupTABLE[c][2][width=3cm]
      \setupTABLE[c][3][width=6cm]
      \switchtobodyfont[small]
      \bTABLE
      \bTR \bTD Attribute                   \eTD \bTD Used for           \eTD \bTD Accepted value \eTD \eTR
      \bTR \bTD \typeKAOSTool{name}         \eTD \bTD Name                 \eTD \bTD String \eTD \eTR
      \bTR \bTD \typeKAOSTool{definition}   \eTD \bTD Used for           \eTD \bTD String \eTD \eTR
      \bTR \bTD \typeKAOSTool{formalspec}   \eTD \bTD Formal specification \eTD \bTD Formal specification \eTD \eTR
      \bTR \bTD \typeKAOSTool{argument}     \eTD \bTD Formal arguments     \eTD \bTD An identifier followed by a colon (\typeKAOSTool{:}) and an identifier \eTD \eTR
      \eTABLE
      }
    
      \placetable[here][tab:entity_attributes]
      {Attributes for entity declaration}
      {
      \setupTABLE[c][each][align={right,lohi},frame=off,offset=0pt]
      \setupTABLE[r][1][style=bold,bottomframe=on,boffset=4pt]
      \setupTABLE[c][1,2,3][roffset=4pt,loffset=4pt]
      \setupTABLE[r][2][toffset=4pt]
      \setupTABLE[c][1][width=3cm]
      \setupTABLE[c][2][width=3cm]
      \setupTABLE[c][3][width=6cm]
      \switchtobodyfont[small]
      \bTABLE
      \bTR \bTD Attribute                   \eTD \bTD Used for   \eTD \bTD Accepted value \eTD \eTR
      \bTR \bTD \typeKAOSTool{name}         \eTD \bTD Name         \eTD \bTD String \eTD \eTR
      \bTR \bTD \typeKAOSTool{definition}   \eTD \bTD Used for   \eTD \bTD String \eTD \eTR
      \bTR \bTD \typeKAOSTool{type}   \eTD \bTD Entity type  \eTD \bTD \typeKAOSTool{software}, \typeKAOSTool{environment} or \typeKAOSTool{shared} \eTD \eTR
      \bTR \bTD \typeKAOSTool{isa}          \eTD \bTD Inheritance  \eTD \bTD Identifier \eTD \eTR
      \bTR \bTD \typeKAOSTool{attribute}    \eTD \bTD Attribute    \eTD \bTD An identifier optionally followed by a colon (\typeKAOSTool{:}) and an identifier \eTD \eTR
      \eTABLE
      }
 
      \placetable[here][tab:association_attributes]
      {Attributes for association declaration}
      {
      \setupTABLE[c][each][align={right,lohi},frame=off,offset=0pt]
      \setupTABLE[r][1][style=bold,bottomframe=on,boffset=4pt]
      \setupTABLE[c][1,2,3][roffset=4pt,loffset=4pt]
      \setupTABLE[r][2][toffset=4pt]
      \setupTABLE[c][1][width=3cm]
      \setupTABLE[c][2][width=3cm]
      \setupTABLE[c][3][width=6cm]
      \switchtobodyfont[small]
      \bTABLE
      \bTR \bTD Attribute                   \eTD \bTD Used for           \eTD \bTD Accepted value \eTD \eTR
      \bTR \bTD \typeKAOSTool{name}         \eTD \bTD Name                 \eTD \bTD String \eTD \eTR
      \bTR \bTD \typeKAOSTool{definition}   \eTD \bTD Used for           \eTD \bTD String \eTD \eTR
      \bTR \bTD \typeKAOSTool{attribute}    \eTD \bTD Attribute            \eTD \bTD An identifier optionally followed by a colon (\typeKAOSTool{:}) and an identifier \eTD \eTR
      \bTR \bTD \typeKAOSTool{link}         \eTD \bTD Linked entity        \eTD \bTD An identifier \eTD \eTR
      \eTABLE
      }  
 
      \placetable[here][tab:type_attributes]
      {Attributes for given type declaration}
      {
      \setupTABLE[c][each][align={right,lohi},frame=off,offset=0pt]
      \setupTABLE[r][1][style=bold,bottomframe=on,boffset=4pt]
      \setupTABLE[c][1,2,3][roffset=4pt,loffset=4pt]
      \setupTABLE[r][2][toffset=4pt]
      \setupTABLE[c][1][width=3cm]
      \setupTABLE[c][2][width=3cm]
      \setupTABLE[c][3][width=6cm]
      \switchtobodyfont[small]
      \bTABLE
      \bTR \bTD Attribute                   \eTD \bTD Used for           \eTD \bTD Accepted value \eTD \eTR
      \bTR \bTD \typeKAOSTool{name}         \eTD \bTD Name                 \eTD \bTD String \eTD \eTR
      \bTR \bTD \typeKAOSTool{definition}   \eTD \bTD Used for           \eTD \bTD String \eTD \eTR
      \eTABLE
      }
    \stopfloatcombination
  \stopplacetable
        
  \noindent {\bf Entity and association attributes.} An entity or association
  attribute is composed of an attribute identifier and a given type. A given
  type is a base type, such as Boolean or String, or a domain-specific type
  such as {\it MotorStatus}. The latter are declared using the
  \typeKAOSTool{type} keyword. \in{Code}[code:object_1] shows an example of
  attribute declaration.
  
  \placecode[here][code:object_1]
    {Declaration of the entity \entity{DieselGenerator}.}
    {\startKAOSTool
    declare entity [ dieselGenerator ]
      name "DieselGenerator"
      attribute Started: bool
    end
    \stopKAOSTool} 
  
  \noindent {\bf Linking associations and entities.} Entities are connected to other
  entities through associations. The attribute \typeKAOSTool{link} declares the
  association connecting entities, optionally with corresponding
  multiplicities \cite[Lam09]. \in{Code}[code:object_2] provides an example
  with an association linking the entities \entity{DieselGenerator} and
  \entity{ElectroValve}. In the association, there is at most one diesel
  generator, but one or more corresponding valve(s).
  
  \placecode[here][code:object_2]
    {Declaration of the association \association{GeneratorValves}.}
    {\startKAOSTool
    declare association [ generator_valves ]
      name "GeneratorValves"
      link [1] dieselGenerator
      link [1,N] electroValve
    end
    \stopKAOSTool} 
  
  \subsubject{Experts and calibration variables}
  
  To support the techniques presented in
  \in{Chapter}[chap:knowledge-uncertainty], the declaration language was
  extended to explicitly declare experts and calibration variables. Experts are
  declared using the keyword \typeKAOSTool{expert} followed by their unique
  identifier. Calibration variables are declared using the keyword
  \typeKAOSTool{calibration} followed by their unique identifier.
  \in{Code}[code:expert_1] declares two experts and one calibration variable.
  \in{Tables}[tab:expert_attributes] and \in{}[tab:cal_attributes] describes
  the attributes availables.
  
  \placecode[here][code:expert_1]
    {Declaration of the association \association{GeneratorValves}.}
    {\startKAOSTool  
    declare expert [ expert1 ] name "Expert 1" end
    declare expert [ expert2 ] name "Expert 2" end

    declare calibration [ calibration1 ]
      name "Push Button Broken"
      esr 4.37%
      esr[expert1] quantile[3.812%, 4.285%, 4.759%]
      esr[expert2] quantile[3.961%, 4.453%, 4.945%]
    end
    \stopKAOSTool} 
 
    \placetable[here][tab:expert_attributes]
    {Attributes for expert declaration}
    {
    \setupTABLE[c][each][align={right,lohi},frame=off,offset=0pt]
    \setupTABLE[r][1][style=bold,bottomframe=on,boffset=4pt]
    \setupTABLE[c][1,2,3][roffset=4pt,loffset=4pt]
    \setupTABLE[r][2][toffset=4pt]
    \setupTABLE[c][1][width=3cm]
    \setupTABLE[c][2][width=3cm]
    \setupTABLE[c][3][width=6cm]
    \switchtobodyfont[small]
    \bTABLE
    \bTR \bTD Attribute                   \eTD \bTD Used for           \eTD \bTD Accepted value \eTD \eTR
    \bTR \bTD \typeKAOSTool{name}         \eTD \bTD Name                 \eTD \bTD String \eTD \eTR
    \eTABLE
    }  

    \placetable[here][tab:cal_attributes]
    {Attributes for declaration of calibration variables.}
    {
    \setupTABLE[c][each][align={right,lohi},frame=off,offset=0pt]
    \setupTABLE[r][1][style=bold,bottomframe=on,boffset=4pt]
    \setupTABLE[c][1,2,3][roffset=4pt,loffset=4pt]
    \setupTABLE[r][2][toffset=4pt]
    \setupTABLE[c][1][width=3cm]
    \setupTABLE[c][2][width=3cm]
    \setupTABLE[c][3][width=6cm]
    \switchtobodyfont[small]
    \bTABLE
    \bTR \bTD Attribute                   \eTD \bTD Used for           \eTD \bTD Accepted value \eTD \eTR
    \bTR \bTD \typeKAOSTool{name}         \eTD \bTD Name                 \eTD \bTD String \eTD \eTR
    \bTR \bTD \typeKAOSTool{definition}   \eTD \bTD Used for           \eTD \bTD String \eTD \eTR
    \bTR \bTD \typeKAOSTool{esr}          \eTD \bTD Estimated Satisfaction Rate \eTD \bTD Number, probability distribution or percentage \eTD \eTR
    \eTABLE
    }
  
  For a calibration variable, the
  reference value is declared using the \typeKAOSTool{esr} attribute without
  declaring an expert. The estimates by experts are declared using the
  \typeKAOSTool{esr} attribute with the corresponding expert between square
  brackets. When declaring the estimated satisfaction rate by an expert in an
  obstacle, the expert is also declared in square brackets (see
  \in{Code}[code:expert_2]).
  
  \placecode[here][code:expert_2]
    {Declaration of the association \association{GeneratorValves}.}
    {\startKAOSTool  
      declare obstacle [ power_cable_failure ]
        name "Power Cabling Failure"
        esr[expert1] quantile[0.015,0.01752,0.02]
        esr[expert2] quantile[0.001,0.002,0.003]
      end
    \stopKAOSTool} 
 
  \subsubject{Additional features}
  
  In addition to the constructs presented before, it is possible to define
  custom attributes to split the model into different files, to declare model managment
  attributes such as the title of the model, the version, and the authors.
  Utilities are also available to help analysts writing declarations.
  
  \noindent {\bf Custom attributes.} All elements can have extra attributes
  not declared by the grammar. These custom attributes must start with
  \typeKAOSTool{$}. Such extensions proved usefull 
  to support specific tools without changing the grammar. For example,
  \in{Code}[code:custom_attr] shows how activation and deactivation procedures
  are declared\emdash{}our runtime adaptation tool did not
  require changes to the declaration language.
  
  \placecode[here][code:custom_attr]
    {Declaration of the association \association{GeneratorValves}.}
    {\startKAOSTool
    declare goal [ speed_acquired_by_camera ]
      name "Achieve [SpeedAcquiredEvery5SecondsByCamera]"
      refinedby images_acquired, 
                tracers_identified, 
                speed_acquired_from_tracers
      $ondeploy "DeployCamera"
      $onwithold "DeployUltrasound"
    end
    \stopKAOSTool} 
  
  \noindent {\bf Splitting the model into multiple files.} It is easier to handle a large
  model by splitting it into multiple files. Sub-models can be imported using
  \typeKAOSTool{import} followed by the path to the sub-model. Adding
  attributes to an element already declared facilitates the splitting into
  multiple files. The element is \quote{declared} using the keyword
  \typeKAOSTool{override}, in place of \typeKAOSTool{declare}, to add an
  attribute. This proved useful to split goals in a specific file and
  obstacles in another one while declaring the obstructions in the file with the
  obstacle declarations. \in{Code}[code:import] shows an example of such
  usage.
  
  \placecode[here][code:import]
    {Splitting model declaration accross multiple files.}
    {\startKAOSTool
    # In the file 'goal.kaos'
    declare goal [ manual_opening_of_valve ]
      name "Achieve [Valve Opened When Unavailable Due To Maintenance]"
      assignedto operator
    end
    ...
    import "obstacle.kaos"
    
    # In the file 'obstacle.kaos'
    override goal [ manual_opening_of_valve ]
      obstructedby manual_opening_valve_failure
    end
    declare obstacle [ manual_opening_valve_failure ]
      name "Valve Not Opened When Valve Control Not Unavailable"
      probability 0.0001
    end
    ...
    \stopKAOSTool}
    
  \noindent {\bf Model attributes.} \in{Code}[code:model_attributes] shows how
  the title, the authors and the version of the considered goal model can be
  declared.
  
    \placecode[here][code:model_attributes]
      {Model attributes.}
      {\startKAOSTool
    @title "Spent Fuel Pool"
    @author "Antoine Cailliau, Axel van Lamsweerde"
    @version "0.1"
    \stopKAOSTool}
  
  \noindent {\bf Utilities.} A \tool{Checker}  tool  is available for checking
  the correct syntax and providing statistics for the model. For example, this
  tool may produce the following:
  
  \startCLI
  $ mono Checker.exe model.kaos 
  Goals: 20
  Root goals: 1
  Leaf goals: 14
  Goal refinements: 6

  Obstacles: 16
  Root obstacles: 6
  Leaf obstacles: 12
  Obstacle refinements: 11

  Resolutions: 10

  Generated goal exceptions: 9 (distributed over 3 goals)
  Generated provided assumption: 23 (distributed over 6 goals)
  \stopCLI
  
  The tool \tool{OmnigraffleExport} generates goal and obstacle diagrams
  corresponding to the graphical syntax of the model provided in textual
  format. The analyst can edit the generated diagrams using OmniGraffle
  \cite[Ols10]. This tool was used to produce all goal and obstacle diagrams
  presented in the thesis. Last, a bundle is available for the TextMate text
  editor \cite[Gra07] to provide syntax highlighting.
  
  \stopsection

  \startsection[reference=sec:kt_arch,title={KAOSTools: architecture}]
  
    \in{Figure}[fig:kt_architecture] shows the architecture of the KAOSTools
    tool suite. Boxes refer to libraries, dashed boxes to external libraries.
    The latter were developed to support KAOSTools and might be used
    outside the framework. Arrows indicate {\it USE} links \cite[Med10].
    The architectural modules are briefly described below.
  
    \placefigure[]
         [fig:kt_architecture]
         {KAOSTools Architecture.}
      {\externalfigure[../images/chap8/architecture.pdf]}
    
    \noindent {\bf Core.} This package contains all classes representing the
    probabilistic goals, domain properties, obstacles, and so forth. It mainly
    supports the techniques in \in{Chapter}[chap:proba-framework] together with
    other libraries through {\it helpers} for selecting, modifying, and
    creating KAOS model elements.
  
    \noindent {\bf Parsing.} This package contains all the classes related to
    the parsing of the KAOSTools declaration language. On parsing, a model is
    built in memory using the helpers provided in \tool{Core}.
  
    \noindent {\bf Propagators.} This package implements the techniques
    presented in \in{Chapter}[chap:assessing]. The classes there implement both
    the Pattern-based and BDD-based computations. The package also supports the
    computation of satisfaction rates with their uncertainty margins. This
    library relies on \tool{BDDSharp} to support the BDD-based computation.
    \tool{BDDSharp} is a freely available library to create and manipulates
    {Reduced Order Binary Decision Diagrams} (ROBDDs) in C\# \cite[Cai17d]. The
    ease of deployment and the availability on multiple OSes motivated the
    creation of the library; If needed, more efficient BDD libraries can
    replace \tool{BDDSharp} in the KAOSTools suite.
  
    \noindent {\bf CriticalObstacles.} This package implements the techniques
    presented in \in{Chapter}[chap:assessing] to highlight most critical
    obstacles. This package also supports such highlighting taking uncertainty
    margins into account as presented in
    \in{Chapter}[chap:knowledge-uncertainty]. This library also generates
    violation diagrams.
  
    \noindent {\bf Integrators.} This package implements the techniques
    presented in \in{Chapter}[chap:controlling_obstacle] for 
    integration of countermeasures. The package supports both {\it hard}
    integration that modifies the goal/obstacle model, and {\it soft} integration
    using the constructs presented in \in{Section}[sec:exception_handling].
  
    \noindent {\bf Optimizers.} This package implements the techniques
    presented in \in{Chapter}[chap:controlling_obstacle] for selection of most
    appropriate countermeasures. Single-value and multi-value leaf obstacle
    estimates are both supported.
    
    \noindent {\bf ExpertCombination.} This package implements the techniques
    presented in \in{Chapter}[chap:knowledge-uncertainty]. It supports the
    combination of multiple experts estimates using both Cook's and
    Mendel-Sheridan's techniques. This library relies on
    \tool{ExpertOpinionSharp} to combine expert opinions.
    \tool{ExpertOpinionSharp} was developed specifically for combining expert
    estimates using calibration variables \cite[Cai17e]. It was developed as a
    separate library; other approaches can thereby reuse these combination
    techniques.
  
    \noindent {\bf Monitoring.} This package supports the techniques presented
    in \in{Chapter}[runtime]. The library relies on \tool{LTLSharp} for the
    generation and update of $LTL_3$ monitors. \tool{LTLSharp} is a library
    that provides model checking algorithms, algorithms for building Buchï
    automata, and $LTL_3$ monitors for C\# libraries and applications
    \cite[Cai17f].
    
  \stopsection

  \startsection[reference=sec:kt_tools,title={KAOSTools: tool suite}]
  
    In addition to the various libraries developed to support our techniques, a series
    of command-line tools enables the analyst to run the techniques on their
    goal/obstacle models. This section briefly describes the tools. More
    detailed descriptions are available through the option {\tttf --help} to
    the command-line tools.

    \noindent {\bf Utils.Propagator.} This tool computes the satisfaction rate
    of the input high-level goals.
    
    \startCLI
    $ mono Propagator.exe --root=make_up_water_provided model.kaos 
    Achieve [Make Up Water Provided When Loss Of Cooling]: 80.31 %
    \stopCLI
    
    \noindent {\bf Utils.UncertaintyPropagator.} This tool computes the
    satisfaction uncertainty and violation uncertainty for the input high-level goals.
    
    \startCLI
    $ mono UncertaintyPropagator.exe --root=make_up_water_provided \
      model-uncertainty.kaos 
    Achieve [Make Up Water Provided When Loss Of Cooling]:
      Mean: 78.67 %
      Required Satisfaction Rate: 80.00 %
      Violation Uncertainty: 71.80 %
      Uncertainty Spread: 0.02694
    \stopCLI
  
    \noindent {\bf Utils.ViolationDiagram.} This tool generates the data
    for violation uncertainty diagrams. The latter
    can then be generated using R \cite[Tea00].
  
    \noindent {\bf Utils.CMIntegrator.} This tool provides an interactive
    console for integrating countermeasures into an original goal model. The
    latter can then be exported.
    
    \startCLI
    $ mono CMIntegrator.exe model.kaos 
    > resolve pump_failure
    [0] Achieve [Cooling System Repaired]
    [1] Achieve [Redundant Pump Motor On When Primary Pump Failure]
    Select the countermeasure goal to integrate: 0 
    > 
    \stopCLI
      
    \noindent {\bf Utils.CMSelector.} This tool returns a list of most
    appropriate countermeasures. {\bf Utils.CMSelectorUncertainty} returns a
    list of most appropriate countermeasures when leaf obstacles are estimated
    together with their uncertainty margins.
    
    \startCLI
    $ mono CMSelectorUncertainty.exe --root=make_up_water_provided \
      model-uncertainty.kaos 
    Violation Uncertainty without countermeasures: 0.718
    Uncertainty Spread without countermeasures: 0.0269443101910917
    Required Satisfaction Rate: 0.8
    Minimal cost to guarantee RSR: 1
    Optimal selections (2):
    * [OptimalSelection: Resolutions={cooling_system_repaired}, 
                         Cost=1, 
                         ViolationUncertainty=0, 
                         UncertaintySpread=0]
    * [OptimalSelection: Resolutions={redundant_pump_started}, 
                         Cost=1, 
                         ViolationUncertainty=0, 
                         UncertaintySpread=0]

    --- Statistics ---
    Number of countermeasure goals: 6
    Number of possible selections: 63
    Number of safe selections: 48
    Number of tested selections (for minimal cost): 63
    Number of tested selections (for optimal selection): 6
    Maximal safe cost: 6
    Time to compute minimal cost: 00:00:00.3229049
    Time to compute optimal selections: 00:00:00.3229049
    \stopCLI
  
    \noindent {\bf Utils.Monitor.} This tool runs the monitoring of
    probabilistic obstacles, selects more appropriate countermeasures and
    triggers the activation/deactivation procedure to be applied.
    \in{Section}[sec:bads] provides concrete details about the monitoring
    infrastructure in action on a case study.
    
  \stopsection

  \startsection[title={Summary}]
  
    This chapter briefly presented the toolset supporting the techniques
    discussed in the previous chapters. The tools accept a textual declaration
    of the goal/obstacle model as inout. The modular architecture of KAOSTools
    enables new techniques to be implemented and integrated while leveraging
    the existing ones.
  
    The tools presented in this chapter were used to generate all diagrams,
    charts, and computation results provided in the various examples throughout
    the thesis. These were also used to evaluate our techniques, as presented
    in the next chapter.
  
  \stopsection

\stopchapter

\stopcomponent
