% !TEX root = thesis.tex

\startcomponent chap-8
\environment common
\product thesis

\startchapter[reference=chap:tool_support,title={Tool support}]

  The techniques presented in the previous chapters might require many
  computations. For example, computing the satisfaction rate of high-level
  goals together with their uncertainty margins typically requires at least
  100.000 propagations to achieve reasonable precision. Manual propagation or
  propagation using standard spreadsheet software is not feasible in practice.
  Other examples include computing the most appropriate countermeasures, which
  requires computing the satisfaction rate for every countermeasure
  combination. Dedicated tools are thereby needed to apply the techniques in an
  industrial and realistic setting. This chapter discusses the tools supporting
  the techniques, used in the presented running examples and evaluated on
  larger models in \in{Chapter}[chap:evaluation].
  
  The chapter is organized as follows. \in{Section}[sec:kt_lang] describes the
  KAOSTools specification language enabling the analyst to provide a textual
  specification of the goal and obstacle models. \in{Section}[sec:kt_arch]
  describes the architecture of the KAOSTools tool suite.
  \in{Section}[sec:kt_tools] details the command-line tools available to the
  risk and requirement analysis for applying the techniques.
  
  All tools described in this chapter are written in C\# and runs on Windows,
  Linux, and MacOS. The source code is freely available under MIT Licence
  \cite[Cai17b,Cai17d,Cai17f]. Packaged versions are also available in NuGet
  \cite[NuGet].

  \startsection[reference=sec:kt_lang,title={KAOSTools Specification Language}]
  
  The techniques presented in the previous chapter rely on a goal and an
  obstacle model. To specify such models, the tools take as an input a textual
  specification of these models. This section presents the specification
  language.
  
  In the language, declaring an element starts with the keyword
  \typeKAOSTool{declare} and ends with the keyword \typeKAOSTool{end}. All
  elements have a unique identifier that is used to reference the element.
  Attributes precisely specify each element. Attributes start with a keyword,
  such as \typeKAOSTool{name} or \typeKAOSTool{definition}, and ends with a
  value, such as a string or an integer. The accepted values depend on the
  attribute.
  
  The following describes the attributes for each element, such as goals and
  obstacles, and provides examples of such specifications.
  
  \subsubject{Goals, soft goals, domain properties, and domain hypotheses}
  
  To declare a goal, the keyword \typeKAOSTool{declare} must be followed by the
  keyword \typeKAOSTool{goal}. \in{Code}[code:goal_1] provides an example to
  specify the goal \goal{Achieve [Make Up Water Provided When Loss Of
  Cooling]}. The unique identifier \typeKAOSTool{make_up_water_provided}
  immediately follows the keyword \typeKAOSTool{goal} between square brackets.
  Comments starts with a dash (\typeKAOSTool{#}) and ends at the end of the
  line. \in{Table}[tab:goal_attributes] describes the attributes for specifying
  goals.
  
    \placecode[][code:goal_1]
    {Specification of the goal \goal{Achieve [Make Up Water Provided When Loss Of Cooling]}.}
    {\startKAOSTool
      declare goal [ make_up_water_provided ]
        name "Achieve [Make Up Water Provided When Loss Of Cooling]"
        definition 
          "Make up water shall be provided when loss of cooling occurs."
        rsr .99 # or rsr 99%
        refinedby response_to_loss_of_cooling, pump_on_when_activated
      end
    \stopKAOSTool}
  
  \placetable[][tab:goal_attributes]
  {Attributes for goal specification}
  {
      \setupTABLE[c][each][align={right,lohi},frame=off,offset=0pt]
      \setupTABLE[r][1][style=bold,bottomframe=on,boffset=4pt]
      \setupTABLE[c][1,2,3][roffset=4pt,loffset=4pt]
      \setupTABLE[r][2][toffset=4pt]
      \setupTABLE[c][1][width=3cm]
      \setupTABLE[c][2][width=3cm]
      \setupTABLE[c][3][width=6cm]
      \switchtobodyfont[small]
      \bTABLE
  \bTR \bTD Attribute                   \eTD \bTD Definition                  \eTD \bTD Accepted value \eTD \eTR
  \bTR \bTD \typeKAOSTool{name}         \eTD \bTD Name                        \eTD \bTD String \eTD \eTR
  \bTR \bTD \typeKAOSTool{definition}   \eTD \bTD Definition                  \eTD \bTD String \eTD \eTR
  \bTR \bTD \typeKAOSTool{refinedby}    \eTD \bTD Goal AND-Refinement          \eTD \bTD List of identifiers \eTD \eTR
  \bTR \bTD \typeKAOSTool{rsr}          \eTD \bTD Required Satisfaction Rate  \eTD \bTD Number or percentage \eTD \eTR
  \bTR \bTD \typeKAOSTool{obstructedby} \eTD \bTD Goal obstruction            \eTD \bTD Identifier \eTD \eTR
  \bTR \bTD \typeKAOSTool{assignedto}   \eTD \bTD Goal assignment             \eTD \bTD Identifier \eTD \eTR
  \bTR \bTD \typeKAOSTool{formalspec}   \eTD \bTD Formal specification        \eTD \bTD Formal specification \eTD \eTR
  \eTABLE
  }
  
  \noindent {\bf Goal refinement.} Goal refinements are specified using the
  keyword \typeKAOSTool{refinedby}. The list of identifiers shall refer to other
  goals, domain properties or domain hypotheses; These forms an AND-Refinement.
  If an identifier is not explicitly defined, a goal with the corresponding
  identifier is added to the model. For example, the \in{Code}[code:goal_1]
  specify a single AND-Refinement with two subgoals,
  \typeKAOSTool{response_to_loss_of_cooling} and
  \typeKAOSTool{pump_on_when_activated}.
  
  Multiple \typeKAOSTool{refinedby} attributes are used to specify alternative
  refinements. Goal refinements might be decorated with their refinement
  pattern using the name of the pattern between square brackets, as seen in
  \in{Code}[code:goal_2]. In addition, a partial entailment can be specified
  for the subgoals; the probability to satisfy the parent goal alone is
  specified between brackets after the subgoal identifier.
  \in{Code}[code:goal_2] specify that subgoal
  \typeKAOSTool{amb_mobilized_when_amb_allocated_on_ road} satisfy its parent
  goal with a probability $.8$.
  
  \placecode[][code:goal_2]
      {Specification of the refinement pattern and partial entailment.}
      {\startKAOSTool
      refinedby [case]
        amb_mobilized_when_amb_allocated_on_road [.8], 
        amb_mobilized_when_amb_allocated_at_station [.2]
      \stopKAOSTool}
      
  Similarly to goals, soft goals can be specified using the keywork
  \typeKAOSTool{softgoal}. The accepted attributes are \typeKAOSTool{name} and
  \typeKAOSTool{definition}. Domain properties and domain hypotheses are
  specified using the keyword \typeKAOSTool{domprop} and \typeKAOSTool{domhyp}
  respectively. The accepted attributes are summarized in
  \in{Table}[tab:dom_attributes].
  
  \placetable[here][tab:dom_attributes]
  {Attributes for domain properties and hypotheses specification}
  {
      \setupTABLE[c][each][align={right,lohi},frame=off,offset=0pt]
      \setupTABLE[r][1][style=bold,bottomframe=on,boffset=4pt]
      \setupTABLE[c][1,2,3][roffset=4pt,loffset=4pt]
      \setupTABLE[r][2][toffset=4pt]
      \setupTABLE[c][1][width=3cm]
      \setupTABLE[c][2][width=3cm]
      \setupTABLE[c][3][width=6cm]
      \switchtobodyfont[small]
      \bTABLE
  \bTR \bTD Attribute                   \eTD \bTD Definition                  \eTD \bTD Accepted value \eTD \eTR
  \bTR \bTD \typeKAOSTool{name}         \eTD \bTD Name                        \eTD \bTD String \eTD \eTR
  \bTR \bTD \typeKAOSTool{definition}   \eTD \bTD Definition                  \eTD \bTD String \eTD \eTR
  \bTR \bTD \typeKAOSTool{formalspec}   \eTD \bTD Formal specification        \eTD \bTD Formal specification \eTD \eTR
  \eTABLE
  }
  
  \subsubject{Obstacles} 
  
  Obstacles are specified using the keyword \typeKAOSTool{obstacle} followed by
  their unique identifier between square brackets. \in{Code}[code:obstacle_1]
  shows the specification for the non-leaf obstacle \obstacle{Valve Not Opened
  And Requested}. \in{Table}[tab:obstacle_attributes] shows the attributes
  available to specify obstacles.
  
  \placetable[here][tab:obstacle_attributes]
  {Attributes for obstacle specification}
  {
      \setupTABLE[c][each][align={right,lohi},frame=off,offset=0pt]
      \setupTABLE[r][1][style=bold,bottomframe=on,boffset=4pt]
      \setupTABLE[c][1,2,3][roffset=4pt,loffset=4pt]
      \setupTABLE[r][2][toffset=4pt]
      \setupTABLE[c][1][width=3cm]
      \setupTABLE[c][2][width=3cm]
      \setupTABLE[c][3][width=6cm]
      \switchtobodyfont[small]
      \bTABLE
  \bTR \bTD Attribute                   \eTD \bTD Definition                  \eTD \bTD Accepted value \eTD \eTR
  \bTR \bTD \typeKAOSTool{name}         \eTD \bTD Name                        \eTD \bTD String \eTD \eTR
  \bTR \bTD \typeKAOSTool{definition}   \eTD \bTD Definition                  \eTD \bTD String \eTD \eTR
  \bTR \bTD \typeKAOSTool{refinedby}    \eTD \bTD Obstacle AND-Refinement     \eTD \bTD List of identifiers \eTD \eTR
  \bTR \bTD \typeKAOSTool{resolvedby}   \eTD \bTD Obstacle resolution         \eTD \bTD Identifier \eTD \eTR
  \bTR \bTD \typeKAOSTool{formalspec}   \eTD \bTD Formal specification        \eTD \bTD Formal specification \eTD \eTR
  \bTR \bTD \typeKAOSTool{esr}          \eTD \bTD Estimated Satisfaction Rate \eTD \bTD Number, probability distribution or percentage \eTD \eTR
  \eTABLE
  }
      
  \noindent {\bf Obstacle refinements.} The specification of obstacle
  refinements is similar to the specification of goal refinements. However,
  identifiers must refer to obstacles, domain properties or domain hypotheses.
  If an identifier is not explicitly defined, a new obstacle with the
  corresponding identifier is added to the model. \in{Code}[code:obstacle_1]
  shows 4 alternative refinements with a single obstacle each.
  
  \placecode[][code:obstacle_1]
      {Specification of the obstacles \obstacle{Valve Not Opened And Requested}.}
      {\startKAOSTool
      declare obstacle [ valve_not_open ]
        name "Valve Not Opened And Requested"
        refinedby valve_failure
        refinedby no_power_available
        refinedby unavailable_due_to_maintenance
        refinedby valve_electronic_failure
      end
      \stopKAOSTool}
  
  \noindent {\bf Goal obstructions.} Obstructions to goals are specified using
  the keyword \typeKAOSTool{obstructedby}. The value of the attribute is a
  single identifier referring to an obstacle\emdash{}if not explicitly
  defined, a new obstacle will be added. \in{Code}[code:goal_3] shows how the
  obstruction of \goal{Achieve [Make Up Pump Motor On When Water Requested]} is
  specified.
  
  \placecode[][code:goal_3]
      {Specification of the goal \goal{Achieve [Make Up Pump Motor On When Water Requested]}.}
      {\startKAOSTool  
        declare goal [ pump_motor_on ]
          ...
          obstructedby motor_not_on
        end
      \stopKAOSTool}
  
  \noindent {\bf Obstacle satisfaction rate.} Leaf obstacles are estimated by
  experts. The specification language supports both single-value and
  multi-value satisfaction rates. \in{Code}[code:obstacle_2] shows the
  specification of the obstacle \obstacle{Valve Mechanical Failure} with a
  single-value $.002$ satisfaction rate, whereas \in{Code}[code:obstacle_3]
  shows the specification of the obstacle \obstacle{Pump Mechanical Failure}
  with a multi-value satisfaction rate. The supported multi-value satisfaction
  rates are summarized in \in{Table}[tab:multisatrate].
  
  \placecode[][code:obstacle_2]
      {Specification of the obstacle \obstacle{Valve Mechanical Failure}.}
      {\startKAOSTool
      declare obstacle [ valve_failure ]
        name "Valve Mechanical Failure"
        resolvedby [restoration:make_up_water_provided] 
                   cooling_system_repaired
        esr 0.002
      end
      \stopKAOSTool}
      
    \placecode[][code:obstacle_3]
        {Specification of the obstacle \obstacle{Valve Not Opened When Valve Control Not Unavailable}.}
        {\startKAOSTool
@experts.quantiles "(10%, 50%, 90%)"
declare obstacle [ manual_opening_valve_failure ]
  ...
  esr quantile[0.001,0.002,0.003]
end
        \stopKAOSTool}
  
  As seen in \in{Chapter}[chap:knowledge-uncertainty], a quantile is a single
  probability value attached to a cumulative probability. To specify the
  cumulative probabilities, model attribute \typeKAOSTool{@experts.quantiles}
  must be specified. For example, \in{Code}[code:obstacle_3] specify that
  $10\%$ of the values are below $0.001$, $50\%$ are below $0.002$ and $90\%$
  of the values are below $0.003$. The model attribute
  \typeKAOSTool{@experts.quantiles} shall only be specified once for the model.
  
  \placetable[here][tab:multisatrate]
  {Supported multi-value satisfaction rates}
  {
  \setupTABLE[c][each][align={right,lohi},frame=off,offset=0pt]
  \setupTABLE[r][1][style=bold,bottomframe=on,boffset=4pt]
  \setupTABLE[c][1,2,3][roffset=4pt,loffset=4pt]
  \setupTABLE[r][2][toffset=4pt]
  \switchtobodyfont[small]
  \bTABLE[option=stretch]
  \bTR \bTD Value                       \eTD \bTD Definition               \eTD \bTD Example \eTD \eTR
  \bTR \bTD \typeKAOSTool{beta}         \eTD \bTD Beta Distribution        \eTD \bTD \typeKAOSTool{beta[2,4]}. A beta distribution with $\alpha = 2$ and $\beta = 4$. See \cite[Vos08] for details on the parameters $\alpha$ and $\beta$. \eTD \eTR
  \bTR \bTD \typeKAOSTool{uniform}      \eTD \bTD Uniform Distribution     \eTD \bTD \typeKAOSTool{uniform[.1,.3]}. A uniform distribution between $.1$ and $.3$. \eTD \eTR
  \bTR \bTD \typeKAOSTool{triangular}   \eTD \bTD Triangular Distribution  \eTD \bTD \typeKAOSTool{triangular[.1,.4,.6]}. A triangular distribution with a $.1$ min-value, $.6$ max-value and a $.4$ mode. \eTD \eTR
  \bTR \bTD \typeKAOSTool{pert}         \eTD \bTD PERT Distribution        \eTD \bTD \typeKAOSTool{pert[.1,.4,.6]}. A PERT distribution with a $.1$ min-value, $.6$ max-value and a $.4$ mode. See \cite[Vos08] for details about the PERT distribution. \eTD \eTR
  \bTR \bTD \typeKAOSTool{quantiles}    \eTD \bTD Quantiles                \eTD \bTD \typeKAOSTool{quantiles[.1,.4,.6]}. A list of quantiles. \eTD \eTR
  \eTABLE
  }

  \noindent {\bf Obstacle resolutions.} Obstacles resolutions are specified
  using the keyword \typeKAOSTool{resolvedby} followed by an identifier to a
  goal\emdash{}if not explicitely defined, a new goal will be added.
  \in{Code}[code:obstacle_2] shows the specification of the resolution by the
  goal \typeKAOSTool{cooling_system_repaired}. The resolution tactic might be
  specified between square brackets (\typeKAOSTool{restoration} in
  \in{Code}[code:obstacle_2]), followed by the anchor goal
  (\typeKAOSTool{make_up_water_provided} in the example).
  \in{Table}[tab:kt_rtact] summarizes the supported resolution tactics.

  \placetable[here][tab:kt_rtact]
  {Supported multi-value satisfaction rates}
  {
  \setupTABLE[c][each][align={right,lohi},frame=off,offset=0pt]
  \setupTABLE[r][1][style=bold,bottomframe=on,boffset=4pt]
  \setupTABLE[c][1,2,3][roffset=4pt,loffset=4pt]
  \setupTABLE[r][2][toffset=4pt]
  \switchtobodyfont[small]
  \bTABLE[option=stretch]
  \bTR \bTD Resolution tactic                 \eTD \bTD Definition               \eTD \eTR
  \bTR \bTD \typeKAOSTool{substitution}       \eTD \bTD Goal substitution        \eTD \eTR
  \bTR \bTD \typeKAOSTool{prevention}         \eTD \bTD Obstacle prevention      \eTD \eTR
  \bTR \bTD \typeKAOSTool{obstacle_reduction} \eTD \bTD Obstacle reduction       \eTD \eTR
  \bTR \bTD \typeKAOSTool{restoration}        \eTD \bTD Goal restoration         \eTD \eTR
  \bTR \bTD \typeKAOSTool{weakening}          \eTD \bTD Weakening                \eTD \eTR
%  \bTR \bTD \typeKAOSTool{mitigation}         \eTD \bTD Mitigation (strong or weak) \eTD \eTR
  \bTR \bTD \typeKAOSTool{weak_mitigation}    \eTD \bTD Weak mitigation          \eTD \eTR
  \bTR \bTD \typeKAOSTool{strong_mitigation}  \eTD \bTD Strong mitigation        \eTD \eTR
  \eTABLE
  }
  
  %\page[yes]
  
  \subsubject{Agents}
  
  Agents are specified using the keyword \typeKAOSTool{agent} followed by their
  unique identifier between square brackets. \in{Table}[tab:agent_attributes]
  summarizes the attributes to specify agents. \in{Code}[code:agent_1] provides
  an example of agent specification.
        
  \placecode[here][code:agent_1]
    {Specification of the agent \agent{Controlling Software}.}
    {\startKAOSTool
    declare agent [ operator ]
      name "Operator"
    end
    \stopKAOSTool}
  
  \noindent {\bf Agent assignments.} An agent assignment to a goal is specified
  using the keyword \typeKAOSTool{assignedto}. \in{Code}[code:agent_2] shows
  how the agent \agent{Operator} is assigned to the goal \goal{Achieve [Valve
  Opened When Unavailable Due To Maintenance]}. If not explicitely defined, an
  agent with the corresponding identifier is added to the model.
  
  \placecode[here][code:agent_2]
    {Specification of the goal \goal{Achieve [Valve Opened When Unavailable Due To Maintenance]}.}
    {\startKAOSTool
    declare goal [ manual_opening_of_valve ]
      name "Achieve [Valve Opened When Unavailable Due To Maintenance]"
      obstructedby manual_opening_valve_failure
      assignedto operator
    end
    \stopKAOSTool}
  
  \placetable[here][tab:agent_attributes]
  {Attributes for agent specification}
  {
      \setupTABLE[c][each][align={right,lohi},frame=off,offset=0pt]
      \setupTABLE[r][1][style=bold,bottomframe=on,boffset=4pt]
      \setupTABLE[c][1,2,3][roffset=4pt,loffset=4pt]
      \setupTABLE[r][2][toffset=4pt]
      \setupTABLE[c][1][width=3cm]
      \setupTABLE[c][2][width=3cm]
      \setupTABLE[c][3][width=6cm]
      \switchtobodyfont[small]
      \bTABLE
  \bTR \bTD Attribute                   \eTD \bTD Definition                  \eTD \bTD Accepted value \eTD \eTR
  \bTR \bTD \typeKAOSTool{name}         \eTD \bTD Name                        \eTD \bTD String \eTD \eTR
  \bTR \bTD \typeKAOSTool{definition}   \eTD \bTD Definition                  \eTD \bTD String \eTD \eTR
  \bTR \bTD \typeKAOSTool{type}         \eTD \bTD Whether the agent is a {\it software} or a {\it environment} agent.
     \eTD \bTD \typeKAOSTool{software} or \typeKAOSTool{environment} \eTD \eTR
  \eTABLE
  }
  
  \subsubject{Predicates and formal specifications}
  
  Formal specification of goals and obstacles can be provided using the keyword
  \typeKAOSTool{formalspec}. The language supports first-order, linear temporal
  logic, assertions. A complete BNF grammar is available in
  \in{Appendix}[chap:bnf]. \in{Code}[code:formalspec_1] provides an example of
  a formal specification.

  \placecode[here][code:formalspec_1]
    {Formal specification of the obstacle \predicate{Diesel Generator Not Started}.}
    {\startKAOSTool
    declare obstacle [ diesel_generator_not_started ]
      name "Diesel Generator Not Started"
      formalspec 
        sooner-or-later ( dieselGeneratorRequested and
          always, for more than 5 minutes, 
          not (exists g: DieselGenerator . generatorStarted(g))
        )
      probability 0.002
    end
    \stopKAOSTool} 
  
  The identifier \typeKAOSTool{generatorStarted} refers to a predicate with a
  single argument. Predicates can be explicitly specified using the keyword
  \typeKAOSTool{predicate} followed by a unique identifier.
  \in{Code}[code:pred_1] shows the specification of the predicate
  \predicate{GeneratorStarted}. \in{Table}[tab:predicate_attributes] provides
  the attributes for specifying predicates.
  
  \placecode[here][code:pred_1]
    {Specification of the predicate \predicate{ValveOpened}.}
    {\startKAOSTool
    declare predicate [ generatorStarted ]
      name "GeneratorStarted"
      argument g: dieselGenerator
      formalspec g.Started
    end
    \stopKAOSTool}  
  
  \noindent {\bf Arguments.} As seen in \in{Code}[code:pred_1], predicates
  might have arguments. The formal specification of the predicate can refer to
  the argument variables. In the example, the formal specification states that
  the predicate is true if the attribute \typeKAOSTool{Started} is true for the
  instance of \typeKAOSTool{DieselGenerator} provided as an argument. When
  referring to a predicate in a formal specification, the actual arguments must
  match the order and the respective entity types specified in the 
  argument list.
  
  \subsubject{Entities, associations and types}
  
  Formal specifications, in the end, refer to entities and associations to
  express their truth value. All the referenced entities, entity attributes,
  associations, association attributes and types are automatically inferred
  from the formal specification of goals, domain properties, domain hypotheses,
  obstacles, and predicates. This revealed to be particularly useful to check
  whether formal specifications refer to the appropriate elements and avoid
  similar names (e.g. \typeKAOSTool{dieselGenerator} and
  \typeKAOSTool{diesel_generator}).
  
  The analyst might enrich the object model by explicitly defining the
  elements. \in{Tables}[tab:entity_attributes],
  \in{}[tab:association_attributes] and \in{}[tab:type_attributes] provides the
  attributes for specifying entities, associations and given types. The
  following briefly shows how an object model can be specified.
  
  \startplacetable[location={page,none}]
    \startfloatcombination[nx=1,ny=4,after={\blank[10mm]}]
   
      \placetable[here][tab:predicate_attributes]
      {Attributes for predicate specification}
      {
      \setupTABLE[c][each][align={right,lohi},frame=off,offset=0pt]
      \setupTABLE[r][1][style=bold,bottomframe=on,boffset=4pt]
      \setupTABLE[c][1,2,3][roffset=4pt,loffset=4pt]
      \setupTABLE[r][2][toffset=4pt]
      \setupTABLE[c][1][width=3cm]
      \setupTABLE[c][2][width=3cm]
      \setupTABLE[c][3][width=6cm]
      \switchtobodyfont[small]
      \bTABLE
      \bTR \bTD Attribute                   \eTD \bTD Definition           \eTD \bTD Accepted value \eTD \eTR
      \bTR \bTD \typeKAOSTool{name}         \eTD \bTD Name                 \eTD \bTD String \eTD \eTR
      \bTR \bTD \typeKAOSTool{definition}   \eTD \bTD Definition           \eTD \bTD String \eTD \eTR
      \bTR \bTD \typeKAOSTool{formalspec}   \eTD \bTD Formal specification \eTD \bTD Formal specification \eTD \eTR
      \bTR \bTD \typeKAOSTool{argument}     \eTD \bTD Formal arguments     \eTD \bTD An identifier followed by a colon (\typeKAOSTool{:}) and an identifier \eTD \eTR
      \eTABLE
      }
    
      \placetable[here][tab:entity_attributes]
      {Attributes for entity specification}
      {
      \setupTABLE[c][each][align={right,lohi},frame=off,offset=0pt]
      \setupTABLE[r][1][style=bold,bottomframe=on,boffset=4pt]
      \setupTABLE[c][1,2,3][roffset=4pt,loffset=4pt]
      \setupTABLE[r][2][toffset=4pt]
      \setupTABLE[c][1][width=3cm]
      \setupTABLE[c][2][width=3cm]
      \setupTABLE[c][3][width=6cm]
      \switchtobodyfont[small]
      \bTABLE
      \bTR \bTD Attribute                   \eTD \bTD Definition   \eTD \bTD Accepted value \eTD \eTR
      \bTR \bTD \typeKAOSTool{name}         \eTD \bTD Name         \eTD \bTD String \eTD \eTR
      \bTR \bTD \typeKAOSTool{definition}   \eTD \bTD Definition   \eTD \bTD String \eTD \eTR
      \bTR \bTD \typeKAOSTool{type}   \eTD \bTD Entity type  \eTD \bTD \typeKAOSTool{software}, \typeKAOSTool{environment} or \typeKAOSTool{shared} \eTD \eTR
      \bTR \bTD \typeKAOSTool{isa}          \eTD \bTD Inheritance  \eTD \bTD Identifier \eTD \eTR
      \bTR \bTD \typeKAOSTool{attribute}    \eTD \bTD Attribute    \eTD \bTD An identifier optionally followed by a colon (\typeKAOSTool{:}) and an identifier \eTD \eTR
      \eTABLE
      }
 
      \placetable[here][tab:association_attributes]
      {Attributes for association specification}
      {
      \setupTABLE[c][each][align={right,lohi},frame=off,offset=0pt]
      \setupTABLE[r][1][style=bold,bottomframe=on,boffset=4pt]
      \setupTABLE[c][1,2,3][roffset=4pt,loffset=4pt]
      \setupTABLE[r][2][toffset=4pt]
      \setupTABLE[c][1][width=3cm]
      \setupTABLE[c][2][width=3cm]
      \setupTABLE[c][3][width=6cm]
      \switchtobodyfont[small]
      \bTABLE
      \bTR \bTD Attribute                   \eTD \bTD Definition           \eTD \bTD Accepted value \eTD \eTR
      \bTR \bTD \typeKAOSTool{name}         \eTD \bTD Name                 \eTD \bTD String \eTD \eTR
      \bTR \bTD \typeKAOSTool{definition}   \eTD \bTD Definition           \eTD \bTD String \eTD \eTR
      \bTR \bTD \typeKAOSTool{attribute}    \eTD \bTD Attribute            \eTD \bTD An identifier optionally followed by a colon (\typeKAOSTool{:}) and an identifier \eTD \eTR
      \bTR \bTD \typeKAOSTool{link}         \eTD \bTD Linked entity        \eTD \bTD An identifier \eTD \eTR
      \eTABLE
      }  
 
      \placetable[here][tab:type_attributes]
      {Attributes for given type specification}
      {
      \setupTABLE[c][each][align={right,lohi},frame=off,offset=0pt]
      \setupTABLE[r][1][style=bold,bottomframe=on,boffset=4pt]
      \setupTABLE[c][1,2,3][roffset=4pt,loffset=4pt]
      \setupTABLE[r][2][toffset=4pt]
      \setupTABLE[c][1][width=3cm]
      \setupTABLE[c][2][width=3cm]
      \setupTABLE[c][3][width=6cm]
      \switchtobodyfont[small]
      \bTABLE
      \bTR \bTD Attribute                   \eTD \bTD Definition           \eTD \bTD Accepted value \eTD \eTR
      \bTR \bTD \typeKAOSTool{name}         \eTD \bTD Name                 \eTD \bTD String \eTD \eTR
      \bTR \bTD \typeKAOSTool{definition}   \eTD \bTD Definition           \eTD \bTD String \eTD \eTR
      \eTABLE
      }
    \stopfloatcombination
  \stopplacetable
        
  \noindent {\bf Entity and association attributes.} An entity or association
  attribute is composed of an attribute identifier and a given type. A given
  type is a base type, such as boolean or string, or a domain-specific type
  such as {\it MotorStatus}. These custom types might be declared using
  \typeKAOSTool{type} keyword. \in{Code}[code:object_1] shows an example of
  attribute specification.
  
  \placecode[here][code:object_1]
    {Specification of the entity \entity{DieselGenerator}.}
    {\startKAOSTool
    declare entity [ dieselGenerator ]
      name "DieselGenerator"
      attribute Started: bool
    end
    \stopKAOSTool} 
  
  \noindent {\bf Linking associations and entities.} Entities connect to other
  entities by associations. The attribute \typeKAOSTool{link} specify the
  association connecting entities, optionally with the corresponding
  multiplicities \cite[Lam09]. \in{Code}[code:object_2] provides an example
  with an association linking the entities \entity{DieselGenerator} and
  \entity{ElectroValve}. In the association, there is at most one diesel
  generator, but there might have more than one corresponding valve.
  
  \placecode[here][code:object_2]
    {Specification of the association \association{GeneratorValves}.}
    {\startKAOSTool
    declare association [ generator_valves ]
      name "GeneratorValves"
      link [1] dieselGenerator
      link [1,N] electroValve
    end
    \stopKAOSTool} 
  
  \subsubject{Experts and Calibration variables}
  
  To support the techniques presented in
  \in{Chapter}[chap:knowledge-uncertainty], the specification language was
  extended to explicitly declare experts and calibration variables. Experts are
  declared using the keyword \typeKAOSTool{expert} followed by their unique
  identifier. Calibration variables are declared using the keyword
  \typeKAOSTool{calibration} followed by their unique identifier.
  \in{Code}[code:expert_1] declares two experts and one calibration variable.
  \in{Tables}[tab:expert_attributes] and \in{}[tab:cal_attributes] describes
  the attributes availables.
  
  \placecode[here][code:expert_1]
    {Specification of the association \association{GeneratorValves}.}
    {\startKAOSTool  
    declare expert [ expert1 ] name "Expert 1" end
    declare expert [ expert2 ] name "Expert 2" end

    declare calibration [ calibration1 ]
      name "Push Button Broken"
      esr 4.37%
      esr[expert1] quantile[3.812%, 4.285%, 4.759%]
      esr[expert2] quantile[3.961%, 4.453%, 4.945%]
    end
    \stopKAOSTool} 
 
    \placetable[here][tab:expert_attributes]
    {Attributes for expert specification}
    {
    \setupTABLE[c][each][align={right,lohi},frame=off,offset=0pt]
    \setupTABLE[r][1][style=bold,bottomframe=on,boffset=4pt]
    \setupTABLE[c][1,2,3][roffset=4pt,loffset=4pt]
    \setupTABLE[r][2][toffset=4pt]
    \setupTABLE[c][1][width=3cm]
    \setupTABLE[c][2][width=3cm]
    \setupTABLE[c][3][width=6cm]
    \switchtobodyfont[small]
    \bTABLE
    \bTR \bTD Attribute                   \eTD \bTD Definition           \eTD \bTD Accepted value \eTD \eTR
    \bTR \bTD \typeKAOSTool{name}         \eTD \bTD Name                 \eTD \bTD String \eTD \eTR
    \eTABLE
    }  

    \placetable[here][tab:cal_attributes]
    {Attributes for calibration variable specification}
    {
    \setupTABLE[c][each][align={right,lohi},frame=off,offset=0pt]
    \setupTABLE[r][1][style=bold,bottomframe=on,boffset=4pt]
    \setupTABLE[c][1,2,3][roffset=4pt,loffset=4pt]
    \setupTABLE[r][2][toffset=4pt]
    \setupTABLE[c][1][width=3cm]
    \setupTABLE[c][2][width=3cm]
    \setupTABLE[c][3][width=6cm]
    \switchtobodyfont[small]
    \bTABLE
    \bTR \bTD Attribute                   \eTD \bTD Definition           \eTD \bTD Accepted value \eTD \eTR
    \bTR \bTD \typeKAOSTool{name}         \eTD \bTD Name                 \eTD \bTD String \eTD \eTR
    \bTR \bTD \typeKAOSTool{definition}   \eTD \bTD Definition           \eTD \bTD String \eTD \eTR
    \bTR \bTD \typeKAOSTool{esr}          \eTD \bTD Estimated Satisfaction Rate \eTD \bTD Number, probability distribution or percentage \eTD \eTR
    \eTABLE
    }
  
  \noindent {\bf Expert satisfaction rate.} For a calibration variable, the
  reference value is specified using the \typeKAOSTool{esr} attribute without
  specifying an expert. The estimates by the experts are specified using the
  \typeKAOSTool{esr} attribute while specifying the expert, between square
  brackets. When specifying the estimated satisfaction rate by an expert in an
  obstacle, the expert is also specified in square brackets (see
  \in{Code}[code:expert_2]).
  
  \placecode[here][code:expert_2]
    {Specification of the association \association{GeneratorValves}.}
    {\startKAOSTool  
      declare obstacle [ power_cable_failure ]
        name "Power Cabling Failure"
        esr[expert1] quantile[0.015,0.01752,0.02]
        esr[expert2] quantile[0.001,0.002,0.003]
      end
    \stopKAOSTool} 
 
  \subsubject{Additional features}
  
  In addition to the constructs presented before, it is possible to define
  custom attributes, split the model into different files, and specify model
  attributes such as the title of the model, the version, and the authors.
  Utilities are also available to help the analyst writing specifications.
  
  \noindent {\bf Custom attributes.} All elements might have extra attributes
  not specified by the grammar. These custom attributes must start with
  \typeKAOSTool{$}. It showed useful for extending the specification language
  to support specific tools without changing the grammar. For example,
  \in{Code}[code:custom_attr] shows how activation and deactivation procedures
  are specified\emdash{}the risk-driven runtime adaptation tools did not
  require changes in the KAOSTools specification language.
  
  \placecode[here][code:custom_attr]
    {Specification of the association \association{GeneratorValves}.}
    {\startKAOSTool
    declare goal [ speed_acquired_by_camera ]
      name "Achieve [SpeedAcquiredEvery5SecondsByCamera]"
      refinedby images_acquired, 
                tracers_identified, 
                speed_acquired_from_tracers
      $ondeploy "DeployCamera"
      $onwithold "DeployUltrasound"
    end
    \stopKAOSTool} 
  
  \noindent {\bf Splitting into multiple files.} It is easier to handle a large
  model by splitting it into multiple files. Sub-models can be imported using
  \typeKAOSTool{import} followed by the path to the sub-model. Adding
  attributes to an element already declared facilitate the splitting into
  multiple files. The element is \quote{declared} using the keyword
  \typeKAOSTool{override}, in place of \typeKAOSTool{declare}, to add an
  attribute. \typeKAOSTool{override} showed useful to split goals in a file and
  obstacles in another while specifying the obstructions in the file with the
  obstacle specifications. \in{Code}[code:import] shows an example of such
  usage.
  
  \placecode[here][code:import]
    {Splitting model specification accross multiple files.}
    {\startKAOSTool
    # In the file 'goal.kaos'
    declare goal [ manual_opening_of_valve ]
      name "Achieve [Valve Opened When Unavailable Due To Maintenance]"
      assignedto operator
    end
    ...
    import "obstacle.kaos"
    
    # In the file 'obstacle.kaos'
    override goal [ manual_opening_of_valve ]
      obstructedby manual_opening_valve_failure
    end
    declare obstacle [ manual_opening_valve_failure ]
      name "Valve Not Opened When Valve Control Not Unavailable"
      probability 0.0001
    end
    ...
    \stopKAOSTool}
    
  \noindent {\bf Model attributes.} \in{Code}[code:model_attributes] shows how
  to specify the title, the authors and the version of the specified goal model.
  
    \placecode[here][code:model_attributes]
      {Model attributes.}
      {\startKAOSTool
    @title "Spent Fuel Pool"
    @author "Antoine Cailliau, Axel van Lamsweerde"
    @version "0.1"
    \stopKAOSTool}
  
  \noindent {\bf Utilities.} The tool \tool{ModelChecker} is available to check
  the correct syntax and provides statistics for the model. For example,
  
  \startCLI
  $ mono ModelChecker.exe model.kaos 
  Goals: 20
  Root goals: 1
  Leaf goals: 14
  Goal refinements: 6

  Obstacles: 16
  Root obstacles: 6
  Leaf obstacles: 12
  Obstacle refinements: 11

  Resolutions: 10

  Generated goal exceptions: 9 (distributed over 3 goals)
  Generated provided assumption: 23 (distributed over 6 goals)
  \stopCLI
  
  The tool
  \tool{OmnigraffleExport} generates several goal and obstacle diagrams
  corresponding to the graphical syntax of the model provided in textual
  format. The analyst might edit the generated diagrams with OmniGraffle
  \cite[Ols10]. This tool was used to produce all goal and obstacles diagrams
  presented in the thesis. Last, a bundle is available for the TextMate text
  editor \cite[Gra07] to provide syntax highlighting.
  
  \stopsection

  \startsection[reference=sec:kt_arch,title={KAOSTools Architecture}]
  
    \in{Figure}[fig:kt_architecture] shows the architecture of the KAOSTools
    tool suite: Boxes shows libraries, dashed boxes are external libraries.
    External libraries were developed to support KAOSTools and might be used
    outside the framework. Arrows indicates {\it USE} links \cite[Med10].
    Brief descriptions of the modules are provided below.
  
    \placefigure[]
         [fig:kt_architecture]
         {KAOSTools Architecture.}
      {\externalfigure[../images/chap8/architecture.pdf]}
    
    \noindent {\bf Core.} This package contains the classes representing the
    probabilistic goals, domain properties, obstacles, and so forth. It mainly
    support the \in{Chapter}[chap:proba-framework]. It supports the other
    libraries by providing a set of helpers for selecting, modifying, and
    creating KAOS elements.
  
    \noindent {\bf Parsing.} This package contains all the classes related to
    the parsing of the KAOSTools specification language. On parsing, a model is
    built in memory using the helpers provided in \tool{Core}.
  
    \noindent {\bf Propagators.} This package implements the techniques
    presented in \in{Chapter}[chap:assessing]. Classes implement both the
    Pattern-based and BDD-based computations. The package also supports the
    computation of satisfaction rate with their uncertainty margins. This
    library relies on \tool{BDDSharp} to support the BDD-based computation.
    \tool{BDDSharp} is a freely available library to create and manipulates
    {Reduced Order Binary Decision Diagrams} (ROBDDs) in C\# \cite[Cai17d].
    Ease of deployment and availability on multiple OSes motivated the creation
    of the library; If required, more efficient BDD libraries might replace
    \tool{BDDSharp} in the KAOSTools suite.
  
    \noindent {\bf CriticalObstacles.} This package implements the techniques
    presented in \in{Chapter}[chap:assessing] to highlight the most critical
    obstacles. This packages also support the highlighting of the most critical
    obstacles with their knowledge uncertainty margins as presented in
    \in{Chapter}[chap:knowledge-uncertainty]. This library generates the
    violation diagrams.
  
    \noindent {\bf Integrators.} This package implements the techniques
    presented in \in{Chapter}[chap:controlling_obstacle] related to the
    integration of countermeasures. The package supports both {\it hard}
    integration (modifying the goal/obstacle model), and {\it soft} integration
    (using the constructs presented in \in{Section}[sec:exception_handling]).
  
    \noindent {\bf Optimizers.} This package implements the techniques
    presented in \in{Chapter}[chap:controlling_obstacle] related to the
    selection of countermeasures. The package implements the selection of most
    appropriate countermeasures for both single-value and multi-value leaf
    obstacle estimates.
    
    \noindent {\bf ExpertCombination.} This package implements the techniques
    presented in \in{Chapter}[chap:knowledge-uncertainty]. It supports the
    combination of multiple experts using both Cook's and Mendel-Sheridan's
    techniques. This library relies on \tool{ExpertOpinionSharp} to combine the
    expert opinions. \tool{ExpertOpinionSharp} was developed specifically for
    combining expert estimates using calibration variables \cite[Cai17e]. It
    was developed as a separate library as other approaches might reuse
    these combination techniques.
  
    \noindent {\bf Monitoring.} This package supports the techniques presented
    in \in{Chapter}[runtime]. The library relies on \tool{LTLSharp} for the
    generation and update of $LTL_3$ monitors. \tool{LTLSharp} is a library
    that provides model checking algorithms, algorithms to build Buchï
    automata, and $LTL_3$ monitors for C\# libraries and applications
    \cite[Cai17f].
    
  \stopsection

  \startsection[reference=sec:kt_tools,title={KAOSTools Tool Suite}]
  
    In addition to the libraries developed supporting the techniques, a series
    of command-line tools enables the analyst to run the techniques on their
    goal/obstacle models. This section briefly describes the tools. More
    detailed descriptions are available by passing the option {\tttf --help} to
    the command-line tools.

    \noindent {\bf Utils.Propagator.} This tool computes the satisfaction rate
    of the high-level goals.
    
    \startCLI
    $ mono Propagator.exe --root=make_up_water_provided model.kaos 
    Achieve [Make Up Water Provided When Loss Of Cooling]: 80.31 %
    \stopCLI
    
    \noindent {\bf Utils.UncertaintyPropagator.} This tool computes the
    satisfaction uncertainty and violation uncertainty for the high-level goals.
    
    \startCLI
    $ mono UncertaintyPropagator.exe --root=make_up_water_provided model-uncertainty.kaos 
    Achieve [Make Up Water Provided When Loss Of Cooling]:
      Mean: 78.67 %
      Required Satisfaction Rate: 80.00 %
      Violation Uncertainty: 71.80 %
      Uncertainty Spread: 0.02694
    \stopCLI
  
    \noindent {\bf Utils.ViolationDiagram.} This tool generates the data
    corresponding to the violation uncertainty diagrams. The violation diagram
    can then be generated using R \cite[Tea00].
  
    \noindent {\bf Utils.CMIntegrator.} This tool provides an interactive
    console to integrate countermeasures into an ideal goal model. The ideal
    goal model can then be exported.
    
    \startCLI
    $ mono CMIntegrator.exe model.kaos 
    > resolve pump_failure
    [0] Achieve [Cooling System Repaired]
    [1] Achieve [Redundant Pump Motor On When Primary Pump Failure]
    Select the countermeasure goal to integrate: 0 
    > 
    \stopCLI
      
    \noindent {\bf Utils.CMSelector.} This tool returns the list of the most
    appropriate countermeasures. {\bf Utils.CMSelectorUncertainty} returns the
    list of the most appropriate countermeasures when leaf obstacles are
    estimated together with their knowledge uncertainty margins.
    
    \startCLI
    $ mono CMSelectorUncertainty.exe --root=make_up_water_provided \
      model-uncertainty.kaos 
    Violation Uncertainty without countermeasures: 0.718
    Uncertainty Spread without countermeasures: 0.0269443101910917
    Required Satisfaction Rate: 0.8
    Minimal cost to guarantee RSR: 1
    Optimal selections (2):
    * [OptimalSelection: Resolutions={cooling_system_repaired}, 
                         Cost=1, 
                         ViolationUncertainty=0, 
                         UncertaintySpread=0]
    * [OptimalSelection: Resolutions={redundant_pump_started}, 
                         Cost=1, 
                         ViolationUncertainty=0, 
                         UncertaintySpread=0]

    --- Statistics ---
    Number of countermeasure goals: 6
    Number of possible selections: 63
    Number of safe selections: 48
    Number of tested selections (for minimal cost): 63
    Number of tested selections (for optimal selection): 6
    Maximal safe cost: 6
    Time to compute minimal cost: 00:00:00.3229049
    Time to compute optimal selections: 00:00:00.3229049
    \stopCLI
  
    \noindent {\bf Utils.Monitor.} This tool runs the monitoring of
    probabilistic obstacles, select the most appropriate countermeasures and
    pushes the activation/deactivation procedure that should be applied into a
    RabbitMQ queue \cite[Vid12]. \in{Section}[sec:bads] provides more details
    about the monitoring infrastructure.
    
  \stopsection

  \startsection[title={Summary}]
  
    This chapter presented the tools supporting the techniques presented in the
    previous chapters. The tools accept a textual specification of the
    goal/obstacle model. The modular architecture of KAOSTools enables new
    techniques to be implemented while leveraging the existing ones.
  
    The presented tools were used to generate all diagrams, charts, and
    computation results provided in the various examples through the thesis.
    The tools were used to evaluate the techniques, as presented in the next
    chapter.
  
  \stopsection

\stopchapter

\stopcomponent
