% !TEX root = thesis.tex

\startcomponent chap-6
\environment common
\product thesis

\chapter[runtime]{Handling Obstacles At System Runtime}

  Software systems are deployed in changing environments. The identification,
  assessment, and control of obstacles through countermeasures is based on
  environment assumptions and obstacle satisfaction rates that are determined at RE
  time. These influencing factors may however turn to be different at system
  runtime \cite[Fic95]. Some assumptions might no longer hold; new
  characteristics might emerge; experts’ estimates at RE time for obstacle
  assessment might prove inaccurate at system runtime; other estimates might
  not be available at RE time; and so forth. For better fit to the system’s
  goals under changing or originally unknown conditions, it might thus be
  better to defer decisions on selecting most appropriate countermeasures to
  system runtime \cite[Fea98].

  This chapter discusses how obstacles and goals may drive the runtime
  adaptation of a software system. In particular, it introduces how
  probabilistic obstacles can be monitored at runtime and how the monitored
  satisfaction rates can drive the dynamic selection of more appropriate
  countermeasures in view of changing conditions.

  The satisfaction rate of an obstacle is defined as the lowest state
  probability of satisfying its specification. Such precise characterization in
  terms of states and behaviors enables the monitoring of state probabilities.
  The satisfaction of a specification is monitored at runtime using
  LTL-monitoring techniques; its satisfaction rate is estimated by counting
  positive and negative occurences of the formula. The monitored satisfaction
  rate of leaf obstacles is then up-propagated to high-level goals in the goal
  model (as seen in \in{Chapter}[chap:assessing]). If the satisfaction rate of
  a high-level goal does not exceed its required satisfaction rate, an
  adaptation of the software system is required. The currently deployed
  countermeasures are reconsidered; more appropriate ones are selected (as in
  \in{Chapter}[chap:controlling_obstacle]) among those identified at RE time,
  in view of runtime information about changing conditions; and the newly
  selected countermeasures are deployed in the running system. The software
  adaptation process thus can be driven by the satisfaction of probabilistic
  goals. This provides traceable criteria for deciding when an adaptation is
  required and what adaptations should be performed to guarantee the
  satisfaction of the considered probabilistic goals.

  The chapter is structured as follows. \in{Section}[sec:overview] presents the
  overall approach for dynamically switching to most appropriate countermeasure
  at runtime. \in{Section}[sec:runtime_running_example] introduces a specific
  running example of runtime adaptation. \in{Section}[sec:monitoring] details
  how probabilistic obstacles can be monitored at run-time.
  \in{Section}[sec:obstaclebased_adaptation] details how probabilistic
  obstacles may drive the runtime adaptation process.
  \in{Section}[sec:sw_update] explains how the running software can be adapted
  from the derived goal model adaptations.
  \in{Section}[sec:update_k_uncertainty] discusses how knowledge uncertainty
  may affect obstacle monitoring and model adaptation.

  \startsection[reference=sec:overview, title={Overview of the approach}]
  
    The objective in this chapter is to let the system dynamically switch to
    more appropriate countermeasures to leaf obstacles in view of evolving
    environment conditions and obstacle satisfaction rates. The satisfaction
    rate of leaf obstacles was estimated at RE time; it can now be observed at
    system runtime. A set of alternative countermeasure goals to those leaf
    obstacles is assumed to be known; the countermeasures are identified and
    specified at RE time.

    A countermeasure to an obstacle should dynamically replace the current one
    when, unlike the latter, it makes the satisfaction rate of high-level goals
    exceed their required satisfaction rate. The satisfaction rate of
    high-level goals is obtained from the monitored satisfaction rate of leaf
    obstacles by up-propagation through the goal/obstacle model, using the
    technique presented in \in{Chapter}[chap:assessing]. The monitored
    satisfaction rate of a leaf obstacle is obtained by counting the observed
    behaviors.

    \noindent More specifically, our approach comprises 7 steps detailed in the
    next sections. The relation among steps is shown in
    \in{Figure}[fig:overview_approach].
    
    \placefigure[bottom]
   	  [fig:overview_approach]
   	  {Overview of the approach.}
      {\externalfigure[../images/chap7/overview_approach.pdf][scale=1000]}
  
    \startitemize[n]

      \item {\bf Build monitors for leaf obstacles.} The probabilistic leaf
      obstacles in obstacle refinement trees need be monitored at system
      runtime. Our approach extends the monitoring technique introduced in
      \cite[Bau11] for non-probabilistic LTL assertions, in order to monitor
      probabilistic LTL assertions at system runtime. This step builds, at RE
      time,  $LTL_3$ monitors for the leaf obstacles. The list of predicates
      to observe at runtime is thereby provided. (See
      \in{Section}[sec:building_monitor])

      \item {\bf Update obstacle monitors.} At runtime, the states of the
      monitored system are observed at a regular pace. (This pace may be chosen
      to fit a specific domain.) At every observation, a new virtual monitor
      for each leaf obstacle is started while existing monitors are updated.
      (See \in{Section}[sec:monitoring_based].)

      \item {\bf Compute high-level goal satisfaction rates.} The monitored
      satisfaction rate of leaf obstacles is up-propagated through
      obstacle/goal refinement trees up to high-level goals. (As explained in
      \in{Chapter}[chap:assessing], however performed here at runtime.)

      \item {\bf Selecte countermeasures.} The comparison between the monitored
      satisfaction rates obtained for those goals and their required
      satisfaction rate (RSR) determines whether the current countermeasures to
      the monitored obstacles are still appropriate. If a monitored goal
      satisfaction rate falls below the goal’s RSR, alternative more
      appropriate countermeasures are selected among those available. (As
      explained in \in{Section}[sec:selecting_cm])

      \item {\bf Update the RE model.} The goal/obstacle model is updated
      accordingly by integrating the new current countermeasures and updating
      the propagation in Step 4. (As explained in \in{Section}[sec:integrating])

      \item {\bf Deploy at runtime.} The software is automatically adapted
      according to the selected countermeasures in the new model. (See
      \in{Section}[sec:sw_update])

    \stopitemize
    
  \stopsection
  
  \startsection
    [reference=sec:runtime_running_example,
     title={Running example for runtime adaptation}]
     
    The running example for this section was inspired from a flood detection
    system \cite[Hug06,Hug06b]. This running example is commonly used to illustrates or evaluates
    other runtime adaptation techniques \cite[Saw10,Ben08,Wel11,Gol08].
    
    {\vskip.8\baselineskip}
    \framed[width=local,align={width},offset=10pt,after={\vskip-\baselineskip}
    ] {\it \quote{Hydrologists have traditionally approached the problem of modeling
    and predicting floods by deploying sensing and logging equipment in areas
    susceptible to flooding. River profile data from these sensors, such as
    depth and rate of flow, is then either transmitted off-site or collected
    manually. This data is then used as the input to complex spatial flood
    prediction models, which typically run on high-performance
    computers and Grid technologies. This data is also used as the basis for
    much simpler single-point models that can run on simpler devices.}\emdash{}\cite[Hug06]}
   
    The example system here monitors the speed and depth of the river. If
    the levels are criticals (approximated here as \quote{\it above a given
    threshold}), the locals are warned as soon as possible. The speed can be
    measured either using an ultrasound sensor or a camera-based sensor.
    \in{Figure}[fig:goal_model] shows a fragment of the goal model;
    \in{Figure}[fig:obstacle_model] shows a fragment of a corresponding obstacle model.
    
    \placefigure[here][fig:goal_model]
      {Goal model fragment for a Flood Detection System.}
      %{\startcombination[1*2]
        {\externalfigure[../images/chap7/goal_model_a.pdf]}{}
        %{\externalfigure[../images/chap7/goal_model_b.pdf]}{}
      %\stopcombination}
   
    \placefigure[bottom][fig:obstacle_model]
      {Obstacle model fragment for a Flood Detection System.}
      {\startcombination[2*1]
        {\externalfigure[../images/chap7/obstacle_model_a.pdf]}{}
        {\externalfigure[../images/chap7/obstacle_model_b.pdf]}{}
      \stopcombination}
     
  \stopsection
  
  \startsection
    [reference=sec:monitoring,
     title={Monitoring probabilistic obstacles}]
  
    At RE time, domain experts estimate the satisfaction rates of the leaf
    obstacles in obstacle refinement trees based on their knowledge of the system or their experience with
    similar systems as discussed in \in{Section}[sec:estimating_satrate]. These
    estimates might prove inaccurate at system runtime\emdash{}they might be
    too rough or environment properties or assumptions might have changed in
    the meantime. Moreover, some variables might be hard to estimate at RE
    time\emdash{}prior data might not be available or might be too costly to
    acquire; too many parameters might be involved; etc. Monitoring the actual
    satisfaction rate of leaf obstacles at system runtime helps filling this
    gap. Former estimates may be made more accurate; missing data may be
    made available.
  
    \startsubsection
      [reference=sec:building_monitor,
       title={Background: Monitoring non-probabilistic LTL assertions}]
  
      This section introduces the necessary background for monitoring
      non-probabilistic LTL assertions. 
      
      Monitoring the runtime satisfaction of an LTL formula relies on the
      finite trace observed so far, and on an automata-based expression of
      formula satisfaction. The approach presented in \cite[Bau11] is chosen as
      it reports LTL formula satisfaction or violation as early as possible.
      
      $LTL_3$, the LTL considered in \cite[Bau11], uses {\it true}, {\it
      false}, and {\it inconclusive} as truth values. A finite trace is labeled
      as {\it true} if any continuation of it satisfies the formula; {\it
      false} if any continuation falsifies the formula; and {\it inconclusive}
      otherwise. A $LTL_3$ formula monitor is a finite state machine (FSM) that
      reads finite traces and outputs the corresponding truth value. As
      suggested by \in{Figure}[fig:ltl3_approach], this FSM is built from the
      formula and its negation \cite[Bau11].
      
      \startitemize
      
        \item The associated non-deterministic Büchi automaton (NBA) is
        generated using a standard algorithm \cite[Bai08].
      
        \item A non-deterministic finite automaton (NFA) is then generated by
        performing an emptiness check for each NBA state. A state $s$ is
        labeled with {\it true} if a continuation satisfying the formula exists
        (that is, if the language corresponding to the NFA with initial state
        $s$ is not empty); it is {\it false} otherwise.
      
        \item The NFA is determinized to produce a deterministic finite
        automaton (DFA) using powersets construction \cite[Bai08].
      
      \stopitemize
    
      \placefigure[top]
     	  [fig:ltl3_approach]
     	  {Steps for computing monitor FSM from input formula $\phi$.}
        {\externalfigure[../images/chap7/ltl3_approach.pdf][scale=1000]}
      
      The monitor FSM results from the product of both DFAs. Let $(s_1,s_2)$
      denote a state in this product, where $s_1$ is the DFA state
      corresponding to the formula $\phi$, and $s_2$ the DFA state corresponding
      to its negation $\neg\phi$. This monitor FSM state is labeled as:
      
      \startitemize
      
        \item {\it true} if $s_1$ is labeled as {\it true} and $s_2$ as {\it
        false} (no continuation exists such that the formula is falsified);
      
        \item {\it false} if $s_1$ is labeled as {\it false} and $s_2$ as {\it
        true} (no continuation exists such that the formula is satisfied);
      
        \item {\it inconclusive} otherwise (a continuation exists such that
        the formula is satisfied, and a continuation exists such that the
        formula is not satisfied.)
      
      \stopitemize
      
      At runtime, the current state of the monitor FSM is updated according to
      the truth value of the observed predicates and state assertions on the FSM
      transitions. More details about the approach can be found in
      \cite[Bau11].
  
    \stopsubsection
  
    \startsubsection
      [reference=sec:monitoring_based,
       title={Monitoring-based estimation of satisfaction rates}]
  
      As seen in \in{Section}[sec:defining_obstacles], the satisfaction rate of
      an obstacle of form $\ltlF(C \wedge \Theta OC)$ is the upper bound among
      the state probabilities of $C \wedge \Theta OC$. We may at runtime count
      the number of observed behaviors satisfying and not satisfying the
      formula $C \wedge \Theta OC$ from states $s$; the ratio between the
      number of observed behaviors satisfying the formula and the number of
      observed behaviors may be used as estimates for the corresponding state
      probability. The automata-based monitoring procedure for $LTL_3$
      determines at runtime whether $C \wedge \Theta OC$ is satisfied from $s$.

      \startdefinition[def:monitored_satrate]{Monitored Satisfaction Rate}
    
        The monitored satisfaction rate of an obstacle or a goal is its actual
        satisfaction rate as observed in the running system.
    
      \stopdefinition
      
      \noindent The process for obtaining the monitored satisfaction rate can
      be summarized as follows:
      
      \startitemize[a]
      
        \item Observed states are collected;
        
        \item (Optional step) Those observed states are filtered;
        
        \item Monitors are created and updated according to the observed states;
        
        \item Monitors labeled with $T$ (true) and $F$ (false) are counted; The
        ratio between the number of $T$- and $F$-monitors
        provides the monitored satisfaction rate.
      
      \stopitemize
      
      \noindent {\bf {\bi (a)} Observing states.} Consider the obstacle \obstacle{Dusty
      Environment} shown in \in{Figure}[fig:obstacle_model].
      \in{Figure}[fig:monitor_execution] shows the process of monitoring the
      satisfaction rate of that obstacle, formalized as $\ltlF(\ltlG_{>2s}
      dustyEnvironment)$, during 8 observations. The squares represent states
      of the observed system. Three states $A$, $B$, $C$ are observed; $A$ and
      $B$ satisfy the predicate $dustyEnvironment$ while $C$ does not. The
      circles show the label of the current state of the $LTL_3$ monitors.
      
      \placefigure[bottom]
     	  [fig:monitor_execution]
     	  {Monitoring the probabilistic obstacle \obstacle{Dusty Environment}.}
        {\externalfigure[../images/chap7/monitor_execution.pdf][scale=1000]}

      As the semantics of our language is synchronous \cite[Let08],
      observations are made at a regular pace. If observations are performed
      every second, the Metric Linear temporal logic (M-LTL) formula $C \wedge
      \Theta OC$ inside the $\ltlF$-operator can be transformed into an LTL
      conjunction. Back to our example, if observations are performed every
      second, we observe:

      \startformula
        \phi : dustyEnvironment \wedge \ltlX dustyEnvironment \wedge \ltlX \ltlX dustyEnvironment
      \stopformula
      
      \noindent {\bf {\bi (b)} Filtering observed states.} As time goes by, the
      number of monitors will increases and the monitored state probability will
      then vary less and less. This problem is acknowledged in other approaches
      such as \cite[Epi09]. Indeed, at the begining, 10 new positive monitors
      might strongly impact the state probability, e.g., 
      
      \startformula 
      
        \text{from } \vfrac{5}{10} = .5 \text{ to } \vfrac{15}{20} = .75
        
      \stopformula
      
      whereas when 1000 observations have been
      made, the state probability would only change 
      
      \startformula
      
        \text{from } \vfrac{5}{1000} = .005 \text{ to } \vfrac{15}{1010} =
        0.014.
      
      \stopformula
      
      This is not a problem if the \quote{actual} satisfaction rate does not
      vary a lot. On the contrary, if the \quote{actual} satisfaction rate is
      changing over time, this will smooth out variations.
      
      To circumvent that effet, the process for obtaining the monitored
      satisfaction rate may be modified to forget old information \cite[Cal11].
      This filtering step keeps only recent monitors, for example the last 100
      monitors. 
      
      The appropriate number of monitors to be kept needs to be determined
      adequatly depending on the domain, how fast the states are observed, and
      the desired innaccuracy margins (see
      \in{Section}[sec:innaccuracy_margins]). Such appropriate number might
      also be automatically set to an optimal aging factor \cite[Cal14].
      
      \noindent {\bf {\bi (c)} Creating and updating monitors.} For a monitored leaf
      obstacle, at each observation of the running system, an $LTL_3$ monitor
      is started to check whether the behavior from the current state satisfies
      $\phi$. As seen below, virtual copies are used in practice to avoid
      creating new monitors at runtime.
      
      \in{Figure}[fig:ltl3_monitor] shows the corresponding $LTL_3$ monitor.
      The top left monitor state labeled with $?$ is the initial state. Each
      transition is labeled with a state formula. The label on states are: $T$
      for {\it true}, $F$ for {\it false}, $?$ for {\it inconclusive}.
      
      \placefigure[bottom]
     	  [fig:ltl3_monitor]
     	  {$LTL_3$ monitor for \obstacle{Dusty Environment}.}
        {\externalfigure[../images/chap7/ltl3_monitor.pdf][scale=1000]}

      \noindent Let us have a closer look at the example in
      \in{Figure}[fig:monitor_execution].

      \startitemize

        \item At observation $0$, we start a monitor $M_0$ to check whether the
        future system behavior satisfies $\phi$.
    
        \item At observation $1$, the monitor $M_0$ is still inconclusive; a new
        monitor $M_1$ is started.
    
        \item At observation $2$, the current state of the monitor $M_0$ is
        updated to $T$ (true). For state $A$, one observed behavior so far
        satisfies $\phi$ (see \quote{\ss 1/1} at the bottom of
        \in{Figure}[fig:monitor_execution]). A new monitor $M_2$ is started.
    
        \item At observation $3$, we get the label $F$ (false) for the current
        state of the monitors started at observations 1 to 3. For state $B$,
        two behaviors were observed to not satisfy $\phi$ (see \quote{\ss 0/2}
        at the bottom of \in{Figure}[fig:monitor_execution]). For state $C$,
        one behavior is seen to not satisfy $\phi$ (\quote{\ss 0/1}).
    
        \item At observation $7$, one observed behavior from state $A$ satisfies
        $\phi$ among the two observed ones (the third one is still
        inconclusive). The two observed behaviors from B violate the formula.
        The three observed behaviors from $C$ also violate the formula.

      \stopitemize
      
      In practice, creating a new $LTL_3$ monitor at each observation is
      clearly unrealistic as the time complexity is $\cal O(2^{2^n})$ where $n$ is
      the size of the formula \cite[Bau11]. To avoid creating multiple
      instances of the same monitor, one $LTL_3$ monitor is built at RE time;
      the monitors being started at runtime are virtual copies of the former. A
      {\it virtual copy} only contains a pointer to the current state of the
      $LTL_3$ monitor. The complexity of starting a "new" virtual monitor there becomes
       $\cal O(1)$.

      The complexity of updating all monitors is $\cal O(n)$ where $n$ is the
      number of virtual monitors. This number depends on the number of
      observations. The worst-case situation corresponds to a system where all
      observed states are unique and all monitors remain inconclusive forever.
      Such system is fairly unlikely. Our running example and the evaluation case
      study in \in{Chapter}[chap:evaluation] suggest that monitors have a short
      life which reduces the cost of updating monitors.

      Other monitoring techniques such as \cite[dAm05, Gia01, Hav04, Tha05]
      might be used to determine the satisfaction of the formula $C\wedge\Theta OC$.
      $LTL_3$ technique, however, reports both violation and satisfaction of the formula
      as early as possible. Its three-value semantics distinguishes cases where
      a formula is satisfied, not satisfied, or none applies. Techniques such
      as \cite[dAm05] amalgamate the last two cases. Other monitoring
      techniques, such as \cite[Bas11], support richer logics such as metric
      temporal logic and first-order logic. These techniques might be used in
      place of $LTL_3$; this might improve the applicability or the technique
      to industrial settings.

      \noindent {\bf {\bi (d)} Obtaining monitored satisfaction rates.} In this
      setting, the {\it monitored state probability} of state $s$ is the ratio
      between the number of monitors started in $s$ whose current state is
      labeled as $T$ (true), and the number of monitors started in $s$ whose
      current state is labeled as $T$ (true) or $F$ (false).

      In our example, the monitored state probability of state $A$ is not
      available for the two first observations; it is equal to $1$ for the five
      next observations, and to $.5$ for the last one. The satisfaction rate
      for our obstacle is the upper bound of these state probabilities. It
      changes from ‘not available’ to 100\% at observation 2, then decreases
      from 100\% to 50\% at observation 7 (see the bottom of
      \in{Figure}[fig:monitor_execution]).

      \in{Figure}[fig:satrate-dusty_environment] shows the evolution of the
      satisfaction rate of the obstacle \obstacle{Dusty Environment} in our
      simulated environment. The black line shows the estimated satisfaction
      rate. The graph was produced by monitoring the system with the tool
      described in \in{Chapter}[chap:tool_support].
    
      \placefigure[bottom]
     	  [fig:satrate-dusty_environment]
     	  {Monitored satisfaction rate for \obstacle{Dusty Environment}.}
        {\externalfigure[../images/chap7/satrate-dusty_environment.pdf][scale=1000]}

      To efficiently obtain the satisfaction rate of leaf obstacles, a list of
      monitors is kept in memory for each observed state; the number of
      behaviors satisfying the formula $C\wedge\Theta OC$ and the number of
      behaviors violating it are kept in registers. Once a monitor reaches a
      monitor state labeled as {\it T} (true) or {\it F} (false), it can be
      removed from the list and the corresponding register is updated.
      Computing the state probability is then reduced to arithmetic operations
      on these registers.
      
    \stopsubsection

    \startsubsection[reference=sec:innaccuracy_margins,title={Bounding inaccuracy margins}]
    
      To mitigate the risk of unnecessary system adaptations, \quote{enough}
      observations should be made before deciding whether or not an adaptation
      is required. Otherwise, decisions might be based on non-statistically
      significant data, possibly leading to adaptations that deteriorate the
      system instead of improving it.
    
      For example, assume 4 behaviors\emdash{}2 positive and 2 negative ones. We
      might be tempted to estimate the state probability to 0.5. However, it is
      very likely that the real probability is not .5. It can be shown that
      there is a 95\% of chance for the \quote{exact} value to lie between $.01$
      and $.99$, which is not very informative.
    
      To address this problem, we may compute the number of observations
      required to achieve a specified level of accuracy. Every observation
      triggers a new virtual monitor which increases the number of available
      data.
    
      For example, let us require that the monitored satisfaction rate should
      differ from the exact value by at most 1\% with 95\% certainty, that is,
      95\% of the values lies within $\pm1.96\sigma$ of the exact value, where
      $\sigma$ is the standard deviation. This interval holds if the variation
      around the exact satisfaction rate has a normal distribution
      \cite[Wac07]. In our example, $1.96\sigma$ must be equal to $.01$. As the
      value of $\sigma$ is not known, the following unbiased estimator provides
      an estimate \cite[Wac07]:

      \startformula
        \hat{\sigma} = \sqrt{\frac{p\times(1-p)}{n}},
      \stopformula
    
      where $p$ is the estimated satisfaction rate. Replacing $\sigma$ by $\hat\sigma$ in
      our example results in the following expression:
    
      \startformula
        1.96\hat\sigma = 1.96 \sqrt{\frac{p\times(1-p)}{n}} = 0.01.
      \stopformula
    
      This expression can be solved to compute the number $n$ of observations
      that must be made to guarantee a satisfactory level of accuracy. If the
      estimated satisfaction rate for the obstacle \obstacle{Dusty Environment}
      is 20\%, we have:
  
      \startformula
        n = \left(\frac{1.96}{0.01}\right)^2 \times .8 \times .2 \approx 6146
      \stopformula
    
      Our accuracy bound thus requires $6146$ observations of the state with
      the lowest state probability. With one observation every second, it
      requires at least 1h 42m 26s. Note that this is a lower bound; as discussed in
      \in{Section}[sec:defining_obstacles], the
      satisfaction rate is the lowest state probability; to reach that
      uncertainty, the system has to stay in the same state for the required
      timespan. This indicates how fast our technique may compute a
      statistically accurate monitored satisfaction rate. (Note that such
      limitation applies to any monitoring technique with the same amount of
      observed data.)
      
      In the opposite direction, we may compute the uncertainty margins given a
      timespan of observation. Assume that we make an observation every second for 30
      minutes; we therefore have 1800 observations. The estimated standard
      deviation is:
      
      \startformula
        \hat{\sigma} = \sqrt{\frac{p\times(1-p)}{n}} = \sqrt{\frac{.8\times.2}{1800}} = 0.009428...
      \stopformula
      
      The \quote{exact} monitored satisfaction rate lies between $78\%$ and
      $81\%$ with 95\% of certainty; this is already a narrow interval. Such
      information might be used at run-time to decide whether it is reasonable
      to adapt the running software.
    
      In our example, we used the estimated satisfaction rate $p = 0.2$. If no
      estimated satisfaction rate $p$ is available, it is common to use $.5$;
      this value maximizes the number $n$ of observations. This guarantees the
      accuracy for any satisfaction rate.
      
      \in{Figure}[fig:satrate-dusty_environment] shows the evolution of the
      95\% confidence interval as more states are observed in dark gray. We can
      see that the confidence interval is large at the begining where few
      observations are available, and narrow around the satisfaction rate when
      more observations are available.
    
    \stopsubsection

  \stopsection
  
  \startsection[reference=sec:obstaclebased_adaptation, title={Obstacle-driven system adaptation}]
  
    This chapter proposes to let probabilistic goals and obstacles drive the
    runtime adaptation process. A system adaptation is required at runtime when
    the current configuration of countermeasures does not guarantee the
    required satisfaction rate (RSR) of the system’s high-level goals. This
    provides a criteria for deciding when and why an adaptation is required. It
    also documents how the software system should adapt.
    
    The {\it actual} satisfaction rate of these goals must therefore be
    determined from the {\it monitored} satisfaction rates of leaf obstacles.
    When this actual satisfaction rate falls below the goal's RSR, alternative
    countermeasures maximizing its satisfaction rate of these goals should
    replace the corresponding ones in the current configuration.
    
    \startitemize
    
      \item The up-propagation procedure described in
      \in{Chapter}[chap:assessing] determine the actual satisfaction rate of
      high-level goals from the monitored satisfaction rate of leaf obstacles.
    
      \item The countermeasure selection procedure described in
      \in{Chapter}[chap:controlling_obstacle] is then used to select most
      appropriate countermeasures.
    
      \item Countermeasures are then integrated in the ideal model and deployed
      in the running system\emdash{}\in{Section}[sec:sw_update] will describe
      next.
    
    \stopitemize
    
    If no selection guarantees the goal's required satisfaction rate, the
    selection that maximizes the satisfaction rate of the high-level goals is
    picked. The monitoring system might also prompt the user for interactively
    defining new countermeasures.
    
    \placefigure[bottom]
   	  [fig:satrate-locals_warned_when_risk_imminent]
   	  {Monitored satisfaction rate for the goal \goal{Achieve [Locals Warned When Risk
      Imminent]}.}
      {\externalfigure[../images/chap7/satrate-locals_warned_when_risk_imminent.pdf]}
  
    \placefigure[top]
   	  [fig:obstacle_adaptation]
   	  {Monitored satisfaction rate for the leaf obstacles.}
      {\startcombination[2*3]
        {\externalfigure[../images/chap7/satrate-dusty_environment-small.pdf]}{\tfx (\obstacle{\tfx Dusty Environment})}
        {\externalfigure[../images/chap7/satrate-echo-small.pdf]}{\tfx (\obstacle{\tfx False Echo})}
        {\externalfigure[../images/chap7/satrate-depth_broken-small.pdf]}{\tfx (\obstacle{\tfx Radar Sensor Broken})}
        {\externalfigure[../images/chap7/satrate-ultrasound_broken-small.pdf]}{\tfx (\obstacle{\tfx Ultrasound Sensor Broken})}
        {\externalfigure[../images/chap7/satrate-noisy_image-small.pdf]}{\tfx (\obstacle{\tfx Noisy Image})}
        {\externalfigure[../images/chap7/satrate-voice_dow-small.pdf]}{\tfx (\obstacle{\tfx Voice Network Overloaded})}
      \stopcombination}
    
    Back to our example,
    \in{Figure}[fig:satrate-locals_warned_when_risk_imminent] shows the
    satisfaction rate for the goal \goal{Achieve [Locals Warned When Risk
    Imminent]}, computed from the monitored satisfaction rates of the leaf
    obstacles. (See \in{Figure}[fig:goal_model] for the goal model.)
    \in{Figure}[fig:obstacle_adaptation] shows the monitored satisfaction rates
    for our obstacles. Every 5 minutes, the optimization process computes a
    most appropriate countermeasure selection. At 08:26 and 09:21, an
    adaptation is required as the goal's satisfaction rate falls below its
    required satisfaction rate.
    
    The peak at 08:47, seen in
    \in{Figure}[fig:satrate-locals_warned_when_risk_imminent] and in
    \in{Figure}{(\obstacle{Noisy Image})}[fig:obstacle_adaptation], illustrate
    the problem with statistically innaccurate satisfaction rates discussed in
    \in{Section}[sec:innaccuracy_margins]. Here, it did not trigger an
    adaptation as the violation was very short (it lasts 2 seconds).
    
  \stopsection
  
  \startsection[reference=sec:sw_update,title={Runtime deployment of most appropriate countermeasures}]
    
    When more appropriate countermeasures are selected and integrated in the
    goal model, the running software system should be adapted to match the
    updated goal model.
    
    The software component responsible for adapting the running system is named
    {\it Adaptor} in the following discussion. This component keeps track of a
    current selection of countermeasures. When new most appropriate
    countermeasures are computed, it identifies the countermeasures to be {\it
    (i)} {\it added}\emdash{}as not found in the current selection but in the
    selection of the most appropriate ones; and {\it (ii)} {\it
    removed}\emdash{}as no longer in the selection of the most appropriate ones.

    To adapt the software system, the {\it Adaptor} calls the activation and
    deactivation procedures associated with the countermeasures to be added or
    removed. The {\it activation} procedure is the code-related procedure responsible
    for the deployment of the corresponding countermeasure in the running
    system. The {\it deactivation} procedure is responsible for its removal. These
    procedures are used to:
    
    \startitemize
    
      \item add, remove or replace a running component; 
      
      \item update configuration parameters;
    
      \item activate hardware components;
      
      \item and so forth.
      
    \stopitemize
      
    \noindent The countermeasure goals are decorated with these procedures.

    For example, let us consider that the monitored satisfaction rate of the
    leaf obstacle \obstacle{Ultrasound Sensor Broken} increases at 08:26, as
    shown in \in{Figure}[fig:obstacle_adaptation]. This increase causes a
    decrease in the monitored satisfaction rate of the high-level goal
    \goal{Achieve [Locals Warned When Risk Imminent]} below its RSR (as seen in
    \in{Figure}[fig:satrate-locals_warned_when_risk_imminent]). This decrease
    causes the countermeasure \goal{Achieve [Speed Acquired Every 5 Sec By
    Camera]} to be selected as most appropriate countermeasure. The activation
    procedure for \goal{Achieve [Speed Acquired Every 5 Sec By Camera]},
    namely, {\it DeployCamera}, is called and replaces the software component
    acquiring the speed by the camera-related component. As
    \in{Figure}[fig:satrate-locals_warned_when_risk_imminent] shows, the
    satisfaction rate of the high-level goal increases above its RSR after
    software adaptation. At 09:21, the satisfaction rate of \goal{Achieve
    [Locals Warned When Risk Imminent]} falls again below the required
    satisfaction rate. This fall is due to an increase in the satisfaction rate
    of the obstacle \obstacle{Noisy Image} for the selected countermeasure
    goal. This cause the countermeasure goal \goal{Achieve [Speed Acquired
    Every 5 Sec By Camera]} to be removed from the selection, causing the
    deactivation procedure {\it DeployUltrasound} to be activated. The
    satisfaction rate then increases back.
  
  \stopsection
  
	\startsection[reference=sec:update_k_uncertainty,title={Updating uncertain probability values at runtime}]
  
    Managing uncertainty at run-time is a critical challenge \cite[Ben10].
    \in{Chapter}[chap:knowledge-uncertainty] discussed how knowledge
    uncertainty can be integrated with obstacle assessment and control. 
    
    Such knowledge uncertainty may be reduced at runtime using the monitored
    information. However, if the observed values vary a lot over time, the
    knowledge uncertainty might increase, as \quote{actual} value is indeed
    uncertain.
    
    To integrate knowledge uncertainty at runtime, the current monitored satisfaction rate
    should no longer be expressed as a single-value but rather as a multi-value
    satisfaction rate, typically a {\it Normal} probability distribution.
    
    \noindent The process for obtaining the monitored satisfaction rate is
    updated as follows.
    
    \startitemize[a]
    
      \item Observed states are collected;
      
      \item (Optional step) Those observed states are filtered;
      
      \item Monitors are created and updated according to the observed states;
      
      \item Monitors labeled with $T$ (true) and $F$ (false) are counted; the
      ratio between the number of $T$- and $F$-monitors
      provides the monitored satisfaction rate.
      
      \sym{\it(e*)} A {\it Beta} probability distribution is fitted to the number of
      monitors.
    
      \sym{\it(f*)} The {\it Beta} distribution updates the current monitored
      satisfaction rate using Bayesian inference. The updated satisfaction rate
      then becomes the new satisfaction rate.
    
    \stopitemize
    
    Step {\it (e*)} is straightforward as the parameters of the {\it Beta}
    distribution are, $\alpha$, the number of positive monitors and, $\beta$,
    the number of negative monitors \cite[Vos08].
    
    The computation in step {\it (f*)} relies on Bayesian Inference; it 
    is a statistical computation in which the Bayes' theorem is used
    to update a probability distribution as more information becomes available
    \cite[Wac07]. The updated satisfaction rate $P(O\mid M)$, where $M$ represents
    what has been last observed, is given by:
    
    \startformula
      P(O\mid M) = \frac{P(M\mid O)\times P(O)}{P(M)}
    \stopformula
    
    where $P(M\mid O)$ is the likelihood of observing the monitored satisfaction
    rate, $P(O)$ is the current monitored satisfaction rate, and $P(M)$ is a
    scaling factor. The Beta distribution obtained at step {\it (e*)} captures the
    likelihood $P(M\mid O)$.
  
    \placefigure[bottom]
   	  [fig:runtime_update]
   	  {Monitored satisfaction rate for the leaf obstacles with knowledge uncertainty.}
      {\externalfigure[../images/chap7/runtime_update.pdf]}
    
    \in{Figure}[fig:runtime_update] shows how the satisfaction rate for the
    obstacle \obstacle{Pump Mechanical Failure} is updated with the observed
    satisfaction rate. The lighest line shows the satisfaction rate provided by
    the experts, as seen in \in{Chapter}[chap:knowledge-uncertainty]. The
    darkest line shows the satisfaction rate updated after 6 observations. The
    other shades show the intermediate steps. For this example, we assumed that
    the monitored satisfaction rate was $.45$ for the first three observations
    and $.55$ for the three last. We can see that the more observations are
    made, the more accurate the satisfaction rate is (the curve gets narrower.)
  
  \stopsection
  
  \startsection[title={Summary}]
  
    Software systems should ideally self-adapt to changing conditions and
    assumptions in order to keep their goals satisfied. This chapter detailed
    an obstacle-driven runtime adaptation approach aimed at increasing the
    actual satisfaction rate of probabilistic system goals. Leaf obstacles are
    monitored at runtime to let the system dynamically switch to more
    appropriate countermeasure goals that increase the satisfaction rate of the
    system’s high-level goals under new conditions. The approach guarantees
    that the required satisfaction rate of high-level goals remains satisfied
    when obstacle satisfaction rates are changing.

    This requires obstacle satisfaction rates to be monitored at system
    runtime. The proposed monitoring technique extends the $LTL_3$ approach in
    \cite[Bau11] to support the monitoring of probabilistic assertions. The
    monitors are built at RE time; at runtime, virtual copies of $LTL_3$
    monitors keep track of obstacle satisfaction. State probabilities for each
    observed state can thereby be estimated; their upper bound provides the
    monitored obstacle satisfaction rates. The latter are propagated through
    the obstacle/goal model up to the system’s high-level goals. The monitored
    satisfaction rates thereby obtained for these goals are compared with their
    required satisfaction rate; when the former fall below the latter, more
    appropriate countermeasures replace the current ones to remain above the
    required threshold.
    
    The next chapter describes the tools supporting the various techniques
    presented in the thesis.
  
  \stopsection
  
\stopcomponent

% TODO
% - Not all obstacles are monitorable. It might be required to refine the obstacles until a software agent is able to monitor the obstacle.
% - Monitoring rare obstacles might underestimate the obstacle satisfaction rate. In particular when not used with prior knowledge.
