% !TEX root = thesis.tex

\startcomponent chap-6
\environment common
\product thesis

\chapter[runtime]{Handling Obstacles At System Runtime}

  Software systems are deployed in changing environment. The selection of the
  most appropriate countermeasures is based on environment assumptions and
  obstacle satisfaction rates determined at RE time. These influencing factors
  may however turn to be different at system runtime \cite[Fic95]. Some
  assumptions might no longer hold; new characteristics might emerge; experts’
  estimates at RE time for obstacle assessment might prove inaccurate at system
  runtime; other estimates might not be available at RE time; and so forth. For
  better fit to the system’s goals under changing or originally unknown
  conditions, it might thus be better to defer decisions on selecting most
  appropriate countermeasures to system runtime \cite[Fea98].

  This chapter details how obstacles and goals might drive the adaptation of a
  software system. The chapter introduces how probabilistic obstacles are
  monitored at runtime and how monitored satisfaction rates drive the selection
  of the most appropriate countermeasures.

  The satisfaction rate of an obstacles is defined as the lowest state
  probability to satisfy its specification. Such precise characterization in
  terms of states and behaviors enables the monitoring of the state
  probabilities. The satisfaction of the specification is monitored at runtime
  using standard LTL-monitoring techniques and the satisfaction rate is
  estimated by counting positive and negative occurences of the formula. The
  monitored satisfaction rate of the leaf obstacles is then propagated up-to
  the high-level goals (as seen in \in{Chapter}[chap:assessing]). If the
  satisfaction rate of a high-level goal do not exceed its required
  satisfaction rate, an adaptation of the software system is required. The most
  appropriate countermeasures are then selected (as seen in
  \in{Chapter}[chap:controlling_obstacle]), using the runtime information, and
  deployed in the running system. As a result, the software adaptation process
  can be driven by the satisfaction of the probabilistic goals. It provides
  traceable criteria to decide when an adaptation is required and what are the
  adaptation that should be performed to guarantee the satisfaction of the
  probabilistic goals.

  The chapter is structured as follows. \in{Section}[sec:overview] presents the
  overall approach to switch to the most appropriate countermeasure at runtime.
  \in{Section}[sec:runtime_running_example] introduces the running example.
  \in{Section}[sec:monitoring] details how probabilistic obstacles are
  monitored at run-time. \in{Section}[sec:obstaclebased_adaptation] details how
  probabilistic obstacles drive the runtime adaptation process.
  \in{Section}[sec:sw_update] explains how the running software is adapted.
  \in{Section}[sec:update_k_uncertainty] discusses how knowledge uncertainty
  affects the monitoring and the adaptations.

  \startsection[reference=sec:overview, title={Overview of the Approach}]
  
    The objective in this chapter is to let the system dynamically switch to more
    appropriate countermeasures to leaf obstacles in view of evolving environment
    conditions and obstacle satisfaction rates. The satisfaction rate of leaf
    obstacles was estimated at RE time and is now being observed at system
    runtime. The alternative countermeasure goals to those leaf obstacles were
    identified and specified at RE time.

    A countermeasure should dynamically replace the current one when, unlike the
    latter, it makes the satisfaction rate of high-level goals exceed their
    required satisfaction rate. The satisfaction rate of high-level goals is
    obtained from the monitored satisfaction rate of the leaf obstacles by
    up-propagation through the goal/obstacle model. The monitored satisfaction
    rate of a leaf obstacle is obtained by counting the observed behaviors.

    \noindent Our approach comprises 7 steps detailed in the next sections. How
    steps relates to each other is show in \in{Figure}[fig:overview_approach].
    
    \placefigure[bottom]
   	  [fig:overview_approach]
   	  {Overview of the approach.}
      {\externalfigure[../images/chap7/overview_approach.pdf][scale=1000]}
  
    \startitemize[n]

      \item {\bf Building Monitors for Leaf Obstacles.} Probabilistic obstacles
      are monitored at system runtime. Our approach extends the monitoring
      technique introduced in \cite[Bau11] for non-probabilistic linear
      temporal logic (LTL) to monitor probabilistic LTL assertions at system
      runtime. This step builds, at RE time, the $LTL_3$ monitors for the leaf
      obstacles. The list of predicates to observe at runtime is thereby
      provided. (See \in{Section}[sec:building_monitor])

      \item {\bf Observing States.} At runtime, the states of the monitored
      system are observed at a regular pace. (The pace may be chosen to fit a
      specific domain.)
      
      \item {\bf Updating Obstacle Monitors.} At every observation, a new
      virtual monitor for each leaf obstacle is started while existing monitors
      are updated. (See \in{Section}[sec:monitoring_based])

      \item {\bf Computing High Level Satisfaction Rates.} The monitored
      satisfaction rate of leaf obstacles is up-propagated through
      obstacle/goal refinement trees up to high-level goals. (As explained in
      \in{Chapter}[chap:assessing].)

      \item {\bf Selecting Countermeasures.} Comparing the monitored
      satisfaction rates obtained for those goals with their RSR determines
      whether the current countermeasures to the monitored obstacles are still
      appropriate. If a monitored goal satisfaction rate falls below the goal’s
      RSR, alternative more appropriate countermeasures are selected among
      those available. (As explained in \in{Section}[sec:selecting_cm])

      \item {\bf Updating the RE Model.} The goal/obstacle model is updated
      accordingly by integrating the new current countermeasures and updating
      the propagation in Step 4. (As explained in
      \in{Section}[sec:integrating])

      \item {\bf Runtime Deployment.} The software is automatically adapted
      according to the selected countermeasures. (See
      \in{Section}[sec:sw_update])

    \stopitemize
    
  \stopsection
  
  \startsection
    [reference=sec:runtime_running_example,
     title={Running Example for Runtime Adaptation}]
     
    The running example for this section was inspired from a flood detection
    system \cite[Hug06,Hug06b]. This running example illustrates or evaluates
    other runtime adaptation techniques \cite[Saw10,Ben08,Wel11,Gol08].
    
    {\vskip.8\baselineskip}
    \framed[width=local,align={width},offset=10pt,after={\vskip-\baselineskip}
    ] {\it \quote{Hydrologists have traditionally approached the problem of modeling
    and predicting floods by deploying sensing and logging equipment in areas
    susceptible to flooding. River profile data from these sensors, such as
    depth and rate of flow, is then either transmitted off-site or collected
    manually. This data is then used as the input to complex spatial flood
    prediction models, which typically run on high-performance
    computers and Grid technologies. This data is also used as the basis for
    much simpler single-point models that can run on simpler devices.}\emdash{}\cite[Hug06]}
   
    Our running example system monitors the speed and depth of the river. If
    the levels are criticals (approximated here as \quote{\it above a given
    threshold}), the locals are warned as soon as possible. The speed can be
    measured either using an ultrasound sensor or a camera-based processing.
    \in{Figure}[fig:goal_model] shows a fragment of the goal model whereas
    \in{Figure}[fig:obstacle_model] shows the fragment of obstacle model.
    
    \placefigure[here][fig:goal_model]
      {Goal model fragment for the Flood Detection System.}
      %{\startcombination[1*2]
        {\externalfigure[../images/chap7/goal_model_a.pdf]}{}
        %{\externalfigure[../images/chap7/goal_model_b.pdf]}{}
      %\stopcombination}
   
    \placefigure[bottom][fig:obstacle_model]
      {Obstacle model fragment for the Flood Detection System.}
      {\startcombination[2*1]
        {\externalfigure[../images/chap7/obstacle_model_a.pdf]}{}
        {\externalfigure[../images/chap7/obstacle_model_b.pdf]}{}
      \stopcombination}
     
  \stopsection
  
  \startsection
    [reference=sec:monitoring,
     title={Monitoring Probabilistic Obstacles}]
  
    At RE time, domain experts estimate the satisfaction rates of the leaf
    obstacles based on their knowledge of the system or their experience with
    similar systems as detailed in \in{Section}[sec:estimating_satrate]. These
    estimates might prove inaccurate at system runtime\emdash{}they might be
    too rough or environment properties or assumptions might have changed in
    the meantime. Moreover, some variables might be hard to estimate at RE
    time\emdash{}prior data might not be available or might be too costly to
    acquire; too many parameters might be involved; etc. Monitoring the actual
    satisfaction rate of leaf obstacles at system runtime helps filling this
    gap; former estimates may be made more accurate, and missing data may be
    made available.
  
    \startsubsection
      [reference=sec:building_monitor,
       title={Background: Monitoring Non-Probabilistic LTL Assertions}]
  
      This section introduces the necessary background for monitoring
      non-probabilistic LTL assertions. Monitoring the runtime satisfaction of
      an LTL formula relies on the finite trace observed so far, and on an
      automata-based modeling the satisfaction of the formula. The approach
      presented in \cite[Bau11] is chosen as it reports LTL formula
      satisfaction or violation as early as possible.
      
      $LTL_3$, the LTL considered in \cite[Bau11], uses {\it true}, {\it
      false}, and {\it inconclusive} as truth values. A finite trace is
      labeled as true if any continuation of it satisfies the formula; false
      if any continuation falsifies the formula; and inconclusive otherwise. A
      $LTL_3$ formula monitor is a finite state machine (FSM) that reads finite
      traces and outputs the corresponding truth value. As suggested by
      \in{Figure}[fig:ltl3_approach], this FSM is built from the formula and
      its negation \cite[Bau11].
      
      \startitemize
      
        \item The associated non-deterministic Büchi automaton (NBA) is
        generated using a standard algorithm \cite[Bai08].
      
        \item A non-deterministic finite automaton (NFA) is then generated by
        performing an emptiness check for each NBA state. A state $s$ is labeled
        with {\it true} if a continuation satisfying the formula exists (that is, if
        the language corresponding to the NFA with initial state $s$ is not
        empty); it is false otherwise.
      
        \item The NFA is determinized to produce a deterministic finite
        automaton (DFA) using powersets construction \cite[Bai08].
      
      \stopitemize
    
      \placefigure[top]
     	  [fig:ltl3_approach]
     	  {Steps for computing monitor FSM from input formula $\phi$.}
        {\externalfigure[../images/chap7/ltl3_approach.pdf][scale=1000]}
      
      The monitor FSM results from the product of both DFAs. Let $(s_1,s_2)$
      denote a state in this product, where $s_1$ is the DFA state
      corresponding to the formula $\phi$ and $s_2$ the DFA state corresponding
      to its negation $\neg\phi$. This monitor FSM state is labeled as:
      
      \startitemize
      
        \item {\it true} if $s_1$ is labeled as {\it true} and $s_2$ as {\it
        false} (no continuation exists such that the formula is falsified);
      
        \item {\it false} if $s_1$ is labeled as {\it false} and $s_2$ as {\it
        true} (no continuation exists such that the formula is satisfied);
      
        \item {\it inconclusive} otherwise (a continuation exists such that
        the formula is satisfied, and a continuation exists such that the
        formula is not satisfied.)
      
      \stopitemize
      
      At runtime, the current state of the monitor FSM is updated according to
      the truth value of the observed predicates and state assertions on the FSM
      transitions. More details about the approach can be found in
      \cite[Bau11].
  
    \stopsubsection
  
    \startsubsection
      [reference=sec:monitoring_based,
       title={Monitoring-Based Estimation of Satisfaction Rates}]
  
      As recall, the satisfaction rate of an obstacle $\ltlF(C \wedge \Theta
      OC)$ is the upper bound among the state probabilities of $C \wedge \Theta
      OC$. We may at runtime count the number of observed behaviors satisfying
      $C \wedge \Theta OC$ from states $s$; this estimates the corresponding
      state probability. The automata-based monitoring procedure for $LTL_3$
      determines at runtime whether $C \wedge \Theta OC$ is satisfied from $s$.

      \startdefinition[def:monitored_satrate]{Monitored Satisfaction Rate}
    
        The monitored satisfaction rate of an obstacle or a goal is its actual
        satisfaction rate as observed in the running system.
    
      \stopdefinition

      Consider the obstacle \obstacle{Dusty Environment} shown in
      \in{Figure}[fig:obstacle_model]. \in{Figure}[fig:monitor_execution] shows
      the process of monitoring the satisfaction rate of that obstacle,
      formalized as $\ltlF(\ltlG_{>2s} dustyEnvironment)$, during 8
      observations. The squares represent states of the observed system. Three
      states $A$, $B$, $C$ are observed; $A$ and $B$ satisfy the predicate
      $dustyEnvironment$ while $C$ does not. The circles show the label of the
      current state of the $LTL_3$ monitors.
      
      \placefigure[bottom]
     	  [fig:monitor_execution]
     	  {Monitoring the probabilistic obstacle \obstacle{Dusty Environment}.}
        {\externalfigure[../images/chap7/monitor_execution.pdf][scale=1000]}

      As the semantics of our language is synchronous \cite[Let08],
      observations are made at a regular pace. If observations are performed
      every second, the MTL formula $C \wedge \Theta OC$ inside the
      $\ltlF$-operator can be transformed into an LTL conjunction. Back to our
      example, if observations are performed every second:

      \startformula
        \phi : dustyEnvironment \wedge \ltlX dustyEnvironment \wedge \ltlX \ltlX dustyEnvironment
      \stopformula
      
      (In the following, $\phi$ denotes the above formula.)
      \in{Figure}[fig:ltl3_monitor] shows the corresponding $LTL_3$ monitor.
      The top left monitor state labeled with $?$ is the initial state. Each
      transition is labeled with a state formula. The label on states are: $T$
      for true, $F$ for false, $?$ for inconclusive.
      
      \placefigure[bottom]
     	  [fig:ltl3_monitor]
     	  {$LTL_3$ monitor for \obstacle{Dusty Environment}.}
        {\externalfigure[../images/chap7/ltl3_monitor.pdf][scale=1000]}

      For a monitored leaf obstacle, at each observation of the running system,
      an $LTL_3$ monitor is started to check whether the behavior from the
      current state satisfies $\phi$. As seen below, virtual copies are used in
      practice to avoid creating new monitors at runtime.

      \noindent Let us have a closer look at the example in
      \in{Figure}[fig:monitor_execution].

      \startitemize

        \item At observation $0$, we start a monitor $M_0$ to check whether the
        future system behavior satisfies $\phi$.
    
        \item At observation $1$, the monitor $M_0$ is still inconclusive; a new
        monitor $M_1$ is started.
    
        \item At observation $2$, the current state of the monitor $M_0$ is
        updated to $T$ (true). For state $A$, one observed behavior so far
        satisfies $\phi$ (see \quote{\ss 1/1} at the bottom of
        \in{Figure}[fig:monitor_execution]). A new monitor $M_2$ is started.
    
        \item At observation $3$, the current state of the monitors started at
        observations 1 to 3 is labeled $F$ (false). For state $B$, two
        behaviors were observed to not satisfy $\phi$ (see \quote{\ss 0/2} at
        the bottom of \in{Figure}[fig:monitor_execution]). For state $C$, one
        behavior is seen to not satisfy $\phi$ (\quote{\ss 0/1}).
    
        \item At observation $7$, one observed behavior from state $A$ satisfies
        $\phi$ among the two observed ones (the third one is still
        inconclusive). The two observed behaviors from B violate the formula.
        The three observed behaviors from $C$ also violate the formula.

      \stopitemize

      In this setting, the {\it monitored state probability} of state $s$ is
      the ratio between {\it (a)} the number of monitors started in $s$ whose
      current state is labeled as $T$ (true), and {\it (b)} the number of
      monitors started in $s$ whose current state is labeled as $T$ (true) or
      $F$ (false).

      In our example, the monitored state probability of state $A$ is not
      available for the two first observations; it is equal to $1$ for the five
      next observations, and to $.5$ for the last one. The satisfaction rate
      for our obstacle is the upper bound of these state probabilities. It
      changes from ‘not available’ to 100\% at observation 2, then decreases
      from 100\% to 50\% at observation 7 (see the bottom of
      \in{Figure}[fig:monitor_execution]).

      \in{Figure}[fig:satrate-dusty_environment] shows the evolution of the
      satisfaction rate of the obstacle \obstacle{Dusty Environment} in our
      simulated environment. The black line show the estimated satisfaction
      rate. The graph was produced by monitoring the system, with the provided
      tool support (see \in{Chapter}[chap:tool_support]).
    
      \placefigure[bottom]
     	  [fig:satrate-dusty_environment]
     	  {Monitored satisfaction rate for \obstacle{Dusty Environment}.}
        {\externalfigure[../images/chap7/satrate-dusty_environment.pdf][scale=1000]}
      
      So, the process for obtaining the monitored satisfaction rate can be
      summarized as follows: {\it (a)} observed states are collected; {\it (b)}
      monitors are created and updated according the observed states; {\it (c)}
      monitors labeled with $T$ (true) and $F$ (false) are counted; {\it (d)}
      the ratio between these number of monitors provides the monitored
      satisfaction rate.
      
      However, as the time goes by, the number of monitors increases and the
      monitored state probability will then vary less and less. This problem is
      acknowledge in other approaches such as \cite[Epi09]. Indeed, at the
      begining, 10 new positive monitors might strongly impact the state
      probability, e.g., from $\vfrac{5}{10} = .5$ to $\vfrac{15}{20} = .75$
      whereas when 1000 observations have been made, the state probability
      would only chage from $\vfrac{5}{1000} = .005$ to $\vfrac{15}{1010} =
      0.014$. If the \quote{real} satisfaction rate does not vary a lot, this
      is not a problem. On the contrary, if the \quote{real} satisfaction rate
      does change over time, this will smooth out the variations. To circumvent
      that effet, the process for obtaining the monitored satisfaction rate is
      modified to forget old information \cite[Cal11]. This filtering step
      (between step {\it (b)} and {\it (c)}) keeps only recent monitors, for
      example the last 100 monitors. The appropriate number of monitors to be
      kept needs to be determined adequatly depending on the domain, how fast
      states are observed, and the desired innaccuracy margins (see
      \in{Section}[sec:innaccuracy_margins]). Such appropriate number might
      also be automatically set to an optimal aging factor \cite[Cal14].

      In practice, creating a new $LTL_3$ monitor at each observation is
      clearly unrealistic as the complexity is $\cal O(2^{2^n})$ where $n$ is
      the size of the formula \cite[Bau11]. To avoid creating multiple
      instances of the same monitor, one $LTL_3$ monitor is built at RE time;
      the monitors being started at runtime are virtual copies of the former. A
      {\it virtual copy} only contains a pointer to the current state of the
      $LTL_3$ monitor. The complexity of starting a "new" virtual monitor is
      thus $\cal O(1)$.

      The complexity of updating all monitors is $\cal O(n)$ where $n$ is the
      number of virtual monitors. This number depends on the number of
      observations. The worst-case situation corresponds to a system where all
      observed states are unique and all monitors remain inconclusive forever.
      Such system is unlikely. Our running example and the evaluation case
      study in \in{Chapter}[chap:evaluation] suggest that monitors have a short
      life which reduces the cost of updating monitors.

      To implement the monitoring of leaf obstacles, a list of monitors is kept
      in memory for each observed state. To increase efficiency, the number of
      behaviors satisfying the formula $C\wedge\Theta OC$ and the number of
      behaviors violating it are kept in registers. Once a monitor reaches a
      monitor state labeled as {\it T} (true) or {\it F} (false), it can be
      removed from the list and the corresponding register updated. Computing
      the state probability is then reduced to arithmetic operations on these
      registers.

      Note that other monitoring techniques such as \cite[dAm05a, Gia01,
      Hav04, Tha05] might be used to determine the satisfaction of
      $C\wedge\Theta OC$. $LTL_3$, however, reports both violation and
      satisfaction of the formula as early as possible. Its three-value
      semantics distinguishes cases where a formula is satisfied, not
      satisfied, or none applies. Techniques such as \cite[dAm05a] amalgamate
      the last two cases. Other monitoring techniques, such as \cite[Bas11],
      support richer logics such as metric temporal logic and first-order
      logic. These techniques might be used in place of $LTL_3$; this might
      improve the applicability or the technique to industrial settings.
      
    \stopsubsection

    \startsubsection[reference=sec:innaccuracy_margins,title={Bounding the Inaccuracy Margins}]
    
      To mitigate the risk of unnecessary system adaptations, \quote{enough}
      observations should be made before deciding whether or not an adaptation
      is required. Otherwise, decisions would be based on non-statistically
      significant data, possibly leading to adaptations that deteriorate the
      system instead of improving it.
    
      For example, assume 4 behaviors\emdash{}2 positives and 2 negatives. We
      might be tempted to estimate the state probability to 0.5. However, it is
      very likely that the real probability is not .5; it can be shown that
      there is 95\% of chance for the \quote{exact} value to lie between $.01$
      and $.99$, which is not very informative.
    
      To address this problem, we may compute the number of observations
      required to achieve a specified level of accuracy. Every observation
      triggers a new virtual monitor which increases the number of available
      data.
    
      For example, let us require that the monitored satisfaction rate should
      differ from the exact value by at most 1\%, with 95\% certainty; i.e.,
      95\% of the values lies within $\pm1.96\sigma$ of the exact value, where
      $\sigma$ is the standard deviation. This interval holds if the variation
      around the exact satisfaction rate has a normal distribution
      \cite[Wac07]. In our example, $1.96\sigma$ must equal $.01$. As the
      value of $\sigma$ is not known, the following unbiased estimator provides
      an estimate \cite[Wac07]:

      \startformula
        \hat{\sigma} = \sqrt{\frac{p\times(1-p)}{n}}
      \stopformula
    
      where $p$ is the estimated satisfaction rate. Replacing $\sigma$ by $\hat\sigma$ in
      our example results in the following expression:
    
      \startformula
        1.96\hat\sigma = 1.96 \sqrt{\frac{p\times(1-p)}{n}} = 0.01
      \stopformula
    
      This expression can be solved to compute the number $n$ of observations
      that must be made to guarantee a satisfactory level of accuracy. If the
      estimated satisfaction rate for the obstacle \obstacle{Dusty Environment}
      is 20\%:
  
      \startformula
        n = \left(\frac{1.96}{0.01}\right)^2 \times .8 \times .2 \approx 6146
      \stopformula
    
      Our accuracy bound thus requires $6146$ observations of the state with
      the lowest state probability. With one observation every second, it
      requires at least 1h 42m 26s. Note that this is a lower bound; the
      satisfaction rate is the lowest state probability, to reach that
      uncertainty, the system has to stay in the same state for the required
      timespan. This indicates how fast our technique may compute a
      statistically accurate monitored satisfaction rate. (Note that such
      limitation applies to any monitoring technique with the same amount of
      observed data.)
      
      In the opposite direction, we can compute the uncertainty margins given a
      timespan of observation. Assume that we observe every second for 30
      minutes, we therefore have 1800 observations. The estimated standard
      deviation is
      
      \startformula
        \hat{\sigma} = \sqrt{\frac{p\times(1-p)}{n}} = \sqrt{\frac{.8\times.2}{1800}} = 0.009428...
      \stopformula
      
      The \quote{exact} monitored satisfaction rate lies between $78\%$ and
      $81\%$, with 95\% of certainty; this is already a narrow interval. Such
      information might be used at run-time to decide whether it is reasonable
      to adapt the running software.
    
      In our example, we used the estimated satisfaction rate $p = 0.2$. If no
      estimated satisfaction rate $p$ is available, it is common to use $.5$;
      this value maximizes the number $n$ of observations. This thereby
      guarantees the accuracy for any satisfaction rate.
      
      \in{Figure}[fig:satrate-dusty_environment] shows the evolution of 95\%
      confidence interval as more states are observed in dark gray. We can see
      that the confidence interval is large at the begining, where few
      observations are available and narrow around the satisfaction rate when
      more observations are available.
    
    \stopsubsection

  \stopsection
  
  \startsection[reference=sec:obstaclebased_adaptation, title={Obstacle-based System Adaptation}]
  
    We propose to drive the runtime adaptation process using the probabilistic
    goals and obstacles. A system adaptation is required at runtime when the
    current configuration of countermeasures does not guarantee the required
    satisfaction rate (RSR) of the system’s high-level goals. This provides a
    criteria for deciding when and why an adaptation is required. It also
    documents how the software system shall adapt.
    
    The actual satisfaction rate of these goals must therefore be determined
    from the monitored satisfaction rates of leaf obstacles. When falling below
    their RSR, alternative countermeasures maximizing the satisfaction rate of
    these goals should replace the current configuration. An up-propagation
    procedure, as described in \in{Chapter}[chap:assessing], determines the
    satisfaction rate of high-level goals. A countermeasure selection
    procedure, as described in \in{Chapter}[chap:controlling_obstacle], is used
    to select the most appropriate countermeasures. Countermeasures are then
    integrated in the ideal model and deployed in the running system\emdash{}As
    \in{Section}[sec:sw_update] will describe next.
    
    If no selection guarantee the required satisfaction rate, the selection
    that maximize the satisfaction rate of the high-level goals is picked. The
    monitoring system might also prompt the user to interactively define new
    countermeasures.
    
    \placefigure[bottom]
   	  [fig:satrate-locals_warned_when_risk_imminent]
   	  {Monitored satisfaction rate for the goal \goal{Achieve [Locals Warned When Risk
      Imminent]}.}
      {\externalfigure[../images/chap7/satrate-locals_warned_when_risk_imminent.pdf]}
  
    \placefigure[top]
   	  [fig:obstacle_adaptation]
   	  {Monitored satisfaction rate for the leaf obstacles.}
      {\startcombination[2*3]
        {\externalfigure[../images/chap7/satrate-dusty_environment-small.pdf]}{\tfx (\obstacle{\tfx Dusty Environment})}
        {\externalfigure[../images/chap7/satrate-echo-small.pdf]}{\tfx (\obstacle{\tfx False Echo})}
        {\externalfigure[../images/chap7/satrate-depth_broken-small.pdf]}{\tfx (\obstacle{\tfx Radar Sensor Broken})}
        {\externalfigure[../images/chap7/satrate-ultrasound_broken-small.pdf]}{\tfx (\obstacle{\tfx Ultrasound Sensor Broken})}
        {\externalfigure[../images/chap7/satrate-noisy_image-small.pdf]}{\tfx (\obstacle{\tfx Noisy Image})}
        {\externalfigure[../images/chap7/satrate-voice_dow-small.pdf]}{\tfx (\obstacle{\tfx Voice Network Overloaded})}
      \stopcombination}
    
    Back to our example,
    \in{Figure}[fig:satrate-locals_warned_when_risk_imminent] shows the
    satisfaction rate for the goal \goal{Achieve [Locals Warned When Risk
    Imminent]} computed from the monitored satisfaction rates of the leaf
    obstacles. (See \in{Figure}[fig:goal_model] for the goal model.)
    \in{Figure}[fig:obstacle_adaptation] shows the monitored satisfaction rates
    for our obstacles. Every 5 minutes, the optimization process compute the
    most appropriate countermeasure selection. At 08:26 and 09:21, an
    adaptation is required as the satisfaction rate is below the required
    satisfaction rate.
    
    The peak at 08:47 as seen in
    \in{Figure}[fig:satrate-locals_warned_when_risk_imminent] and in
    \in{Figure}{(\obstacle{Noisy Image})}[fig:obstacle_adaptation] illustrate
    the problem with statistically innaccurate satisfaction rates discussed in
    \in{Section}[sec:innaccuracy_margins]. Here, it did not trigger an
    adaptation as the violation was very short (it lasts 2 seconds).
    
  \stopsection
  
  \startsection[reference=sec:sw_update,title={Runtime Deployment of Most Appropriate Countermeasures}]
    
    When most appropriate countermeasures are selected and integrated in the
    goal model, the running software system must be adapted to match the
    updated goal model. The software component responsible for adapting the
    running system is named {\it Adaptor} in the following discussion. This
    component keeps track of a current selection of countermeasures. When the
    most appropriate countermeasures are computed, it identifies the
    countermeasures that are {\it (i)} added\emdash{}that is, not found in the
    current selection but in the selection of the most appropriate ones; and
    {\it (ii)} removed\emdash{}that is, no longer in the selection of the most
    appropriate ones.

    To adapt the software system, the {\it Adaptor} calls the activation and
    deactivation procedures associated with the countermeasures to be added or
    removed. The activation procedure is the code-related procedure responsible
    for the deployment of the corresponding countermeasure in the running
    system. The deactivation procedure is responsible for its removal. These
    procedures are used to: add, remove or replace a running component; update
    configuration parameters; activate hardware components; and so forth. The
    countermeasure goals are decorated with these procedures.

    For example, let us consider that the monitored satisfaction rate of the
    leaf obstacle \obstacle{Ultrasound Sensor Broken} increases at 08:26, as
    shown in \in{Figure}[fig:obstacle_adaptation]. This increase causes a
    decrease in the monitored satisfaction rate of the high-level goal
    \goal{Achieve [Locals Warned When Risk Imminent]} (as seen in
    \in{Figure}[fig:satrate-locals_warned_when_risk_imminent]) below its RSR.
    This decrease causes the countermeasure \goal{Achieve [Speed Acquired Every
    5 Sec By Camera]} to be selected as most appropriate countermeasure. The
    activation procedure for \goal{Achieve [Speed Acquired Every 5 Sec By
    Camera]} {\it DeployCamera} is called and replaces the software component
    acquiring the speed by the camera-related component. As
    \in{Figure}[fig:satrate-locals_warned_when_risk_imminent] shows, the
    satisfaction rate of the high-level goal increases above its RSR after
    software adaptation. At 09:21, the satisfaction rate of \goal{Achieve
    [Locals Warned When Risk Imminent]} fall again below the required
    satisfaction rate. This fall is due to an increase in the satisfaction rate
    for the obstacle \obstacle{Noisy Image} of the selected countermeasure
    goal. This cause the countermeasure goal \goal{Achieve [Speed Acquired
    Every 5 Sec By Camera]} to be removed from the selection, causing the
    deactivation procedure {\it DeployUltrasound} to be activated. The
    satisfaction rate increases back.
  
  \stopsection
  
	\startsection[reference=sec:update_k_uncertainty,title={Updating Expert Estimates At Run-Time}]
  
    Managing uncertainty at run-time is a critical challenge \cite[Ben10].
    \in{Chapter}[chap:knowledge-uncertainty] presented how knowledge
    uncertainty integrates with the obstacle assessment and control step. Such
    knowledge uncertainty might be reduced at runtime using the monitored
    information. Note that, however, if the observed values vary a lot over
    time, the knowledge uncertainty might increase.
    
    To integrate knowledge uncertainty, the current monitored satisfaction rate
    is no longer modeled as a single-value satisfaction rate but as a multi-value
    satisfaction rate, typically a Normal probability distribution.
    
    The process for obtaining the monitored satisfaction rate is updated as
    follows: {\it (a)} observed states are collected; {\it (b)} monitors are
    created and updated according the observed states; (if required, monitors
    are filtered to keep recent monitors only.) {\it (c)} monitors labeled
    with $T$ (true) and $F$ (false) are counted; {\it (d)} a Beta probability
    distribution is fitted to the number of monitors; {\it (e)} The Beta
    distribution updates the current monitored satisfaction rate using Bayesian
    inference; {\it (f)} The updated satisfaction rate becomes the new current
    satisfaction rate.
    
    The step {\it (d)} is straightforward as the parameters of the Beta
    distribution are, $\alpha$, the number of positive monitors and, $\beta$,
    the number of negative monitors \cite[Vos08]. The computation in step {\it
    (e)} relies on Bayesian Inference. Bayesian Inference is a statistical
    computation in which the Bayes' theorem is used to update a probability
    distribution as more information becomes available \cite[Wac07]. The
    updated satisfaction rate $P(O\mid M)$ ($M$ represents what has has been
    observed last) is given by
    
    \startformula
      P(O\mid M) = \frac{P(M\mid O)\times P(O)}{P(M)}
    \stopformula
    
    where $P(M\mid O)$ is the likelihood to observe the monitored satisfaction
    rate, $P(O)$ is the current monitored satisfaction rate, and $P(M)$ a
    scaling factor. The Beta distribution obtained at step {\it (d)} models the
    likelihood $P(M\mid O)$.
    
    \in{Figure}[fig:runtime_update] shows how the satisfaction rate for the
    obstacle \obstacle{Pump Mechanical Failure} is updated with the observed
    satisfaction rate. The lighest line shows the satisfaction rate provided by
    the experts, as \in{Chapter}[chap:knowledge-uncertainty] showed. The
    darkest line show the satisfaction rate updated after 6 observations. The
    other shades show the intermediate steps. For the example, we assumed that
    the monitored satisfaction rate was $.45$ for the three first observations
    and $.55$ for the three last. We can see that the more observations are
    made, the more accurate the satisfaction rate (the curve is narrower.)
  
    \placefigure[here]
   	  [fig:runtime_update]
   	  {Monitored satisfaction rate for the leaf obstacles using knowledge uncertainty.}
      {\externalfigure[../images/chap7/runtime_update.pdf]}
  
  \stopsection
  
  \startsection[title={Summary}]
  
    Software systems should adapt to changing conditions and assumptions in
    order to keep satisfying their goals. This chapter detailed an
    obstacle-driven runtime adaptation approach aimed at increasing the actual
    satisfaction rate of probabilistic system goals. Leaf obstacles are
    monitored at runtime to let the system dynamically switch to more
    appropriate countermeasure goals that increase the satisfaction rate of the
    system’s high-level goals under the current conditions. The approach
    guarantees that the required satisfaction rate of high-level goals remains
    satisfied when obstacle satisfaction rates are changing.

    The presented monitoring technique extends the $LTL_3$ approach to support
    the monitoring of probabilistic assertions. The monitors are built at RE
    time and, at runtime, virtual copies of $LTL_3$ monitors keep track of
    obstacle satisfaction. State probabilities for each observed state can
    thereby be estimated; their upper bound provides monitored obstacle
    satisfaction rates. The latter are propagated through the obstacle/goal
    model up to the system’s high-level goals. The monitored satisfaction rates
    obtained for these goals are compared with their required satisfaction
    rate; when the former fall below the latter, more appropriate
    countermeasures replace the current ones to remain above the required
    threshold.
    
    The next chapter will present the tools supporting the techniques presented
    in the thesis.
  
  \stopsection
  
\stopcomponent

% TODO
% - Not all obstacles are monitorable. It might be required to refine the obstacles until a software agent is able to monitor the obstacle.
% - Monitoring rare obstacles might underestimate the obstacle satisfaction rate. In particular when not used with prior knowledge.
