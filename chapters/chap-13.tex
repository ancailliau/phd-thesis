% !TEX root = thesis.tex

\startcomponent chap-13
\environment common
\product thesis

\chapter {Random Notes and Thoughts}

  \section {Generalizing the aggregation of state probabilities}

    The current technique aggregate the state probabilities by taking
    the smallest probability (for goals), and the largest probability (for obstacles).
    
    This gives a pessimistic view of the system, i.e. in the worst case, our system 
    provides a satisfaction rate of XXX.
    
    An other approach might be interrested in an optimistic view of the system, i.e.
    in the best case, our system provides a satisfaction rate of XXX (which, interestingly might not be 1.)
    In the context of anti-goal, this might make more sense.
    
    We might be interested in averaging the probabilities (possibly by weighting
    the states according the frequency we are likely to visit these). 
    
    More generaly, we might aggregate the state probabilities using a {\it r-norm}:
    
    \startformula
      P(G) = \left(\sum_{s \in States} w_s\cdot p_s^r\right)^{1/r}
    \stopformula
    
    Somes combinations are likely to preserve somes properties while violating others.

\stopcomponent
