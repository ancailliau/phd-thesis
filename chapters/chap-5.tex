% !TEX root = thesis.tex

\startcomponent chap-5
\environment common
\product thesis

\chapter[chap:controlling_obstacle]{Controlling Obstacles}

  Once identified and assessed, risks shall be adressed somehow. Critical and
  likely risks shall be cost-effectively controlled through appropriate
  countermeasures. Alternative countermeasures are identified and most
  appropriate countermeasures are selected. The selection process of most
  appropriate countermeasures might benefit from the requirement structure to
  ensure cost-effective selections. In addition, a precise characterization of
  appropriate countermeasures might guarantee that requirements are more
  complete and more accurate as the countermeasures are integrated in the ideal
  requirement model.

  This chapter details what are the most appropriate countermeasures goals, how
  to compute cost-effective selection of countermeasures goals, and how
  selected countermeasure goals are integrated in the ideal model.

  An appropriate countermeasure selection is a selection of valid
  countermeasures that ensure the satisfaction rate of high-level goals is
  higher than their required satisfaction rate, while maintaining a correct
  model. A valid countermeasure guarantee progress\emdash{}that is, once
  integrated in the model, the satisfaction rate of the high-level goals
  increases. Two integration schemas guarantee that only required parts of the
  model are modified. Lastly, propagation of changes ensure that correct model
  are still correct models after the integration process. As a result, the most
  appropriate countermeasure goals can be selected and integrated in the ideal
  model to produce more complete and more accurate models.

  This chapter is organized as follows. \in{Section}[sec:identifying_cm]
  details what is a valid countermeasure and how these countermeasures can be
  generated. \in{Section}[sec:iterating_oa] discuss how to determine whether an
  new cycle of obstacle analysis is required on the countermeasure goals.
  \in{Section}[sec:selecting_cm] details how most appropriate countermeasures
  are assessed and selected. \in{Section}[sec:integrating] describes how
  countermeasures are integrated into the ideal goal model.

  \startsection[reference=sec:identifying_cm,title={Identifying Valid Countermeasures}]
  
    The critical and likely obstacles shall be controlled by appropriate
    countermeasures \cite[Lam09a]. Countermeasures are typically identified
    using heuristics \cite[Lam00a], or by automatically generating goal
    revisions \cite[Alr16a].
    
    A {\it valid countermeasures} ensures that its integration will increase
    the satisfaction rate of high-level goal, in other words, its integration
    guarantee the progress towards more robust systems. To that aim, a valid
    countermeasure must satisfy the two following conditions:
  
    \startitemize

      \item {\bf Non-obstruction.} Either the countermeasure goal $CG$ is not
      obstructed by the obstacle $O$ it resolves

      \startformula
        \{O, Dom\} \models\not \neg CG \hskip1cm \text {(non-obstruction)}
      \stopformula

      or $O \wedge Dom$ is no longer realizable by the environment (non-feasibility).

      \item {\bf Ancestor entailment.} The countermeasure goal $CG$ must guarantee
      a deidealized version of an ancestor of the leaf goal obstructed by
      $O$. 

    \stopitemize
    
    \noindent The non-obstruction condition for a probabilistic goal is
    generalized as:
    
    \startformula
      P(CG\mid O,Dom) > 0 
    \stopformula
    
    The non-feasibility condition is derived from the realizability condition
    described in \cite[Lam09a,De-07a]. The environment mentionned in the
    condition monitors all the variables constrained by the countermeasure goal
    and the obstructed goal and controls all the others. This non-feasibility
    condition is critical the correctness of the {\it obstacle avoidance}
    resolution tactic. Assume an obstacle $O$, formalized as $\ltlF(C\wedge
    \ltlG OC)$, obstructing a goal $OG$, formalized as $C \Rightarrow \ltlF T$,
    assigned to an agent $Ag$. The condition $O \wedge Dom$ must be realizable
    by the environment monitoring the variables controlled by $Ag$ and
    controlling the others; otherwise, the obstacle would never happen. In our
    example, $OC$ is controlled by the environment. Assume that this obstacle
    is resolved by the use of a countermeasure goal $Avoid[O]$, formalized as
    $C \Rightarrow \neg OC$. This countermeasure goal will be refined until
    single agents might be assigned to the leaf goals. These agents will
    collaboratively control the variables constrained by the goal $Avoid[O]$,
    in our example $OC$. As a variable is not be controlled by multiple agents,
    the environment has no longer the control and monitoring capabilities to
    realize $O \wedge Dom$. The obstacles resolved by using the {\it avoid
    obstacle} tactic therefore satisfy the non-feasibility condition.
    
    Other obstacle resolution tactics such as {\it mitigate obstacle}, {\it
    weaken goal}, or {\it substitute goal} \cite[Lam09a] can be shown to
    produce valid countermeasures modulo propagations of their effect through
    the refinement trees in which they are involved (such propagation will be
    explained in \in{Section}[sec:integrating]).
    
    To make the ancestor entailment condition further precise we need the
    following definitions.

    \startdefinition[def:deidealized-goal]{Deidealized goal}
      
      A goal $G'$ is a {\it deidealized} version of goal $G$ if $G \models G'$.
  
    \stopdefinition

    \startdefinition[def:valid-cm]{Acceptable deidealization}
    
      A deidealized version $G'$ of $G$ is {\it acceptable} if, for every goal
      refinement with $G$ as a child, there exists an acceptable deidealized
      version of its siblings and parents such that the corresponding
      refinement still meets the completeness, consistency and minimality
      conditions recalled in \in{Chapter}[chap-background].
  
    \stopdefinition

    \noindent The ancestor entailment condition can now be formulated as
    follows:
  
    \startformula
      \{CG, G_1', ... , G_n', Dom'\} \models PG' 
        \hskip1cm \text {(ancestor-entailment)}
    \stopformula
  
    for some acceptable deidealized version $PG'$ of ancestor $PG$, where $PG$
    is an ancestor of the obstructed goal $G$ and $G_1', ..., G_n'$ are
    acceptable deidealized versions of descendants of $PG$.

    \startdefinition[def:valid-cm]{Valid countermeasure}
    
      A countermeasure goal $CG$ against obstacle $O$ is said to be {\it valid}
      if it satisfies the non-obstruction or non-feasibility condition together
      with the ancestor-entailment conditions.
  
    \stopdefinition

    For example, the countermeasure goal \goal{Achieve [Redundant Pump Motor On
    When Primary Pump Failure]} to the obstacle \obstacle{Pump Electrical
    Failure}, shown in \in{Figure}[fig:redundant_pump], is valid. The obstacle
    \obstacle{Pump Electrical Failure} does not obstruct it and this
    countermeasure together with \goal{Achieve [Valve Opened When Water
    Requested]} guarantee the satisfaction of the parent goal \goal{Achieve
    [Make Up Water Provided When Requested]}. In this example, the parent goal
    is not deidealized.

    \placefigure[bottom]
    	[fig:redundant_pump]
    	{A countermeasure goal to the obstacle \obstacle{Pump Electrical Failure}.}
      {\externalfigure[../images/chap5/redundant_pump.pdf][scale=1000]}

    Multiple candidate ancestors might be considered for the
    ancestor-entailment condition. In our example, \goal{Achieve [Make Up Water
    Provided When Requested]} and \goal{Achieve [Make Up Water Provided When
    Loss Of Cooling]} are potential candidates with respect to the goal
    \goal{Achieve [Redundant Pump Motor On When Primary Pump Failure]}. The
    nearest candidate to this goal appears preferable for more local model
    change\emdash{}that is, the goal \goal{Achieve [Make Up Water Provided When
    Requested]}.

    The anchor for a countermeasure goal $CG$ is the lowest ancestor goal $PG$
    meeting the ancestor-entailment condition. It is the goal through which the
    countermeasure goal is integrated, as discussed in
    \in{Section}[sec:integrating].

    A valid countermeasure is a sufficient condition to guarantee an increase
    in the satisfaction rate of some high-level goal.

    \startproposition{Progress}
  
      For any valid countermeasure goal CG, the probability of satisfaction
      of its anchor PG increases:

      \startformula P (PG') > P (PG) \stopformula

      where $PG'$ in $M'$ corresponds to $PG$ in $M$.
      
    \stopproposition

    \startproof

      This can be proved {\it ab absurdo}. Assume there is no such increase:
      $P(PG') \leq P(PG)$. Introducing conditional probabilities, we have:

      \startformula
        \startalign
          \NC P(PG') = \NC P(O) \times P(PG'\mid O) + P(\neg O) \times P(PG'\mid \neg O)\NR
          \NC P(PG) =  \NC P(O) \times P(PG\mid O) + P(\neg O) \times P(PG\mid \neg O)\NR
        \stopalign
      \stopformula

      Given the obstruction of $PG$, we have: $P(PG\mid O) = 0$. Therefore,

      \startformula
        P(O) \times P(PG'\mid O) + P(\neg O) \times P(PG'\mid \neg O) \leq P(\neg O) \times P(PG\mid \neg O) 
      \stopformula
  
      Since $PG'$ is a deidealized version of $PG$, we have:

      \startformula
        P(PG'\mid \neg O) \geq P(PG\mid \neg O).
      \stopformula
  
      Therefore, we can multiply both sides by the positive term $P(\neg O)$
  
      \startformula
        P(\neg O) \times P(PG'\mid \neg O) \geq P(\neg O) \times P(PG\mid \neg O).
      \stopformula
  
      As $P(PG'\mid O) > 0$, our initial assumption gets contradicted.
    %
    \stopproof
    
  \stopsection
  
  \startsection[reference=sec:iterating_oa,title={Iterating Obstacle Analysis}]

    As \in{Figure}[fig:redundant_pump] shows, a countermeasure goal should
    itself be refined down to leaf goals assignable to single agents. These
    leaf goals might be obstructed by new obstacles, a new cycle of obstacle
    analysis may be needed at RE time. To determine whether a new obstacle
    analysis cycle is required, the maximal change in satisfaction rate of the
    considered high-level goal should be considered. At best, the satisfaction
    rate of the countermeasure goal is $1$ (no possible obstruction of it). By
    up-propagation, this satisfaction rate leads to a satisfaction rate $sr_1$
    for the high-level goal. At worst, the satisfaction rate is $0$. This leads
    to another satisfaction rate $sr_0$ for the high-level goal. The
    countermeasure's maximal impact is obtained by taking the difference $sr_1
    - sr_0$. Based on this, the analyst may decide at RE time whether a new
    cycle is needed or not.

    \startluacode
      sr0 = 81.74
      sr1 = 94.94
      max_sr = sr1 - sr0
    \stopluacode

    For example, consider the countermeasure goal \goal{Achieve [Redundant Pump
    Motor On When Primary Pump Failure]}. If the satisfaction rate of this
    countermeasure is $0$, the satisfaction rate of the high-level goal
    \goal{Achieve [Make Up Water Provided When Loss Of Cooling]} is
    \ctxlua{round2(sr0)}\%. If the satisfaction rate of this countermeasure is
    1, the satisfaction rate of the high-level goal is \ctxlua{round2(sr1)}\%.
    The maximal impact on the high-level goal is thus
    \ctxlua{round2(max_sr)}\%. As the impact is important, it may be worth
    spending time in studying obstacles to this countermeasure goal at RE time
    to better estimate the impact of its deployment.
  
  \stopsection
  
	\section[sec:selecting_cm]{Selecting Most Appropriate Countermeasures}

    Thanks to the progress property introduced in the previous section, once a
    countermeasure is integrated into the goal model, the satisfaction rate of
    some high-level goal must improve. This improvement quantifies the impact
    of a countermeasure and is the key to select the most appropriate ones, as
    we aimed at maximizing this impact.

    \in{Section}[sec:assessing_impact] describes how the impact of alternative
    countermeasures is assessed. \in{Section}[sec:most_appropriate] clarifies
    what are \quote{most appropriate} countermeasures to be selected and
    deployed when an adaptation is required. \in{Section}[sec:selecting_cm]
    explains how such selection is computed.

		\subsection[sec:assessing_impact]{Assessing the Impact of Countermeasures}
  
      Integrating a countermeasure goal into the goal model impacts the
      satisfaction rate of a high-level goal. As \in{Section}[sec:integrating]
      will show, depending on the selected obstacle resolution strategy and
      associated countermeasure integration schema, the obstructed goal is
      either removed from the goal model or kept. In the former case, the
      obstructed goal is replaced by the countermeasure goal; in the latter
      case, a new refinement is introduced involving both the obstructed goal
      and the countermeasure goal. The satisfaction rate of the high-level goal
      is computed as described next to evaluate the impact of the
      countermeasure.

      \noindent {\bf Assessing countermeasure impact when the obstructed goal
      is replaced.} In this case, when the monitored satisfaction rate of the
      high-level goal is computed, the propagation procedure computing the
      satisfaction rate of a high-level goal shall use the satisfaction rate of
      the countermeasure goal in place of the satisfaction rate of the replaced
      goal.
  
      For example, the goal \goal{Achieve [Make Up Water Automatically
      Requested When Alarm Raised]} was generated using the agent substitution
      strategy. When integrated, it replaces the goal \goal{Achieve [Make Up
      Water Requested When Alarm Raised]}. The computation of the satisfaction
      rate of the parent goal \goal{Achieve [Make Up Water Requested When Loss
      Of Cooling]} uses the satisfaction rate of this countermeasure goal
      rather than the satisfaction rate of the replaced goal. When integrated,
      the satisfaction rate for the top goal increases from 81.74\% to 81.77\%.
    
%     To enable efficient computation of the impacts, the BDD-based propagation
%     technique is extended to support resolutions. We introduce a new binary 
%     variable $R_i$ that indicates whether the resolution $i$ is selected.
%     As recall, the obstruction superset for a goal $PG$ refined with two
%     subgoals $G_1$ and $G_2$ is given by
%
%     \startformula
%       OS(PG) = OS(G_1) \cup OS(G_2)
%     \stopformula
%     
%     Assume a resolution $i$ with anchor goal $PG$, countermeasure goal $CM$
%     and whose corresponding obstructed goal is $G_2$. If resolution $i$ is
%     integrated, the obstruction superset for $PG$ should be:
%
%     \startformula
%       OS(PG) = OS(G_1) \cup OS(CM)
%     \stopformula
%     
%     Using an extra variable $R_i$, the obstruction superset is given by
%     
%     \startformula\startalign[n=3]
%       \NC OS(PG) \NC = \NC (OS(G_1) \cup OS(G_2)) \times \{\neg R_i\} \NR
%       \NC        \NC   \NC \cup (OS(G_1) \cup OS(CM)) \times \{R_i\}  \NR
%     \stopalign\stopformula
%     
%     When computing the probability of the obstruction superset, $R_i$
%     is assigned $1$ if the resolution is selected, $0$ otherwise.
%     
%     \noindent In the example above, before integrating the resolution, we
%     have:
%
%     \startformula\startalign[n=1,align=left]
%       \NC OS(\goal{Achieve [Make Up Water Requested When Loss Of Cooling]}) \NR\noalign{\vskip-.5em}
%       \NC \hskip1cm = \{\{\obstacle{Alarm Not Raised And Low Water}\}, \NR\noalign{\vskip-.5em}
%       \NC \hskip2cm \{\obstacle{Make Up Water Not Requested And Alarm Raised}\}\} \NR
%     \stopalign\stopformula
%     
%     When the resolution is integrated, we have:
%
%     \startformula\startalign[n=1,align=left]
%       \NC OS(\goal{Achieve [Make Up Water Requested When Loss Of Cooling]}) \NR\noalign{\vskip-.5em}
%       \NC \hskip1cm = \{\{\obstacle{Alarm Not Raised And Low Water}\}\} \NR
%     \stopalign\stopformula
%     
%     Using the extra variable $R$ to represent the selection of the
%     aforementioned resolution, we have
%
%     \startformula\startalign[n=1,align=left]
%       \NC OS(\goal{Achieve [Make Up Water Requested When Loss Of Cooling]}) \NR\noalign{\vskip-.5em}
%       \NC \hskip1cm = \{\{\neg R, \obstacle{Alarm Not Raised And Low Water}\}, \NR\noalign{\vskip-.5em}
%       \NC \hskip2cm \{\neg R, \obstacle{Make Up Water Not Requested And Alarm Raised}\} \NR\noalign{\vskip-.5em}
%       \NC \hskip2cm \{R, \obstacle{Alarm Not Raised And Low Water}\}\} \NR
%     \stopalign\stopformula
  
      \noindent {\bf Assessing countermeasure impact when the obstructed goal
      is kept.} In this case, the anchor goal is refined into two subgoals. As
      a recall, one subgoal is the obstructed goal conjoined with the negation
      of the obstacle condition; the other subgoal is the countermeasure goal.
      The satisfaction rate for the anchor goal is therefore a combination of
      the satisfaction rates of those two subgoals.
  
      For example, integrating the countermeasure goal \goal{Achieve [Redundant
      Pump Motor On When Primary Pump Failure]} causes the anchor goal to be
      refined in two subgoals: \goal{Achieve [Make Up Water Provided When
      Requested And No Pump Failure]} and \goal{Achieve [Redundant Pump Motor
      On When Primary Pump Failure]}. The satisfaction rate of the root goal
      increases from 81.74\% to 94.94\%: The down-propagation of changes in the
      goal graph reduces the formal specification of the obstacles
      \obstacle{Pump Electrical Failure} and \obstacle{Pump Mechanical Failure}
      to false; The satisfaction rate of the latter is therefore $0$.
    
%     Similarly to assessing when the obstructed goal is removed, the BDD-based propagation 
%     is extended to support these resolutions. As in \in{Figure}[fig:integration_remove],
%     assume a resolution $i$ with anchor goal $AG$, countermeasure goal $CM$
%     and whose corresponding leaf obstacle is $LO$. If resolution $i$ is
%     integrated, the obstruction superset for $PG$ should be:
%
%     \startformula
%       OS(AG') = \{os\setminus\{LO\}\mid os\in OS(AG)\} \cup OS(CM)
%     \stopformula
%     
%     The leaf obstacle is removed from the obstruction set of the obstructed
%     goal $AG$ and obstacles to the countermeasure goal are taken into
%     account. Using an extra variable $R_i$, the obstruction superset is given
%     by
%     
%     \startformula\startalign[n=3]
%       \NC OS(AG') \NC = \NC (OS(AG) \times \{\neg R_i\}) \NR
%       \NC         \NC   \NC \cup (\left[\{os\setminus\{LO\}\mid os\in OS(AG)\} \cup OS(CM)\right] \times \{R_i\}) \NR
%     \stopalign\stopformula
  
      Such computation is very easy when goal exception and goal replacement
      annotation are used. The resolutions are integrated and removed to
      evaluate the impact of the selection. This however requires the BDD to be
      built at each combination, which slow down the optimization process.
      Improvements are expected as a single BDD including all possible
      combinations using extra variables to encode which resolution is
      selected, could be constructed; Such BDD might be computed once at
      RE-time.
  
		\subsection[sec:most_appropriate]{What are Most Appropriate Countermeasures?}
	
      Countermeasure selection should increase the satisfaction rate of
      high-level goals above their RSR but cost reasonably.
    
      \startdefinition{Safe Selection}
    
        A {\it safe selection} of countermeasures is a set of countermeasures
        that, once integrated, guarantees that the satisfaction rate of the
        high-level goals is greater than their respective RSR.
    
      \stopdefinition
    
      A selection of countermeasures must be consistent, as conflicting
      countermeasures shall not be selected together:
    
      \startformula
        \{CM_i, CM_j, Dom\} \nvDash false \text{ for } i \neq j \hskip1cm \text{(selection-consistency)}
      \stopformula
    
      Countermeasures come with a cost. The latter are elicited with regard to
      the countermeasure's contributions to optimization goals in the goal
      model\emdash{}such as \goal{Maximize[Operator Safety]}, \goal{Minimize[Power
      Consumption]} or \goal{Maximize[Speed Computation]}. Without loss of
      generality, only single costs are considered here. Multiple costs are
      combined by a weighted sum where weights correspond to the priorities of
      the considered soft goals.

      \startdefinition{Resolution Cost}

        The {\it resolution cost} of a selection is the sum of the costs associated
        with the selected countermeasures.

      \stopdefinition

      A selection is {\it cost-optimal} if no other selection increases the
      probability of satisfaction of the high-level goal without increasing the
      resolution cost. The selection of most appropriate countermeasures is
      defined as the one having the lowest resolution cost such that the
      satisfaction rate for the high-level goals is maximized while being safe,
      consistent, and cost-optimal.
        
		\subsection[sec:selecting_cm]{Selecting Countermeasures}

      The selection of most appropriate countermeasures amounts to solving two
      optimization problems:
    
      \startitemize

        \item Finding the minimal cost for guaranteeing the RSR of the
        high-level goals,

        \item Finding the selections that maximizes the satisfaction rate of
        the high-level goals given this cost.

      \stopitemize
    
      \noindent {\bf Computing the minimal cost.} We may iteratively generate
      all possible selections and keep the selection that minimizes cost while
      guaranteeing that the RSRs are met. The complexity of this naïve approach
      is $O(2^n)$ where is $n$ the number of countermeasures.
    
      \noindent {\bf Computing the cost-optimal selection.} We may then
      generate all possible selections, compute their cost, and keep the
      selection with a minimal cost and the largest satisfaction rate for the
      high-level goals. The complexity of this naïve computation is $O(2^n)$
      for $n$ countermeasures.
    
      In our example, the minimal cost for guaranteeing the RSR of 95\% for the
      goal \goal{Achieve [Make Up Water Provided When Loss Of Cooling]} is 1.
      With 6 possible resolutions, there are 63 possible selections. Among them
      24 are safe selections that guarantee the RSR with costs ranging from 1
      to 6 (we used unitary costs for the countermeasure goals). A single
      selection of two countermeasure goals maximize the satisfaction rate of
      \goal{Achieve [Make Up Water Provided When Loss Of Cooling]}:
      \goal{Achieve [Cooling System Repaired]} and \goal{Achieve [Power
      Supplied When Primary Power Supply Failure]}. On this running example,
      the best satisfaction rate that can be achieve is 97.18\%.
    
      Our naïve approach to selecting the most appropriate countermeasures
      could be improved by exploring different directions:
    
      \startitemize
    
        \item The problem of finding the cost-optimal selection shares
        similarities with the NP-hard Knapsack optimization problem
        \cite[Cor09a]. The latter is concerned with filling a bag with valued
        items without exceeding a maximal weight while maximizing value. The
        problem here differs in that {\it (i)} the value and weight of items do
        not simply sum; and {\it (ii)} adding a new countermeasure goal does
        not necessarily increase the satisfaction rate of a high-level goal.
        Improvements of our naïve algorithms are however expected as a
        pseudo-polynomial algorithm exists for the Knapsack problem
        \cite[Cor09a].
  
        \item Other techniques such as \cite[Men03a] might also improve
        selections. Incremental control learning is a technique that present,
        at each iteration, a set of key variables (in our case, countermeasure
        goals) that have the most impact on the system. This technique has
        already been applied to select risk mitigations in the DDP
        framework showing promising results. (See \in{Chapter}[chap:relatedwork]
        for a detailed discussion about DDP.)
    
        \item Lastly, Cross-Entropy method is an optimization technique for
        solving combinatorial problems \cite[De-05a]. Applied to our problem,
        the method can be broken in two steps: {\it (1)} an initial selection
        is generated; {\it (2)} the parameters of the random generation are
        updated according the score of the generated selection. The steps are
        run a large number of times. Such technique has been successfully
        applied for selecting software adaptations at runtime \cite[Mor17a].
    
      \stopitemize

  
  \startsection
    [reference=sec:integrating,
     title={Integrating Countermeasures in the Goal Model}]
	
    Selected countermeasures shall be integrated in the ideal goal model.
    Integrating a countermeasure goal to an obstacle in a goal model means:
    {\it (a)} connecting this goal to a parent node in the model; {\it (b)}
    adding other sibling subgoals in the refinement if necessary; {\it (c)}
    propagating the corresponding changes along refinement trees in which the
    countermeasure goal is involved; and {\it (d)} refining this goal if
    necessary.

    More precisely, an integration $Int_{CG,O}(M)$ of a countermeasure $CG$ to
    an obstacle $O$ in goal model $M$ maps $M$ to a new model $M'$ so as to
    satisfy the following desired properties.
      
    \startitemize[n]

      \item {\bf Progress.} The application of the integration operator
      $Int_{CG,O}$ should increase the probability of satisfaction of some root
      goal $G$ at least, that is, there must be at least one root goal $G'$ in
      $M'$ corresponding to $G$ in $M$ such that $P_{M'} (G') > P_M (G)$, where
      $P_M(G)$ refers to the satisfaction rate of $G$ in model $M$. As seen in
      \in{Section}[sec:identifying_cm], this is enforced by valid
      countermeasures.

      \item {\bf Minimal change.} The application of $Int_{CG,O}$ should
      preserve prescribed behaviors in $M$ that are not affected by $O$, that
      is, for any goal $G$ in $M$ such that $\{O, Dom\} \models\not \neg G$ and
      corresponding goal $G'$ in $M'$, we should have: $G' \models G$.

      \item {\bf Correctness preservation.} The correctness of goal refinements
      in $M$ should be preserved in $M'$, that is, if all refinements in $M$
      are complete, consistent, and minimal then those in $M'$ are complete,
      consistent, and minimal as well.
    
    \stopitemize

    \in{Section}[sec:integration_schema] shows how the minimal change property
    is guaranteed by the integration of a countermeasure.
    \in{Section}[sec:propagation] details how correctness is preserved through
    the integration. These two sections introduces the countermeasure goals in
    the ideal goal model by modifying the refinement structure.
    \in{Section}[sec:exception_handling] details how explicit modeling of
    exceptions improves the documentation of these exceptional cases.
	
    \startsubsection
      [reference=sec:integration_schema,
       title={Ensuring Minimal Changes: Integration Schemas}]
        
      A single valid countermeasure goal ensures progress towards a complete
      model; its integration in the model should ensure that the minimal change
      property is met, i.e. nothing more than necessary is modified.

      Two alternative integration schemas may be used for this, dependent on
      the obstacle resolution tactic being selected \cite[Lam09a,Lam00a]. The
      first schema removes the obstructed goal; it should be applied when the
      {\it substitute goal} or {\it weaken goal} tactic is used for resolving
      the obstacle. The second integration schema keeps the obstructed goal in
      the model; it should be applied when the {\it avoid obstacle}, {\it
      reduce obstacle likelihood}, or {\it mitigate obstacle} tactic is used.

      \noindent {\bf Removing the obstructed goal.}
      \in{Figure}[fig:integration_remove] shows a first integration schema
      expressed as a model rewriting rule. In this first schema, the refinement
      of anchor goal $AG$, containing at least one obstructed goal, is replaced
      with a new refinement. The latter contains the countermeasure goal $CG$
      to leaf obstacle $LO$ together with all non-obstructed children. (An
      anchor's obstructed children are those being directly or indirectly
      obstructed by $LO$.) The anchor $AG$ may need to be deidealized; in this
      case, $AG'$ replaces $AG$ in the new goal model.

      \placefigure[here]
      	[fig:integration_remove]
      	{Integration schema with obstructed goal being removed.}
        {\externalfigure[../images/chap5/integration_remove.pdf][scale=1000]}

      This first integration schema has a precondition for use, namely, the
      countermeasure goal $CG$ and the non-obstructed children are sufficient for
      satisfying the anchor goal:
      
      \startformula
        \{CG, Children(AG) \setminus ObstructedChildren\} \models AG'
      \stopformula

      Otherwise, the new refinement would not be complete. 

      For example, the goal \goal{Achieve [Make Up Water Requested When Alarm
      Raised]} assigned to the agent \agent{Controlling Software} is a
      countermeasure produced through the {\it agent substitution} tactic
      \cite[Lam09a,Lam00a]. Its anchor goal is \goal{Achieve [Make Up Water
      Requested When Loss Of Cooling]}. The following refinement is complete,
      consistent, and minimal for this anchor goal:

      \startkaosspec
        Achieve [Make Up Water Requested When Loss Of Cooling]\blank[none]
        \rightarrow\hskip.5em Achieve [Make Up Water Requested When Alarm Raised]\blank[none]
        \rightarrow\hskip.5em Achieve [Alarm Raised When Low Water]
      \stopkaosspec

      \noindent We may therefore replace the old refinement with this one.

      It is easy to see that this first integration schema meets our minimal
      change property. Non-obstructed goals are composed from non-obstructed
      goals in the refinements of $AG$ and its descendants, and in the siblings
      of $AG$ and its ancestors. The former are kept as is whereas the latter
      might need to be deidealized through change propagation (see
      \in{Section}[sec:propagation]). All these goals thus satisfy $G' \models
      G$.

      \noindent {\bf Keeping the obstructed goal.}
      \in{Figure}[fig:integration_keep] shows a second integration schema. In
      this schema, a new refinement is introduced; it includes a modified
      version of the obstructed anchor and the countermeasure goal. The
      obstructed anchor is deidealized for removing the obstruction. The
      negation of the leaf obstacle is added to form a decomposition by cases.

      \placefigure[here]
      	[fig:integration_keep]
      	{Integration schema with obstructed goal being kept.}
        {\externalfigure[../images/chap5/integration_keep.pdf][scale=1000]}

      This second integration schema has a precondition for use as well; the
      countermeasure goal must be sufficient for satisfying the anchor goal
      when the obstacle occurs:

      \startformula
        \{LO, CG, Dom\} \models AG'
      \stopformula
      
      For example, the countermeasure goal \goal{Achieve [Cooling System
      Repaired]} entails its anchor \goal{Achieve [Make Up Water Provided When
      Loss Of Cooling]} when the obstacle \obstacle{Valve Mechanical Failure}
      occurs. This second rule then produces the following refinement:

      \startkaosspec
        Achieve [Make Up Water Provided When Loss Of Cooling]\blank[none]
        \rightarrow\hskip.5em Achieve [Make Up Water Provided When Loss Of Cooling\blank[none]
        \hskip4em And No Valve Mechanical Failure]\blank[none]
        \rightarrow\hskip.5em Achieve [Cooling System Repaired]
      \stopkaosspec

      Note that the anchor goal is not deidealized in this example. The new
      goal \goal{Achieve [Make Up Water Provided When Loss Of Cooling And No
      Valve Mechanical Failure]} is a de-idealization of the anchor goal. The
      negation of the obstacle is added to the goal antecedent:

      \startformula
        LossOfCooling \hskip.7ex\graymath{\wedge\hskip.5ex \neg MakeUpValveFailure}\hskip.7ex 
        \Rightarrow \ltlF_{\leq 55h} MakeUpWaterProvided
      \stopformula

      This second integration schema can be seen to meet our minimal change
      property as well; the reasoning is similar to the first schema. Only
      siblings of the anchor goal and its ancestor may need to be deidealized.
      The other goals were obstructed by the resolved obstacle and are
      therefore not concerned with the minimal change property.
    
    \stopsubsection
    
    \startsubsection
      [reference=sec:propagation,
       title={Preserving Refinement Correctness: Change Propagation}]
      
      Goal deidealizations and countermeasure integrations may require
      corresponding changes to be propagated along refinement trees in which
      the countermeasure goal is involved. Such propagations are intended to
      ensure our third property on integrations, that is, the resulting model
      must remain complete, consistent, and minimal. This section discusses how
      deidealizations and change propagations are performed.

      \noindent {\bf Deidealization by strengthening the goal's antecedent.} A
      first way of deidealizing a goal of form $C\Rightarrow\Theta T$ is to add
      an adequate conjunct to its antecedent:

      \startformula
        AddConjunct (C \Rightarrow \Theta T, \Theta EC) 
          = \[C \wedge \Theta EC\] \Rightarrow \Theta T
      \stopformula

      For example, the goal \goal{Achieve [Make Up Pump Motor On When Water
      Requested]} may be deidealized so as to exclude pump failure from pump
      actuation:

      \startformula
        WaterRequested \wedge \neg PumpFailure \Rightarrow \ltlF_{\leq 5m} PumpMotorOn
      \stopformula

      \noindent {\bf Deidealization by weakening the goal's consequent.} A
      second way of deidealizing the goal $C\Rightarrow\Theta T$ is to add an
      adequate disjunct to its consequent:

      \startformula
        AddDisjunct (C \Rightarrow \Theta T, \Theta EC) 
          = C \Rightarrow \[\Theta T \vee \Theta EC\]
      \stopformula

      For example, the goal \goal{Achieve [Make Up Pump Motor On When Water
      Requested]} might be deidealized so as to require the pump or the
      redundant pump to be actuated:

      \startformula
        WaterRequested \Rightarrow \ltlF_{\leq 5m} PumpMotorOn 
          \vee \ltlF_{\leq 10m} RedundantPumpMotorOn
      \stopformula

      \noindent {\bf Change propagation.} When a goal is deidealized, the
      change must be propagated along the refinement trees in which this goal
      is involved\emdash{}both up and down such trees. When applying the {\it AddConjunct}
      or {\it AddDisjunct} operators the corresponding syntaxic changes
      are applied recursively
      to the other goals up and down refinement links. 

      \placefigure[here]
      	[fig:cooling_repaired]
      	{A countermeasure goal to the obstacle \obstacle{Valve Mechanical 
         Failure}.}
        {\externalfigure[../images/chap5/cooling_repaired.pdf][scale=1000]}

      For example, the integration of the countermeasure goal \goal{Achieve
      [Cooling System Repaired]} for the obstacle \obstacle{Valve Mechanical
      Failure}, as shown in \in{Figure}[fig:cooling_repaired] requires change
      propagation to the descendants of the goal \goal{Achieve [Make Up Water
      Provided When Loss Of Cooling]}. First, this goal is modified as
      previously shown:
        
      \startformula
        LossOfCooling \wedge \neg MakeUpValveFailure 
          \Rightarrow \ltlF_{\leq 55h} MakeUpWaterProvided
      \stopformula
      
      Next, the change is down-propagated, leading to an application of {\it
      AddConjunct} to the goal \goal{Achieve [Make Up Water Provided When
      Requested]}:
        
      \startformula\startalign[n=1,align={left}]
        \NC MakeUpWaterRequested \wedge \neg MakeUpValveFailure \NR\noalign{\vskip-.5em}
        \NC \hskip1em\Rightarrow \ltlF_{\leq 24h} MakeUpWaterProvided \NR
      \stopalign\stopformula
      
      The next refinement instantiates the divide-and-conquer-driven refinement
      pattern. The {\it AddConjunct} operator is applied to the goal
      \goal{Achieve [Valve Opened When Water Requested]}. The operator could
      have been applied to the goal \goal{Achieve [Make Up Pump Motor On When
      Water Requested]} but this was assumed irrelevant. We obtain:

      \startformula\startalign[n=1,align={left}]
        \NC MakeUpWaterRequested \wedge \neg MakeUpValveFailure  \NR\noalign{\vskip-.5em}
        \NC \hskip1em\Rightarrow \ltlF_{\leq 15m} MakeUpValveOpen \NR
      \stopalign\stopformula

      Since this goal is a leaf goal, the propagation ends.

      Change propagation in the general case proceeds as follows. When {\it
      AddConjunct} is applied to a goal for deidealization, the specification
      pattern followed by the formal specification of the goal is identified
      and used to extract the antecedant. The de-idealized antecedant is
      obtained by adding the conjunction. The ideal antecedant is then replaced
      by the de-idealized one. The process is applied recursively to the
      subgoals (resp. parents) until leaf goals (resp. root goals) are reached.
      The procedure is similar when applying {\it AddDisjunct}.

      Such propagation technique produces formal specification that can often
      be simplified. Moreover, other mechanisms are required for change
      propagation to goals not matching a known specification pattern. Even
      though pattern-based propagation can be performed semi-automatically
      interacting with the user when no specification pattern can be matched.
      The general problem of automatic change propagation through arbitrary
      refinement structures remains open.
    
    \stopsubsection
    
		\startsubsection[reference=sec:exception_handling,title={Obstacle Resolution as Exception Handling}]

      The integration of countermeasure goals through new, explicit refinements
      in the original model raises several issues.
      
      \startitemize
        
        \item The goal graph might undergo significant changes each time a new
        obstacle is identified.
        
        \item Normal situations would be mixed with exceptional ones; it might
        be hard to distinguish the former from the latter without domain
        expertise.
        
        \item Goal specifications become increasingly more complex.
        
        \item As new countermeasures are introduced, the ordered nesting of
        exceptional cases along refinements may lead to a combinatorial blow-up
        of special cases.

      \stopitemize
      
      \noindent This section shows how dedicated constructs helps solving these issues.
      
      \noindent {\bf Extending the Goal Specification Language.} Dedicated
      constructs are provided for encapsulating the required modifications
      while documenting each exceptional case separately.
    	  
        \noindent {\it Except.} A first construct links a countermeasure goal to its anchor
        goal:
      
        \startkaosspec
          \GoalName {$AG$}
          \KaosAttribute {FormalSpec} {$C\Rightarrow\Theta T$}
          \KaosAttribute {Except} {$O$ {\cronos\bf Then} $CG$}
        \stopkaosspec
        
        \noindent where $AG$ denotes the anchor goal $C \Rightarrow \Theta T$
        of a countermeasure goal $CG$ to an obstacle $O$. Semantically, this
        implicit specification is fully equivalent to the refinement in
        \in{Figure}[fig:except_semantic]. This construct may be used under the
        following precondition:

        \startformula
          \{O, CG, Dom\} \models AG
        \stopformula

        \placefigure[top]
        	[fig:except_semantic]
        	{Semantic equivalent of {\it Except}.}
          {\externalfigure[../images/chap5/except_semantic.pdf][scale=1000]}

        For example, consider the goal \goal{Achieve [Make Up Water Provided When Loss Of
        Cooling]}.
        Under the exceptional condition of a mechanical failure of the make up
        valve, the goal is guaranteed through repairing the cooling system. We
        may therefore specify the following:
     
        \startkaosspec
          \GoalName {Achieve [Make Up Water Provided When Loss Of Cooling]}
          \KaosAttribute {FormalSpec} {$C\Rightarrow\Theta T$}
          \KaosAttribute {Except} {Valve Mechanical Failure {\cronos\bf Then} Achieve [Cooling System Repaired]}
        \stopkaosspec
       
        This specification is logically equivalent to the refinement illustrating
        the second integration schema in \in{Section}[sec:integrating].

        Multiple {\it Except} annotations may be attached to a single goal to
        cope with different obstacles; the latter may therefore be introduced
        incrementally. Compared with the complexity of an equivalent explicit
        specification, the complexity of an implicit goal specification with
        multiple {\it Except} annotations remains linear in the number of
        exceptions. The specification of the ideal goal remains unchanged.
        Moreover, multiple annotations sharing the same countermeasure goal may
        be factored out to simplify the model.

        \noindent {\it Provided.} A second construct specifies an extra conjunct
        on the antecedent of an ideal goal $G$ to produce a countermeasure:
     
        \startkaosspec
          \GoalName {$AG$}
          \KaosAttribute {FormalSpec} {$C\Rightarrow\Theta T$}
          \KaosAttribute {Provided} {$EC$}
        \stopkaosspec
        
        \noindent where $EC$ denotes an extra conjunct to be added to the goal
        antecedent for resolving the considered obstacle. Semantically, this
        implicit specification is equivalent to:
     
        \startkaosspec
          \GoalName {$AG$}
          \KaosAttribute {FormalSpec} {$C \wedge EC \Rightarrow \Theta T$}
        \stopkaosspec

        The {\it Provided} construct is typically used for deidealizing goals.
        For example, the deidealization of the goal \goal{Achieve [Make Up Pump
        Motor On When Water Requested]} may be specified by highlighting the
        normal situation as follows:
     
        \startkaosspec
          \GoalName {Achieve [Make Up Pump Motor On When Water Requested]}
          \KaosAttribute {Provided} {$\neg PumpMechanicalFailure$}
        \stopkaosspec
            
        \noindent It often proves convenient to write {\it ProvidedNot $EC$} instead of
        {\it Provided $EC$}.
        
        \noindent {\it RelaxedTo.} Symmetrically to {\it Provided}, this
        construct specifies an extra disjunct on the consequent of an ideal goal
        G to produce a countermeasure:
        
        \startkaosspec
          \GoalName {$AG$}
          \KaosAttribute {FormalSpec} {$C\Rightarrow\Theta T$}
          \KaosAttribute {RelaxedTo} {$ED$}
        \stopkaosspec

        \noindent where $ED$ denotes an extra disjunct to be added to the goal consequent
        for resolving the considered obstacle. Semantically, this goal is
        equivalent to:
        
        \startkaosspec
          \GoalName {$AG$}
          \KaosAttribute {FormalSpec} {$C \Rightarrow \Theta T \vee ED$}
        \stopkaosspec

        This construct is useful for deidealizing goals as well. For example,
        another deidealization of the same goal \goal{Achieve [Make Up Pump Motor
        On When Water Requested]} might be specified by highlighting the normal
        situation as follows:
        
        \startkaosspec
          \GoalName {Achieve [Make Up Pump Motor On When Water Requested]}
          \KaosAttribute {RelaxedTo} {$RedundantPumpMotorOn$}
        \stopkaosspec
        
        Multiple {\it Provided} and {\it RelaxedTo} annotations may be attached
        to a single goal to introduce multiple countermeasures.
        
        \noindent {\it Replaces.} This construct appears useful for tracing
        previous versions of a goal:
        
        \startkaosspec
          \GoalName {$AG$}
          \KaosAttribute {Replaces} {$AG'$}
        \stopkaosspec
        
        \noindent Such traceability helps readers understand the rationale behind
        the final goal.
      
      \noindent {\bf Model Refactoring for Goal Exceptions.} In practice, the
      analyst should decide at some point whether a countermeasure goal refers
      to an exceptional situation or to a normal one to be considered in the
      ideal model. Such a decision might depend on various factors such as the
      satisfaction rate of the resolved obstacle, the criticality of the
      obstructed goal, domain-specific culture, stakeholders wishes, and so
      forth. To make the decision flexible and easily reversible, we presents
      here model refactoring operators for attaching or detaching the
      annotations introduced in the previous section to/from a goal model.
      Three operators are available for transforming an annotated model portion
      into a standard one.

        \placefigure[bottom]
        	[fig:refactoring_except]
        	{Implicit and explicit countermeasure integration.}
          {\startcombination[2*1]
            {\externalfigure[../images/chap5/refactoring_except_a.pdf]}{\tfx(a)}
            {\externalfigure[../images/chap5/refactoring_except_b.pdf]}{\tfx(b)}
           \stopcombination
          }
        
        \noindent {\it Detach-Except}, applied to an annotated goal, produces a
        new model where the goal is no longer annotated with a specific {\it
        Except} clause. The operator introduces a new refinement with two
        children: the countermeasure goal and a deidealization of the original
        goal (see \in{Figure}[fig:except_semantic] from left to right). The
        children of the original goal are then children of the deidealized
        goal. Back to an earlier example, the operator takes the model fragment
        in \in{Figure}{a}[fig:refactoring_except] to produce the model fragment
        in \in{Figure}{b}[fig:refactoring_except].

        \noindent {\it Detach-Provided}, applied to an annotated goal, produces
        a new model where a specific {\it Provided} annotation is
        \quote{compiled} into its equivalent formal specification. For example,
        after application of this operator the goal \goal{Achieve [Make Up Pump Motor On When Water Requested]}
         is specified without its {\it Provided} annotation as follows:
       
       \startkaosspec
         \GoalName {Achieve [Make Up Pump Motor On When Water Requested]}
         \KaosAttribute {FormalSpec} {
           $WaterRequested \wedge \neg PumpMechanicalFailure$\blank[none]
           \hskip2.5cm$\Rightarrow \ltlF_{\leq 5m} PumpMotorOn$
         }
       \stopkaosspec
       
        \noindent {\it Detach-RelaxedTo}, applied to an annotated goal,
        produces a new model where a specific {\it RelaxedTo} annotation is
        removed. Back to an earlier example, the application of this operator
        to the goal \goal{Achieve [Make Up Pump Motor On When Water Requested]}
        yields the following goal specification:
       
       \startkaosspec
         \GoalName {Achieve [Make Up Pump Motor On When Water Requested]}
         \KaosAttribute {FormalSpec} {$WaterRequested \Rightarrow [ \ltlF_{\leq 5m} PumpMotorOn $\blank[none]
           \hskip2.5cm$\vee\hskip.5em\ltlF_{\leq 10m} RedundantPumpMotorOn ]$}
       \stopkaosspec
         
        Similarily, three operators are available for transforming a standard
        model portion into an annotated one\emdash{}namely, {\it
        Attach-Except}, {\it Attach-Provided} and {\it Attach-RelaxedTo}. These
        operators are the reverse of the {\it Detach} ones.
            
      \noindent {\bf Exception Diagrams.} Textual goal specifications with {\it
      Except} and {\it Replaces} annotations may be graphically represented in
      an exception diagram. \in{Figure}[fig:exception_diagram] shows a two
      portions of such a diagram for the goals \goal{Achieve [Make Up Water
      Requested When Alarm Raised]} and \goal{Achieve [Make Up Pump Motor On
      When Water Requested]}. This diagram captures that {\it (a)} this goal replaces
      the similar goal assigned to the agent \agent{Operator}; and {\it (b)} when the
      obstacle \obstacle{Pump Mechanical Failure} or \obstacle{Pump Electrical
      Failure} occurs the countermeasure goal \goal{Achieve [Redundant Pump
      Motor On When Primary Pump Failure]} will guarantee this goal.

        \placefigure[page]
        	[fig:exception_diagram]
        	{Exception diagrams.}
          {\startcombination[1*2]
            {\externalfigure[../images/chap5/exception_diagram_a.pdf]}{\tfx(a) Exception diagram with a {\it replace} annotation.}
            {\externalfigure[../images/chap5/exception_diagram_b.pdf]}{\tfx(b) Exception diagram with two {\it except} annotations.}
           \stopcombination
          }
      
      \stopsubsubsection
      
    \stopsubsection
  
  \stopsection
  
	\section{Summary}
  
    This chapter presented techniques for controlling critical and likely
    obstacles resolved by valid countermeasures in order to produce more
    complete requirements. Countermeasures are selected to maximize the
    satisfaction rate of high-level goals while minimizing the resolution cost.
    The integration operator guarantee three properties: {\it progress} towards
    more complete requirements, {\it minimal changes}, and {\it correctness
    preservation}. A valid countermeasure goals guarantee {\it progress}, i.e.,
    once the countermeasure is integrated, the satisfaction rate of a
    high-level goal increase. Such increase quantifies the impact of a
    countermeasure and drives the selection of appropriate countermeasures. The
    chapter proposed two integration schemas, picked according the resolution
    tactic, guaranteeing {\it minimal change} to the ideal model. Propagating
    the changes guarantee the correctness of the new model. However, such
    integrations mixes ideal and exceptional cases. To mitigate this,
    extensions to the goal specification language are proposed to specify
    exceptions and de-idealization to goals together with refactoring
    operators. Exceptions might then be visualized in exception diagrams.
    
    The techniques presented so far always assume a single-value for the
    satisfaction rate. The next chapter introduce uncertainty about the
    satisfaction rates and details how such uncertainty impacts the assessment
    of the most critical obstacles.
      
    % TODO 
      % BDD with all possible resolutions
      % Draft the BDD with partial-entailement refinements + explain in the text
    
\stopcomponent