\environment common

\startproduct thesis

\startstandardmakeup
	
    \framed[align=normal,frame=off]{%
      \cronos\tfd%
      Software Requirements Engineering:\blank[small]
      A Risk-driven Approach
    }
    
    \blank[2*line]

    Requirements Engineering (RE) is the branch of Software Engineering
    concerned with the elicitation, evaluation, specification, analysis and
    evolution of the objectives, functionalities, qualities and constraints of
    a software-intensive system. Our natural inclination to conceive idealized
    systems often results in lack of anticipation of exceptional conditions.
    Risk analysis aims at identifying, assessing and controlling these
    conditions.
    
    \blank[line]
    
    Goals are a natural abstraction for the objectives to be satisfied by the
    software-to-be. Obstacle analysis was introduced as a goal-oriented form of
    risk analysis; an obstacle to a goal is a precondition for the
    non-satisfaction for this goal. Obstacle analysis is structured in three
    steps: {\it identification} of as many obstacles as possible, {\it
    assessment} of the likelihood and criticality of the obstacles and {\it
    resolution} of the critical and likely obstacles.
    
    \blank[line]
    
    The thesis presents a model-based quantitative framework for obstacle
    assessment and control. In the proposed framework, experts estimate
    fine-grained obstacles, with their uncertainty margins. Theses estimates
    are up-propagated through the obstacle and goal model. Comparing the
    estimated satisfaction rate of high-level goals to their required
    satisfaction rate determines the criticality of the obstacles. As a result,
    the most critical and likely obstacles are highlighted. Countermeasures to
    these obstacles are then identified. The countermeasures maximizing the
    satisfaction rate of high-level goals while minimizing the resolution cost
    are selected to be integrated in the ideal model, thereby documenting
    exceptional cases. Using these techniques, the thesis proposes to drive the
    runtime adaptation of software systems by the satisfaction of high-level
    goals, monitoring low-level obstacles to determine the most appropriate
    countermeasures to deploy in order to guarantee the required satisfaction
    rates.
    
\stopstandardmakeup

% Remerciement
% staff ingi
% iba
% EMTs

\startfrontmatter
	\component chap-0
\stopfrontmatter

% remmerciement
% yves de sadeleer
   
\startbodymatter		
    \setcounter[userpage][1]
    \component chap-1
    \component chap-2
    \component chap-3
    \component chap-4
    \component chap-5
    \component chap-6
    \component chap-7
    \component chap-8
    \component chap-9
    \component chap-10
    \component chap-11
\stopbodymatter

\startbackmatter
	\component chap-12
\stopbackmatter

\startappendices
 \startchapter[title={Case-studies}]
   \component app-bas
   \component app-cool
 \stopchapter
 \component bnf
\stopappendices

\stopproduct
