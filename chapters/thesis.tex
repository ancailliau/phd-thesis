\environment common

\startproduct thesis

\startstandardmakeup
	
    \framed[align=normal,frame=off]{%
      \cronos\tfd%
      Software Requirements Engineering:\blank[small]
      A Risk-Driven Approach
    }
    
    \blank[2*line]

    Requirements Engineering (RE) is the branch of Software Engineering
    concerned with the elicitation, evaluation, specification, analysis and
    evolution of the objectives, functionalities, qualities and constraints of
    a software-intensive system. Our natural inclination to conceive idealized
    systems often results in a lack of anticipation of exceptional conditions
    that may prevent the consider system from achieving its intented mission.
    Risk analysis at RE time aims at identifying, assessing and controlling
    such conditions.
    
    \blank[line]
    
    In a software development project, {\it goals} are a natural abstraction
    for the objectives to be satisfied by the system to be developped. A {\it
    goal model} is an AND/OR refinement graph showing how goals contribute to
    each other. {\it Obstacle analysis} has been introduced as a goal-oriented
    form of risk analysis: An {\it obstacle} to a goal is a precondition for
    the non-satisfaction for this goal. An {\it obstacle model} is a set of
    AND-refinement trees showing how obstacles contribute to each other.
    Following standard risk analysis cycles, obstacle analysis is structured in
    three steps: the {\it identification} of as many obstacles as possible, the
    {\it assessment} of the likelihood and criticality of the identified
    obstacles, and the {\it control} of likely and critical obstacles.
    
    \blank[line]
    
    The thesis presents a model-based quantitative framework for assessing and
    controlling obstacles to {\it probabilistic goals}, that is, goals required
    to be satisfied in a specified percentage of cases at least. In the
    proposed framework, domain experts estimate the likelihood of fine-grained
    obstacles, together with their uncertainty margins for these estimates.
    Theses estimates are up-propagated through the obstacle and goal models.
    Comparing the estimated satisfaction rate of high-level goals to their
    required satisfaction rate helps determining the criticality of the
    obstructing obstacles. As a result, the most likely and critical obstacles
    are highlighted. Countermeasures to these obstacles are then identified.
    The countermeasures that maximize the satisfaction rate of high-level goals
    while minimizing their resolution cost are selected for expansion of the
    ideal model. Exceptional cases are thereby documented. These techniques are
    extended from development time to system runtime for runtime adaptation
    towards better satisfaction of high-level goals. Based on the fine-grained
    obstacles, most appropriate countermeasures are dynamically selected in
    order to guarantee the required satisfaction rates.
    
\stopstandardmakeup

% Remerciement
% yves de sadeleer
% staff ingi
% iba
% EMTs
% communauté RE

\startfrontmatter
	\component chap-0
\stopfrontmatter
   
\startbodymatter		
    \setcounter[userpage][1]
    \component chap-1
    \component chap-2
    \component chap-3
    \component chap-4
    \component chap-5
    \component chap-6
    \component chap-7
    \component chap-8
    \component chap-9
    \component chap-10
    \component chap-11
\stopbodymatter

\startbackmatter
	\component chap-12
\stopbackmatter

\startappendices
 \startchapter[title={Case-studies}]
   \component app-bas
   \component app-cool
 \stopchapter
 \component bnf
\stopappendices

\stopproduct
