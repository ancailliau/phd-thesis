% !TEX root = thesis.tex

\startcomponent chap-10
\environment common
\product thesis

\chapter[chap:relatedwork]{Related Work}
    
  This chapter reviews the relevant work on risk analysis models and
  techniques, quantitative requirements, uncertainty in requirements, exception
  handling in requirements and requirements-driven runtime adaptation. The
  related work presented here is restricted to software systems.
  
  The chapter is structured as follows. \in{Section}[sec:comparing_ra] reviews
  the work related to risk analysis models and techniques not explicitly
  connected to requirements, such as Fault Tree Analysis (FTA), HAZOP/HAZAN,
  FMEA/FMECA, and CORAS. \in{Section}[sec:comparing_qre] discusses the quantitative
  requirements approaches. \in{Section}[sec:comparing_rdre] presents and
  discusses the models and techniques for risk analysis related to
  requirements. \in{Section}[sec:comparing_um] reviews uncertainty
  management for requirement models and techniques.
  \in{Section}[sec:comparing_cr] discusses models and techniques for exception
  handling. Last, \in{Section}[sec:comparing_as] details other runtime
  adaptation models and techniques and compares these to our approach.
  
  Our review do not include security risk analysis techniques such as {\it
  attack trees} and {\it threat trees} \cite[Sch99a,Hel02a,Sch11a], {\it
  UMLSec} \cite[Jur01a,Jur02a], {\it SecureUML} \cite[Lod02a], {\it Abuse Case}
  \cite[McD99a], or {\it Misuse Case} \cite[Sin05a]. These frameworks deals
  specifically with security concerns. Future research is required to better
  understand how these relates to the presented approaches.
  
  \startsection[reference=sec:comparing_ra,title={Risk Analysis}]
    
    This section briefly presents three classical risk analysis models and
    techniques commonly found in the industry plus the CORAS approach. A more
    detailed overview of conventional risk analysis approaches can be read in
    \cite[Lev01a].
    
    \noindent {\bf Fault Tree Analysis.} Fault Tree Analysis (FTA) is a very
    widely used method to analyze risks related to safety
    \cite[Ves81a,Bed01a,Lev01a,Lam09a,Rui15a]. Application domains range from
    aeronautics to software systems \cite[Lev83a]. \cite[Ves81a] provides a
    comprehensive introduction to FTA. We briefly discuss here classical fault
    trees and dynamic fault trees and refer the reader to \cite[Rui15a] for a
    more detailed discussion about state-of-the-art techniques for FTA.
      
    Fault trees (FTs) models how failures propagate through a system, i.e., how
    a basic event leads to a system failure. An FT is an acyclic graph with two
    types of nodes: events and gates. The events of an FT model the failures
    whereas the gates (the non-leaf nodes) model failure propagation. A leaf
    event is called a {\it basic event}, a non-leaf event is known as an {\it
    intermediate event}. Gates includes {\it AND} and {\it OR} gates modeling,
    respectively, the AND-combinations (output is $true$ iff all inputs are
    $true$) and OR-combinations (output is $true$ if at least one input is
    $true$) of the child events (called inputs). An other gate is the {\it k/N}
    gate that models a {\it voting} gate where at least $k$ child events shall
    be $true$, among $N$, are required for the output to be $true$.
    
    Analysis of fault tree is either {\it qualitative} or {\it quantitative}.
    Qualitative analysis focuses on the structure of the fault tree; this type
    of analysis includes identifying {\it cut-sets} (i.e., a combination of
    basic events leading to the failure of the system). Different techniques
    are available to efficiently compute these cut sets, such as a bottom-up
    propagation, or a BDD-based computation \cite[Rui15a]. Quantitative
    analysis focuses on computing failures probabilities such as {\it system
    reliability} (i.e., the probability that a system fails given the horizon
    $t$), or {\it mean time between failures}, (i.e., the average duration
    between two subsequent failures). Low-level events are annotated with
    probabilities that are propagated from causing events to their consequences
    in the fault tree. The probabilities can be single-value or multi-value
    estimates such as probability distribution. A large number of techniques
    are available to compute the probability of the root events \cite[Rui15a].
    
    Standard Fault Trees (SFTs) provides a simple model for risk analysis but
    lacks expressivity. Dynamic Fault Trees (DFTs) extends SFTs to support
    finer-grained analysis. DFTs include additional gates such as {\it
    Priority-AND} (inputs shall be $true$ in a specific order), {\it FDEP}
    (Function DEPendency, inputs influence other inputs) and {\it SPARE} (model
    components replaced by others). The quantitative analysis of a DFT is
    usually performed by transforming the DFT into a Markov Chain that can be
    further analyzed using standard techniques such as Model Checking
    \cite[Bai08a].
  
    \noindent {\bf HAZOP/HAZAN.} The approach HAZOP focuses on hazard and
    operability problem identification \cite[Kle99a,Lev01a]. It deals with
    deviation from the intended design (called {\it hazards} \cite[IEC61882]),
    their possible causes and their consequences \cite[IEC61882]. Originally
    designed for chemical plants, the HAZOP approach has been extended and used
    in different settings such as administrative procedures, medical devices
    \cite[IEC61882], and software systems
    \cite[Ear92a,Chu93a,Bur93a,Fen94a,McD94a,McD95a,McD02a,Par07a].
    
    An accurate and detailed design is required to carry out an HAZOP analysis.
    At a later stage, it is, however, more costly and sometimes impossible to
    introduce major changes in the design; \cite[Kle99a] reports the
    application of HAZOP analysis on preliminary designs.
    
    The system under scrutiny is divided into parts. (The size of the parts
    depends on the complexity.) Each part and characteristic are examined and
    challenged by the use of guide words such as {\ss NO OR NOT}, {\ss MORE},
    and {\ss LATE} to feed the elicitation process. Specific guide words were
    developed for software systems \cite[Bur93a,McD94a,McD95a,McD02a,Han04a].
    In the context of software systems, UML attributes and entities might be
    used as a basis for the application of the guide-words \cite[Han04a].
  
    In the chemical process context, \cite[Vai96a] extends an expert system,
    aimed at generating HAZOP analysis, with quantitative information about the
    processes and the materials used. The consequences of a hazard are then
    ranked according to their severities.
  
    Once identified, the hazards might be assessed. The assessment might be on
    standard consensus techniques or rely on Fault Tree Analysis to compute
    more accurate estimates \cite[Kle99a]. Once assessed, the consequences for
    the employees, the public and the environment, and the plant and profit are
    estimated. The hazard assessment and the consequence assessment are then
    compared to target criteria \cite[Kle99a].
  
    \noindent {\bf FMEA/FMECA.} Failure Modes and Effects Analysis (FMEA) is a
    systematic technique for identifying failure modes of a system (or a
    subsystem) and for evaluating the consequences of the failure modes on the
    higher levels \cite[Haa02a,Lev01a]. Failure Modes, Effects and Criticality
    Analysis (FMECA) extends FMEA with quantitative analysis. The identified
    failure effects are organized according to their criticality (on an $x$-axis)
    and likelihood (on a $y$-axis) in a {\it criticality matrix}. Such
    criticality matrix helps the risk engineers to identify critical and likely
    failures. We refer the reader to \cite[Haa02a] for a detailed survey
    related to the application of FMEA to software systems.
    
    \noindent {\bf CORAS}. CORAS is a model-driven risk analysis approach with
    a strong focus on defensive security analysis \cite[Lun10a]. The framework
    is centered around protecting a set of identified assets. The approach is
    structured in eight steps. Steps 1 to 4 focuses on the preparation of the
    risk analysis. These are important steps but are considered as out of scope
    for this review, Step 5 to 8 however directly relates to the
    identify-assess-control cycles: (Step 5) Risk identification using threats
    diagrams; (Step 6) Risk estimation using threats diagrams; (Step 7) Risk
    evaluation using risk diagrams; (Step 8) Risk treatments using treatments
    diagrams. More details about these steps can be found in \cite[Sol11a]. The
    approach is supported by a dedicated tool support \cite[See12a]. The tool
    enables the analyst to build the various diagrams used in the approach and
    to run most of the analysis proposed in \cite[Sol11a]. A wide variety of
    case-studies \cite[Bra07c,Bra07a,Ref15a,Sol13a] from various industries and
    of different size illustrates and validates the practical applicability of
    the approach.
        
    Risk identification (Step 5) is driven by the protection of the assets.
    This step produces a threat diagram identifying the risks harming the
    identified assets. An {\it asset} is defined as {\it something to which a
    stakeholder assigns value and hence for which the stakeholder requires
    protection}. This general concept might be physical objects (such as
    computers), virtual elements (such as customer database, or online store),
    or desirable properties (such as compliance, or reputation). The risks are
    further refined in categories \cite[Sol11a]. In the framework, an {\it
    unwanted incident} is an event that harms or reduces the value of an asset;
    a {\it threat} is a potential cause of an unwanted incident; a {\it threat
    scenario} is a chain or series of events that is initiated by a threat, and
    that may lead to an unwanted incident; a {\it vulnerability} is a weakness,
    flaw or deficiency that opens, for, or may be exploited by, a threat to
    cause harm to or reduce the value of an assets. Threat scenarios are
    connected to other threats scenarios and threat by {\it leads-to} links. A
    vulnerability decorates either a link between a party and a threat
    scenario, either a {\it leads-to} relation. These elements and their
    relations are identified during (Step 5) by a structured workshop where
    experts and risk analysts identify as many elements as possible.
        
    After the identification and documentation of the risks, (Step 6) focuses
    on risk estimation. During a structured workshop, the unwanted incidents
    are annotated with a likelihood. A {\it likelihood} is defined as the
    frequency or the probability of something to occur. The estimates might be
    qualitative or quantitative. To help to support the estimation, a calculus
    is provided to compute the likelihood of a dependent artifact. The calculus
    might be used to check whether the estimate of the unwanted incident is
    coherent with the other estimates of threats and threats scenarios. Such
    technique might be useful to identify incomplete, inconsistent or ambiguous
    risk analysis \cite[Ref15a]. \cite[Ref09a] suggests specifying likelihoods
    using equations combining key indicators whose value might change over
    time. However, little support is provided to identify such key indicators,
    to combine these in likelihood and how these key indicators should be
    monitored.
        
    The uncertainty of the estimates might be captured using intervals instead
    of single-point values \cite[Sol11a]. The calculus is extended to support
    these intervals by applying the operation (such as addition, multiplication
    or subtraction) on the bounds of the interval. Distributions might replace
    intervals for a more refined analysis but, to the best of our knowledge,
    this has not been done.
        
    A {\it consequence} is the impact of an unwanted incident on an asset in
    terms of harm or reduced asset value. The consequences are estimated on a
    qualitative or quantitative scale. Most commonly, the consequences are
    classified as insignificant, minor, moderate, major or catastrophic.
    However, for specific assets, a quantitative value, such as monetary or
    time, might be used.
        
    Estimated risks are then evaluated and prioritized (Step 7). The unwanted
    incidents are displayed on a risk matrix, whose rows are the frequencies
    and columns are the consequences, to identify critical and likely risks.
    The risk matrix indicates risks that are acceptable and risks that are not
    acceptable. Which cell is acceptable is determined by the analyst and the
    customer based on a risk evaluation criteria.
        
    (Step 8) is concerned with the identification, assessment, and selection of
    countermeasures to unacceptable risks. In CORAS, countermeasures are called
    {\it treatments}. A {\it treatment} is an appropriate measure to reduce
    risk. Using a structured workshop, the analyst and experts identify
    possible countermeasures. Using countermeasures categories, such as avoid,
    reduce consequences, reduce likelihood, transfer, or retain helps to
    generate alternative countermeasures. Once countermeasures are identified,
    their cost and reduction effect in both likelihood and consequence are
    estimated. The benefit for a countermeasure regarding an asset is computed
    as a combination of the risk probability, the risk reduction, and the
    consequences. Best countermeasures are then selected during the workshop.
    \cite[Sol13b] proposes a technique for systematic countermeasures selection
    when estimates are quantitative. The procedure is structured in three
    steps: {\it (a)} the countermeasures are annotated with their cost and
    reduction effect; {\it (b)} for each unwanted incident, a decision diagram
    display the countermeasures combined cost and the reduced probability of
    the incident; {\it (c)} the countermeasures minimizing their combined cost
    the probability of the incident are selected.
    
    \noindent {\bf Comparing with our approach.} We can observe some
    differences between these approaches and ours:
    
    \startitemize
      
      \item The lack of connection with a goal model may make it hard to
      identify the root events for starting backward causal analysis. Fault
      Tree analysis is often anchored on Event Trees \cite[Lev01a] or used as a
      complimentary technique to provide more precise estimates for the risks
      identified using HAZOP/HAZAN or FMEA/FMECA. CORAS anchors the risk
      analysis on assets; however, this appears, to us, to be vague and
      unprecise to start a risk analysis. For example, what does it mean to
      protect a user database?
    
      \item Unlike the presented approaches, our approach is anchored in a
      refinement structure on probabilistic goal/obstacle assertions. In Fault
      Tree Analysis and CORAS, the risk trees are not grounded on a formal
      framework; their nodes are just event names and their causal links have
      no precise semantics. As a consequence, the correctness of event
      decompositions cannot be established, and the propagation rules may
      appear ad hoc. In HAZOP/HAZAN and FMEA/FMECA, there is no structure at
      all among risks.
      
      \item In these approaches, the likelihoods and their contributions have
      no precise semantics that would enable formal reasoning or monitoring of
      these likelihoods.
      
      \item Our approach is integrated into an overall method for risk
      identification, assessment, and resolution. For example, FTA proposes no
      mechanism for risk resolution; no approach presented here integrates the
      selected countermeasures back into the model. Lastly, none of these
      approaches support risk analysis cycle where risks to countermeasures are
      identified, assessed and controlled.
    
    \stopitemize
      
  \stopsection
  
  \startsection[reference=sec:comparing_qre,title={Quantitative Requirements Engineering}]
  
    The related work in this section focuses on models and techniques for
    quantitative requirements that are not explicitly connecting requirements
    and risks. This section discusses exclusively quantitative analysis. We
    refer the user to \cite[Hor11a] for available qualitative techniques to
    analyze requirements models.
      
    \noindent {\bf KAOS-based Approaches.} \cite[Let04a] proposes an approach
    to probabilistically reason on goal models for selecting the most
    appropriate alternatives. In the proposal, goals are annotated with quality
    variables, such as {\it Failure Rate} or {\it Response Time}. On leaf
    goals, these quality variables are estimated using probability
    distributions that are analytically combined for obtaining probability
    distributions for the quality variables of top goals. The combinations are
    obtained by applying domain-specific propagation equations that annotate
    the refinement structure. High-level goals are decorated with objective
    functions to maximize or minimize, together with a target value.
  
    \cite[Lut12a,Lut12b,Ell14a] discusses how quantitative goal-oriented
    requirements engineering helped in designing a molecular system.
    \cite[Lut12b] introduces {\it THRESHOLD} refinement nodes. In such
    refinement, the parent goal is satisfied if \quote{enough} subgoals are
    satisfied. \quote{Enough} subgoals are satisfied if at least $k$ subgoals
    are satisfied over all $N$ possible subgoals.
    
    \cite[Dar15a,Pon17a] evaluates how quantitative reasoning can be used to
    assess accessibility in physical buildings. The approach differs from
    common approaches by generating a domain specific assessment tool for
    assessing the satisfaction of goals.
      
    \noindent {\bf Tropos.} \cite[Gio02a,Gio03a] introduces a formal framework
    for reasoning with goal models. This formal framework was then applied in
    \cite[Gio05a] to the framework Tropos \cite[Bre04a]. In addition to
    refinement links, the goal model includes qualitative relationships between
    goals. These relations include $++,+,-,--$:

    \vskip-.09cm
    \startitemize[packed]
         \item $G_1 \xrightarrow{}{++} G_2$: the satisfaction of $G_1$ implies the satisfaction of $G_2$.
         \item $G_1 \xrightarrow{}{+} G_2 $: there is evidence that the satisfaction of $G_1$ implies the satisfaction of $G_2$.
         \item $G_1 \xrightarrow{}{-} G_2 $: there is evidence that the satisfaction of $G_1$ implies the denial of $G_2$.
         \item $G_1 \xrightarrow{}{--} G_2$: the satisfaction of $G_1$ implies the denial of $G_2$.
    \stopitemize
    
    The relations $+,-$ might be annotated with a weight, enabling quantitative
    analysis, representing the strength by which the satisfiability/deniability
    of the source goal influences the satisfiability/deniability of the target
    goal.
    
    In the framework, goals are annotated with two attributes, {\it SAT} and
    {\it DEN}, that quantify the value of evidence for the goal being satisfied
    or denied, respectively. The values of these attributes are qualitative and
    ranges from {\it Full} to {\it Partial} to {\it None}, or from real
    variable $inf$ to real variable $sup$ when using quantitative analysis
    (these needs to be specified by the user).
    
    A propagation algorithm uses propagation rules to compute the values {\it
    SAT} and {\it DEN} of the root nodes. The algorithm starts from leaf goals
    and propagates the values to the goals directly connected to these leaf
    goals. The process is then repeated until a fixpoint is reached.
    
    \noindent {\bf i* models for selecting architecture components.}
    \cite[Fra03a,Fra04a] reports the application of i* goal models \cite[Yu97a]
    to select architecture components. Architecture components are modeled as
    actors, and the selection of the most appropriate components is driven by
    the satisfaction of goals and soft goals. Soft goals are quantified using
    architecture-specific metrics such as diversity, packaging, uniformity,
    connectivity. These architecture-specific metrics are then computed for the
    alternative architectures to drive the selection of the most appropriate
    one. It is not clear how these metrics are estimated.
      
    \noindent {\bf Quantitative GRL.} \cite[Amy10a] introduces a quantitative
    reasoning layer to Goal-oriented Requirement Language (GRL). The reasoning
    is aimed at comparing the relative effectiveness of alternative designs.
    Satisfaction values are assigned to intentional elements (nodes of the goal
    model graph) and propagated to the other intentional elements through the
    decomposition, contribution and dependency links. The propagation might be
    forward (from leaf nodes to root nodes) or backward (from root nodes to
    leaf nodes) or a mix of both. Forward analysis answers \quote{what if?}
    questions whereas backward analysis focus on \quote{is it possible?}
    questions. A qualitative assessment and a hybrid assessment (mixing both
    qualitative and quantitative information) are also proposed.
      
    \noindent {\bf Approach by G. Chatzikonstantinou and K. Kontogiannis.} In
    \cite[Cha13a], the degree of satisfaction for a goal is determined by the
    degree of satisfaction of its subgoals, and the degree of satisfaction of
    other goals that are connected through {\it Contribution Links}.
    Contribution links indicate whether the source goal contributes to the
    satisfaction or violation of the target goal. These links are weighted
    according to the probability a source goal satisfy or deny the target goal.
    The approach \cite[Cha13a] generates a Markov Logic Network (MLN) \emdash
    an MLN is similar to a Bayesian network \emdash from a goal model. The MLN
    is then trained with past data to obtain the numerical values for the
    contributions links.
    
    \noindent {\bf Comparing our approach.} As these approaches are not
    connected to risks, the focus of these is more on selecting alternative
    design, such as selecting alternative goal refinements or selecting
    alternative agents to satisfy high-level objectives.
    
    Unlike all approaches presented in this section, our approach links the
    partial satisfaction of the goals to the satisfaction of the risks. (Next
    section will discuss models and methodologies incorporating both a
    requirement and a risk model.) The satisfaction of the requirement is
    computed from the risks preventing their satisfaction. With a risk
    AND/OR-refinement structure for propagating probabilities from fine-grained
    risks, the estimates are easier to obtain and interpret in terms of
    application-specific phenomena. Also, it documents the rationale for the
    partial satisfaction of the goals. A notable exception is \cite[Let04a]
    where quality variables are easy to obtain and interpret.
      
    Our approach differs in several directions from the presented approaches,
    specifically:
    
    \startitemize
           
      \item Unlike \cite[Gio02a, Gio03a, Gio05a, Fra03a, Fra04a, Amy10a,
      Cha13a], our approach relies on a precise semantic in terms of behavior.
      This enables the estimates to be grounded on real-world phenomena.
      
      \item The approaches proposed in \cite[Gio02a, Gio03a, Gio05a, Amy10a,
      Cha13a] propagate the partial satisfaction through refinements and
      contribution links. Our approach does not include such contribution
      links. As the goals appear to have no precise definitions in terms of
      behaviors, it is unclear how contributions links are defined and differ
      from partial refinements, in a quantitative setting.
      
      \item Unlike \cite[Gio02a, Gio03a, Gio05a, Fra03a, Fra04a, Amy10a,
      Cha13a], our approach compares the {\it estimated} satisfaction rate to
      {\it required} satisfaction rate. In \cite[Let04a], objective functions
      are decorated with a target value, similar to our RSR.
      
      \item Our framework takes a probabilistic approach; goals and obstacles
      are estimated with a single value indicating both their satisfaction
      ($sr$) and their non-satisfaction ($1-sr$). \cite[Gio02a, Gio03a, Gio05a,
      Amy10a, Cha13a] takes a different approach where goals are estimated with
      2 values indicating the satisfaction ($SAT$) and their non-satisfaction
      ($DEN$). This increases the number of elements to be elicited by experts.
      Also, such theory might not be as well understood as the probability
      theory. In particular, experts might overlook that $SAT \neq 1-DEN$.
      
      \item The approaches proposed in \cite[Let04a,Fra03a,Fra04a] rely on
      domain-specific variables and equations. These approaches appear to be
      finer-grained as it relies on domain-specific knowledge. However, these
      variables and the equations must be defined in an ad-hoc manner.
      
      \item The domain specific assessment tools presented in
      \cite[Dar15a,Pon17a] could easily integrate with this approach.
    
    \stopitemize
      
    \cite[Lut12a,Lut12b,Ell14a] introduces a new type of refinement node: the
    {\it THRESHOLD} node. This is somewhat similar to a the $k/N$ gate in Fault
    Tree Analysis. {\it THRESHOLD}-refinements are not probabilistic
    refinements, the parent goal is satisfied if \quote{enough} subgoals {\it
    are} satisfied, not if \quote{enough} subgoals {\it might be} satisfied;
    this differs from our approach.
      
    The backward reasoning applied in \cite[Amy10a] for identifying the leaf
    goals that are sufficient for satisfying the parent goal is not relevant in
    our framework as we do not consider alternatives. However, the sets of leaf
    obstacles that are adequate for preventing the satisfaction of the parent
    goal can be computed in our approach with the obstruction superset (see
    \in{Section}[sec:computing_bdd]).
    
  \stopsection
  
  \startsection[reference=sec:comparing_rdre,title={Risk-Driven Requirements Engineering}]
  
    This section reviews the approaches where requirements and risks are
    integrated in a unified qualitative or quantitative framework.
      
    \noindent {\bf DDP.} The process "Defect Detection and Prevention" (DDP)
    originated from NASA, Jet Propulsion Lab \cite[Cor01a, Fea03a]. It is
    designed as an aid to the decision making process in the early phases of a
    project. In theses phases, information on which decisions are made are
    often incomplete and uncertain. The process is structured in 4 steps: {\it
    (1)} Overview the problem and DDP methodology; {\it (2)} Develop the impact
    information; {\it (3)} Develop the effect information; {\it (4)} Decision
    making. The main concepts of the approach are {\it requirements} (what the
    system shall satisfy), {\it failure modes} (what could prevent the
    requirements to be satisfied) and {\it PACTs} (what can be done to reduce
    the likelihood or impact of a failure mode). PACT stands for "Preventions,
    Analyzes, process Controls, and Tests." A detailed presentation of the
    concepts and the process can be found in \cite[Fea03a]. The approach is
    supported by a tool, with a strong focus on decision making and
    visualization to help these decisions \cite[Fea03a]. The process has been
    used internally at NASA for a long period in various settings, for examples
    see \cite[Sha06a,Fea08b,Fea08c].
  
    A {\it requirement} is defined as \quote{\it whatever the system under
    scrutiny is to achieve, and the constraints under which it must operate}.
    Requirements are specified in the natural language. Requirements are
    organized in trees that can be seen as folders grouping requirements
    according to the modeler's conventions. The trees do not convey a
    structural meaning. Leaf requirements are decorated with a weight,
    indicating their relative importance about the other requirements. The
    weight of the leaf goals might be inferred from the weight of the
    high-level goals \cite[Mes04a].
  
    A {\it failure mode} is defined as \quote{\it a thing, that, should it
    occurs, will lead to loss of requirement}. Failure modes are informally
    specified using natural language. They are organized in OR-tree, indicating
    an alternative way to cause a parent failure mode to happen. Failures mode
    are identified during structured workshops. During these workshops,
    previous DDP analysis helps the experts to determine the risks from similar
    systems or setting \cite[Fea03a].
  
    The identified failure modes impact requirements. An impact matrix
    quantitatively specifies the proportion of the requirement that would be
    lost if the failure mode occurs (between $0$ and $1$). If multiple failure
    modes impact a requirement, the impacts are summed, leading to values that
    might be above $1$. In that case, the sums are capped to $1$ when
    evaluating how the requirements are attained. Summing without capping
    better emphasis the amount of risk to overcome to satisfy the requirement.
    For example, an impact of $1.8$ might requires more effort to be reduced
    than an impact of $1.1$.
          
    Failure modes are decorated with a priori likelihood and a cost of repair.
    Likelihoods and costs are specified by experts, during workshops and
    agreeing on a single value. For each failure mode, we can compute its
    global impact by summing the impact of the weighted requirements. This
    helps to identify the most critical failure modes. By taking the PACTs into
    account, it helps to identify the failure modes that are not enough
    mitigated.
  
    A {\it PACT} is \quote{\it anything that could be done to reduce, the
    likelihood of a failure mode and/or reduce their impact on requirements}.
    These include preventive measures, detections, and alleviations: {\it
    Preventive measures} includes, for example, training, standards, selection
    of high-quality parts; {\it Detections} discover instances of the failure
    mode through analysis or test prior to their use or release; and {\it
    alleviations} reduce the severity of the failure mode such as defensive
    programming. A PACT is decorated with a cost that might be a measure of
    budget, schedule, physical resources (such as weight or power), scarce
    resources (such as time from skilled experts) or a combination of these.
  
    PACTs influence the failure modes likelihood. The effect matrix
    quantitatively specifies the effect of a PACT on a failure mode. The effect
    matrix contains a row for each PACTs and a column for each failure mode;
    Each combination of PACT and failure mode needs to be assessed. The effects
    are multiplicative, and the order in which they are applied is not relevant.
  
    The selection of the most appropriate PACTs is supported by a wide
    selection of visualizations \cite[Fea06a]. These appear to facilitate the
    decision making process \cite[Cor02a]. Efforts have also been provided to
    support the semi-automated selection of best PACTs given a budget. A large
    number of combinations requires techniques such as genetic algorithms
    \cite[Cor02a]. A dedicated technique helps to automatically identify the
    key PACTs, even with important uncertainty \cite[Men03a]. The techniques
    rely on a learning framework to iteratively select the PACTs that mostly
    reduces the uncertainty and increases the requirements satisfaction.
      
    \noindent {\bf Goal-Risk Framework.} \cite[Asn06a,Asn11a] extends the
    Tropos framework with risks and countermeasures. In the framework, a {\it
    risk} is an event having an adverse impact on goals whereas a {\it
    countermeasure} is a treatment that can be adopted to mitigate the effect of the
    risk. The framework is structured in three layers: a {\it asset
    layer} (called {\it goal layer} in
    \cite[Asn06a,Asn06b,Asn06c,Asn07a,Asn07b,Asn07c]), an {\it event layer} and
    a {\it treatment layer} with {\it impact} links between events and goals,
    and {\it effect} links between treatments and events. An event is annotated
    with a (risk) likelihood, and the (risk) severity is modeled as a
    contribution (positive or negative) from an event to goal. Events and
    treatments can be AND-/OR-refined and can contribute to other goals,
    events, or treatments. \cite[Asn06b] proposes strategies for identifying
    possible treatments. \cite[Asn06a] further extends the framework proposed
    in \cite[Asn06a] to support treatments mitigating the impact of an event.
    \cite[Asn07a] extends the framework to account for trust between the actors
    of the system. \cite[Asn07c] adds support for quantitative reasoning.
      
    Events and treatments are annotated with {\it SAT} and {\it DEN} attributes
    that quantify the evidence that the event, respectively the treatment, is
    satisfied or denied. A treatment is {\it effective} for an event if it
    introduces a conflict between the {\it SAT} and the {\it DEN} values of the
    event (e.g., the event is fully satisfied, and fully denied).

    The analysis proposed in \cite[Asn06a] aims at identifying the solutions
    that satisfy the high-level goals desired values while minimizing the total
    cost. A solution is a set of leaf goals and treatments. As an input, it
    takes the desired satisfaction values (i.e., minimal SAT-value that is
    accepted) and the acceptable risk values (i.e. maximal DEN-value that is
    accepted) for the top goals. Using backward analysis \cite[Hor10a,Seb04a],
    it identifies the leaf goals that are sufficient to satisfy the root goals.
    It then assess, using forward analysis \cite[Gio05a] (i.e. propagates the
    estimated SAT/DEN-values of the events to the root goals), whether
    treatments shall be introduced. The algorithm identifies the admissible
    values for the events and then the possible countermeasures values that
    guarantee these event values. \cite[Asn07c] adapts this qualitative
    analysis to quantitative analysis.
    
    \cite[Sha12a,Sha13a] propose an extension to the Goal-Risk framework for
    prioritizing risks with a Risk/Cost analysis. It is, however, unclear how
    prioritization is achieved and how costs are obtained.
    
    \noindent {\bf KAOS-based Approaches.} In \cite[Sab11a], KAOS goal models
    are extended with probabilities and propagation rules for technology
    qualification. Experts assess the probability of satisfaction of leaf
    obstacles. These probabilities are then propagated bottom-up up to the root
    goals using propagation rules.
        
    \noindent {\bf RiskML.} \cite[Sie14a] introduces a risk-oriented framework
    for selecting Open Source Software (OSS) components. The approach
    introduces a language, called {\it RiskML}, to relate risks to goals. A
    {\it risk} is defined as \quote{{\it a composed concept modeling a lack of
    knowledge about some happening and what could be the negative
    consequences}} \cite[Sie14a]. Risks are not modeled as a first-class entity
    but as a triplet of situations (in which the events occurs), events (that
    prevent the satisfaction of goals) and goals (that are not satisfied). A
    {\it situation} is a state of the world that exposes or prevents events,
    and increases or decreases their probability of occurrence. {\it
    Indicators} quantifies the occurrence of situations; these need to be
    estimated by experts. A propagation algorithm, inspired by \cite[Gio03a],
    computes scores representing the exposure of the goals to the identified
    risks, using the estimated indicators. These scores are a relative measure
    for comparing alternatives but are not meant to be interpreted as a
    probability or degree of belief. \cite[Cos15a] extends the approach to
    connect RiskML and i* models, enabling the degree of belief of higher-level
    goals to be computed from the estimated indicators.
    
    \noindent {\bf Comparing with our approach.} These approaches, like our,
    integrates both requirements and their related risks. However, some
    differences can be observed:
    
    \startitemize
      
      \item Unlike \cite[Fea03a,Asn11a,Sab11a,Sha12a,Sha13a,Sie14a], our
      approach relies on a precise semantic in terms of behavior. This enables
      the estimates to be grounded on real-world phenomena. In general, these
      numbers are often very subjective, with no or little physical
      interpretation in the application domain, e.g., in \cite[Kai02a]. In
      \cite[Sab11a], the lack of precise semantics in terms of behavior limits
      precise reasoning and questions the correctness of propagations. For
      example, the AND-propagation rule does not apply to case-driven or
      non-minimal refinements; the OR-propagation rule for obstacles can be
      made simpler thanks to obstacle disjointness.
      
      \item Our framework takes a probabilistic approach;
      \cite[Asn11a,Sha12a,Sha13a] adopts a different approach where goals are
      estimated with two values indicating the satisfaction ($SAT$) and their
      non-satisfaction ($DEN$).
    
      \item Unlike \cite[Fea03a,Asn11a,Sab11a,Sha12a,Sha13a,Sie14a], our
      probabilistic goals are supported in terms of {\it estimated} vs. {\it
      required} satisfaction rates. In \cite[Sab11a], the notion of required
      satisfaction rate does not appear relevant to the objectives of this
      approach; obstacle prioritization is therefore different.
      
      \item Unlike \cite[Fea03a], the refinement structure of goals and
      obstacles convey a structural meaning enabling the propagation of
      probabilities through risks and consequences at various levels of
      granularity.
    
    \stopitemize
    
  \stopsection
    
  \startsection[reference=sec:comparing_um,title={Requirements Uncertainty Managment}]
    
    Work on handling uncertainty in RE has been mainly devoted to physical uncertainty
    rather than knowledge uncertainty (as defined in
    \in{Chapter}[chap:knowledge-uncertainty]). For a detailed discussion and
    state-of-the-art related to the uncertainty in the specific context of
    self-adaptive systems, we refer the reader to \cite[Esf13a].
    
    \noindent {\bf Fault Tree Analysis.} Knowledge uncertainty is supported in
    Fault Tree Analysis through probabilistic distributions on leaf events and
    Monte-Carlo simulation for propagation to root causes \cite[Rui15a].
    
    \noindent {\bf DDP.} In \cite[Men03a], a dedicated technique helps to
    automatically identify the most important countermeasures in the presence
    of uncertainty. The approach iteratively selects the countermeasures to
    reduce the uncertainty and increases the requirements satisfaction.
    
    \noindent {\bf KAOS-based approaches.} Related work on handling physical
    uncertainty through obstacle analysis was discussed in the previous
    section. The approach \cite[Let04a] is extended in \cite[Hea11a] for
    selecting best system design alternatives in presence of uncertainty. The
    quality variables are random variables whose value is modeled as a
    probability distribution. The top probability distribution is obtained
    through repeated single-value propagation. The approach is supported by a
    tool \cite[Bus17a]. Based on a framework for statistical decision making,
    the technique in \cite[Let14a] evaluates the gain of perfect information
    and helps identify alternatives maximizing benefits while reducing project
    risks.
    
    \cite[Sab11a] extends the KAOS goal/obstacle models with a probabilistic
    layer for technology qualification (see previous
    \in{Section}[sec:comparing_ra]). The probability of satisfaction for leaf
    obstacles is estimated by a triangular distribution. The distribution for
    the root goal is then computed by Monte-Carlo simulation and a single-value
    propagation algorithm.
      
    \noindent {\bf Fuzzy-based approaches.} Fuzzy goals may be introduced for
    coping with satisfaction uncertainty. The available approaches are more
    oriented towards self-adaption at system runtime. The concern there is to
    reason in terms of proximity of goals being fully satisfied\emdash{}rather
    than in terms of probabilities of satisfaction.
        
    {\it RELAX} is a structured natural language including operators to
    explicit capture environmental uncertainty \cite[Whi09a,Whi10a]. Operators,
    such as {\ss\it MAY}, {\ss\it AS EARLY AS POSSIBLE}, or {\ss\it AS MANY AS
    POSSIBLE}, constraints how the goal can be relaxed at runtime. The semantic
    of RELAX is defined in terms of fuzzy logic. \cite[Che09c] introduces a
    variation of threat modeling, based on obstacles, to identify the sources
    of uncertainty in the satisfaction of the goals. The countermeasures either
    changes the goal model either RELAXes the goal specifications. In
    \cite[Ram12a,Fre14a], RELAXed goals are automatically learned using an
    executable specification of the system. The approach can be summarised as
    follows: {\it (a)} A search process generates semi-randomly RELAXed goal
    models; {\it (b)} the RELAXed goals are scored by running the executable
    specification and monitoring the satisfaction of the goals; {\it (c)} if
    enough iterations have been performed, the process stops and outputs a
    RELAXed model, otherwise, new RELAXed models are generated and {\it (b)} is
    executed again. In \cite[Ram12b], claims, introduced to document and
    support decisions \cite[Wel09a,Wel10a,Wel11a], are RELAXed to improve the
    number of required adaptations.
    
    \cite[Bar10a] introduces {\it fuzzy goals} to provide more flexibility
    necessary in systems where goals might not be measured or whose
    specification is not entirely known. Crisp goals are goals that must be
    fully satisfied, in oppositions to fuzzy goals where fuzzy membership
    function quantifies the degree of goal satisfaction. \cite[Cha16a,Cha16b]
    presents a fuzzy logic-based approach combining both crisp and fuzzy goals.
    A bottom-up propagation algorithm computes the degree of satisfaction for
    both types of goals. The degree of satisfaction of a fuzzy goal depends on
    the degree of satisfaction of the children in the refinements, and on the
    degree of satisfaction of the goals that positively and negatively
    contribute to that goal. The algorithm might be parallelized to enable fast
    computation on large graphs \cite[Cha16b].
        % How values for leaf goals is obtained is not actually discussed...
      
    \noindent {\bf Dempster-Shafer theory-based approach.} The theory of belief
    function (or Dempster-Shafer) is an other approach to reason about
    uncertainty. In this approach, beliefs about possibilities (element of the
    domain of interest) are bounded by two values: belief and plausibility.
    \cite[Liu05a] proposes to evaluate risks using Dempster-Shafer theory. This
    theory allows experts to independently assess the risk likelihoods and how
    risks combine. Experts can also assess their uncertainty about the
    estimated values. The experts' opinions are then combined to provide a
    unified assessment of the risks with their uncertainty margins.
    
    \noindent {\bf Model uncertainty.} Other approaches focus on
    \quote{design-time} uncertainties which enable reasoning about the presence
    or absence of model elements. In \cite[Hor14a], the i* goal models
    presented in \cite[Hor16a] are integrated with the MAVO framework
    \cite[Sal12a] to highlight the uncertainties in the model structure, i.e.,
    in terms of absence and presence of model elements, that must be resolved
    for meeting the desired goal satisfaction levels. In \cite[Fra06a], a
    framework for metrics on i* is introduced to reason quantitatively about
    the properties of an i* model. An other approach \cite[Esp13a] defines
    metrics over KAOS models to measure their complexity and completeness.
    
    \noindent {\bf Comparing our approach.} Our approach differs from the
    presented approaches in various aspects:
    
    \startitemize
    
      \item Contrary to our approach, none of the presented approaches supports
      assessment for multiple experts. These approaches could benefit from
      integrating multiple sources of assessment to reduce uncertainty margins.
      
      \item Unlike \cite[Whi09a,Che09c,Whi10a,Bar10a,Cha16a,Cha16b], our
      approach is not based on fuzzy logic. Our approach might be applied there
      to elicit membership functions and regulate the fuzziness of goal
      satisfaction.
    
      \item The approach proposed in \cite[Liu05a], based on Dempster-Shafer
      theory, allows experts to disagree on both the estimated probabilities
      and the risk model structure, which is not supported in our framework.
      However, the Dempster-Shafer theory is less known by risk analysis
      experts; this might prevent the approach to be applied in a real,
      industrial setting.
    
    \stopitemize
    
    \noindent Also, regarding the previous comparison, we might note that
    
    \startitemize
    
      \item Regarding FTA, as seen in \in{Section}[sec:comparing_ra], our
      approach exploits a goal refinement graph to assess the severity of risk
      consequences. However, the approach for computing satisfaction rates of
      high-level goals is very similar to the BDD-based propagation techniques
      for computing the probability of a root event \cite[Bed01a].
    
      \item Unlike \cite[Sab11a], our approach is not limited to one triangular
      distribution per obstacle, and we support multiple experts. Our
      propagation procedure appears more efficient as it only requires a single
      propagation through the obstacle/goal model.

      \item As pointed in the previous section, the technique proposed in
      \cite[Let04a,Hea11a] appears more finer-grained at the cost of eliciting
      domain-specific propagation variables and equations.
        
      \item The approach in \cite[Let14a] might be adapted for selecting most
      critical obstacles.
    
    \stopitemize
  
  \stopsection
  
  \startsection[reference=sec:comparing_cr,title={Models and Techniques for Controlling Risks}]
  
    This section discusses the relevant work related to controlling risks and
    more precisely to our technique presented in
    \in{Chapter}[chap:controlling_obstacle].
    
    \noindent {\bf Generating Countermeasures.} \cite[Sut98b] proposes
    scenario-based heuristics to generate risks together with generic
    requirements that could mitigate these.

    As a recall from Background Chapter, in the context of KAOS, goal-oriented
    formal techniques are available, such as regression calculus \cite[Lam00a]
    or a learning-based approach \cite[Alr16].
    
    In the context of Tropos, \cite[Bry06b,Bry06c] extends the framework Secure
    Tropos\emdash{}a variant of Tropos with specific constructs dedicated to
    the modeling of security-related concepts \cite[Gio05b]\emdash{}to
    automatically generate alternative models. The problem of generating
    alternatives is formulated as a planning problem; Actions correspond to the
    application of model primitives, such as adding or removing links whereas
    the Planning Objective is to satisfy predicates encoding \quote{correct}
    models. The planning actions do not introduce new intentional elements such
    as goals or actors. A planner then produces a plan that, once applied to
    the initial model, corresponds to an alternative design satisfying the
    initial goals. Alternative designs might then be evaluated to identify the
    most promising ones \cite[Bry07a]. \cite[Bry06a] further extends the
    approach to generate alternative configuration at runtime, when
    reconfiguration is required. \cite[Asn06d] combines the planning approach
    proposed in \cite[Bry06b,Bry06c] with the quantitative Goal-Risk Framework
    \cite[Asn11a] to generate, automatically, alternative designs that minimize
    the risk of some specified high-level goals.
  
    \noindent {\bf Integrating Countermeasures.} We are not aware of any work
    on the systematic integration of countermeasures in a requirement model
    with a clear, precise semantics. However, the relevance and importance of
    default-based reasoning have been recognized in the context of elaborating
    requirements or specifications.

    In \cite[Zow97a], a formal framework is proposed for reasoning about
    evolving requirements. The framework is based on belief revision and
    default theory; operators for adding new requirements and retracting
    requirements from a model are defined together with formal conditions for
    their valid application. The tracing of exceptional requirements is not
    discussed there.
    
    \cite[Bro06a] proposes an extension to the KAOS framework to specify
    changes in the goal's formal specifications based on A-LTL \cite[Zha06a].
    A-LTL is an adaptation-based extension to linear temporal logic that
    introduces an extra {\it adapt} operator. Using the operator, the user can
    specify changes in specification over time.
  
    In \cite[Rya93a], a specification is structured through axioms and {\it
    Overrides} relations. Such relations are derived from the structural
    decomposition of the system. Specific axioms predominate more general ones
    when a conflict occurs. This framework comes with formal foundations and
    well-defined procedures for identifying conflicts and predominance among
    axioms. These appear more oriented towards specification elaboration. In
    \cite[Sch93a], default specifications are introduced together with
    exceptions to increase the completeness of algebraic specifications.
    
    \noindent {\bf Comparing our approach.} Our approach mainly differs from
    those previous efforts in the following directions.
    
    \startitemize

      \item Unlike \cite[Rya93a,Sch93a,Zow97a,Bro06a], our techniques operate
      at requirements level and benefit from the refinement structure of a goal
      model. This structure helps in building a model where exception handling
      is integrated and in propagating required changes throughout the model.
  
      \item Unlike \cite[Rya93a,Sch93a,Zow97a,Bro06a,Asn11a], new requirements
      for a more robust system are incrementally integrated through obstacle
      analysis. The model updates are traceable back to the identified
      obstructed goals and their obstacles.

      \item The {\it But} relation introduced in \cite[Sch93a] somewhat
      corresponds to our {\it Except} relation. Also, our {\it Replace}, {\it
      RelaxedTo} and {\it Provided} annotations could be encoded into the
      specification using the {\it adapt} A-LTL operator as in \cite[Bro06a].
      However, this would clutter the specification as exceptional cases would
      be mixed in the ideal one.
  
    \stopitemize
  
    At the programming level, aspects may be used for separating exception
    handling from normal code \cite[Lip00a]. At modeling level, \cite[Ali12]
    convincingly shows how aspects can be used for separating exceptional
    behaviors from normal ones. As an alternative to the approach advocated in
    this thesis, robustness aspects might be incorporated in a goal model by
    use of constructs similar to the ones presented in \cite[Gil09a, Mus07a,
    Mor13a]. To the best of our knowledge, we are not aware of aspect-oriented
    requirements engineering models or techniques that focus on exceptions
    handling as proposed in \in{Chapter}[chap:controlling_obstacle].
  
  \stopsection
  
  \startsection[reference=sec:comparing_as,title={Requirements-Driven Runtime Adaptation}]
  
    The literature on runtime adaptation of software system is vast and covers
    a diversity of concerns. We refer the reader to
    \cite[Che09b,Saw10a,De-13a,Yan14a,Jur15a] for an overview of the
    state-of-the-art techniques and approaches. We discuss the approaches
    closely related to our risk-driven runtime adaptation techniques presented
    in \in{Chapter}[runtime].
  
    \noindent {\bf Early (KAOS-Based) Monitoring Approaches.} Upon
    \cite[Fic95a], \cite[Fea98a] proposes an approach that builds on KAOS to
    monitor the satisfaction of the requirements at runtime. Monitored
    parameters, identified at RE-time, are monitored at runtime; when a
    violation is observed, a reconciliation tactic, identified at RE-time, is
    chosen and deployed.

    \noindent {\sc\bf ReqMon.} \cite[Rob03a,Rob05a,Rob06a] builds on
    \cite[Fic95a,Fea98a] to monitor requirements and obstacles. Based on a
    goal/obstacle, monitor specifications are extracted from the obstacles
    assigned to monitoring agents. These specifications are then turned
    automatically into executable monitors within the {\sc ReqMon}
    infrastructure; Specification patterns drives the transformation. The
    approach aimed at raising alerts. The approach was extended so that goals
    can be formalized using UCM Object Constraint Language (OCL) \cite[Rob07a].
  
    \noindent {\bf Adaptive goals.} \cite[Bar10b] extends KAOS with adaptive
    goals. Adaptive goals prescribe a set of possible adaptations that can be
    performed when goals are violated. Adaptations include adding and removing
    goals, modifying goal specifications, etc.
    
    \noindent {\bf Context-based Approaches.} \cite[Ali09,Ali10] details a
    tool-supported approach where goals, refinements, agent assignment and
    contributions to soft goals can be decorated with context conditions; the
    approach introduce {\it Context Goal Models}. These context conditions are
    Boolean AND/OR formulas that are then monitored, at runtime, to switch to
    more appropriate alternative refinements. In \cite[Ali11], a context is
    both a target (desirable) state to be reached by a set of requirements and
    a current state that enables requirements (like a precondition) or activate
    requirements (like a trigger condition). Based on this notion of context,
    \cite[Ali11] identifies 4 type of assumptions: {\it (a)} {\it Activation
    Assumptions} captures that a context requires a set of specific
    requirements; {\it (b)} {\it Adaptability Assumptions} captures that a context is
    needed by a requirement; {\it (c)} {\it Refinement Assumptions} captures that
    subgoals satisfy their parent goal; {\it (d)} {\it Requirements Achievement
    Assumptions} captures that a requirement is satisfied if a target variant
    is achieved. These assumptions are represented as a Boolean CNF formula of
    atomic propositions that are monitored at runtime. At runtime, the system
    decides what variant of the system shall be selected and deployed. Two
    metrics drive the decision process: {\it Assumption Validity} that captures
    the statistical evidence of a variant to satisfy the root goal; and {\it
    Assumption Criticality} that captures the extend to which a context
    condition prevents a variant to satisfy the root goal.
  
    \cite[Gui15a,Gui17a] extends {\it Contextual Goal Models} with {\it
    Pragmatic Goals}. A {\it pragmatic goal} captures variations in the goal
    interpretation. It focuses on the variation of the quality variables such
    as time, errors, delays, etc. For example, consider the requirement
    \quote{\it Ambulances shall be on scene in a timely fashion}; the
    interpretation of \quote{\it timely fashion} would vary from one context (a
    simple dizziness for example) to another (a heart attack). The framework
    proposes to annotate goals with context conditions and their quality
    constraints. A pragmatic goal is satisfied if it satisfies the quality
    constraints of the current context. In the example above, the goal might be
    annotated with $time < 8m$ in the context $CriticalIncident$ and with $time
    < 14m$ otherwise. A bottom-up propagation algorithm is proposed to select
    the tasks that would guarantee the requirements given a current context
    \cite[Gui15a,Gui17a]. If multiple plans are available, the best plan
    regarding the quality variable is selected.
  
    \cite[Men16a] combines {\it Runtime Goal Models} \cite[Dal13a] and {\it
    Contextual Goal Models} \cite[Ali09,Ali10]. Based on the goal refinement
    annotations, a Discrete Time Markov Chain (DTMC) is build from the leaf
    goals to the root goals. The DTMC for a leaf goal is a simple DTMC with a
    success and a failure probability. The DTMC for a non-leaf goal is obtained
    by combining the DTMC of the subgoals according to the annotation (parallel
    combination, sequential combinations, number of retries, etc.) Once the
    DTMC for the root goal is obtained, a probabilistic model checker provides
    an estimate of the probability to satisfy the root goal; the model checker
    computes the probability to satisfy $\ltlF(\wedge_i G_i)$ where $G_i$ are
    the leaf goals. A parametric model checker computes a parametric expression
    representing the probability to satisfy the root goal to enable computation
    at runtime; the parameters are the values of the success and failure
    probabilities for the leaf goals. \cite[Men14a] briefly discusses the
    impact of context on the failure probabilities.
  
    \noindent {\bf Product-line based approach.} \cite[Epi09a] proposes an
    approach to learn transition probabilities in a Discrete Time Markov Chain
    (DTMC). At runtime, transitions between the nodes of the DTMC are
    observed, and probabilities for these transitions are computed. These
    probabilities then feed the Bayesian inference to learn new, revised,
    probabilities taking into account both the previously observed
    probabilities and the last observed one. A probabilistic model checker than
    then provides whether the reliability and performance are satisfied.
    \cite[Ghe09a] builds upon to detect and predict requirements failures.
    
    In \cite[Ghe13a], the software is modeled as a dynamic product line (DSPL),
    and the different configurations define the possible space for the
    adaptations. From the DSPL models, a parametric Markov Chain is generated.
    Efficient runtime verification technique then drives the selection of the
    most appropriate configuration at system runtime \cite[Fil16a].
    
    In \cite[Fil11a,Fil14a,Fil15b], a control-theoretical approach
    automatically generates controllers for self-adaptive systems. The user
    provides a running software, a parameter that impacts some non-functional
    requirements. The approach first learns a linear-model binding the
    parameter to the non-functional requirements; a controller is then
    generated to tune the parameter according to the non-functional
    requirement. \cite[Fil15c] extends the approach to multiple non-functional
    requirements and multiple parameters.
    
    \cite[Fil15a] discusses an approach for filtering the monitored
    probabilities to reduce the noise of the observations while maintaining an
    accurate tracking for the real probabilities.
    
    \noindent {\bf Awareness and Evolution Requirements.} \cite[Sou11b]
    introduces {\it Awareness Requirements} ({\it AwReq}) to reason about
    requirement satisfaction. {\it Awareness Requirements} are
    meta-requirements about the system requirements that are monitored at
    runtime, using {\it ReqMon}. \cite[Sou12a] introduces {\it Evolution
    Requirements} ({\it EvoReq}) to reason about the evolution of requirements.
    {\it Evolution Requirements} are meta-requirements contraining how system
    requirements shall be change when required. {\it AwReq} and {\it EvoReq}
    complements each other, in a framework called {\it Zanshin}, to specify a
    feedback loop for self-adaptive systems \cite[Sou13a]. {\it AwReq}
    indicates the situations in which an adaptation is required whereas {\it
    EvoReq} prescribes what adaptation shall be deployed. \cite[Sou11a]
    discusses how awareness requirements interacts in a control-theoritic
    approach for self-adaptive systems.
    
    \noindent {\bf Tropos for Adaptive Systems.} \cite[Mor08a,Mor17b] extends
    the Tropos framework specifically for automated systems. It proposes three
    dimensions: a goal dimension, an environment dimension, and a failure
    dimension. In the framework, goals are decorated with conditions formally
    specifying the goal; conditions includes {\it creation condition}, {\it
    pre-conditions}, {\it target conditions}, etc. The conditions are expressed
    in a non-temporal propositional logic. The failure model captures the
    exceptional situations and possible mitigations. A failure model contains
    {\it Failure} that models states where goals are not satisfied; {\it
    Errors} are events\emdash{}formalised as boolean conditions\emdash that
    lead to failures (called {\it symptoms} in \cite[Mor08a]); and {\it
    Recovery actions} are activities that anticipate, prevent and recover from
    errors. The reasoning about errors is not quantitative and no dedicated
    algorithm is available for selecting the most appropriate recovery actions.
    
    \noindent {\sc\bf Olympus.} \cite[Ram11a] proposes a genetic algorithm to
    choose the best configuration for an overlay network. \cite[Ram10a]
    describes an approach to generate sequences of actions to change a software
    system from one configuration to another. In \cite[Ram11b] utility
    functions are generated from a goal model with RELAXed requirements to
    enable the monitoring of goals. Depending on the type of goal, the
    technique automatically generates a state-based (invariant goal),
    metric-based (real-valued goal) or fuzzy logic-based (RELAX goal) utility
    function that, fed with monitored values, provides an estimate for the goal
    satisfaction. Utility scores are then propagated in the goal model from the
    monitored goals to the root goals. \cite[Che13a] illustrates how these
    techniques intertwine to enable self-adaptation of software systems.
    
    \noindent {\bf Self-adaptation for Service-based systems.} Approach in
    \cite[Ori12a,Qur09a,Qur10a,Qur10b,Qur11a,Qur11b] introduces a framework for
    requirements on self-adaptive, service-based systems. At design-time a
    forest of goal models is elicited to cover a wide variety of refinement
    alternatives. At runtime, the best alternatives are selected to satisfy
    high-level requirements and maximize the user preferences. User preferences
    are determined explicitly by the user or might be detected by the adaptive
    software.
    
    \noindent {\bf Surprise-based approach.} \cite[Ben14a,Ben15a] drafts a
    preliminary approach where the need for adaptation is inferred from
    surprise. A surprise is defined as \quote{\it how much monitored information
    differs from previously monitored information}.
  
    \noindent {\bf Comparing our approach.} Some comparison from the previous
    section also applies:
    
    \startitemize
    
      \item Our approach builds on a precise semantic defined in terms of
      behaviors. This is particularly important when monitoring; otherwise,
      these efforts are somewhat limited by lack of precise characterization in
      terms of observed states and behaviors. Unlike \cite[Ram11a, Ali10,
      Gui17a, Men16a, Sou13a], we are able to monitor behaviors and not just
      state predicates. We might, therefore, detect complex behaviors (such as
      $P$ occurs between $S$ and $Q$) that appear to be hard to model with
      other approaches.
      
      \item Unlike some of the presented approaches, such as
      \cite[Rob03a,Rob05a,Rob06a,Ori12a,Ghe13a,Fil16a], we propagate the
      monitored satisfaction rate through the goal/obstacle refinement trees.
      This enables the computation of the satisfaction rate of goals that are
      not directly monitorable. Our risk-driven adaptation, therefore, benefits
      from the goal/obstacle refinement structure.
    
      \item Unlike most of the approaches presented here, our approach does not
      focus on selecting alternatives or configuration parameters but on
      selecting appropriate countermeasures to risks. The approach in
      \cite[Rob07a] focuses on raising alerts. Our approach appears to be
      closer to \cite[Bar10a], as the goal model itself might be adapted.
    
    \stopitemize
    
    More specifically related to runtime adaptation approaches, our techniques
    differ for the following reasons:
    
    \startitemize
    
      \item Our approach shares similarities with \cite[Ghe09a,Ghe13a,Fil16a]
      in terms of probabilistic formalization and use of formal verification
      techniques. It does however not require building an explicit behavior
      model. The approach \cite[Men16a] also relies on probabilistic model
      checking, but the Markov Chain do not model the behavior of the system.
  
      \item The focus in \cite[Sou11b,Sou12a] is on meta-requirements about
      other requirements. Our approach limits to a single meta-requirement,
      guaranteeing that the estimated satisfaction rate is above the required
      satisfaction rate, and a single evolution meta-requirement that is
      selecting the most appropriate countermeasures.
      
      \item Unlike \cite[Rob07a,Mor08a,Ram11b], our monitoring procedure does
      not rely on specification patterns or type of goals. We also support
      temporal operators, such as $\ltlU$ or $\ltlW$. However, the monitored
      leaf obstacles shall be formalized in Linear Temporal Logic.
      
      \item Unlike \cite[Ori12a,Qur09a,Qur10a,Qur10b,Qur11a,Qur11b], our
      approach is not specific to self-adaptive, service-based systems. The
      type of adaptation is therefore different.
      
      \item Unlike \cite[Mor17b,Ali09,Ali10,Sou13a], our approach is
      quantitative. This enables a different, finer-grained, set of techniques.
      
    \stopitemize
    
    Aside from these frameworks for software adaptation, other monitoring
    techniques could relate to our approach. Monitoring techniques such as
    \cite[Bar04a,Sam05a,Tha05a] focus on failure detection; our focus in on
    probabilistic assessment. In \cite[Sam05a,Lee07a], the observed trace is
    divided into smaller samples so that properties can be checked on each
    sample; statistical reasoning is then applied to extract a probability. Our
    approach proposes an alternative without sampling the observed behavior
    into fixed-size samples. These approaches are not integrated with a
    requirement or risk models; they are restricted to monitoring LTL
    assertions.
    
    Also, our approach could benefit from the approach proposed in
    \cite[Fil15a] to smooth out monitored satisfaction rates. The approach
    drafted in \cite[Ben14a,Ben15a] could be applied to our monitored
    satisfaction rates to detect critical changes. This, however, needs to be
    further studied as no precise semantic appears to be available.
    
    The problem of deploying countermeasures and computing the sequence of
    actions required to adapt the running software system is more complex than
    the simplistic approach presented in \in{Section}[sec:sw_update].
    Approaches such as \cite[Ram10a] could integrate to take the complexity of
    the problem into account.
    
    Other approaches to runtime adaptation, such as Rainbow
    \cite[Gar03a,Gar04a,Gar09a], focuses on architectural models. These
    approaches tend to focus on low-level requirements and do not benefit from
    the refinement structure enabling more complex adaptations. \cite[Ang13a]
    compares the requirements-based approach Zanshin to the architecture-based
    approach Rainbow \cite[Gar04a]. Other approaches, such as
    \cite[Che14a,Kra07a], mixes both requirements and architectural model to
    propose better adaptations \cite[Saw10a,Nus01a]. \cite[Cas11a] tackles a
    different problem trying to localize faulty components by monitoring a
    software system.
    
    Last, crossing the related work presented in the previous sections,
    highlighted some interesting connections.
    
    \startitemize
    
      \item The approach presented in \cite[Asn11a] could be combined with
      their previous effort \cite[Asn06d] to achieve risk-driven software
      adaptation similar to our techniques.
    
      \item In \cite[Bar10a,Bar10b], adaptive goals prescribes the possible
      adaptation that can be performed when a goal is violated. Our integration
      schemas changing the goal model, as introduced in
      \in{Section}[sec:integrating], might be seen as strategies that can be
      applied to the adaptive goal monitoring the obstructed leaf goal and the
      obstructing leaf obstacle.
      
      \item The approach proposed in \cite[Dal13a] appears to be similar to
      Dynamic Fault Trees with temporal requirements as described in
      \cite[Rui15a]. The proposed approach in \cite[Men16a] to compute the
      probability to satisfy a root goal based on a bottom-up propagation of
      Markov Chains appears to be similar to the computation procedure used in
      Dynamic Fault Trees \cite[Dug92a, Rao09a].
    
    \stopitemize
    
  \stopsection
  
    \section {Summary}

    This section presented the relevant work related to our approach. Our
    approach mainly distinguishes from the other approaches in the following
    aspects. Our approach proposes a quantitative risk analysis framework
    anchored on requirements; The risks are identified, assessed and controlled
    regarding these requirements. The tree structure in the goal/obstacle
    models enables finer-grained obstacles to be assessed by the experts or
    monitored at runtime while enabling obstacles to be prioritized or best
    countermeasures to be selected regarding high-level objectives. The
    semantic of our probabilistic goals and obstacles are precisely defined in
    terms of observable states and behaviors. Lastly, multiple cycles of risk
    analysis might be performed where risks to countermeasures are identified,
    assessed and controlled.

    Next chapter will conclude and discuss the open issues and perspectives.
  
\stopcomponent
