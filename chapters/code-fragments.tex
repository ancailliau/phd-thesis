
   \framed[toffset=.2em,width=local,frame=off,before=,after=,align=middle]{%
      {\setupTABLE[c][each][align={flushleft},frame=off,offset=2pt]
      \bTABLE
      \bTR
        \bTD[nc=2]	{\cronos\bf Goal} \framed[location=top,frame=off,align=flushleft,width=8cm]{Achieve [Make Up Water Provided When LOC \\
        \spoiler{Achieve [} And No Valve Mechanical Failure]} \eTD
      \eTR
      \bTR
        \bTD[loffset=10pt] {\cronos\bf FormalSpec}	\eTD
        \bTD \startformula[align=right] \startalign[n=1,align={left}]
          \NC LossOfCooling \wedge \neg MakeUpValveFailure \NR  \noalign{\vskip-.5em}
          \NC \Rightarrow MakeUpWaterProvided \NR
        \stopalign \stopformula \eTD
      \eTR
      \eTABLE}
    }\vskip-1.8em





\startluacode

function drawHistories (histories)
    context.startMPcode()
	local vspacing = 1.4
	for i, h in ipairs(histories) do
		drawHistory (h, (vspacing * #histories) - vspacing * i)
	end
    context.stopMPcode()
end

function drawHistory(items, position)
	local circle_size = 2
	local histories = #items
	
	context ("labeloffset := 2mm;")
	print (position)

	context( "draw" )	
	for i=1,(histories-2) do
		context( "(%fcm, %fcm) -- ", i - 1, position )
    end
	context( "(%fcm, %fcm);", histories-1, position )
	
    
	for i, item in ipairs(items) do
		context( "pair a ; a := (%fcm, %fcm); ", i - 1, position)
		if (item.color == 0) then
			context( "fill fullcircle scaled %imm shifted a withcolor red;", circle_size)
		else
			context( "fill fullcircle scaled %imm shifted a withcolor .5green;", circle_size)
		end
		context( "label.top(btex %s etex,a);", item.top)
		context( "label.bot(btex %s etex,a);", item.bottom)
    end
end

\stopluacode



% Realizability

\startsubsection[title={To Merge with previous}]
    
        In the non-probabilistic framework, a goal defines a set of 
        intended behaviours. In the probabilistic framework, a goal defines a set of 
        computation trees. The underlying agent model presented in \cite[Let02a] 
        needs to be revised to be consistent with the probabilistic framework.
        
        A computation tree is a tree of states. A computation tree can be 
        represented, with a labelling function, as a set of sequences of states that 
        is prefix closed, i.e. all prefixes of a sequences are member of the set 
        \cite[Kup00a]. Consider the following computation tree where $s_i$ are 
        states of the system:
        
        \midaligned{
        \startMPcode
        		def connectcircle (expr a, b, circle_size) =
        			numeric ang; ang := angle(b-a);
            		drawarrow (a + getanchor(ang, circle_size))
            				  -- (b - getanchor(ang, circle_size));
        		enddef ;
        		
        		def getanchor (expr ang, circle_size) =
        			(cosd(ang) * circle_size/2, sind(ang) * circle_size/2)
        		enddef ;
        
        		hspacing := 8mm;
        		vspacing := 8mm;
        		size := 6mm;
        		
        		draw fullcircle scaled size shifted (1*hspacing,0) withcolor black;
        		draw fullcircle scaled size shifted (2*hspacing,0) withcolor black;
        		draw fullcircle scaled size shifted (3*hspacing,0) withcolor black;
        		draw fullcircle scaled size shifted (4*hspacing,0) withcolor black;
        		draw fullcircle scaled size shifted (5*hspacing,0) withcolor black;
        		
        		draw fullcircle scaled size shifted (2*hspacing,1*vspacing) withcolor black;
        		draw fullcircle scaled size shifted (4.5*hspacing,1*vspacing) withcolor black;
        		
        		draw fullcircle scaled size shifted (3.25*hspacing,2*vspacing) withcolor black;
        		
        		connectcircle ((2*hspacing,1*vspacing), (1*hspacing,0), size);
        		connectcircle ((2*hspacing,1*vspacing), (2*hspacing,0), size);
        		connectcircle ((2*hspacing,1*vspacing), (3*hspacing,0), size);
        		connectcircle ((4.5*hspacing,1*vspacing), (4*hspacing,0), size);
        		connectcircle ((4.5*hspacing,1*vspacing), (5*hspacing,0), size);
        		connectcircle ((3.25*hspacing,2*vspacing), (2*hspacing,1*vspacing), size);
        		connectcircle ((3.25*hspacing,2*vspacing), (4.5*hspacing,1*vspacing), size);
        		
        		label(btex $s_{11}$ etex, (1*hspacing,0));
        		label(btex $s_{12}$ etex, (2*hspacing,0));
        		label(btex $s_{13}$ etex, (3*hspacing,0));
        		label(btex $s_{21}$ etex, (4*hspacing,0));
        		label(btex $s_{22}$ etex, (5*hspacing,0));
        		label(btex $s_1$ etex, (2*hspacing,1*vspacing));
        		label(btex $s_2$ etex, (4.5*hspacing,1*vspacing));
        		label(btex $s_\epsilon$ etex, (3.25*hspacing,2*vspacing));
        \stopMPcode
        }
        
        \noindent This tree can be represented by the set of behaviours
        
        \startformula
        \lbrace s_\epsilon, s_\epsilon s_1, s_\epsilon s_1s_{11}, s_
        \epsilon s_1s_{12}, s_\epsilon s_1s_{13}, s_\epsilon s_2, s_\epsilon 
        s_2s_{21}, s_\epsilon s_2s_{22} \rbrace
        \stopformula
        
        For a more concise representation, we can introduce a sequence of 
        natural numbers representing the computation tree above, as the indices 
        suggest. The sequence of natural numbers $11$ represents the sequence of 
        state $s_\epsilon s_1 s_{11}$. The set of sequences can be represented as
        \startformula
        \lbrace \epsilon, 1, 11, 12, 13, 2, 21, 22 \rbrace
        \stopformula
        
        In the following, for a computation tree $\tau$:
        \startitemize
        \item $\tau_0$ denotes the root of the computation tree. In our 
        		example, $\tau_0$ denotes $s_\epsilon$.
        \item $\tau_{[t]}$ denotes the sequence of states corresponding to 
        		the sequence $t$. In our example, $\tau_{[11]}$ denotes the sequence $s_
        		\epsilon s_1 s_{11}$.
        \item $\tau_{t}$ represents the state corresponding to the sequence 
        $t$. In our example $\tau_{11}$ represents the state $s_{11}$. $\tau_{t\cdot 
        s}$ represents the state corresponding to the sequence $t$ followed by the 
        sequence $s$. In our example, $\tau_{2\cdot 2}$ represents the state 
        $s_{22}$.
        \stopitemize
        
        An agent is composed by
        \startitemize
        \item An interface, stating the variables that are monitored and 
        	controlled by the agent. 
        \item A transition system, composed by a condition on the initial 
        states and a transition function mapping a sequence of monitored states to a 
        next state of controlled variables. 
        \item A set of goal the agent is responsible for. 
        \stopitemize
        
        In the sequel, $Mon(ag)$ denotes the set of monitorable variables 
        for the agent $ag$ and $Ctrl(ag)$ denotes the set of controllable variables 
        for the agent $ag$, and $Voc(ag)$ denotes the union of the two. $V^{|
        Ctrl(ag)}$, $V^{|Mon(ag)}$ respectively, denotes the projection of the set 
        of variables $V$ on the controlled, monitored respectively, variables by the 
        agent.
        
        denoted $Init(ag)$
        
        $Next(ag)$
        
        A computation tree over a set of variable $V$ is a finite or 
        infinite tree of states of $V$. The set of all computation trees over a set 
        of variable is denoted $Trees(V)$.
        
        The runs of an agent is a set of $Run(ag) \subseteq Trees(V)$ such 
        		that, for each computation tree $\tau \in Runs(ag)$,
        \startitemize
        \item $\tau_0^{|Ctrl(ag)} \in Init(ag)$ ,
        \item $\langle\tau_{[i]}^{|Voc(ag)}, \tau_{i\cdot j}^{|Ctrl(ag)}
        		\rangle \in Next(ag)$ for all sequences $i\cdot j$ in $\tau$.
        \stopitemize
        
        The set of infinite computation trees of an agent $ag$ is denoted 
        		$Behaviour(ag)$.
        
        \startdefinition{Responsibility Consistency Rule}
        \startformula
        \text{for all } ag \in Agent, g \in Goal, \text{ if } Resp(ag, g) 
        		\text {, then } Behaviour(ag) \subseteq g
        \stopformula
        \stopdefinition
        
        \startdefinition{Realizability}
        A goal $G$ is realizable by an agent $ag$ if there exists a 
        		transition system with
        \startitemize
        \item $Init(ag) \subseteq State(Ctrl(ag))$
        \item $Next(ag) \subseteq Paths(Voc(ag)) \times Ctrl(ag)$
        \stopitemize
        such that $Behaviour(ag) = G$.
        \stopdefinition
        
        \stopsubsection
        
        

        % \subsubsection {Limitation of PCTL formalisation}
        
        %LTL and CTL have an incomparable expressiveness. There exists a 
        %formula in LTL that has no equivalent in CTL, and vice-versa. For example, 
        %the reset property stating that from an error state, we can always reach a 
        %state where the error is recovered is not expressible in LTL. On the 
        %contrary, the stability property, stating that at some point a property is 
        %true forever (until the program stops) cannot be expressed in CTL. CTL can 
        %only express that we can reach a point where the property is true forever 
        %(which is too permissive) and that all computation reach a state where the 
        %property is true (which is too strong). It is not so obvious for the second 
        %claim. Consider the following system:
        %
        %\midaligned{
        %\startMPcode
        %		def connectcircle (expr a, b, circle_size) =
        %			numeric ang; ang := angle(b-a);
        %    		drawarrow (a + getanchor(ang, circle_size))
        %    				  -- (b - getanchor(ang, circle_size));
        %		enddef ;
        %		
        %		def getanchor (expr ang, circle_size) =
        %			(cosd(ang) * circle_size/2, sind(ang) * circle_size/2)
        %		enddef ;
        %	
        %		pair a, b, c;
        %		numeric circle_size;
        %		circle_size := 8mm;
        %		
        %		width := 6mm;
        %		height := 6mm;
        %		
        %		a := (0, 0);
        %		b := (1.5cm, 0);
        %		c := (3cm, 0);
        %		
        %		draw unitsquare xscaled width yscaled height smoothed .5mm shifted (c-(width/2,height/2)) withcolor black;
        %		draw unitsquare xscaled width yscaled height smoothed .5mm shifted (a-(width/2,height/2)) withcolor black;
        %		draw unitsquare xscaled width yscaled height smoothed .5mm shifted (b-(width/2,height/2)) withcolor black;
        %		
        %		label(btex $s_0$ etex scaled .75, a);
        %		label(btex $s_1$ etex scaled .75, b);
        %		label(btex $s_2$ etex scaled .75, c);
        %		
        %		label.top(btex $\lbrace a\rbrace$ etex scaled .75, a + (0, height/2));
        %		label.top(btex $\emptyset$ etex scaled .75, b + (0, height/2));
        %		label.top(btex $\lbrace a\rbrace$ etex scaled .75, c + (0, height/2));
        %		
        %		drawarrow 
        %			a + (width/2, 0) .. b - (width/2, 0);
        %			
        %		drawarrow
        %			b + (width/2, 0) .. c - (width/2, 0);
        %			
        %		drawarrow 
        %			a + (getanchor(-65, circle_size) yscaled 0 - (0,height/2))
        %			.. a + getanchor(-90, circle_size) * 1.5
        %			.. a + (getanchor(-105, circle_size) yscaled 0 - (0,height/2));
        %			
        %		drawarrow 
        %			c + (getanchor(-65, circle_size) yscaled 0 - (0,height/2))
        %			.. c + getanchor(-90, circle_size) * 1.5
        %			.. c + (getanchor(-105, circle_size) yscaled 0 - (0,height/2));
        %\stopMPcode}
        %
        %In $s_0$, the LTL formula is satisfied, as eventually, a path 
        %remains in $s_2$ or in $s_0$. However, the CTL formula $\forall\lozenge
        %\forall\square a$ does not hold in $s_0$. Indeed, in $s_0$, $\forall\square 
        %a$ is not satisfied (because $s_1$ does not satisfy $a$), and so $s_0$ does 
        %not satisfy $\forall\lozenge\forall\square a$ as the path that remains in 
        %$s_0$ never reach a state where $\forall\square a$. A more detailed 
        %discussion comparing both logic can be found in \cite[Var01a].
        
        % PCTL vs CTL
        
        
        
        
        
        
        
        
        
        
        
        
        
        
        
        
        
			
        			Note that we are not interested in the proportion of behaviors where
        			$C\Rightarrow\Theta T$ is satisfied among all possible behaviors. In fact,
        			the chance of observing a behavior satisfying $C\Rightarrow\Theta T$
        			tends towards $0$ in the long run. Imagine a very small system sending
        			messages with a probability of $\vfrac{9}{10}$ to deliver a message, as the following
        			figure suggest:
			
        			\vskip1em\midaligned{\startMPcode
                    		def connectcircle (expr a, b, circle_size) =
                    			numeric ang; ang := angle(b-a);
                        		drawarrow (a + getanchor(ang, circle_size))
                        				  -- (b - getanchor(ang, circle_size));
                    		enddef ;
            		
                    		def getanchor (expr ang, circle_size) =
                    			(cosd(ang) * circle_size/2, sind(ang) * circle_size/2)
                    		enddef ;
            	
                    		pair a, b, c, d;
                    		numeric circle_size;
                    		circle_size := 8mm;
            		
                    		width := 1.4cm;
                    		height := 6mm;
            		
                    		a := (2.5cm, 1cm);
                    		b := (2.5cm, 0cm);
                    		c := (0cm, 0cm);
                    		d := (5cm, 0cm);
            		
                    		draw unitsquare xscaled width yscaled height smoothed .5mm 
        						shifted (a-(width/2,height/2)) withcolor black;
                    		draw unitsquare xscaled width yscaled height smoothed .5mm 
        						shifted (b-(width/2,height/2)) withcolor black;
                    		draw unitsquare xscaled width yscaled height smoothed .5mm 
        						shifted (c-(width/2,height/2)) withcolor black;
                    		draw unitsquare xscaled width yscaled height smoothed .5mm 
        						shifted (d-(width/2,height/2)) withcolor black;
            		
                    		label(btex Start etex scaled .75, a);
                    		label(btex Try etex scaled .75, b);
                    		label(btex Delivered etex scaled .75, c);
                    		label(btex Lost etex scaled .75, d);

        					drawarrow a - (0, height/2) -- b + (0, height/2);
        					drawarrow b + (width/2, 0) .. b + (d-b)/2 + (0, height/2) .. d - (width/2, 0);
        					drawarrow d - (width/2, 0) .. b + (d-b)/2 - (0, height/2) .. b + (width/2, 0);
        					drawarrow b - (width/2, 0) -- c + (width/2, 0);
        					drawarrow c + (0, height/2) .. controls (c + (0, 1cm))
        								and (a - (1cm, 0)) .. a - (width/2, 0);
								
                    		label.top(btex $\frac{1}{10}$ etex scaled .75, b + (d-b)/2 + (0, height/2));
                    		label.top(btex $\frac{9}{10}$ etex scaled .75, b + (c-b)/2);
                    	\stopMPcode}\vskip1em
			
        			To satisfy $Try\Rightarrow\bigcirc Delivered$, every message must be successfully
        			delivered. The probability to deliver one message is $\vfrac{9}{10}$,
        			and $n$ messages is $\vfrac{9}{10}^n$. When
        			$n \rightarrow \infty$, $\vfrac{9}{10}^n \rightarrow 0$. However, the probability
        			to satisfy $Try\rightarrow\bigcirc Delivered$ is $1$ in \italic{Start},
        			\italic{Delivered} and \italic{Lost} states, and $\vfrac{9}{10}$ in
        			\italic{Try} state. Therefore, the satisfaction rate for $Try\Rightarrow\bigcirc Delivered$
        			is $\vfrac{9}{10}$.
            
            
                    \startdefinition[dfn:satisfaction-rate]{Satisfaction Rate}
                      	The satisfaction rate of a goal $C\Rightarrow \Theta T$ 
                       	is the maximal probability $p$ such that
        				\startformula
        					Pr(\lbrace\pi\in behaviors(s)\ |\ \pi\models C\rightarrow \Theta T\rbrace) \geq p
        				\stopformula
        				in all states $s$.
                    \stopdefinition	
            
                    \startexample
        %    			Consider a very simple system for a mine pump where we only have 4 
        %    			possible finite behaviors of two states:
        %            
        %        	    \startformula \startalign
        %            	 	\NC p_1 \NC = (LowWater, PumpOff), (LowWater, PumpOff) \NR
        %             		\NC p_2 \NC = (HighWater, PumpOff), (LowWater, PumpOn) \NR
        %    	         	\NC p_3 \NC = (HighWater, PumpOff), (HighWater, PumpOn) \NR
        %        	    	\NC p_4 \NC = (HighWater, PumpOff), (HighWater, PumpOff) \NR
        %    	        \stopalign \stopformula
        %            
        %          		The requirement $HighWater \Rightarrow X PumpOn$ is satisfied on 
        %    			three of the possible behaviors: $p_1$, $p_2$, and $p_3$. The behavior 
        %    			$p_4$ does not satisfy the requirement and $p_1$ vacuously satisfy our 
        %    			requirement. Intuitively, if all behaviors are observed at the same rate, 
        %    			the satisfaction rate for this requirement is $\fraction{2}{3}$.
                    \stopexample
                    
                    
                    
                    
                    
                    
                    
                    
                    
                    
                    
                    
                    
    
            
            % For example, the assertion `{\bf sooner-or-later} $P$' where $P$
            % is a set of state corresponds to the behaviors
            % $s_0s_1...s_{n-1}s_n$ where $s_0,...,s_{n-1} \in\not P$ and $s_n
            % \in P$. In other words, this corresponds to the behaviors
            % $(S\setminus B)^*B$ where $S$ is the set of possible states. As
            % this is countable, we have
            % 
            % \startformula
            %   Pr(\text{'\bf sooner-or-later }P') = \sum_{\hat\pi \in (S\setminus B)^*B} Pr(Cyl(\hat\pi))
            % \stopformula
            % 
            % As example, consider the Markov chain presented in
            % \in{Figure}[fig:incorrect-pctl]. We want to compute the
            % probability of the assertion `{\bf sooner-or-later} PumpOn' for
            % the initial state. The behaviors corresponding to the assertion
            % have the form $(s_0)^ns_1$ where $n$ is a natural number. We have
            % 
            % \startformula
            %   Pr(\text{'\bf sooner-or-later }PumpOn') = \sum_{i=0}^\infty (.5)^n\times.5 = 1
            % \stopformula
            
          
%            \noindent\bold{Probabilistic Goals.} Based on the formal definition 
%            of probability measure over behaviors, the satisfaction rate of a goal is 
%            formally defined as:
%            
%            \startdefinition[dfn:satisfactionrate]{Satisfaction Rate}
%                The satisfaction rate of a goal $\square\phi$, is defined as 
%                $Pr(\{\pi \in behaviors(s)\ |\ \pi \models \phi\})$ 
%                where $behaviors(s)$ are the possible behaviors in the system starting in any possible state $s$.
%            \stopdefinition	
%            
%            We assume that the behavior of the system can be captured by a 
%    		Markov Chain, even if not explicitly provided. Extending the definition 
%    		above, $P(G|H)$ is formally characterized as:
%            
%            \startformula
%            	Pr(\{\pi \in \{\pi'\in behaviors\ |\ \pi' \models H\}\ |\ \pi \models G \})
%            \stopformula
%            
%            We focus on non-vacuous satisfaction of our requirements. The 
%            problem of antecedent failure was recognized to be a problem is application 
%            of formal verification \cite[Bea94a]. Vacuous satisfaction potentially hides 
%            faults in complex models as parts of the assertion are not necessary for the 
%            satisfaction. Techniques exists to detect vacuity in formalized requirements 
%            \cite[Kup03a]. We assume that the requirements are not vacuously satisfied 
%            by the system of interest.
%            
            %The key idea behind the detection is to replace sub-formula by 
    		%$True$ or $False$ depending on their polarity and check the transformed 
			%assertion against the model. 
            
            %\startexample
            %Assume the requirement $HighWater \Rightarrow_{\geq 95\%} 
			%\lozenge_{\leq 2min} PumpOn$.
            %\startitemize
            %\item Replacing $HighWater$ by $True$ leads to $Pr_1[\square 
            %P_{\geq 95\%}[\lozenge_{\leq 2min} PumpOn]]$. If this is true in our system, 
            %it means that the motor is always turned on within two minutes, whether the 
            %water is high or not does not impact the satisfaction, and our requirement 
            %is vacuously satisfied.
            
            %\item Replacing $\lozenge_{\leq 2min} PumpOn$ by $False$ leads to 
            %$Pr_1[\square P_{\geq 95\%}[\neg HighWater]] \equiv Pr_1[\square \neg 
            %HighWater]$. If this is true in our system, water is never high, and our 
            %requirement is vacuously satisfied.
            %\stopitemize
            %\stopexample
            
            
			
%			The probability of satisfaction of a goal $C \Rightarrow \Theta T$ 
%			is the proportion between (a) the number of possible behaviors satisfying 
%			the goal's antecedent $C$ and consequent $\Theta T$ and (b) the number of 
%			possible behaviors satisfying the condition $C$.
            
            %For a behavioral goal $C\Rightarrow\Theta T$, we are obviously 
            %interested in non-vacuous satisfaction, leaving aside those trivial cases 
            %where the goal is satisfied due to C being false. We therefore focus our 
            %attention on behaviors where the goal antecedent C is satisfied.
        
        
        
      
      \stopsubsection
          
      \startsubsection[title={Formal Specification of Probabilistic Obstacles}]
      
    
      A probabilistic obstacle $\ltlF (C \wedge \neg\Theta T)$ with a satisfaction rate $x$
      might be formalized using PCTL* as
        
      \startformula
        
        \ltlF \mathbb{P}_{> x} (C\wedge\neg\Theta T)
        
      \stopformula
        
      For example, assume the goal \goal{Achieve[OwnerWarnedWhenPHLevelCritical]} 
      with a required degree of satisfaction of $.95$
      formalized as
      
      \startformula
        \ltlG \mathbb{P}_{\geq .95}(PHLevelCritical \rightarrow \ltlF_{< 5 \text{min}} OwnerWarned)
      \stopformula
      
      It states that the owner shall be warned within 5 minutes when the PH
      level of her aquarium is critical in 95\% of cases. An obstacle to this
      expresses that there is more than 5\% of cases where the owner is not
      warned within the time bounds.
      
      The minimal state probability to observe
      
      \startformula PHLevelCritical \rightarrow \ltlF_{< 5 \text{min}} OwnerWarned \stopformula
      
      shall be $.95$. The state with such state probability has the maximal state probability to satisfy the negation 
      
      \startformula PHLevelCritical \wedge \ltlG_{\geq 5 \text{min}} OwnerWarned \stopformula
      
      Therefore, the goal is not satisfied if the maximal probability is strictly
      greater than $.05$:
      
      \startformula
        \ltlF \mathbb{P}_{> .05}(PHLevelCritical \wedge \ltlG_{\geq 5 \text{min}} OwnerWarned)
      \stopformula
      
      \stopsubsection
      

      
            \stopsubsection
      
            % Proof for the Pr_{\geq p} [ True ] \equiv True               
            % \startformula\startalign[n=3,align={left, left, right}]
            %   \NC Pr_{\geq p} [ True ] 
            %   \NC \equiv Pr (\{\pi \in behaviors\ |\ \pi \models True\}) \geq p 	
            %   \NC \hskip.5cm\text{(Def. $Pr_{\geq p}$)}\NR
            %   \NC \NC \equiv Pr (\{\pi \in behaviors\ |\ \pi_0 \models True\})\geq p
            %   \NC \hskip.5cm\text{(Sat. relation for AP)}\NR
            %   \NC \NC \equiv Pr (\{\pi \in behaviors\}) \geq p 											
            %   \NC \hskip.5cm\text{(Def. $\models True$)}\NR
            %   \NC \NC \equiv 1 \geq p 																
            %   \NC \hskip.5cm\text{(Def. $Pr$)}\NR
            %   \NC \NC \equiv True 																	
            %   \NC \hskip.5cm\text{($0\leq p\leq1$)}\NR
            % \stopalign\stopformula
  
      %    \startsubsection
      %    	[reference=sec:patterns-probabilistic-goals,
      %		   title={Specification Patterns for Probabilistic Goals}]
      %    
      %      Formalising probabilistic requirements is a tedious and error-prone task,
      %      although essential for formal and automated reasoning. Specifications
      %      patterns encodes common formalisation to ease the specification of
      %      probabilistic goals. Refinement patterns encodes known tactics for
      %      decomposing goals into subgoals. We present here the probabilistic
      %      specification and refinement patterns inspired from the specification
      %      patterns presented in \cite[Dar95a].
      %      

      
      %      \startexample
      %            For example, the goal \goal{Achieve [AmbulanceOnSceneWhenMobilized]} 
      %            stating that an ambulance shall be on incident scene within 10 minutes when 
      %            mobilized, can be formalized using the \italic{Achieve} specification 
      %            pattern. Using LTL, the goal might be formalized as follow:
      %          
      %            \startformula
      %            	\forall i: Incident, a: Ambulance \centerdot (Mobilized(a, i) \Rightarrow \ltlF_{\leq 10 min} OnScene (a, i))
      %            \stopformula
      %          
      %            Back to our example, if ambulance is expected to be on scene within 
      %            10 minutes only in 95\% of cases, the probabilistic achieve specification 
      %            pattern produces the following formalisation :
      %          
      %            \startformula\startalign[n=1, align={left}]
      %                \NC \forall i: Incident, a: Ambulance \centerdot \NR
      %                \NC \hskip 2em Mobilized(a, i) \Rightarrow_{\geq 95\%} \ltlF_{\leq 10 min} OnScene (a, i) \NR
      %            \stopalign\stopformula
      %          
      %            Formally, the PCTL assertion states that we are almost sure that in 
      %            every state reachable from the initial state, 95\% of the behaviors starting in 
      %            these states are such that a mobilized ambulance is on scene within 10 
      %            minutes.
      %       
      %      \stopexample
      %      

      
      %    \stopsubsection
      %      
      %    \startsubsection[title={Specification Patterns for Probabilistic Obstacles}]
      %  	\stopsubsection
      
