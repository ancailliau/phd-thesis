% !TEX root = thesis.tex

\startcomponent chap-4
\environment common
\product thesis

\chapter
	[reference=chap:assessing,
   list={Assessing Likelihood and Criticality of Obstacles},
   bookmark={Assessing Likelihood and Criticality of Obstacles},
   marking={Assessing Likelihood and Criticality of Obstacles},
	 title={\framed[align=normal,frame=off]{Assessing Likelihood \\ and Criticality of Obstacles}}]

  The identified risks should be assessed in terms of their likelihood and the
  likelihood of their consequences. Identifying critical and likely risks is an
  important step to determine appropriate control measures next. Prioritization of
  the risks is commonly performed by comparing the risk likelihoods, the
  likelihood of the consequences and the severity of the consequences. In
  model-based requirements engineering, consequences might be expressed as the
  loss in satisfaction for some objective. Therefore, the prioritization of the
  risks might benefit from the connection between the risks and the
  requirements they impact.

  This chapter presents how critical and likely obstacles are identified. It
  explains how the satisfaction rate of probabilistic obstacles are estimated,
  and details how the satisfaction rate of high-level probabilistic goals are
  computed from the estimated satisfaction rate of low-level probabilistic
  obstacles.

  Leaf obstacles are estimated by experts as a single-value probability that
  can be combined with other single-value probabilities for more accurate
  estimates. The satisfaction rate of a high-level goal is computed by
  up-propagation from the leaf obstacles to the root obstacles; from the root
  obstacles to the leaf goals; and from the leaf goals to the high-level goals.
  Two propagation techniques are proposed: a BDD-based propagation that
  provides fast satisfaction rate computation at an increased pre-computation
  cost; and a Pattern-based propagation that provides finer-grained estimates
  at an increased specification cost. Critical obstacles are obstacles that
  causes a major drop in the satisfaction rate of high-level goals. The
  severity of an obstacle is therefore derived from this drop and the
  importance of the impacted goal. Identifying critical and likely obstacles
  amounts at finding the likely obstacle combinations that maximize that drop.
  As a result, the next resolution step might then focus on the critical and
  likely obstacles to increase the completeness, accuracy and adequacy of the
  requirements.

  This chapter is organized as follows. \in{Section}[sec:estimating_satrate]
  details how leaf obstacles are estimated. \in{Section}[sec:computing_satrate]
  details how the satisfaction rate of high-level goals is computed from the
  estimated leaf obstacles.
  \in{Section}[sec:indentifying_most_critical_obstacles] details how comparing
  the satisfaction rate of the high-level goals to their respective required
  satisfaction rate identify critical and likely risks.
    
  \startsection
    [title={Estimating Satisfaction Rate of Leaf Obstacles},
     reference=sec:estimating_satrate]

    The first step is to collect the estimates for the satisfaction rate of the
    leaf obstacles. Theses will be up-propagated next to compute the
    satisfaction rate of high-level goals. The satisfaction rate for the leaf
    obstacles might be available from historical data, measurements,
    specification sheets provided by contractors, etc. However, for cost
    reasons, technical difficulties, the uniqueness of the system or some
    practical limitation, such data migth not be available. The estimation of
    the satisfaction rates thereby relies on expert judgement.

    \startsubsection
      [title={General Approach for Estimating Satisfaction Rate}]

      We propose the following approach, inspired from \cite[OHa06a, Otw92],
      for estimating the satisfaction rate of leaf obstacles:
    
      \startitemize[n]

        \item Select the leaf obstacles to be estimated by the experts. These
        are the ones for which no data is available.

        \item Identify, select, and train the experts. The selected experts
        shall expose a diversity of opinions and methods and be independent in
        their knowledge and shall be trained to make sure they understand the
        models and the techniques. \cite[Vos08,OHa06a,Otw92] discusse in
        details how to select experts. \cite[Coo91,Vos08] provides a detailed
        discussion on the elicitation techniques, biais identification, and
        de-biasing techniques.

        \item Elicit the satisfaction rate for the selected leaf obstacles.
        \cite[OHa06a,Vos08] detail techniques and pitfalls for such
        elicitation sessions. The structure of the goal and obstacle model
        might be modified by the experts during this assessement step.

        \item Aggregate and analyze the results. Next section provides an
        overview of techniques for combining multiple expert assessments.

        \item Document all elements from this process such as, what obstacles
        were estimated, how experts were selected and trained, and what are the
        raw estimates provided by the experts. This enables future evaluation
        by independent experts.

      \stopitemize
    
    \stopsubsection
    
    \startsubsection
      [title={Combining Estimated Satisfaction Rates},
       reference=sec:combining_sat_rate]
      
      The use of multiple sources or multiple experts is generally recognized
      to increase the accuracy of estimates \cite[Coo91]. The problem is then
      to combine the satisfaction rates provided by multiple experts into a
      single satisfaction rate. This single satisfaction rate will be
      up-propagated in the next step. Informal techniques are often used in
      practice to reach a consensus on a single value agreed by all experts
      \cite[OHa06a, Vos08]. More mathematical approaches are however
      recognized to produce more accurate results than informal ones
      \cite[Bed01].
      
      Among mathematical approaches, we mainly distinguish two combination
      schemas: bayesian combinations and non-bayesian combination.
      
      Bayesian combination techniques relies on the Bayes' theorem for
      combining probabilities. However, theses techniques require and produce a
      probability distribution. We defer this discussion to the
      \in{Section}[sec:eliciting_more_accurate_estimates] where we includes
      uncertainties in the estimates of satisfaction rates.
      
      Non-Bayesian combination techniques however does not require probability
      distributions. These include standard combination such as taking the
      smallest satisfaction rate, the largest satisfaction rate or average the
      satisfaction rates. These combination might be generalised by a {\it
      r-norm} \cite[Bed01,Coo91]. We present here $r$-norm based combinations
      specialized to our probabilistic obstacle framework. The {\it r-norm
      satisfaction rate}, denoted $P_r(O)$, for an obstacle $O$ is given by
      
      \startformula
        P_r(O) = \left(\sum_{e \in Experts} w_e \left[P_e(O)\right]^r\right)^{1/r}
      \stopformula
      
      where $w_e$ is the weight assigned to the expert $e$ and $P_e(O)$ is the
      probability estimated by the expert $e$ for the obstacle $O$. Depending
      on the value given to the parameter $r$, the norm corresponds to
      different combinations \cite[Bed01,Coo91]. For example, consider the
      obstacle \obstacle{Power Cabling Failure} in
      \in{Figure}[fig:or_and_obstacle_refinement] estimated by two experts $e,
      e'$:

      \startformula 
        P_{e}(\text{\obstacle{Power Cabling Failure}}) = .3\hskip.5cm\text{ and }\hskip.5cmP_{e'}(\text{\obstacle{Power Cabling Failure}}) = .1
      \stopformula

      \placefigure[top]
      	[fig:or_and_obstacle_refinement]
      	{An obstacle refinement tree.}
        {\externalfigure[../images/chap4/or_and_obstacle_refinement.pdf]}

      \startitemize[packed,nowhite]
        
        \item Using the $\infty\text{-norm}$, it amounts at taking the largest
        satisfaction rate. The combined satisfaction rate is
        
          \startformula P_\infty(\text{\obstacle{Power Cabling Failure}}) = .3 \stopformula
        
        \item Using the $-\infty\text{-norm}$, it amounts at taking the
        smallest satisfaction rate. The combined satisfaction rate is
        
          \startformula P_{-\infty}(\text{\obstacle{Power Cabling Failure}}) = .1 \stopformula
          
        \item Using the $0\text{-norm}$, it corresponds to a geometric mean.
        The combined satisfaction rate is
              
          \startformula 
            P_0(\text{\obstacle{Power Cabling Failure}}) = \sqrt{.3\times.1} = 
              \ctxlua{tex.print(string.format("\letterpercent.2f", math.sqrt(.3*.1)))}
          \stopformula

        \item Using the $1\text{-norm}$, it corresponds to an arithmetic mean.
        The combined satisfaction rate is

          \startformula 
            P_1(\text{\obstacle{Power Cabling Failure}}) = \frac{.3 + .1}{2} =
              \ctxlua{tex.print(string.format("\letterpercent.2f", (.3+.1)/2))}
          \stopformula

      \stopitemize
      
      As shown in the example above, the combined satisfaction rate depends on
      the parameter $r$. Which {\it $r$-norm satisfaction rate} to chose
      depends on the required properties for the combined estimate.
      
      \startitemize

        \item {\bf Zero Preservation}. If all experts estimates that the
        satisfaction rate of an obstacle is zero, the combined satisfaction
        rate shall be zero. This property is satisfied for all {\it r-norms}.

        \item {\bf Marginalization}. The combined satisfaction rate does not
        depend on how the obstacle is OR-Refined. In other words, given an
        obstacle $O$ OR-refined by two disjoint subobstacles $SO_1$, $SO_2$,
        the combined satisfaction rate satisfies
        
        \startformula 
          P_r(O) =  P_r(SO_1) + P_r(SO_2)
        \stopformula
        
        Consider that two experts estimates the subobstacles of the
        OR-Refinement in \in{Figure}[fig:or_and_obstacle_refinement]. The first
        expert estimates $.1$ for \obstacle{Power Cabling Failure} and $.2$ for
        \obstacle{UPS Battery Failure}. The second expert estimates $.2$ for
        \obstacle{Power Cabling Failure} and $.1$ for \obstacle{UPS Battery
        Failure}. (For the simplicity of the presentation, we ignore the
        subobstacle \obstacle{Power Supply Failure}.) As we will show in
        \in{Section}[sec:computing_satrate], they both estimate that the
        satisfaction rate of the parent obstacle \obstacle{No Power Available}
        is $.1 + .2 = .3$. Using a {\it $0$-norm} for combining the
        satisfaction rate of the subobstacles gives $\sqrt{.2 \times .1} =
        .14$ for both subobstacles. With these combined estimates, the
        satisfaction rate for the parent obstacle is $.14 + .14 = .28$.
        However, with the {\it $1$-norm}, the combined satisfaction rate for
        both subobstacle is $.5\times(.2 + .1) = .15$. In that case, the
        satisfaction rate for the parent obstacle is $.3$, as the experts have
        estimated independently. The {\it $1$-norm} is the only {\it $r$-norm
        probability} to exhibit the marginalization property.

        \item {\bf Independence Preservation}. The combined satisfaction rate
        does not depend on how the obstacle is AND-Refined. In other words,
        given an obstacle $O$ AND-refined by two independent subobstacles
        $SO_1$, $SO_2$, the combined satisfaction rate satisfies:
        
        \startformula 
          P_r(O) =  P_r(SO_1) \times P_r(SO_2)
        \stopformula
        
        Consider the AND-refinement depicted in
        \in{Figure}[fig:or_and_obstacle_refinement]. Assume that the first
        expert, respectively the second, estimates $.1$ for the obstacle
        \obstacle{Diesel Generator Failure}, respectively $.2$, and $.2$ for
        \obstacle{Primary Power Supply Down}, respectively $.1$. Considering
        experts independently, they both estimates the satisfaction rate for
        the parent obstacle \obstacle{Power Supply Failure} to be $.1 \times .2
        = .02$. Using {\it $1$-norm} to combine their estimates, we obtain
        $.15$ for both subobstacles. The satisfaction rate for the parent
        obstacle is thereby $.15 \times .15 = .0225$, which differs from their
        the estimate they provided independently. Using {\it $0$-norm}, the
        combined satisfaction rate for the subobstacle is $\sqrt{.2\times
        .1}$. The resulting satisfaction rate for the parent obstacle $O$ is
        $\sqrt{.2\times .1} \times \sqrt{.2\times .1} = .02$. The {\it $0$-norm
        probability} has the independence preservation property, whereas {\it
        $1$-norm probability} does not.
        
      \stopitemize
      
      In the $r$-norm, experts are assigned a weight. Picking the weight for
      the experts is an important step. In the example above, we used equal
      weight for all experts. \cite[OHa06a, Bed01, Coo91] discuss how to
      assign weight to experts. The
      \in{Section}[sec:eliciting_more_accurate_estimates] shows how weight can
      be systematically determined when estimates are uncertain.
    
    \stopsubsection
  
  \stopsection
    
	\section
    [title={Computing Satisfaction Rates of High-Level Goals},
     reference=sec:computing_satrate]

    Previous section showed how the satisfaction rate of leaf obstacles are
    estimated. Once estimated, the second step consists of determining the
    satisfaction rate of the high-level goals from these estimated satisfaction
    rate by up-propagation. This section shows two propagation techniques for
    computing the satisfaction rate of high-level goals.
    
    \startsubsection
      [reference=sec:computing_bdd,title={BDD-based Computation}]

      The BDD-based procedure for computing the satisfaction rate for a
      top-level goal $G$ is outlined as follows: First, compute the set of
      AND-combinations of leaf obstacles obstructing $G$ called the obstruction
      superset. This superset is encoded as a binary decision diagram (BDD).
      Second, based on the estimated satisfaction rate for the leaf obstacles,
      compute the probability $p$ for the obstruction superset. The satisfaction
      rate for the goal is $1-p$.
            
  		\startsubsubsection
        [title={Computing Obstruction Sets},
         reference=sec:computing_obstruction_set]

        For a given goal $G$, we compute all AND-combinations of leaf obstacles
        and domain hypotheses that may obstruct $G$. More precisely, an {\it
        obstruction set} for a goal $G$ is a set of obstacles $O_i$ and domain
        hypotheses $DH_j$ such that the obtruction set obstruct the goal
        and the obstruction set is consistent

        \startformula
          \startalign[n=1]
            \NC \{O_1, O_2, ..., DH_1, DH_2, ...\} \vDash \neg G \NR[+]
            \NC \{O_1, O_2, ..., DH_1, DH_2, ... \} \nvDash false \NR[+]
          \stopalign
        \stopformula

        An obstruction set $OS$ should be minimal, that is, all its elements are
        required for falsifying the goal:
      
        \startformula
          \startalign
            \NC \text{for all } O_i \text{ in } OS: \NC OS \setminus O_i \nvDash \neg G	\NR[eq:minimal_o_os]
            \NC \text{for all } DH_j  \text{ in }  OS: \NC OS \setminus DH_j \nvDash \neg G \NR[eq:minimal_d_os]
          \stopalign
        \stopformula 

        For example, the goal \goal{Achieve [Make Up Pump Motor On When Water
        Requested]}, shown in \in{Figure}[fig:make_up_water_pump_on], can be
        obstructed by the following four obstruction sets:

        \startitemize[packed]
          \item \{\obstacle{Pump Mechanical Failure}\}, 
          \item \{\obstacle{Diesel Generator Failure}, \obstacle{Primary Power Supply Down}\}, 
          \item \{\obstacle{Power Cabling Failure}\}, 
          \item \{\obstacle{UPS Battery Failure}\}.
        \stopitemize
        
        \placefigure[bottom]
        	[fig:make_up_water_pump_on]
        	{Obstacles for \goal{Achieve [Make Up Pump Motor On When Water Requested]}.}
          {\externalfigure[../images/chap4/make_up_water_pump_on.pdf]}

        As the example show, a goal might have multiple alternative obstruction
        sets. The OR-combination of all alternative obstruction sets for a goal
        $G$ is called obstruction superset for $G$, denoted by $OS (G)$. In the
        example, the obstruction superset for \goal{Achieve [Pump A Activated
        Until Measured Water Level Appropriate]} is the set containing all the
        obstruction set above.
      
      % The obstruction sets in the obstruction superset should ideally be
      % independent:
      % 
      % \startformula
      %   \text{for all } OS, OS’ \text{ in } OS(G):  OS \cap OS’ = \varnothing
      % \stopformula
      % 
      % Such dependences thus arise when obstruction sets share common obstacles
      % or domain hypotheses.

        The obstruction superset for a goal is computed by up-propagation from
        the leaf obstacles and domain hypotheses in the obstacle trees to the
        root obstacle; then from the root obstacle to the obstructed leaf goal;
        and finally from the leaf goals in the goal trees to the considered
        top-level goal. The following details each step.

        \noindent {\bf Step 1: From leaf obstacles to root obstacles.} Consider
        the obstacle AND/OR refinement tree anchored on a leaf goal $LG$ in the
        goal model. To obtain the obstruction superset $OS (LG)$, we proceed by
        structural induction. Let $OS (LG \mid O)$ denote the obstruction superset
        for $LG$ obtained by considering all obstacles and domain hypotheses in
        the obstacle subtree rooted on $O$.
      
        \startitemize
          \item For a leaf obstacle or a domain hypothesis LO:
            \startformula
              OS (LG \mid LO) = \{LO\}
            \stopformula
          \item For an AND-refinement of $O$ in subobstacles $SO_1$ and $SO_2$: 
            \startformula
              OS(LG \mid O) =  OS(LG \mid SO_1) \times OS(LG \mid SO_2),
          \stopformula
      
          where $\times$ denotes the cartesian product over sets. The
          generalization to more subobstacles is straightforward.

          \item For an OR-refinement of $O$ in subobstacles $SO_1$ and $SO_2$:
            \startformula
              OS (LG \mid O) = OS (LG \mid SO_1) \cup OS (LG \mid SO_2).
            \stopformula
            
            \noindent The generalization to more subobstacles is straightforward too.
          
        \stopitemize

        For example, the obstruction superset for the obstacle \obstacle{Power
        Supply Failure} in \in{Figure}[fig:or_and_obstacle_refinement] is
        \{\{\obstacle{Diesel Generator Failure}, \obstacle{Primary Power Supply
        Down}\}\}; The obstruction superset for the obstacle \obstacle{No Power
        Available} is \{\{\obstacle{Diesel Generator Failure},
        \obstacle{Primary Power Supply Down}\}, \{\obstacle{Power Cabling
        Failure}\},\{\obstacle{UPS Battery Failure}\}\}. The obstruction
        superset for the root obstacle \obstacle{Pump Motor Not On And
        Requested} in \in{Figure}[fig:make_up_water_pump_on] contains the two
        extra obstruction sets \{\obstacle{Pump Mechanical Failure}\} and
        \{\obstacle{Pump Electrical Failure}\}.

        \noindent {\bf Step 2: From root obstacles to leaf goals.} The
        obstruction superset for a leaf goal $LG$ obstructed by a root obstacle
        $RO$ is:

        \startformula
          OS (LG) = OS (LG \mid RO)
        \stopformula
      
        In our example, the obstruction sets for the goal \goal{Achieve [Make
        Up Pump Motor On When Water Requested]} are the obstruction sets of
        \obstacle{Pump Motor Not On And Requested}.

        \noindent {\bf Step 3: From leaf goals to top goals.} The obstruction
        superset of a parent goal depends on the obstruction superset of its
        subgoals. (The presentation hereafter considers refinements consisting
        of two subgoals only, without any loss of generality.) For a parent
        goal $PG$ AND-refined into subgoals $SG_1$ and $SG_2$ we have:

        \startformula 
          OS (PG) = OS (SG_1) \cup OS (SG_2)
        \stopformula

        We do not consider OR-Refinement of goals; OR-Refinements of goals
        represent alternative systems and we focus on a single system.

        The obstruction sets in the superset thereby obtained are not necessarily
        independent. This arises from obstacle refinement trees sharing common
        obstacles, resulting in a non-empty intersection of obstruction sets.
        Such dependencies must be taken into account when computing the
        probability of satisfaction of obstruction sets; this can be achieved
        automatically, see \in{Section}[sec:computing_sat_rate]. By construction,
        independent obstruction sets are minimal.

        Considering the goal refinement presented in
        \in{Figure}[fig:make_up_water_provided], the obstruction superset for
        the goal \goal{Achieve [Make Up Water Provided When Loss Of Cooling]}
        may be computed from the obstruction supersets obtained for the
        subgoals \goal{Achieve [Make Up Water Requested When Loss Of Cooling]}
        and \goal{Achieve [Make Up Water Provided When Requested]}. Recursively
        the obstruction set for the goal \goal{Achieve [Make Up Water Requested
        When Loss Of Cooling]} may be computed from the obstruction set for the
        goals \goal{Achieve [Alarm Raised When Low Water]} and \goal{Achieve
        [Make Up Water Requested When Alarm Raised]}.

        \placefigure[bottom]
        	[fig:make_up_water_provided]
        	{Refinement of \goal{Achieve [Make Up Water Provided When Loss Of Cooling]}.}
          {\externalfigure[../images/chap4/make_up_water_provided.pdf]}

        An obstruction superset somewhat corresponds to the cut set of a Fault
        Tree \cite[Bed01,Lam09]. A first difference is that a cut set yields
        all combinations of leaf events causing the root event to occur whereas
        an obstruction superset yields all combinations of leaf obstacles
        causing the corresponding goal in the goal model to be obstructed. The
        propagation must therefore continue bottom-up through the goal model
        with specific propagation rules to assess the severity of obstruction
        consequences\emdash{}see Step 2 and Step 3, not found in Fault Tree Analysis.
        Another difference is that the tree nodes here are formalizable
        goal/obstacle specifications, linked by entailment relationships among
        levels, rather than event labels.
        
      \stopsubsubsection
      
  		\startsubsubsection
        [title={Computing Satisfaction Rates from Obstruction Sets},
         reference=sec:computing_sat_rate]
         
        Previous section showed how the obstruction superset is obtained for a
        high\-level goal. The satisfaction rate of such top-level goal is
        computed from its obstruction superset and the estimated satisfaction
        rate for the leaf obstacles.
    
        An obstruction set is satisfied if all its obstacles and domain
        hypotheses are satisfied. An obstruction superset is satisfied if at
        least one of its obstruction sets is satisfied. The satisfaction rate
        of a goal $G$ is given by the probability that its obstruction superset
        is not satisfied:
       
        \startformula
          P (G) =  1 - Pr  [OS (G)],
        \stopformula

        where $Pr [OS(G)]$ denotes the probability that the obstruction superset
        $OS (G)$ is satisfied.

        To compute the probability of satisfaction of an obstruction superset,
        a binary decision diagram (BDD) is built that represents the
        corresponding boolean formula where each leaf obstacle and domain
        hypothesis appears as a variable. This boolean formula encodes the
        AND/OR-combination of leaf obstacles and domain hypotheses through the
        disjunction of the conjunction of elements in each obstruction set. As
        the ordered BDD is canonical, equivalent formulas result in the same
        BDD; this makes specific treatments of dependent obstruction sets
        unnecessary. In our example, for the goal \goal{Achieve [Make Up Pump
        Motor On When Water Requested]} (see
        \in{Figure}[fig:or_and_obstacle_refinement] and
        \in{Figure}[fig:make_up_water_pump_on]) and its obstruction sets given
        before, the corresponding boolean formula is:
      
        \startformula\openup -5pt \startalign[n=1,align=left]
          \NC \text{\obstacle{Pump Mechanical Failure}} \NR
          \NC \vee (\text{\obstacle{Diesel Generator Failure}} \wedge \text{\obstacle{Primary Power Supply Down}})\NR
          \NC \vee \text{\obstacle{Power Cabling Failure}} \NR
          \NC \vee \text{\obstacle{UPS Battery Failure}} \NR
          \NC \vee \text{\obstacle{Pump Electrical Failure}}\NR
        \stopalign\stopformula

        Efficient algorithms are available for building compact BDDs
        \cite[Ebe05,Mei12]. Such BDDs enable fast computation of single-point
        probability values from single-point probability values for the
        variables \cite[Bed01]. \in{Figure}[fig:bdd] shows a BDD corresponding
        to the preceding formula. Each node represents a leaf obstacle. By
        following the solid edges if the leaf obstacle is satisfied or the
        dotted edges otherwise, we can determine whether the corresponding
        Boolean formula is true or false. The terminal nodes indicate whether
        the formula is satisfied (1) or not (0). For example, if
        \obstacle{Diesel Generator Failure} and \obstacle{Primary Power Supply
        Down} are both satisfied, the obstruction set is satisfied since the
        BDD path ends at node (1).

        \placefigure[bottom]
        	[fig:bdd]
        	{BDD for the obstruction set of \goal{Achieve [Make Up Pump Motor On When Water Requested]}.}
          {\externalfigure[../images/chap4/bdd.pdf]}

        Every edge in the BDD has a probability label. The probability label for
        a solid edge correspond to the estimated satisfaction rate for the
        obstacle at its source; For a dotted edge, the probability label
        correspond to 1 minus this probability. 

        The probability of a non-terminal node represents the probability that
        the formula corresponding to the sub-tree is satisfied. It is given by
        the product of the probability on the edge and the probability of the
        target node. For a terminal node, the probability is given by its value
        ($0$ or $1$).

        \startluacode
          eps_o_diesel = 0.01
          eps_o_primary = 0.05
          eps_o_cabling = 0.02
          eps_o_battery = 0.02 
          eps_o_mech = 0.14
          eps_o_elec = 0.01
        \stopluacode
        
        \placetable[top][tab:obstacle_value_bdd]{Satisfaction rate for obstacles.}
        {\setupTABLE[c][each][align={right,lohi},frame=off,offset=0pt]
        \setupTABLE[r][1][style=bold,bottomframe=on,boffset=4pt]
        \setupTABLE[r][2][toffset=4pt]
        \setupTABLE[c][1][roffset=4pt]
        \setupTABLE[c][2][loffset=4pt,roffset=4pt]
        \setupTABLE[c][3][loffset=4pt]
        \switchtobodyfont[small]
        \bTABLE
        \bTR \bTD Obstacle                  \eTD \bTD SatRate   \eTD \bTD 1 - SatRate   \eTD \eTR
        \bTR \bTD Diesel Generator Failure  \eTD \bTD $\ctxlua{round2(eps_o_diesel)}$ \eTD \bTD $\ctxlua{round2(1-eps_o_diesel)}$  \eTD \eTR
        \bTR \bTD Primary Power Supply Down \eTD \bTD $\ctxlua{round2(eps_o_primary)}$ \eTD \bTD $\ctxlua{round2(1-eps_o_primary)}$  \eTD \eTR
        \bTR \bTD Power Cabling Failure     \eTD \bTD $\ctxlua{round2(eps_o_cabling)}$ \eTD \bTD $\ctxlua{round2(1-eps_o_cabling)}$ \eTD \eTR
        \bTR \bTD UPS Battery Failure       \eTD \bTD $\ctxlua{round2(eps_o_battery)}$ \eTD \bTD $\ctxlua{round2(1-eps_o_battery)}$   \eTD \eTR
        \bTR \bTD Pump Mechanical Failure   \eTD \bTD $\ctxlua{round2(eps_o_mech)}$ \eTD \bTD $\ctxlua{round2(1-eps_o_mech)}$  \eTD \eTR
        \bTR \bTD Pump Electrical Failure   \eTD \bTD $\ctxlua{round2(eps_o_elec)}$ \eTD \bTD $\ctxlua{round2(1-eps_o_elec)}$ \eTD \eTR
        \eTABLE
        }
        
        
        For example, assume the satisfaction rates summarized in \in{Table}[tab:obstacle_value_bdd].
        Based on thesevalues, the probability of
        the node {\ss Pump Electrical Failure} is given by:

        \ctxlua{node_elect   = (1-eps_o_elec) * 0 + eps_o_elec * 1}
        \ctxlua{node_mech    = (1-eps_o_mech) * node_elect + eps_o_mech * 1}
        \ctxlua{node_battery = (1-eps_o_battery) * node_mech + eps_o_battery * 1}
        \ctxlua{node_cabling = (1-eps_o_cabling) * node_battery + eps_o_cabling * 1}
        \ctxlua{node_primary = (1-eps_o_primary) * node_cabling + eps_o_primary * 1}
        \ctxlua{node_diesel  = (1-eps_o_diesel) * node_cabling + eps_o_diesel * node_primary}
        
        \startformula
          P_{Pump Electrical Failure} = \ctxlua{round2(1-eps_o_elec)} \times 0 
            + \ctxlua{round2(eps_o_elec)} \times 1 
            = \ctxlua{round3(node_elect)}
        \stopformula

        The probability of the node {\ss Pump Mechanical Failure} is:

        \startformula
          P_{Pump Mechanical Failure} = \ctxlua{round2(1-eps_o_mech)} \times \ctxlua{round3(node_elect)} 
            + \ctxlua{round2(eps_o_mech)} \times 1 
            = \ctxlua{round3(node_mech)}
        \stopformula

        Similarily, we compute the probability for the nodes {\ss UPS Battery Failure},
        {\ss Power Cabling Failure} and {\ss Primary Power Supply Down}

        \startformula\openup -5pt\startalign[n=1,align=left]
          \NC P_{UPS Battery Failure} = \ctxlua{round2(1-eps_o_battery)} \times \ctxlua{round3(node_mech)} 
            + \ctxlua{round2(eps_o_battery)} \times 1 
            = \ctxlua{round3(node_battery)} \NR
          \NC P_{Power Cabling Failure} = \ctxlua{round2(1-eps_o_cabling)} \times \ctxlua{round3(node_battery)} 
            + \ctxlua{round2(eps_o_cabling)} \times 1 
            = \ctxlua{round3(node_cabling)} \NR
          \NC P_{Primary Power Supply Down} = \ctxlua{round2(1-eps_o_primary)} \times \ctxlua{round3(node_cabling)} 
            + \ctxlua{round2(eps_o_primary)} \times 1 
            = \ctxlua{round3(node_primary)} \NR
        \stopalign\stopformula

        We finally obtain the probability for the root node {\ss Diesel Generator Failure}:

        \startformula
          P_{Diesel Generator Failure} = \ctxlua{round2(1-eps_o_diesel)} \times \ctxlua{round3(node_cabling)} 
            + \ctxlua{round2(eps_o_diesel)} \times \ctxlua{round3(node_primary)}  
            = \ctxlua{round3(node_diesel)}
        \stopformula

        From there we derive the probability of satisfaction for the goal
        \goal{Achieve [Make Up Pump Motor On When Water Requested]}:

        \startformula
        	P (\text{\goal{Achieve [Make Up Pump Motor On ...]}}) 
            = 1 - \ctxlua{round3(node_diesel)} 
            = \ctxlua{context((1-math.ceil(node_diesel*1000)/1000)*100)}\%
        \stopformula
        
      \stopsubsubsection
    
    \stopsubsection
     
		\subsection {Pattern-Based Computation}
    
      An alternative propagation algorithm computes the satisfaction rate of
      the obstacle or goal at each propagation step. Similarly to the approach
      presented in the previous section, the procedure propagates the
      satisfaction rate from leaf obstacles to root obstacles, from root
      obstacle to obstructed goal, and from leaf goals to root goals. On the
      contrary, it does not compute obstruction supersets. This section
      presents the Pattern-based propagation procedure.
      
      \noindent{\bf Step 1: From leaf obstacles to root obstacles.} For an
      obstacle $O$ refined into two subobstacles $SO_1, SO_2$, without loss of
      generality, the satisfaction rate of the parent obstacle is given by:
      
      \startformula
        P(O) = P(SO_1)\times P(SO_2) \times P(O\mid SO_1, SO_2)
      \stopformula
      
      The term $P(O\mid SO_1, SO_2)$ captures partial entailement, as the
      combination of the subobstacles might not entail $O$ (See
      \in{Definition}[def:partial-obstacle-entailment]). This quantity shall be
      estimated by experts and annotate the refinement.
      
      For a OR-Refinement of the obstacle $O$ into two subobstacles $SO_1, SO_2$ 
      the satisfaction rate of $O$ is given by:
      
      \startformula\openup -5pt
        \startalign
          \NC P(O) = 1 - \NC \left[1 - P(SO_1)\times P(O\mid SO_1)\right]\NR
          \NC            \NC \times \left[1 - P(SO_2)\times P(O\mid SO_2)\right]\NR
        \stopalign
      \stopformula
      
      The terms $P(O\mid SO_i)$ captures partial entailment. 
      
      \noindent{\bf Step 2: From root obstacles to leaf goals.} For an
      obstructed goal $OG$ and the corresponding obstacle $RO$, the
      satisfaction rate is obtained using the following rule.
      
      \startformula
        P(OG) = 1 - P(RO)\times P(\neg OG\mid RO)
      \stopformula
      
      The term $P(\neg OG\mid RO)$ captures a partial obstruction.
            
      \noindent{\bf Step 3: From leaf goals to top goals.} Without loss of
      generality, we restrict ourselves to goal refinement with two subgoals.
      The generalization is straightforward. In the most general
      case the parent goal is satisfied if the two subgoals are satisfied, or
      the satisfaction of the first is sufficient for satisfying the parent, or
      the satisfaction of the second is sufficient for satisfying the parent.
      This leads to the following general propagation rule for AND-refinements:
      
      \startformula\openup -5pt
        \startalign
          \NC P (G) = \NC P (SG_1, SG_2) \times P (G \mid SG_1, SG_2)  \NR
          \NC \NC + P (SG_1, \neg SG_2) \times P (G \mid SG_1, \neg SG_2)  \NR
          \NC \NC + P (SG_2, \neg SG_1) \times P (G \mid SG_2, \neg SG_1)  \NR
          \NC \NC + P (\neg SG_1, \neg SG_2) \times P (G \mid \neg SG_1, \neg SG_2) \NR
        \stopalign
      \stopformula
      
      As we focus our attention on a single system, no alternative
      OR-refinements are considered. The satisfaction rate for a goal $G$
      given that none of the subgoals is satisfied is then equal to zero and
      the last term disappears. Moreover, if the refinement is 
      complete, we have
      
      \startformula
        P (G \mid SG_1, SG_2) = 1
      \stopformula
      
      The AND-propagation rule then reduces to:
      
      \startformula\openup -5pt
        \startalign
          \NC P (G) = \NC P (SG_1, SG_2) \NR
          \NC         \NC + P (SG_1, \neg SG_2) \times P (G \mid SG_1, \neg SG_2) \NR
          \NC         \NC + P (SG_2, \neg SG_1) \times P (G \mid SG_2, \neg SG_1) \NR
        \stopalign
      \stopformula
      
      The terms $P(G\mid ...)$ captures that a single subgoal may entail
      the parent goal. Such probability shall be estimated by the experts
      and annotate the refinement. Depending on the type
      of refinement and goal, this propagation rule can be made further
      specific. \in{Table}[tab:propagation-rules] gives propagation rules for a
      sample of common refinement patterns known to be complete, consistent and
      minimal \cite[Dar95]; the subgoals there are therefore independent.
      
      \placetable[top][tab:propagation-rules]
        {Propagation rules for common goal refinement patterns.}
        {\setupTABLE[c][each][align={right,lohi},frame=off,offset=0pt]
        \setupTABLE[r][1][style=bold,bottomframe=on,boffset=4pt]
        \setupTABLE[r][2][toffset=4pt]
        \setupTABLE[c][1][roffset=4pt]
        \setupTABLE[c][2][loffset=4pt]
        \switchtobodyfont[small]
        \bTABLE
          \bTR \bTD Refinement Pattern        \eTD \bTD Propagation Rule \eTD \eTR
          \bTR \bTD Milestone-driven          \eTD \bTD $P(G) = P(SG_1) \times P(SG_2)$                        \eTD \eTR
          \bTR \bTD Case-driven               \eTD \bTD $P(G) = P(CS) \times P(SG_1) + (1–P(CS)) \times P(SG_2)$ \eTD \eTR
          \bTR \bTD Guard introduction        \eTD \bTD $P(G) = P(SG_1) \times P(SG_2) \times P(SG_3)$         \eTD \eTR
          \bTR \bTD Divide-and-conquer        \eTD \bTD $P(G) = P(SG_1) \times P(SG_2)$                        \eTD \eTR
          \bTR \bTD Unmonitoribility-driven   \eTD \bTD $P(G) = P(SG_1) \times P(SG_2)$                        \eTD \eTR
          \bTR \bTD Uncontrollability-driven  \eTD \bTD $P(G) = P(SG_1) \times P(SG_2)$                        \eTD \eTR
        \eTABLE}
      
      As an example, for a milestone-driven refinement, the satisfaction of a
      single milestone-based subgoal is not sufficient for satisfying the
      parent goal. The propagation rule therefore reduces to:
      
      \startformula
        P (G) = P (SG_1) × P (SG_2)
      \stopformula
      
      However, for a case-driven refinement, the parent goal is satisfied when
      one of the subgoals is satisfied. The probability of satisfying the case
      condition $CS$, denoted $P(CS)$, equals $P(G\mid SG_1,\neg SG_2)$. Therefore,
      assuming two disjoint cases, the propagation rule becomes:
      
      \startformula
        P (G) = P (CS) \times P (SG_1) + (1 - P (CS)) \times P (SG_2) 
      \stopformula
      
      The specific simplification of the generic propagation rule thus depends on the goal refinement
      pattern used. This information is available in the annotation of the
      refinement node \cite[Lam09].
      
    \subsection[sec:compare_propagations]{Discussion}
      
      This section compares the approaches presented in the two previous
      sections for computing the satisfaction rate of a top-level goal.
        
      \startluacode
        eps_o_power_supply_failure = 1 - ((1 - (eps_o_diesel * eps_o_primary)) * (1 - eps_o_cabling) * (1 - eps_o_battery))
      \stopluacode
      
      The propagation using obstruction supersets supports dependent goals and
      obstacles. The Pattern-based propagation procedure does not. For example,
      consider the goal model fragment in \in{Figure}[fig:shared_obstacle]. The
      satisfaction rate for the obstacle \obstacle{No Power Available} is
      $\ctxlua{round3(eps_o_power_supply_failure)}$. With the Pattern-based
      propagation procedure, the satisfaction rate for both subgoals is
      $\ctxlua{round3(1 - eps_o_power_supply_failure)}$. The satisfaction rate
      for the parent goal \goal{Achieve [Make Up Water Provided When
      Requested]} is computed using the specialized propagation rule for {\it
      divide-and-conquer} refinement, leading to $\ctxlua{round3(1 -
      eps_o_power_supply_failure)} \times \ctxlua{round3(1 -
      eps_o_power_supply_failure)} = \ctxlua{round3((1 -
      eps_o_power_supply_failure) * (1 - eps_o_power_supply_failure))}$. This
      however count the obstacles twice. On the contrary, the obstruction
      superset for the parent goal is $\{\{\text{\obstacle{No Power
      Available}}\}\}$. The probability to satisfy the obstruction superset is
      $\ctxlua{round3(eps_o_power_supply_failure)}$. The satisfaction rate for
      the goal is therefore $\ctxlua{round3(1 - eps_o_power_supply_failure)}$.

      \placefigure[top]
      	[fig:shared_obstacle]
      	{Goal/obstacle model fragment for \goal{Achieve [Make Up Water Provided When Requested]}.}
        {\externalfigure[../images/chap4/shared_obstacle.pdf]}
        
      The Pattern-based propagation procedure appears to be finer-grained as it
      support partial entailement and partial obstructions. These needs however
      to be elicited by the experts. Using more specific propagation rules lead
      to more precise estimates, in particular when the refinement follow the
      {\it by-case} refinement pattern. For example, consider the
      (hypothetical) refinement in \in{Figure}[fig:by_case]. Given independent
      obstruction sets, the satisfaction rate for the goal \goal{Maintain
      [Water Level Above LOW]} will equal the product of the satisfaction rate
      of the subgoals. However, the refinement with only one subgoal satisfy
      the partial entailment condition. In other words, the probability that
      \goal{Maintain [Water Level Above LOW]} is satisfied is not $0$ when
      only \goal{Achieve [Water Level Above LOW When Alarm Raised]} is
      satisfied. Using the specific propagation rule gives a satisfaction rate
      for the parent goal that will take the partial satisfaction into account,
      resulting in an higher satisfaction rate.

      \placefigure[top]
      	[fig:by_case]
      	{By-case refinement for \goal{Maintain [Water Level Above LOW]}.}
        {\externalfigure[../images/chap4/by_case.pdf]}
      
      The main difference between the two propagation procedure is {\it when}
      most of the computation occurs. With the BDD-based propagation technique,
      most of the computation cost occurs at the construction of the
      obstruction supersets. The computation of the probability given an
      obstruction superset is very cheap in comparison. With the Pattern-based
      approach, the cost cannot be splitted in two smaller procedures.
      
      The BDD-based approach enables fast computation of probability
      distributions (as it will be seen in
      \in{Chapter}[chap:knowledge-uncertainty]). To compute probability
      distributions, we compute many obstruction superset probabilities on the
      same obstruction superset, taking advantage of the separation in two
      steps. It also enables the runtime monitoring of satisfaction rates (as
      it will be seen in \in{Chapter}[runtime]) as the obstruction superset is
      not computed at runtime.
      
      On a randomly generated model with 10.000 goals and 10.000 obstacles
      (with 1000 obstruction links), the Pattern-based computation takes about
      13 seconds, and the BDD-based computation takes about 18 seconds.
      However, computing the probability of the obstruction set takes only 5
      milliseconds. Each experiment was performed 80 times and the observed
      variance was very small (below 60 ms for the two propagations, and below
      a tenth of a millisecond for the probability computation). All benchmarks
      were performed using {\it BenchmarkDotNet} \cite[BDNET] on an Apple
      MacBook Pro with a 3 GHz Intel Core i7, 16 GB of 1600 MHz DDR3 Ram, and
      an Apple SSD drive.
      
    % \in{Figure}[fig:nb_goals] shows the mean time to compute the satisfaction
    % rate of a high-level goal on a randomly generated goal/obstacle model.
    % The $y$-axis indicates the mean time to compute the satisfaction rate. The
    % $x$-axis indicates the number of obstacles (top), and the number of goals
    % (bottom). The gray line with circle points shows the time for the
    % BDD-based approach; the orange line with triangular points shows the time
    % for BDD-based approach without accounting the time to build the BDD; the
    % blue line with square points shows the time for the Pattern-based
    % approach. Some data points are missing, these corresponds to unrealistic
    % models; e.g. 1000 obstructions with 10 goals and 10 obstacles.
    % 
    % As models are randomly generated, this might not reflect the time
    % required on a real model as the structure is likely to be different.
    % However, it provides useful information to compare the propagation
    % techniques. Each experiment was performed multiple time and the observed
    % variance was very small. All benchmarks were performed using {\it
    % BenchmarkDotNet} \cite[BDNET] on an Apple MacBook Pro with a 3 GHz Intel
    % Core i7, 16 GB of 1600 MHz DDR3 Ram, and an Apple SSD drive.
    % 
    % As we can see on \in{Figure}[fig:nb_goals], it is not always better to
    % use the BDD-based propagation; on some plots, the blue line is lower than
    % the gray line. However, as the orange line shows, if multiple probability
    % computations are required, the BDD-based approach is faster. The
    % computation time of a satisfaction rates increase with the number of
    % obstacles, as expected. The computation time of a satisfaction rate also
    % increase with the number of goals, but the computation time is constant
    % once the obstruction superset is built (for the BDD-based approach).
    %
    % \placefigure[page]
    % 	[fig:nb_goals]
    % 	{Mean time to compute a satisfaction rate in a random goal/obstacle model.}
    %   {\startcombination[1*2]
    %     {\externalfigure[../images/chap4/nb_obstacles.pdf]}{}
    %     {\externalfigure[../images/chap4/nb_goals.pdf]}{}
    %   \stopcombination
    %   }
        
  \startsection
    [reference=sec:indentifying_most_critical_obstacles,title={Identifying Most Critical Obstacles}]

    This section shows how critical and likely risks are highligthed based on
    the estimated statisfaction rates for leaf obstacles, as estimated by the
    experts (see \in{Section}[sec:estimating_satrate]), and the satisfaction
    rate for the high-level goals, as computed by propagation (see
    \in{Section}[sec:computing_satrate]).
    
    ---
    
    An obstacle is said to be {\it critical} for a goal if it causes the
    satisfaction rate of this goal to fall below the required satisfaction
    rate.
    
    \startdefinition[def:critical_obstacle]{Critical obstacle}
    
      A probabilistic obstacle $O$ is {\it critical} for a goal $G$ if
      obstacle $O$ results in a goal satisfaction rate such that $P(G) < RSR(G)$.
    
    \stopdefinition
    
    To identify whether an obstacle $O$ is critical, the severity of
    violation is computed with the ESR of all obstacles but $O$ set to $0$
    (as a recall, $SV(G) = RSR(G) - P(G)$); the obstacle is then critical
    if $SV(G) < 0$. A set of obstacle is {\it critical} if the obstacles
    together causes the violation severity to drop below $0$. 
    
    An obstacle is said {\it acceptable} for a goal if it results in a
    satisfaction rate for this goal.
    
    \startdefinition{Acceptable obstacle}
    
      A probabilistic obstacle $O$ is {\it acceptable} for a goal $G$ if
      obstacle $O$ does not result in a goal satisfaction rate such that
      $P(G) < RSR(G)$.
    
    \stopdefinition
    
    For example, the obstacle \obstacle{UPS Battery Failure} causes the
    satisfaction rate for the root goal \goal{Achieve [Make Up Water
    Provided When Loss Of Cooling]} to fall to $.989$. With a required
    satisfaction rate of $.99$, the obstacle \obstacle{UPS Battery Failure}
    is critical for \goal{Achieve [Make Up Water Provided When Loss Of
    Cooling]}.
    
    An obstacle is said to be {\it tolerable} if the goal's satisfaction
    rate falls below some tolerable satisfaction rate but still above the
    required satisfaction rate.
    
    \startdefinition{Tolerable obstacle}
    
      A probabilistic obstacle $O$ is {\it tolerable} for a goal $G$ if it
      results in a satisfaction rate of the goal $G$ such that $TSR(G) <
      P(G) < RSR(G)$.
    
    \stopdefinition
    
    ---

    The severity of an obstacle is defined as the loss in satisfaction rate of
    some root goal. In our running example, the obstacle \obstacle{UPS Battery
    Failure} causes, alone, a loss of $0.011$ for the high-level goal
    \goal{Achieve [Make Up Water Provided When Loss Of Cooling]}. To evaluate
    obstacles consequences, we may proceed in two ways:
    
    \startitemize

      \item {\bf Global impact analysis:} the satisfaction rate for all leaf
      obstacles are together propagated bottom-up in the goal graph to see how
      much the resulting statisfaction rate of higher-level goals deviates from
      their required satisfaction rate.

      \item {\bf Local impact analysis:} the consequence of a single leaf obstacle is
      evaluated by up-propagation of the satisfaction rate for this leaf
      obstacles, all other leaf obstacles being assigned a probability of $0$
      (meaning that they are all assumed to never happen).
    
    \stopitemize
    
    In between, we can evaluate the consequences of a combination of leaf
    obstacles. By increasing the size of the combinations, we range from local
    impact analysis to global impact analysis.
    
    An obstacle combination is {\it unnacceptable} for a high-level goal if it
    causes the satisfaction rate of that goal to fall below the required
    satisfaction rate. An obstacle combination is {\it acceptable} if above the
    required satisfaction rate. An obstacle combination is {\it tolerable} if
    the satisfaction rate fall below the tolerable satisfaction rate but above
    the required satisfaction rate. For example, assume that the required
    satisfaction rate for \goal{Achieve [Make Up Water Provided When Loss Of
    Cooling]} is $.99$ and its tolerable satisfaction rate is $.6$. The
    obstacle \obstacle{UPS Battery Failure} leads to a $.989$ root goal's
    satisfaction rate. This obstacle is therefore tolerable and not critical.
    
    Risks can be assessed and presented using a violation diagram for early
    criticality assessment. The risks are represented in a two-dimensionnal
    plot of the satisfaction rate for the high-level goal and the probability a
    combination occurs. \in{Figure}[fig:violation_diagram_single_value] shows
    such violation diagram for the goal \goal{Achieve [Make Up Water Provided
    When Loss Of Cooling]}. The $x$-axis reports the probability of the
    combination of obstacles occurs. This is the product of the estimated
    satisfaction rate for the obstacles in the combination. The $y$-axis
    reports the violation severity (as a recall $SV (G) = RSR(G) – P(G)$) for
    the goal given the obstacles in the combination.
    
    \placefigure[top]
    	[fig:violation_diagram_single_value]
    	{Violation Diagram for \goal{Achieve [Make Up Water Provided When Loss Of Cooling]}.}
      {
        \externalfigure[../images/chap4/violation_diagram_single_value.pdf]
      }
      
    In \in{Figure}[fig:violation_diagram_single_value], we can see, most of the
    obstacles alone are not critical for the system. Three obstacles are
    critical and shall be investigated; These shall be resolved in the next
    step.
    
    Criticality assessment should be a dynamic process. Once the most critical
    obstacles are resolved, the violation diagram shall be generated for pairs
    to identify the next obstacle combinations to focus on. And so forth, until
    no obstacle combination are critical and the number of tolerable obstacle
    combinations agrees with the domain best practive, regulations and/or
    standards.
    
    \placefigure[top]
    	[fig:violation_diagram_single_value2]
    	{Violation Diagram for \goal{Achieve [Make Up Water Provided When Loss Of Cooling]} with pair of obstacles.}
      {
        \externalfigure[../images/chap4/violation_diagram_single_value2.pdf]
      }
      
    In practice, it appears that critical combinations of many obstacles are
    caused only by a subset of the obstacles.
    \in{Figure}[fig:violation_diagram_single_value2] shows the violation
    diagram for pairs of obstacles. As we can see, critical pairs contains a
    critical obstacle identified with the violation diagram in
    \in{Figure}[fig:violation_diagram_single_value].
    
    There is a multi-criteria optimization problem here as we are looking for
    minimal sets of leaf obstacles that maximize the severity of goal
    violations. To solve the optimization problem, we can naïvely generate all
    possible leaf obstacle combinations. The violation severity $SV (G)$ is
    then computed for the root goal $G$. The most critical combinations are
    identified by sorting the leaf obstacle combinations by violation severity.

    The set of combinations that maximizes the severity defines a Pareto front;
    efficient algorithms for generating them are available
    \cite[Bor01,Kun75]. These algorithms and the specific structure of our
    problem indicates that our generation of leaf obstacle combinations and
    their ranking by severity could thereby be optimized in order to scale up
    for larger systems.
    
    The techniques presented in the chapter consider only a single high-level
    goal. However, in practice, it is likely that several high-level goals are
    competing. The satisfaction rate for these high-level goal are combined
    using a normalized weighted sum to apply the techniques as if there were
    only a single high-level goal. The weights correspond to the relative
    priority of these high-level goals.

  \stopsection
  
  \startsection[title={Summary}]
  
    This chapter presented how critical and likely risks are highlighted to
    focus the resolution step on the important combinations. To that aim, the
    satisfaction rate of leaf obstacles are estimated by experts, possibly
    combined to produce more accurate estimates. The leaf estimates are
    propagated through the obstacle and the goal model to compute the
    satisfaction rate of the high-level goals. The chapter presented two
    propagation techniques: a BDD-based approach and a Pattern-based approach.
    BDD-based approach works faster if multiple probability computation are
    required whereas the Pattern-Based approach provides finer-grained
    estimates. The estimated satisfaction rate of high-level goals is compared
    to their respective required satisfaction rates. Obstacles causing the
    estimated satisfaction rate to fall below the required satisfaction rate of
    these high-level goals are highlighted in a violation diagram.
    
    The propagation techniques presented in this chapter provides the basis to
    identify the most appropriate countermeasures, as the next chapter will
    present. In addition, the BDD-based propagation approach enables the
    techniques presented in \in{Chapter}[chap:knowledge-uncertainty] (to cope
    with uncertainty margin) and \in{Chapter}[runtime] (to monitor satisfaction
    rates at runtime) to be applied in practice.
  
  \stopsection

\stopcomponent
