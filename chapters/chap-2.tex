% !TEX root = thesis.tex

\startcomponent chap-2
\environment common
\product thesis

\chapter
    [reference=chap-background,
     title={Background: Goal-Oriented Requirements Engineering}]
   
  This chapter introduces some necessary background in goal-oriented
  requirements engineering. It introduces the KAOS framework extended in the
  following chapters. \in{Section}[sec:background_goal] presents how system
  objectives are modeled as goals in the KAOS framework.
  \in{Section}[sec:background_obstacle] introduces obstacle analysis as a mean
  for anticipating what could go wrong.

    \startsection[reference=sec:background_goal,title={Modelling system objectives with KAOS}]
  
    Goal-oriented requirements engineering is now recognized as a major
    modeling technique for engineering requirements of complex,
    mission-critical software systems. Unlike other approaches, it also capture
    the {\it why} behind the resulting software specifications \cite[Lam09].
    Among the available goal-oriented frameworks, KAOS supports a formal
    approach where the objectives of a system can be specified in a precise,
    clear-cut sense \cite[Lam09].
    
    KAOS also supports a multi-view modeling approach where objects,
    agent responsibilities, operations, and behaviors are represented in
    connection with goals. We focus here on goal modeling only; a detailed
    presentation of the framework can be found in \cite[Lam09].
    
    \subsection{Goals, requirements, and environment expectations}
    
      A {\it goal} is a prescriptive statement of intent to be satisfied by the
      agents forming the system. An {\it agent} is an active system component
      having responsibilities in goal satisfaction. An agent has capabilities;
      these are conditions the agent can monitor or control. The word {\it
      system} refers to the software-to-be together with its environment,
      including pre-existing software, devices such as sensors and actuators,
      people, etc.
      
      For example, a goal in a nuclear spent fuel pool system may express that makeup
      water shall be provided when there is a loss of cooling in the pool:

      \startkaosspec
          \GoalName {Make Up Water Provided When Loss Of Cooling}
          \KaosAttribute {Definition} {
            Cold makeup water shall be provided when cooling is no longer guaranteed.
        }
      \stopkaosspec
    
      A goal may be {\it behavioral} or {\it soft} dependent on whether it can
      be satisfied in a clear-cut sense or not. In the context of risk
      analysis, the thesis focuses on behavioral goals. We briefly present soft
      goals in a later subsection.
    
      A behavioral goal captures a maximal set of intended behaviors
      declaratively and implicitly; a {\it behavior} is a sequence of system
      states. A behavior thus violates a goal if it is not among those
      prescribed by the formal specification of the goal \cite[Lam09]. The
      notation $\pi \vDash G$ expresses that the behavior $\pi$ satisfies the
      goal $G$.
      
      Goals should be refined until they are assignable as responsibilities of
      single agents. In such refinements, leaf goals assigned to a software
      agent are {\it requirements} whereas leaf goals assigned to an
      environment agent are {\it expectations}.
      
      Each goal has a {\it name}, a natural language {\it definition} and an
      optional {\it formal specification} (we present formal specification
      later in this section). The above goal is named \goal{MakeUp Water
      Provided When Loss Of Cooling}.
    
    \subsection {Domain properties and domain hypotheses}
    
      Unlike goals, {\it domain properties} are descriptive statements about
      the problem world (such as physical laws). A domain property is
      considered valid in the system-to-be. As seen later, domain properties
      are often involved in refinements of goals and obstacles. An example of
      domain property in our running example is the fact that an electrical
      pump cannot operate without electrical power.

      \startkaosspec
          \DomPropName {Power Needed For Motor Pump On}
          \KaosAttribute {Definition} {
            Electrical power is needed to turn the motor on.
        }
      \stopkaosspec
      
      Expectations, as seen in the previous section, are prescriptive
      assumptions on the environment. Descriptive assumptions, called {\it
      domain hypotheses}, may be needed as well. Domain hypotheses differ from
      domain properties as they are not necessarily always valid in the
      system\emdash{}but are assumed, by the analyst, to be true. For example,
      the fact that a pump might not be turned on when its internal electrical
      relay is broken is captured as a domain hypothesis:

      \startkaosspec
          \DomHypName {No Electrical Failure For Motor Pump On}
          \KaosAttribute {Definition} {
            The pump motor being on assumes no electrical failure.
        }
      \stopkaosspec
    
      A behavioral goal must obviously be consistent with all known domain
      properties, that is,
    
      \startformula
      \{G, Dom\} \vDash\not false    \hskip1.5cm\text{(domain-consistency)}
      \stopformula
        
    \subsection {Specifying goals in Metric Linear Temporal Logic}
    
      Metric Linear Temporal Logic (M-LTL) \cite[Koy92,Man92] may be used for
      formalizing behavioral goals to enable their analysis. The goals then
      take the general form
    
      \startformula
        C \Rightarrow \Theta T
      \stopformula
    
      where $\Theta$ represents a LTL operator such as: $\ltlX$ (in the next
      state), $\ltlF$ (sometimes in the future), $\ltlF_{\leq d}$ (sometimes in
      the future before deadline $d$), $\ltlG$ (always in the future),
      $\ltlG_{\leq d}$ (always in the future up to deadline $d$), $\ltlW$
      (always in the future unless), $\ltlU$ (always in the future until), and
      where $C \Rightarrow \Theta T$ means $\ltlG(C \rightarrow \Theta T)$. The
      following standard logical connectives are used: $\wedge$ (and), $\vee$
      (or), $\neg$ (not), $\rightarrow$ (implies), $\leftrightarrow$
      (equivalent).
      
      A behavioral goal can be of type {\it Achieve} or {\it Maintain/Avoid}
      \cite[Lam09]. The specification pattern for an {\it Achieve} goal is:
      
      \startformula
        \text{{\ss {\bf if} C {\bf then sooner-or-later} T},}\hskip1em\text{that is,}\hskip1emC \Rightarrow \ltlF T,
      \stopformula
        
      where $C$ denotes a current condition and $T$ a target condition, with
      obvious particularizations to {\it Immediate Achieve} goals ($C
      \Rightarrow \ltlX T$) and {\it Bounded Achieve} goals ($C \Rightarrow
      \ltlF_{\leq d} T$) \cite[Lam09].
    
      \noindent The specification pattern for a {\it Maintain} goal is:
      
      \startformula
        \text{{\ss {\bf always} [{\bf if} C {\bf then}] G},}\hskip1em\text{that is,}\hskip1em\ltlG\ ([C \Rightarrow]\ G);
      \stopformula
        
      the specification pattern for an {\it Avoid} goal is:
        
      \startformula
        \text{{\ss {\bf never} [{\bf if} C {\bf then}] B},}\hskip1em\text{that is,}\hskip1em\ltlG\ ([C \Rightarrow]\ \neg B),
      \stopformula
        
      where $G$ and $B$ denote a good condition and a bad condition,
      respectively. (Brackets are used here to delimit optional parts in those
      patterns.)
    
    \subsection {Modelling goals as AND/OR graphs}
    
      A goal model is an AND/OR graph showing how goals contribute positively
      or negatively to each other \cite[Gio03,Lam09]. Parent goals are obtained
      by abstraction, e.g., through {\it why} questions, whereas child goals
      are obtained by refinement, e.g., through {\it how} questions. Refinement
      paths connect parent goal nodes in this graph to their descendant goal
      nodes, possibly together with domain properties and hypotheses. Leaf
      goals are assigned to single system agents as {\it software requirement}
      or {\it environment expectation}. Graphically, goals are represented by
      parallelograms, domain properties by \quote{home} shapes and agents by
      hexagons; see the goal model fragment in
      \in{Figure}[fig:background_make_up_water_provided].

      \placefigure[]
          [fig:background_make_up_water_provided]
          {Refinement of \goal{Achieve [Make Up Water Provided When Loss Of Cooling]}.}
        {\externalfigure[../images/chap4/make_up_water_provided.pdf]}

      Refinement patterns are available to help building goal models. They
      encode different refinement tactics, e.g., the {\it Milestone-Driven},
      {\it Case-Driven}, {\it Guard-Introduction}, {\it
      Unmonitorability-Driven}, {\it Uncontrollability-Driven}, or {\it
      Divide-and-Conquer} refinement patterns \cite[Dar96,Lam09].
    
      In \in{Figure}[fig:background_make_up_water_provided], the top refinement
      is {\it Milestone-Driven} (with {\ss Make Up Water Requested} as the
      milestone condition). The next left refinement follows the {\it
      Milestone-Driven} refinement pattern (with {\ss Alarm Raised} as the
      milestone condition). The right refinement follows the {\it
      Divide-and-Conquer} refinement pattern.
      
      \noindent The top goal is more precisely specified as follows:
    
      \startformula
        LossOfCooling \Rightarrow \ltlF_{\leq 55h} MakeUpWaterProvided
      \stopformula    
    
      Its two subgoals are obtained by application of a formalized version of
      the {\it Milestone-Driven} refinement pattern with $MakeUpWaterRequested$
      as milestone condition:
    
      \startformula\startalign[n=1]
        \NC LossOfCooling \Rightarrow \ltlF_{\leq 5h} MakeUpWaterRequested \NR 
        \NC MakeUpWaterRequested \Rightarrow \ltlF_{\leq 50h} MakeUpWaterProvided, \NR 
      \stopalign\stopformula
      
      \noindent where $h$ denotes the \quote{hour} time unit.
    
      AND-refinements in a goal model should be {\it correct}, that is, complete,
      consistent and ideally minimal \cite[Lam09].
 
      \startitemize
    
        \item A refinement is {\it complete} if the satisfaction of all
        subgoals is sufficient for the satisfaction of the parent goal in view of 
        known domain properties:
    
        \startformula
            \{SG_1, SG_2, ..., SG_n, Dom\} \vDash PG    \hskip2cm\text{(complete refinement)}
        \stopformula
    
        \item A refinement is {\it consistent} if no subgoal contradicts other
        subgoals in the domain:

        \startformula
            \{SG_1, SG_2, ..., SG_n, Dom\} \vDash\not false \hskip2cm\text{(consistent refinement)}
        \stopformula
    
        \item A refinement is {\it minimal} if all the subgoals are needed for
        the satisfaction of the parent goal:

        \startformula
            \{\bigwedge_{j\neq i}SG_j, Dom\} \vDash\not G \hskip1em 
            (1 \leq i \leq n)  \hskip1cm\text{(minimal refinement)}
        \stopformula
    
      \stopitemize
    
      A goal refinement obtained by instantiation of a refinement pattern is
      formally guaranteed to be complete, consistent and minimal
      \cite[Dar95,Lam09]; the correctness proof is done once for all on the
      temporal logic formalization of the pattern. The partial goal model in
      \in{Figure}[fig:background_make_up_water_provided] shows three
      AND-refinements that are complete, consistent, and minimal.
        
    \subsection[sec:goal_conflict]{Goal conflicts}
    
      The techniques for assessing and controlling probabilistic obstacles in
      \in{Chapter}[chap:assessing] and \in{Chapter}[chap:controlling_obstacle] are
      easily extendable for assessing and resolving probabilistic conflicts.
  
      Goals are potentially conflicting (or divergent) if there exists a
      satisfiable and non-trivial boundary condition $B$ making them logically
      inconsistent in the domain \cite[Lam09], that is:

      \startformula\startalign[n=1,align={left}]
          \NC \{ B, SG_1, SG_2, ..., SG_n, Dom \} \vDash false,\hskip2cm\text{(conflict)}\NR
        \NC \{ B, Dom \} \vDash\not false         \NR
      \stopalign\stopformula
  
      As the last condition indicates, the condition $B \wedge Dom$ has to be
      realizable by the environment.
        
    \subsection[sec:soft_goal]{Soft goals as optimisation criteria}
    
      The selection of most appropriate countermeasures requires to
      comparatively evaluate alternative designs. Soft goals document the
      criteria for such comparison.
    
      In constrast with behavioral goals, soft goals are goals that are not
      satisfied in a clear-cut sense \cite[Myl92,Myl99]\emdash{}e.g.,
      \goal{User interface usable} or \goal{Fast computation}. Soft goal are
      more or less satisfied depending on alternative subgoal combinations
      contributing positively or negatively to the soft goal.
      
      Soft goals are used as optimisation criteria to enable the comparison and
      selection alternative system designs \cite[Let02,Lam09,Hea11]. Typically,
      they are used in minimization/maximization functions to be optimized by
      the system-to-be, e.g., \goal{Maximize[Computation Speed]} or
      \goal{Minimize[Energy consumption]}. 
    
  \stopsection
  
    \startsection[reference=sec:background_obstacle,title={Anticipating what could go wrong}]
  
    In goal-oriented modeling frameworks, obstacles were introduced as a
    natural abstraction for risk analysis \cite[Pot95,Ant98,Lam00]. This
    section defines obstacles, how they relate to goals and other obstacles. It
    also briefly reviews techniques for identifying, assessing and controling
    obstacles preventing the system from meeting its goals.
    
    \subsection {What are obstacles?}

      An {\it obstacle} to a goal is a domain-satisfiable precondition for the
      non-satisfaction of this goal \cite[Lam00,Lam09]:
      
      \startformula\startalign[n=2,align={left,left}]
        \NC \{O, Dom\} \vDash \neg G \hskip2cm \NC \text{(obstruction)}\NR
        \NC \{O, Dom\} \vDash\not false \NC \text{(domain-feasibility)}\NR
      \stopalign\stopformula
      
      For our spent fuel pool system, an obstacle to the goal
      \goal{Achieve [Make Up Pump Motor On When Water Requested]} is
      \obstacle{No Power Available}:
      
      \startkaosspec
          \ObstacleName {No Power Available}
          \KaosAttribute {Definition} {
            No electrical power is available at the pump.
        }
      \stopkaosspec
      
      \noindent This obstacle satisfies both the obstruction condition (if
      there is no power, the motor pump cannot be on) and the
      domain-feasibility condition (it might be possible that no power is
      available.)
      
      As a {\it consequence} of an obstacle occuring, some goals and ancestor
      goals are obstructed. As in risk analysis, an obstacle is more or less
      {\it likely} and its consequences or more or less {\it severe}. A likely
      obstacle with severe consequence is said to be {\it critical}.
      
      Each obstacle has a {\it name}, a natural language {\it definition} and
      an optional {\it formal specification}. The above obstacle's name is
      \obstacle{No Power Available}.
        
    \subsection {Specifying obstacles in Metric Linear Temporal Logic}
    
      Similarly to goals, obstacles may be formalized using Metric Linear
      Temporal Logic (M-LTL) \cite[Koy92,Man92]. An obstacle to a goal of form
      $C \Rightarrow \Theta T$ takes the general form
    
      \startformula
        \ltlF (C \wedge \neg \Theta T)
      \stopformula
    
      where $\Theta$ represents a LTL operator. The general form $\ltlF(C
      \wedge \Theta OC)$ is often used where $OC$ denotes the {\it obstacle
      condition}.
      
      An obstacle to an {\it Achieve} goal has the specification pattern:
      
      \startformula
        \text{{\ss {\bf sooner-or-later} C {\bf and never} T},}\hskip1em\text{that is,}\hskip1em\ltlF(C \wedge \ltlG \neg T),
      \stopformula
      
      where $C$ denotes a current condition and $T$ a target condition
      \cite[Lam09]. An obstacle to a {\it Maintain} goal has the specification
      pattern:
      
      \startformula
        \text{{\ss {\bf sooner-or-later} [C {\bf and}] {\bf sooner-or-later not} G},}\hskip1em\text{that is,}\hskip1em\ltlF([C \wedge]\ \neg G)
      \stopformula
       
      whereas an obstacle to an {\it Avoid} goal has the specification pattern:
      
      \startformula
        \text{{\ss {\bf sooner-or-later} [C {\bf and}] {\bf sooner-or-later} B},}\hskip1em\text{that is,}\hskip1em\ltlF([C \wedge]\ B)
      \stopformula
       
      where $G$ and $B$ denote a good condition and a bad condition,
      respectively.
    
    \startsubsection[title={Modelling obstacles as AND/OR graph}]
    
      Similarly to goals, obstacles can be AND/OR refined into subobstacles,
      resulting in a goal-anchored form of a risk refinement tree \cite[Lam09].
      In such obstacle tree,
    
      \startitemize[packed]
    
        \item the root obstacle is the negation of the associated leaf goal in
        the goal model; for a root obstacle $RO$ to a leaf goal $LG$, we have:
        
        \startformula
          \{RO, Dom\} \vDash \neg LG\hskip2cm\text{(obstruction)}
        \stopformula
    
        \item an AND-refinement captures a combination of subobstacles
        entailing the parent obstacle;
    
        \item an OR-refinement captures alternative ways of entailing the
        parent obstacle—and, recursively, of obstructing the corresponding leaf
        goal;
    
        \item the {\it leaf obstacles} are single, fine-grained obstacles whose
        likelihood can be easily estimated.
    
      \stopitemize
    
      AND-refinements are ideally disjoint. This condition is useful for
      exploring specific resolutions to distinct non-overlapping causes.
    
      Considering $n$ OR-Refinements to a parent obstacle $PO$, each with a
      single subobstacle $SO_i$ (for $1 \leq i \leq n$), the conditions above
      can be formally stated as follows \cite[Lam09]. (The generalization to
      more subobstacles in a given OR-Refinement is straightforward.)
      
      \startitemize
      
      \item The subobstacle $SO_i$ in an OR-refinement must entail the parent
      obstacle:
    
      \startformula
        \{ SO_i, Dom \} \vDash PO \hskip2cm\text{(entailment)}
      \stopformula
        
      \item OR-refinements should ideally be domain complete and disjoint:
      
      \startformula\startalign[n=2,align={left,left}]
        \NC \{\neg SO_1, ..., \neg SO_n, Dom\} \vDash \neg PO \hskip2cm \NC \text{(domain-completeness)}\NR
        \NC \text{for all } i \neq j, \{SO_i, SO_j, Dom\} \vDash false \hskip1cm\NC \text{(disjointness)}\NR
      \stopalign\stopformula
      
      \stopitemize
    
    \subsection[sec:obstacle_analysis]{Obstacle analysis}
    
    Obstacle analysis consists of anticipating the conditions under which goals
    from the goal model might not be satisfied. Its main objective is to
    increase requirements completeness through countemeasure goals enriching
    the goal model. It consists of three steps \cite[Lam00]:
    
    \startitemize[a]
    
      \item {\it Obstacle Identification:} as many obstacles as possible to
      every leaf goal in the goal refinement graph should be identified from
      relevant domain properties.
    
      \item {\it Obstacle Assessment:} the likelihood and severity of each
      obstacle should be determined.
    
      \item {\it Obstacle Resolution:} likely and critical obstacles should be
      resolved through appropriate countermeasures to be integrated into the
      goal model.
    
    \stopitemize
    
    \noindent A new cycle of obstacle analysis may then be performed on the
    countermeasure goals. 
    
    \noindent {\bf Obstacle Identification.} Formal and heuristic techniques
    are available for the identification of obstacles
    \cite[Ant98,Sut98b,Lam00,Lam09,Alr12].
    
    In particular, for the {\it Achieve} and {\it Maintain/Avoid} goal
    specification patterns introduced in \in{Section}[sec:background_goal],
    we may look for specific domain properties taking the form:
      
    \startformula
      \text{{\ss {\bf always} {\bf if} Q {\bf then} N},}\hskip1em\text{that is,}\hskip1emQ\Rightarrow N,
    \stopformula
    
    where $N$ denotes a necessary condition for the consequent $Q$ of a leaf
    goal $P\Rightarrow \Theta Q$; $Q$ is the target condition of an {\it
    Achieve} goal $C\Rightarrow \ltlF Q$ or the good condition of a {\it
    Maintain} goal $C\Rightarrow Q$. Such domain properties are equivalent to
    $\neg N \Rightarrow \neg Q$ (by contraposition). A one-step regresion
    through them of the goal negations $\ltlF (C\wedge \ltlG \neg Q)$
    (respectively $\ltlF (C\wedge \neg Q)$) yields the obstacle:
    
    \startformula
      \text{{\ss {\bf sooner-or-later} (C {\bf and never} N)},}\hskip1em\text{that is,}\hskip1em\ltlF(C \wedge \ltlG\neg N),
    \stopformula
      
    for an {\it Achieve} goal, or 
    
    \startformula
      \text{{\ss {\bf sooner-or-later} (C {\bf and not} N)},}\hskip1em\text{that is,}\hskip1em\ltlF(C \wedge \neg N),
    \stopformula
      
    for a {\it Maintain} goal. See \cite[Lam09] for details.
  
    Consider the goal \goal{Achieve [Make Up Pump Motor On When Water
    Requested]} in \in{Figure}[fig:background_make_up_water_provided] whose
    target condition is {\ss PumpMotorOn}:
    
    \startformula
      WaterRequested \Rightarrow \ltlF_{\leq 5m} PumpMotorOn
    \stopformula
    
    Negating this goal yields the root obstacle: 
    
    \startformula
      \ltlF (WaterRequested \wedge \ltlG_{\geq 5m} \neg PumpMotorOn)
    \stopformula
        
    A necessary condition for the target is the following:
    
    \startformula
      PumpMotorOn \Rightarrow \neg PumpElectricalFailure
    \stopformula
    
    This yields the bottom left subobstacle in
    \in{Figure}[fig:background_make_up_water_pump_on], namely:
    
    \startformula
      \ltlF (WaterRequested \wedge \ltlG_{\geq 5m} PumpElectricalFailure)
    \stopformula
        
    \placefigure[]
        [fig:background_make_up_water_pump_on]
        {Obstacles to \goal{Achieve [Make Up Pump Motor On When Water Requested]}.}
      {\externalfigure[../images/chap4/make_up_water_pump_on.pdf]}

    \noindent {\bf Obstacle Assessment.} This step is aimed at identifying the
    likely and critical obstacles to be resolved during the next step of
    obstacle control. Very little work is available regarding obstacle
    assessment; \in{Chapter}[chap:assessing] will detail our contribution to
    this step.
    
    \noindent {\bf Obstacle Control.} The identified obstacles, if likely and
    critical, shall be resolved through appropriate countermeasures to be
    integrated in the goal model. The countermeasure goals needs to be refined
    in turn until their descendants are assignable to single agents.
    Alternative countermeasures are to be identified; \quote{Most appropriate}
    ones need then to be selected according to a variety of criteria such as
    the number of obstacles resolved, the likelihood and criticality of
    resolved obstacles, the countermeasure cost, etc. \cite[Lam09]. The
    selected countermeasures are then deployed at RE time or deferred until
    runtime.
         
    Formal and heuristic techniques are
    available for the identification of alternative countermeasure goals
    \cite[Lam00,Alr16]. In particular, a variety of tactics are available for exploring
    alternative ways of resolving obstacles to a goal such as:
    
    \startitemize
    
      \item {\it Avoid obstacle} (adds a new goal requiring the obstacle to be
      avoided),
    
      \item {\it Reduce obstacle likelihood} (adds a new goal that reduces the
      likelihood of occurence for the obstacle),
    
      \item {\it Mitigate obstacle} (adds a new goal to mitigate the
      consequence of the obstacle, ensuring a weakened version of the
      obstructed goal or of one of its ancestor),
    
      \item {\it Weaken goal} (weakens the obstructed goal so that the
      obstruction no longer occurs),
    
      \item {\it Substitute goal} (replace the obstructed goal by a new goal,
      not obstructed by the obstacle under consideration),
    
      \item {\it Restore goal} (adds a new goal to ensure that the obstructed
      goal is satisfied in a reasonable future), or
    
      \item {\it Substitute agent} (replace the agent responsible for the
      obstructed goal so that the obstacle is no longer feasible)
    
    \stopitemize
    
    \noindent See \cite[Lam00,Lam09] for details.
    
    For example, the obstacle mitigation tactic applied to the preceding
    obstacle \obstacle{Pump Electrical Failure} generates the following
    countermeasure goal:
    
    \startkaosspec
        \GoalName {Achieve [Redundant Pump Motor On When Primary Pump Failure]}
        \KaosAttribute {FormalDef} {
          $WaterRequested \wedge PumpElectricalFailure$\blank[none]
          \hskip2.5cm$\Rightarrow \ltlF_{\leq 5m} RedundantPumpMotorOn$
      }
    \stopkaosspec 
    
    \noindent This countermeasure goal ensures the ancestor goal \goal{Achieve
    [Make Up Water Provided When Requested]} in
    \in{Figure}[fig:background_make_up_water_provided] is guaranteed even when
    the primary pump fails. In this specific case, the mitigation does not
    require a weaking of the ancestor goal.
    
    Very little work is available on countermeasure selection and integration;
    \in{Chapter}[chap:controlling_obstacle] and \in{Chapter}[runtime] will
    propose obstacle resolution techniques both at RE time and runtime.
    
  \stopsection
  
  \startsection[title={Summary}]
  
    This chapter introduced the KAOS goal-oriented requirement engineering
    framework together with its obstacle analysis component. Next chapter will
    introduce a probabilistic extension of this framework on which the obstacle
    assessment and control techniques presented in the following chapters will
    rely.
  
  \stopsection

\stopcomponent
