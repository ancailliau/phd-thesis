% !TEX root = thesis.tex

\startcomponent chap-0
\product thesis

\title{Software Requirements Engineering:\\ A Risk-driven Approach}

Requirements Engineering (RE) is the branch of Software Engineering concerned with the elicitation, evaluation, specification, analysis and evolution of the objectives, functionalities, qualities and constraints of a software-intensive system. Our natural inclination to conceive idealized systems often results in lack of anticipation of exceptional conditions. Risk analysis aims at identifying, assessing and controlling these conditions.

Obstacle analysis was introduced as a goal-oriented form of risk analysis; an obstacle to a goal is a precondition for the non-satisfaction for this goal. The presentation will provide an overview of our techniques for obstacle identification, assessment and control. In particular, we will focus on our probabilistic framework grounded on system-specific phenomena. 

We extended the goal and obstacle specification language with a probabilistic layer enabling the identification and control of most critical and likely obstacles. In our framework, experts estimate fine-grained obstacles, with their uncertainty margins. Theses estimates are propagated through the obstacle and goal model. As a result, we highlight the most critical and likely obstacles. Countermeasures to these obstacles are identified and integrated into the model for more robust systems. 

\title{Table of content}

\start
\setupinterlinespace[line=1.5ex]
\placecontent
\stop
			
\stopcomponent