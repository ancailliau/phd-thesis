% !TEX root = thesis.tex

\startcomponent chap-9
\environment common
\product thesis

\chapter[chap:evaluation] {Evaluation}

  This chapter illustrates the techniques proposed in the previous chapters
  while showing them in action on real case-studies of significant size. Our
  techniques are evaluated according to five hierarchical criteria. Each
  criterion is a prerequisite for the next \cite[Dam14,Let14]. The criteria are
  the following:
  
  \startitemize
    
    \item {\it Correctness}: Are the proposed techniques correct with regard to
    their specifications?
  
    \item {\it Performance and Scalability}: Are the techniques efficient for
    realistically-sized problems?
  
    \item {\it Applicability}: Are the techniques applicable to real-world
    problems?
  
    \item {\it Utility}: Are the techniques solving a real problem?
  
    \item {\it Usability}: Are the techniques usable by risk analysts and
    requirements engineers?
  
  \stopitemize
  
  \noindent These criteria here refer to the techniques presented in the
  thesis, namely:
  
  \startitemize
  
    \item the propagation procedures as seen in \in{Section}[sec:computing_bdd]
    and \in{Section}[sec:pattern_based_computation];
    
    \item the procedure for identifying critical obstacles as seen in
    \in{Section}[sec:indentifying_most_critical_obstacles] and
    \in{Section}[sec:finding_critical_obstacles_uncertainty];
    
    \item the selection of most appropriate countermeasures as seen in
    \in{Section}[sec:selecting_cm];
    
    \item the integration of countermeasures into the goal model as seen in
    \in{Section}[sec:integrating];
    
    \item the assessment of leaf obstacles together with uncertainty margins as
    seen in \in{Section}[sec:assessment_k_uncertainty];
    
    \item the runtime monitoring of probabilistic obstacles as seen in
    \in{Section}[sec:obstaclebased_adaptation].
  
  \stopitemize
  
  As for applicability, the case studies considered for application are real,
  mission-critical systems with non-trivial requirements and significant risks
  that could prevent their correct operation. They includes a Car Pooling
  System, an industrial Yoke Lifting System, and an Ambulance Dispatch System.
  The case-studies are
  
  The chapter is structured as follows. \in{Section}[sec:correctness] discusses
  the correctness of our techniques. \in{Section}[sec:performance] briefly
  presents the complexities and the performance of the proposed approach.
  \in{Section}[sec:applicability] describes the three case-studies, evaluates
  the applied techniques regarding their specific objectives, and briefly
  discusses the applicability of the techniques. \in{Section}[sec:utility]
  discusses the extent to which our techniques solve a real problem.
  \in{Section}[sec:usability] shows how our techniques are usable by risk
  analysts and requirements engineers.

  \startsection[reference=sec:correctness,title={Correctness}]
    
    This section shows that the techniques proposed in
    \in{Chapters}[chap:assessing]-\in{}[runtime] meet their specification.
      
    In particular, the propagation procedures in \in{Chapter}[chap:assessing]
    compute \quote{safe} estimates, that is, lower bound estimates. Otherwise,
    the computed goal satisfaction rates might overestimate the actual one. As
    a consequence, the goal would be stated to meet its required satisfaction
    rate even if does not.
      
    \startproposition
      
      In a model with complete, consistent, and minimal goal refinements and
      with domain-complete obstacle refinements, the satisfaction rate of the
      considered high-level goals is greater or equal to the estimated
      satisfaction rate computed by the propagation procedures in
      \in{Chapter}[chap:assessing].
      
    \stopproposition
      
    \startproof 
    
      (Ab absurdum.) Assume a high-level goal and its lowest descendant $G$
      whose actual satisfaction rate is lower than the computed satisfaction
      rate. The latter goal is either refined or is leaf goal.
      
      \startitemize

        \item If goal $G$ is refined, its subgoals have an actual satisfaction
        rate greater or equal to their respective computed satisfaction rate.
        If not, $G$ is not the lowest descendant as there exists a subgoal with
        an actual satisfaction rate lower than its respective computed
        satisfaction rate.
        
        \startitemize
        
          \item In the pattern-based computation, the satisfaction rate is
          computed using a specific propagation equation. Assuming that the
          satisfaction rates for the subgoals are correct, if the estimated
          satisfaction rate is lower than the computed satisfaction rate, then
          at least a term in the generic propagation equation was not taken
          into account; the appropriate specific propagation equation was
          therefore not applied.
        
          \item In the BDD-based computation, the satisfaction rate would be
          lower than the computed one if and only if the obstruction supersets
          include fewer obstacles; however, by construction, all obstacles are
          included; there is no unidentified obstacles as the model is assumed
          to be domain-complete.
        
        \stopitemize
        
        \item If goal $G$ is a leaf goal, its satisfaction rate is lower than
        the estimated one. The actual satisfaction rate for the root obstacle
        is then higher than the computed satisfaction rate. However, the
        computed satisfaction rate of the root obstacle might be higher than
        the actual one if and only if obstacles are missing. If obstacles are
        missing the model is not domain-complete.%
      %
      \stopitemize%
      \vskip\dimexpr(-2\lineheight-2pt)
    \stopproof 
      
    \startproposition
      
      The obstacles returned by the critical obstacle identification procedure
      cause a loss of satisfaction for the specified goal.
      
    \stopproposition
      
    \startproof 
      
      By definition of violation severity (\in{Definition}[dfn:vs]), obstacles
      with a positive violation severity cause a loss in the computed
      satisfaction rate of the high-level goal. The critical obstacle
      identification procedure presented in
      \in{Section}[sec:indentifying_most_critical_obstacles] returns all
      obstacles with positive violation severity.
      %
    \stopproof 
      
    \startproposition
      
      Any countermeasure not returned by the procedure for selection most
      appropriate countermeasures costs more or would result in a lower computed
      satisfaction rate for the considered high-level goal.
      
    \stopproposition
      
    \startproof 
    
      If a countermeasure not returned by the procedure would cost less, its
      cost is lower than the minimal one returned by the first optimization
      discussed in \in{Section}[sec:selecting_cm]. The minimal cost is
      therefore not minimal. If this countermeasure results in a higher
      computed satisfaction rate for the considered high-level goal, the second
      optimization would not return the highest solution.
      %
    \stopproof 
      
    \startproposition
      
      The integration of a countermeasure in a complete, consistent and minimal
      model produces a complete, consistent and minimal model.
      
    \stopproposition
      
    \startproof 
    
      This is trivially enforced by the {\it correctness preservation}
      property, as seen in \in{Section}[sec:integrating].
      %
    \stopproof 
      
    As for our monitoring-based techniques, we want the monitored satisfaction
    rate to be \quote{safe}, that is, the monitored satisfaction rate for the
    leaf obstacles must be an upper bound. Otherwise, the monitored
    satisfaction rate might underestimate the actual satisfaction rate, while
    would lead to unsafe conclusions.
      
    \startproposition
      
      For a given state decomposition, the monitored satisfaction rate for a
      leaf obstacle is lower or equal to the actual satisfaction rate obtained
      by the monitoring procedure in \in{Section}[sec:monitoring].
      
    \stopproposition
      
    \startproof 
      
      (Ab absurdum.) By definition, the monitored satisfaction rate for the
      formula $\ltlF (C \wedge \Theta OC)$ is the highest state probability to
      satisfy the obstacle condition $C \wedge \Theta OC$. If the actual
      satisfaction rate for a leaf obstacle is greater than the monitored
      satisfaction rate, there is a state probability greater than the
      monitored statisfaction rate; the monitored satisfaction rate is
      therefore not the highest state probability.
      %
    \stopproof 
    
    \noindent {\it Note.}\hskip1em As seen in \in{Section}[sec:monitoring], the
    monitoring procedure relies on observed states. However, the same system
    might be observed with a different state decomposition. The state
    decomposition might impact state probabilities. For example, consider the
    flooding system, and assume two states $RiverHigh$ and $RiverLow$. Both
    have a state probability to satisfy the obstacle \obstacle{Dusty
    Environment}, say $p_{high}$ and $p_{low}$. A decomposition with more
    states or with fewer states changes the values of state probabilities;
    this impacts the monitored satisfaction rate of obstacles and goals.
      
    Consider a decomposition with one state {\it RiverLowHigh} instead of two.
    As a decomposition forms a partition over all possible states, this single
    state is a combination of the two states in the previous decomposition. The
    state probability $p_{low/high}$ for the single state is a combination of
    the two state probabilities:

    \startformula
      p_{low/high} = t_{high} \times p_{high} + t_{low} \times p_{low},
    \stopformula
    
    where $t_{high}$ and $t_{low}$ represent the fraction of time spent in the
    corresponding state, respectively. However, given that the system might be
    in one of the two states but cannot be in both states, $t_{high} + t_{low}
    = 1$. No combination of the probabilities $p_{high}$ and $p_{low}$ can be
    higher than $p_{high}$ or $p_{low}$. The state probability $p_{low/high}$
    for the single state is therefore at least smaller than the state
    probabilities for the two states, i.e.,
      
    \startformula
      p_{low/high} \leq p_{high} \hskip1.5cm p_{low/high} \leq p_{low}.
    \stopalign\stopformula
          
    Instead of a single state, consider now a decomposition with three states.
    Reversing the above reasoning, it is possible that one of the states in
    this decomposition has a state probability higher than the state
    probabilities $p_{high}$ or $p_{low}$. The satisfaction rate of the leaf
    goal would then be higher, leading our technique to underestimate the
    satisfaction rate of an obstacle. This limitation calls for a finer-grained
    analysis of monitored satisfaction rates.
    
  \stopsection

  \startsection[reference=sec:performance,title={Performance and scalability}]
    
    This section discusses the theoretical complexity of the procedures for
    propagating satisfaction rate (\in{Section}[sec:computing_satrate]), for
    indentifying critical obstacles
    (\in{Section}[sec:indentifying_most_critical_obstacles]), for selecting
    most appropriate countermeasures (\in{Section}[sec:selecting_cm]), for
    integrating selected countermeasures into the goal model
    (\in{Section}[sec:integrating]), and for monitoring probabilistic obstacles
    (\in{Section}[sec:monitoring]), together with their performance in
    realistic problems.
      
    \noindent {\bf Propagation procedures.} The {\it Pattern-based computation}
    performs at most $g + o$ computations where $g$ is the number of goals and
    $o$ the number of obstacles. The complexity of this procedure is therefore
    $\cal O(g + o)$.

    The {\it BDD-based computation} performs at most $r_g + o_b + r_o$ BDD
    operations where $r_g$ is the number of goal refinements, $o_b$ is the
    number of obstructions, and $r_o$ is the number of obstacle refinements.
    The complexity of a BDD operation is linear in the size of the combined
    BDDs \cite[Mei12]. However, the size of the BDDs might be exponential in
    the number of variables \cite[Mei12]. The size of our BDDs is $\cal
    O(2^o)$, where $o$ is the number of obstacles. Computing the satisfaction
    rate of an obstruction superset is linear in the number of nodes of the
    BDD. Therefore, the complexity of the BDD-based computation is bounded by
    $\cal O((r_g + o_b + r_o) \times 2^o + 2^o)$. In practice, however, BDDs
    often have a much smaller size than $\cal O(2^o)$. Given the specific
    structure of the encoded formula, we may expect tighter bounds on the size
    of the BDD.
      
    In practice, the computation cost of the satisfaction rate for the
    considered high-level goals is often small. For example, computing the
    satisfaction rate for the high-level goal in the Ambulance Dispatch System
    and in the Car Pooling System discussed in \in{Sections}[sec:bads] and
    \in{}[sec:cool] hereafter took less than a second. The timing includes the
    time to start the .NET virtual machine, read, and parse the model. As seen
    in \in{Section}[sec:compare_propagations], the time to compute the
    satisfaction rates of high-level goals on large, randomly generated
    goal/obstacle models is small. As a recall, on models with 10.000 goals and
    10.000 obstacles, it takes 13s for the Pattern-based computation and 18s
    for the BDD-based computation (including 5ms for computing the probability
    of an obstruction superset.)
      
    When computing satisfaction rates with their uncertainty margins, the
    propagation procedure is repeated. The complexity of the repeated procedure
    is bounded by $\cal O(n\times2^o)$ where $n$ is the number of samples
    computed. In practice, a value such as $n = 1,000,000$ provides
    sufficiently accurate satisfaction rates in less than 5 seconds for the
    Ambulance Dispatching System in \in{Sections}[sec:bads]. Note that on a
    model with 10.000 goals and 10.000 obstacles, it would take about 1h20m;
    this would call for specific techniques to be developped event though such
    model size appears questionable in practice.
      
    \noindent {\bf Identification of critical obstacles.} To identify likely
    and critical obstacles, the procedure in \in{Section}
    [sec:indentifying_most_critical_obstacles] generates all combinations of
    size $1$, then $2$, etc. The complexity for such identification of obstacle
    combinations of size $i$ is $\cal O(o^i)$, where $o$ is the number of
    obstacles.
      
    In practice, the identification of the critical obstacles and pairs of
    obstacles for the Car Pooling System in \in{Section}[sec:cool] took about a
    second. The identification of critical triplets took about 3 seconds
    whereas the identification of 4-tuples took 30 seconds. Similar times were
    observed for the running example, for the Ambulance Dispatch System in
    \in{Section}[sec:bads] and for the C230 Yoke Lifting System in
    \in{Section}[sec:iba]. Again, these timing indications includes the time to
    start the .NET virtual machine and read and parse the models.
          
    \noindent {\bf Selection of most appropriate countermeasures.} The
    selection of most appropriate countermeasure solves two combinatorial
    optimization problems, as seen in \in{Section}[sec:selecting_cm]. The
    complexity of solving these problems is bounded by $\cal O(2^c)$ where $c$
    is the number of countermeasures. As discussed in
    \in{Section}[sec:selecting_cm], future improvements are expected to reduce
    the computation effort.
      
    In practice, selecting most appropriate countermeasures for the Ambulance
    Dispatching System in \in{Section}[sec:bads] took about 2 minutes.

    \noindent {\bf Integration of countermeasures.} Countermeasure integration
    using the constructs presented in \in{Section}[sec:exception_handling] is
    in $\cal O(1)$. However, changes may need to be propagated in the
    goal/obstacle model; this requires, in the worst case, $\cal O(g + o)$ {\it
    Provided} annotations.
      
    In practice, the time to integrate all countermeasures in the Ambulance
    Dispatching System \in{Section}[sec:bads] or in the running example took
    less than a second.
          
    \noindent {\bf Runtime monitoring of probabilistic obstacles.} As discussed
    in \in{Section}[sec:monitoring_based], the building of monitors is in $\cal
    O(2^{2^n})$ where $n$ is the size of the formula \cite[Bau11]. The
    complexity of updating all monitors at runtime is $\cal O(n)$ where $n$ is
    the number of virtual monitors.
      
    As suggested by the Ambulance Dispatch System case-study in
    \in{Section}[sec:bads] and the Flooding System in
    \in{Section}[sec:runtime_running_example] the approach appears to be
    applicable in practice. For the former, building the required monitors
    takes about 10 seconds. Updating the monitors at runtime takes less than 50
    ms, which allows monitoring to occur every second without any problem. The
    monitoring overhead appears very small and could be balanced on different
    computers.
    
  \stopsection
  
  \startsection[reference=sec:applicability,title={Applicability}]
    
    This section shows our proposed techniques in action in realistic
    situations. It describes three case-studies and the application of our
    techniques.
    
    \startitemize
    
      \item The first case study, a Car Pooling System (CPS), was chosen to evaluate
      the applicability of the proposed obstacle assessment techniques as seen
      in \in{Chapter}[chap:assessing]. An existing goal and obstacle models
      with a large number of obstacles calls for our obstacle prioritization
      techniques. (See \in{Section}[sec:cool].)
    
      \item The second case study evaluates the application of the obstacle
      assessment techniques presented in \in{Chapter}[chap:assessing] in an
      industrial setting. Our techniques were applied to an industrial yoke
      lifting system from a leading medical technology company and compared to
      a risk analysis previously performed by engineers of the company. (See
      \in{Section}[sec:iba].)
      
      \item Last, the techniques proposed in \in{Chapter}[chap:assessing] to
      \in{}[runtime] are applied to a benchmark commonly used for evaluating
      obstacle analysis techniques \cite[Fin96,Lam00,Let02,Alr12,Alr16] and
      other techniques such as self-adaptation \cite[Sou12]: an Ambulance
      Despatching System (ADS). This case study was chosen for the availability
      of existing goal and obstacles models, domain experts, and for the
      experience of the author in this type of systems. (See
      \in{Section}[sec:bads].)
    
    \stopitemize
    
    \startsubsection[reference=sec:cool,title={A car pooling system}]
    
      The techniques presented in \in{Chapter}[chap:assessing] were evaluated
      on a first case study of a car pooling system. This section focuses on
      evaluating the techniques in
      \in{Section}[sec:indentifying_most_critical_obstacles] for determining
      the likely and critical obstacles. The system is briefly described as
      follows.
      
      {\vskip.8\baselineskip}
      \framed[width=local,align={width},offset=10pt,after={\vskip-\baselineskip}
       ] {\it The system should act as a marketplace for drivers to offer empty
      seats in real time and for travellers to use them under agreed
      conditions. A driver is matched in real time with anyone searching for a
      ride along a common route. Effective carpooling may critically depend on
      marketplace size; the system should therefore be attractive to drivers,
      in particular by not over-constraining them. Drivers are assumed to have
      a GPS-based navigation device and a PDA/iPhone-like touch screen.}

      \noindent A mode detailed description of the system can be found in
      \cite[Dam10].
    
      \noindent {\bf Goal and Obstacle model.} The goal model for this system
      contains 44 goals with 2 root goals refined through 20 refinements
      leading to 24 leaf goals. The obstacle model contains 109 obstacles with
      12 root obstacles and 69 leaf obstacles. The model contains 108 obstacle
      refinements. The complete model is found in \in{Section}[app:cool].
      
      The root goals are \goal{Achieve [Ride Need Eventually Served]} and
      \goal{Achieve [Ride Proposal Eventually Served]}. Both have a required
      satisfaction rate of $95\%$.
      
      \startkaosspec
        \GoalName {Achieve [Ride Need Eventually Served]}
        \KaosAttribute {Def} {Passengers that want to go to a destination shall
        arrive at that destination within the specified time constraints.}
        \KaosAttribute {RSR} {$95\%$}
      \stopkaosspec
    
      \startkaosspec
        \GoalName {Achieve [Ride Proposal Eventually Served]}
        \KaosAttribute {Def} {Rides proposed by drivers shall be fulfilled
        within time constraints. A ride is fulfilled if the driver did not
        travel alone.}      
        \KaosAttribute {RSR} {$95\%$}
      \stopkaosspec
      
      The goal \goal{Achieve [Ride Need Eventually Served]} is refined in 3
      subgoals whereas \goal{Achieve [Ride Proposal Eventually Served]} is
      refined in two subgoals; both refinements follow the {\it
      milestone-driven} refinement pattern. \in{Figure}[fig:goal_cool1] and
      \in{Figure}[fig:goal_cool2] show the corresponding goal model fragments.
      The non-assigned goals are refined in turn using a variety of refinement
      patterns \cite[Lam09] until all goals are assigned to single agents. The
      system includes 3 agents: \agent{Driver}, \agent{Passenger} and
      \agent{Software}.
      
      \placefigure[]
           [fig:goal_cool1]
           {Goal model fragment for the Car Pooling System.}
        {\externalfigure[../images/chap9/cool_goal_1.pdf]}
      
      \placefigure[]
           [fig:goal_cool2]
           {Goal model fragment for the Car Pooling System.}
        {\externalfigure[../images/chap9/cool_goal_2.pdf]}
      
      The root obstacles were generated by negating the leaf goals from the
      goal model. For example, the goal \goal{Maintain [Ride Offer Accurate]}
      is obstructed by the obstacle \obstacle{Ride Offer Not Accurate}:
      
      \startkaosspec
        \GoalName {Maintain [Ride Offer Accurate]}
        \KaosAttribute {Def} {Ride offers proposed by the drivers shall
        always be as accurate as possible.}
      
        \ObstacleName {Ride Offer Not Accurate}
        \KaosAttribute {Def} {A ride offer encoded in the system does not
        accurately correspond to a ride proposal in the real world.}
      \stopkaosspec

      These obstacles are recursively refined until the satisfaction rate of
      the leaf obstacles can be estimated by experts using available techniques
      \cite[Lam09]. \in{Figure}[fig:cool_obstacle_1] shows a fragment of the
      obstacle model.
      
      \placefigure[]
           [fig:cool_obstacle_1]
           {Obstacle model fragment for the Car Pooling System.}
        {\externalfigure[../images/chap9/cool_obstacle_1.pdf]}
      
      \noindent {\bf Obstacle Assessment.} The 69 leaf obstacles were then
      annotated with estimates of their satisfaction rate. As they are grounded
      on the domain, such estimates can be elicited from users' experience or
      from statistical data. As the system provides specific features for ride
      evaluation by riders, the evaluation questionnaire might be designed for
      non-positive evaluations to reflect the obstacle model so as to acquire
      relevant data.

      \in{Table}[tab:cool_estimates] provides leaf obstacle estimates based on
      personal carpooling experience by colleagues and us. These estimates
      should be refined from statistical data when available. For example, for
      the obstacle \obstacle{Drop Point Inaccessible By Car} 5 behaviors at
      least were estimated to satisfy the obstacle condition out of 1,000
      possible behaviors, the satisfaction rate of this leaf obstacle is
      therefore estimated to be $.5\%$.
    
      \placetable[new][tab:cool_estimates]
        {Estimates for identified leaf obstacles.}
        {\setupTABLE[c][each][align={left,lohi},frame=off,offset=0pt]
        \setupTABLE[r][1][style=bold,bottomframe=on,boffset=4pt]
        \setupTABLE[c][1,3][align={right,lohi},width=6cm]
        \setupTABLE[c][2,4][alignmentcharacter={text->.},aligncharacter=yes,align={middle,lohi}]
        \setupTABLE[2,4][1][align={left,lohi},aligncharacter=no]
        \setupTABLE[c][2][roffset=4pt]
        \setupTABLE[c][3][loffset=4pt]
        \setupTABLE[r][2][toffset=4pt]
        \switchtobodyfont[script]
        \bTABLE
\bTR\bTD Obstacle \eTD\bTD SatRate \eTD\bTD Obstacle \eTD\bTD SatRate \eTD\eTR
\bTR\bTD Fake Ride Offer\eTD\bTD 0.1\% \eTD
\bTD Ride Offer Outdated\eTD\bTD 1.0\% \eTD\eTR
\bTR\bTD Fake Ride Request\eTD\bTD 0.1\% \eTD
\bTD Ride Request Out-dated\eTD\bTD 1.0\% \eTD\eTR
\bTR\bTD Departure Date/Time Miscoded\eTD\bTD 0.9\% \eTD
\bTD Departure Point Miscoded\eTD\bTD 1.5\% \eTD\eTR
\bTR\bTD Arrival Date/Time Miscoded\eTD\bTD 0.9\% \eTD
\bTD Arrival Point Miscoded\eTD\bTD 1.5\% \eTD\eTR
\bTR\bTD No Request Matching For That Time\eTD\bTD 2.0\% \eTD
\bTD No Request Matching For That Position\eTD\bTD 1.0\% \eTD\eTR
\bTR\bTD No Offer Matching For That Time\eTD\bTD 2.0\% \eTD
\bTD No Offer Matching For That Position\eTD\bTD 1.0\% \eTD\eTR
\bTR\bTD Ride Offer Cancelled And Ride List Proposed\eTD\bTD 2.0\% \eTD
\bTD Ride Request Cancelled And Ride List Proposed\eTD\bTD 2.0\% \eTD\eTR
\bTR\bTD Ride List Not Convenient\eTD\bTD 1.0\% \eTD
\bTD Ride Offer Modified\eTD\bTD 1.0\% \eTD\eTR
\bTR\bTD Ride Request Modified\eTD\bTD 1.0\% \eTD
\bTD Ride Offer Cancelled And Selected\eTD\bTD 1.0\% \eTD\eTR
\bTR\bTD Ride Request Cancelled And Selected\eTD\bTD 1.0\% \eTD
\bTD Ride Offer Cancelled And Elected\eTD\bTD 1.0\% \eTD\eTR
\bTR\bTD Ride Request Cancelled And Elected\eTD\bTD 1.0\% \eTD
\bTD No Pickup Point Near Departure Point\eTD\bTD 0.5\% \eTD\eTR
\bTR\bTD Detour Too Important For Reaching Pickup Point\eTD\bTD 1.0\% \eTD
\bTD No Pickup Point Accessible To Driver\eTD\bTD 0.5\% \eTD\eTR
\bTR\bTD No Pickup Point Accessible To Passenger\eTD\bTD 0.5\% \eTD
\bTD No Drop Point Near Destination\eTD\bTD 1.0\% \eTD\eTR
\bTR\bTD Detour Too Important For Reaching Drop Point\eTD\bTD 0.5\% \eTD
\bTD No Drop Point Accessible To Driver\eTD\bTD 0.5\% \eTD\eTR
\bTR\bTD No Drop Point Accessible To Passenger\eTD\bTD 0.5\% \eTD
\bTD Pickup Time Incompatible With Road And Pickup Point\eTD\bTD 0.5\% \eTD\eTR
\bTR\bTD Pickup Time Incompatible With Other Pickups\eTD\bTD 0.5\% \eTD
\bTD Journey Longer Than Proposed Time\eTD\bTD 0.5\% \eTD\eTR
\bTR\bTD Drop Time Incompatible With Road And Drop Point\eTD\bTD 0.5\% \eTD
\bTD Drop Time Incompatible With Other Pickups\eTD\bTD 0.5\% \eTD\eTR
\bTR\bTD Wrong Contact Information\eTD\bTD 7.0\% \eTD
\bTD Failed Communication\eTD\bTD 3.0\% \eTD\eTR
\bTR\bTD Ride Offer Cancelled And Instructions Sent\eTD\bTD 1.5\% \eTD
\bTD Ride Request Cancelled And Instructions Sent\eTD\bTD 1.5\% \eTD\eTR
\bTR\bTD Instruction In Wrong Language\eTD\bTD 0.5\% \eTD
\bTD Printing Failed\eTD\bTD 1.0\% \eTD\eTR
\bTR\bTD Ride Offer Cancelled Last Minute\eTD\bTD 3.0\% \eTD
\bTD Ride Request Cancelled Last Minute\eTD\bTD 2.0\% \eTD\eTR
\bTR\bTD Too Many Luggage\eTD\bTD 0.5\% \eTD
\bTD Too Many Riders\eTD\bTD 0.5\% \eTD\eTR
\bTR\bTD Not Enough Seats\eTD\bTD 0.5\% \eTD
\bTD Riders Do Not Recognize\eTD\bTD 1.0\% \eTD\eTR
\bTR\bTD Location Imprecise\eTD\bTD 2.0\% \eTD
\bTD Passenger Late At Pickup Point\eTD\bTD 6.0\% \eTD\eTR
\bTR\bTD Driver Late At Pickup Point\eTD\bTD 2.0\% \eTD
\bTD Ride Request Cancelled And Driver At Pickup Point\eTD\bTD 1.0\% \eTD\eTR
\bTR\bTD Ride Offer Cancelled And Passenger At Pickup Point\eTD\bTD 1.0\% \eTD
\bTD Passenger Forgot\eTD\bTD 1.0\% \eTD\eTR
\bTR\bTD Driver Forgot\eTD\bTD 2.0\% \eTD
\bTD Passenger Get Lost\eTD\bTD 2.0\% \eTD\eTR
\bTR\bTD Driver GPS Broken Down\eTD\bTD 20.0\% \eTD
\bTD Driver Get Lost To Pickup Point When GPS Broken\eTD\bTD 30.0\% \eTD\eTR
\bTR\bTD Instructions Not Received In Time\eTD\bTD 2.0\% \eTD
\bTD Pickup Point Not Reachable To Passenger\eTD\bTD 0.5\% \eTD\eTR
\bTR\bTD Pickup Point Not Reachable To Driver\eTD\bTD 0.5\% \eTD
\bTD Pickup Point Confused\eTD\bTD 1.0\% \eTD\eTR
\bTR\bTD Instructions Not Clear\eTD\bTD 0.5\% \eTD
\bTD Other Passenger Late\eTD\bTD 0.5\% \eTD\eTR
\bTR\bTD Detour On Planned Road\eTD\bTD 1.5\% \eTD
\bTD Stuck In Traffic Jams\eTD\bTD 10.0\% \eTD\eTR
\bTR\bTD Wrong instructions Sent\eTD\bTD 0.1\% \eTD
\bTD Car Broken Down\eTD\bTD 0.5\% \eTD\eTR
\bTR\bTD Driver Get Lost To Drop Point When GPS Broken Down\eTD\bTD 20.0\% \eTD
\bTD Drop Point Inaccessible By Car\eTD\bTD 0.5\% \eTD\eTR
\bTR\bTD Drop Point Confused\eTD\bTD 1.0\% \eTD
 \bTD\eTD\bTD\eTD\eTR

\eTABLE
        }
        
      \noindent {\bf Global Impact Analysis.} The propagation of the estimated
      satisfaction rates from the leaf obstacles to the root obstacles, then
      from the root obstacles to the leaf goals, and then from the leaf goals
      to the root goals yields the satisfaction rate for the two high-level
      goals. Using the techniques in
      \in{Section}[sec:pattern_based_computation], the satisfaction rate for
      \goal{Achieve [Ride Need Eventually Served]} is found to be only only
      $37.28\%$. The satisfaction rate for the goal \goal{Achieve [Ride
      Proposal Eventually Served]} is $81.02\%$.
      
      The satisfaction rate for the high-level goal \goal{Achieve [Ride Need
      Eventually Served]} means that the probability of serving a ride request
      is only about 40\% if all leaf obstacles were correctly estimated and no
      countermeasures to likely problems are provided; there would thus be
      approximately 2 chances out of 5 to serve a ride request. This is far
      from the RSR of 95\% prescribed on this main goal.
       
      Given the low goal's satisfaction rates obtained by propagating leaf
      obstacle estimates, likely and critical obstacles need be resolved. Due
      to a large number of leaf obstacles (68), we need to prioritize them so
      that we focus our resolutions on most critical ones.
      
      \noindent {\bf Local Impact Analysis.} To complement our global analysis,
      we may therefore consider one single leaf obstacle at a time, setting the
      satisfaction rates of all the other leaf obstacles to $0$. By
      up-propagation of these satisfaction rates, we obtain the violation
      severity for the root goal. Among the 68 single leaf obstacles, only
      three revealed to cause a severe violation, as \in{Table}[tab:cool_sv]
      depicts.
            
      \placetable[top][tab:cool_sv]
        {Critical obstacles for the Car Pooling System.}
        {\setupTABLE[c][each][align={left,lohi},frame=off,offset=0pt]
        \setupTABLE[r][1][style=bold,bottomframe=on,boffset=4pt]
        \setupTABLE[c][1,3][align={right,lohi},loffset=4pt,roffset=4pt]
        \setupTABLE[c][2,3][align={middle,lohi}]
        \setupTABLE[r][2][toffset=4pt]
        \setupTABLE[c][1][width=8cm]
        \setupTABLE[c][2][width=2cm]
        \setupTABLE[c][3][width=2cm]
        \switchtobodyfont[small]
        \bTABLE
          \bTR \bTD Obstacle                           \eTD \bTD SatRate \eTD \bTD Violation Severity \eTD \eTR
          \bTR \bTD Stuck In Traffic Jams              \eTD \bTD 10\%          \eTD \bTD 5\%               \eTD \eTR
          \bTR \bTD Wrong Contact Information          \eTD \bTD 7\%            \eTD \bTD 2\%               \eTD \eTR
          \bTR \bTD Passenger Late At Pickup Point    \eTD \bTD 6\%            \eTD \bTD 1\%               \eTD \eTR
        \eTABLE
        }
    
      \placefigure[bottom]
           [fig:violation_diagram_cool_1]
           {Violation Diagram for the Car Pooling System with single obstacle.}
        {\externalfigure[../images/chap9/cool_violation_diagram_1.pdf]}
    
      As seen in \in{Table}[tab:cool_sv], the leaf obstacle \obstacle{Stuck In
      Traffic Jams}, with an estimated satisfaction rate of 10\%, causes a
      violation severity of 5\% for the root goal. This means that only 90\% of
      requested rides would then be served. The leaf obstacle \obstacle{Wrong
      Contact Information}, with an estimated satisfaction rate of 7\%, causes
      a violation severity of 2\% for the root goal. The leaf obstacle
      \obstacle{Passenger Late At Pickup Point}, with an estimated satisfaction
      rate of 6\%, yields a violation severity of 1\% with respect to the root
      goal's prescribed RSR of 95\%. \in{Figure}[fig:violation_diagram_cool_1]
      shows the violation diagrams for these obstacles.
      
      A similar table and violation diagram can be generated for the second
      root goal \goal{Achieve [Ride Proposal Eventually Served]}. However, no
      obstacle alone is sufficient here for dropping the satisfaction rate
      below the required satisfaction rate.
      
      After having found the single leaf obstacles that are critical, we need
      to turn our attention to the critical obstacle pairs. Assuming the
      preceding critical singletons are selected for resolution, the critical
      pairs should not redundantly include them. Using the brute-force approach
      discussed in \in{Section}[sec:indentifying_most_critical_obstacles], we
      found 206 pairs of leaf obstacles that are critical with respect to the
      root goal out of the 2415 generated pairs in less than a second. Among
      those 206 pairs, 2 do not include any of the critical singletons in
      \in{Table}[tab:cool_sv]. The most critical pairs among these are shown in
      \in{Table}[tab:cool_sv2]. Together with the critical singletons, they are
      the ones having higher priority for resolution. Once resolved, if the
      required satisfaction rate for the high-level goals is not yet reached,
      triple combinations excluding the preceding critical singletons and pairs
      could reveal more critical obstacle combinations.
            
      \placetable[here][tab:cool_sv2]
        {Critical pairs of obstacle for the Car Pooling System.}
        {\setupTABLE[c][each][align={left,lohi},frame=off,offset=0pt]
        \setupTABLE[r][1][style=bold,bottomframe=on,boffset=4pt]
        \setupTABLE[c][1][width=8cm]
        \setupTABLE[c][2][width=2cm]
        \setupTABLE[c][3][width=2cm]
        \setupTABLE[c][1,3][align={right,lohi},loffset=4pt,roffset=4pt]
        \setupTABLE[c][2,3][align={middle,lohi}]
        \setupTABLE[r][2][toffset=4pt]
        \setupTABLE[c][2,3][alignmentcharacter={text->.},aligncharacter=yes,align={middle,lohi}]
        \setupTABLE[2,3][1][align={middle,lohi},aligncharacter=no]
        \switchtobodyfont[small]
        \bTABLE
          \bTR \bTD Obstacles                                                                 \eTD \bTD SatRate \eTD \bTD Violation Severity  \eTD \eTR
          \bTR \bTD Ride Offer Cancelled Last Minute, Failed Communication                   \eTD \bTD 0.09\%          \eTD \bTD 0.91\%               \eTD \eTR
          \bTR \bTD Driver Get Lost To Pickup Point When GPS Broken, \blank[none] \hskip.5em Driver GPS Broken Down  \eTD \bTD 6.00\%            \eTD \bTD 1.00\%                  \eTD \eTR
        \eTABLE
        }

      Note that the ranking of leaf obstacles by their satisfaction rate is not
      the same as their ranking by the resulting violation severity for the
      root goal. The obstacle pair \{\obstacle{Ride Offer Cancelled Last
      Minute},\obstacle{Failed Communication}\} has a lower satisfaction rate
      compared to the pair \{\obstacle{Driver Get Lost To Pickup Point When GPS
      Broken}, \obstacle{Driver GPS Broken Down}\} even though their violation
      severities are close to each other.
      
      \noindent {\bf Discussion.} Our global impact analysis revealed that the
      original model is not compliant with our probabilistic requirement of
      serving 95\% of encoded ride requests. Prioritization was required to
      handle the large number of leaf obstacles obtained. Our local impact
      analysis revealed critical single obstacles and critical obstacle pairs
      to be resolved first in the next phase of risk control.

      Some of the leaf obstacles appeared more critical than others, even if
      they have a small satisfaction rate, especially combined with other
      obstacles. For example, the obstacle \obstacle{Ride Request Cancelled And
      Ride List Proposed}, with a satisfaction rate of 0.02, was seen to
      potentially obstruct the leaf goals \goal{Achieve [Suggestions Selected
      When Proposed by Passenger]} and \goal{Achieve [Suggestions Selected When
      Proposed by Driver]}. Even if its estimated satisfaction rate is low, the
      obstacle may be critical.
        
      Leaf obstacles with lower satisfaction rate may thus be more critical
      than other ones, with a higher satisfaction rate when they obstruct more
      goals; an increase in their satisfaction rate may have a major impact on
      root goal satisfaction.

      Critical combinations with one or two leaf obstacles appeared to include
      most of the critical obstacles. Combinations with more obstacles were
      often supersets of those critical combinations. The resolution of
      critical singletons and pairs is therefore expected to substantially
      reduce the number of critical combinations of a larger size.

      In short, the large number of obstacles made it quite difficult to
      identify most critical obstacles to be considered first for
      resolution. The prioritized list of obstacles produced by our technique
      helped significantly in that direction.
      
      Most likely and critical obstacles should be resolved next; this case
      study however focused on evaluation the techniques for {\it obstacle
      assessment}. The techniques for {\it obstacle control} are evaluated in
      \in{Section}[sec:bads] on a different case study.
      
    \stopsubsection
        
    \startsubsection[reference=sec:iba,title={The IBA yoke lifting system}]
    
      We applied the techniques from \in{Chapter}[chap:assessing] to a second
      case-study. This section focuses on evaluating the applicability of the
      techniques for {\it obstacle assessment} in an industrial setting.
        
      This second case study is an industrial system in the medical domain. IBA
      is a medical technology company based in Louvain-la-Neuve (BE), founded
      in 1986, and active in the field of proton therapy. IBA is currently the
      world leader in proton therapy solutions for cancer treatment
      \cite[Lin11]. Among their products, cyclotrons accelerate particles using
      a high-frequency alternating voltage applied between electrodes inside a
      vacuum chamber. A two-part yoke encloses the necessary equipment; {\it
      yoke} refers to the pieces sustaining the distance between the
      electrodes. The upper part of the yoke can be lifted to access the inside
      of the cyclotron for operation and maintenance purposes. Given the weight
      (> 110 tons) and size of the upper yoke, the lifting system is critical
      for operators safety. \in{Figure}[fig:iba_kiube] shows a IBA ‎Cyclone
      230\fontchar{registered} cyclotron open (left) and closed (right),
      showing the yoke lifting system in action.
      
      \placefigure[]
           [fig:iba_kiube]
           {IBA ‎Cyclone 230\fontchar{registered} Cyclotron.}
        {\startcombination[2*1]
          {\externalfigure[../images/chap9/c230_open.jpg][width=5.5cm]}{}
          {\externalfigure[../images/chap9/c230_closed.jpg][width=5.5cm]}{}
        \stopcombination}
      
      Our risk analysis focussed on the human safety aspects for the IBA
      Cyclotron Service Engineers, named {\it operators} in the following, on
      the IBA C230 Yoke Lifting System (C230-YLS).
        
      The goal and obstacle models were built and analyzed in collaboration
      with G. Gérard (IBA R&D Requirement Engineer) and P. Cailliau (IBA R&D
      C230 System Owner). 
      
      \noindent The purpose of the analysis was two-fold:
      
      \startitemize[a]
      
        \item Identifying potential failures, assessing their criticality and
        defining mitigations when required;

        \item Evaluating the proposed approach for {\it obstacle assessment} on
        this kind of risk analysis.
        
      \stopitemize
      
      \noindent A first risk analysis (FMECA) was available before building the
      goal/obstacle models.
      
      \noindent {\bf Building the goal and obstacle models.} A generic goal
      model was built for human safety requirements based on an available
      document entitled {\it Occupational Injury and Illness Classification
      Manual} \cite[Sta92]. Our model contains 220 goals with 47 refinements
      and 173 leaf goals. The model provides a set of detailed safety
      requirements. Most of the goals in the model are {\it Avoid} goals such
      as \goal{Avoid [Contact with hot objects or substances]} or \goal{Avoid
      [Fall from moving vehicle or mobile equipment]}. These goals were
      directly mapped to categories of injuries and illness from the available
      classification manual; this manual provides definitions and appropriate
      examples. \in{Figure}[fig:iba_goal_1] shows the refinement for the root
      goal \goal{Avoid [Occupational Injury and Illness]}.
      
      \placefigure[]
           [fig:iba_goal_1]
           {The refinement of \goal{Avoid [Occupational Injury and Illness]}.}
        {\externalfigure[../images/chap9/iba_goal_1.pdf]}
      
      The goal model was then specialized to identify the specific human safety
      requirements for the C230-YLS. Thanks to the refinement structure, large
      parts of the generic model were pruned. For instance, as no transport is
      involved in the system, the goal \goal{Avoid [Transportation Accident]}
      in \in{Figure}[fig:iba_goal_1] could be removed from the model together
      with its descendants. The high-level goal \goal{Avoid [Assault and
      violent acts]} was considered to be out of scope; this and its
      descendants were removed too. Such pruning was applied recursively on the
      refinement structure, to eliminate goals that are either not relevant or
      out of scope. The resulting specific goal model contains 22 goals refined
      through 9 goal refinements into 13 leaf goals. These are the specific
      human safety goals to be enforced by the C230-YLS.
      
      Obstacle analysis was anchored on those leaf goals. From the negation of
      these leaf goals, obstacles unique to the C230-YLS were produced based on
      prior experience, existing risk analysis and incident reports.
      \in{Figure}[fig:iba_obstacle_1] shows one of the obstacle trees. The
      obstacle model contains 43 obstacles and 28 obstacle refinements. Thanks
      to a systematic obstacle identification process, we identified risks that
      were absent from the prior risk analysis performed by IBA engineers. The
      latter used Failure Modes, Effects, and Criticality Analysis (FMECA)
      \cite[Lev01].
      
      \placefigure[]
           [fig:iba_obstacle_1]
           {Obstacle tree fragment for the goal \goal{Avoid [Caught in running equipment or machinery]}.}
        {\externalfigure[../images/chap9/iba_obstacle_1.pdf]}
  
      \noindent {\bf Obstacle Assessment.} To assess the risks, FMECA tables
      were generated from the obstacle refinement trees (See
      \in{Section}[sec:comparing_ra] for a detailed comparison between our
      techniques and FMECA.) Each row in a FMECA table corresponds to a failure
      mode. A FMECA table contains the following columns:
      
      \startitemize[packed]

        \item {\bf ID:} a unique identifier.

        \item {\bf Failure Mode Description:} describes the unsatisfied
        requirement.

        \item {\bf Cause:} describes why the requirement is not satisfied.

        \item {\bf Effect:} describes the consequences of not satisfying of the
        requirement.

        \item {\bf Risk:} assessment of the risk likelihood and criticality.

        \item {\bf Mitigated by:} describes the mitigations for the risk.

        \item {\bf Risk after:} assessment of the risk likelihood and
        criticality after the mitigations.

      \stopitemize
    
      \noindent The FMECA-obstacle mapping is as follows. The {\it Failure Mode
      Description} corresponds to the obstructed leaf goal; {\it cause} was
      generated from the obstacle AND-refinements containing leaf goals, {\it
      effect} was initially generated from the ancestors of the obstructed leaf
      goal but was later reformulated manually to match IBA practices.
      
      We generated 5 tables corresponding to the level-$1$ goals refining the
      root goals in \in{Figure}[fig:iba_goal_1], namely, \goal{Avoid [Bodily
      reaction]}, \goal{Avoid [Exposure to harmful substances or
      environments]}, \goal{Avoid [Falls]}, \goal{Avoid [Contact with object
      and equipement]}, \goal{Avoid [Fires and explosion]}.
    
      \noindent {\bf Assessing the likelihood and criticality of obstacles.}
      Risk assessment was performed by IBA engineers based on the FMECA tables.
      This assessment was introduced back into the goal/obstacle model
      afterwards. Risk likelihoods were assessed using a standard
      semi-quantitative scale ranging from A to F \cite[Ayy14]. To enable our
      techniques, we converted this scale into quantitative values using
      reference values provided in \cite[Ayy14]. \in{Table}[tab:iba_scale]
      shows the correspondence between the two scales. The required
      satisfaction rate (RSR) for the top-level goal was defined to $99.99\%$.
      Interestingly, during risk assessment process engineers split some of the
      leaf obstacles into subobstacles to support more fine-grained evaluation;
      the changes were reflected in our models.
        
      \placetable[bottom][tab:iba_scale]
        {Semi-quantitative scale for risk assessment at IBA.}
        {\setupTABLE[c][each][align={left,lohi},frame=off,offset=0pt]
        \setupTABLE[r][1][style=bold,bottomframe=on,boffset=4pt]
        \setupTABLE[c][1,2][align={right,lohi}]
        \setupTABLE[c][3,4,5,6,7][alignmentcharacter={text->.},aligncharacter=yes,align=middle]
        \setupTABLE[3,4,5,6,7][1][align={left,lohi},aligncharacter=no]
        \setupTABLE[c][1][roffset=4pt]
        \setupTABLE[c][2][roffset=4pt]
        \setupTABLE[c][3,4,5,6,7][roffset=4pt,loffset=4pt]
        \setupTABLE[r][2][toffset=4pt]
        \switchtobodyfont[small]
        \bTABLE
        \bTR \bTD Category \eTD \bTD Description \eTD \bTD Numeric value \eTD \eTR
        \bTR \bTD A \eTD \bTD Frequent \eTD \bTD $.1$ \eTD \eTR
        \bTR \bTD B \eTD \bTD Probable \eTD \bTD $.01$ \eTD \eTR
        \bTR \bTD C \eTD \bTD Occasional \eTD \bTD $.001$ \eTD \eTR
        \bTR \bTD D \eTD \bTD Remote \eTD \bTD $.0001$ \eTD \eTR
        \bTR \bTD E \eTD \bTD Improbable \eTD \bTD $.00001$ \eTD \eTR
        \bTR \bTD F \eTD \bTD Incredible \eTD \bTD $.000001$ \eTD \eTR
        \eTABLE
        }
        
      \noindent {\bf Global and local impact analysis.} Our global analysis
      revealed that the satisfaction rate for \goal{Avoid [Injury and Illness
      Caused By YLS]} is 91.46\%. This does not meet the required satisfaction
      rate. Applying our techniques for local analysis revealed 12 obstacles
      that were sufficient alone to cause the satisfaction rate of the
      high-level goal to fall below its required threshold. Two groups of
      obstacles could be identified: obstacles with a $.01$ satisfaction rate
      and obstacles with a $.001$ satisfaction rates. All obstacles identified
      as likely and critical by our techniques were considered as intolerable
      using the IBA Risk Criticality Matrix. This matrix classifies risks in
      three categories: {\it Intolerable} (I), {\it As low as reasonably
      practicable} (II) and {\it Broadly Acceptable} (III).
      
        % Caught in or crushed by collasped accessory during yoke movement:
        %   vs = 0.000899999999999901, cp = 0.001
        % Caught in or crushed by collapsed yoke during yoke movement: 
        %   vs = 0.000899999999999901, cp = 0.001
        % Struck by falling pillar during positioning/manipulation: 
        %   vs = 0.000899999999999901, cp = 0.001
        % Fall to floor On Command Cables: 
        %   vs = 0.000899999999999901, cp = 0.001
        % Contact with electrical current from electrovalve: 
        %   vs = 0.000899999999999901, cp = 0.001
        % 
        % Struck by falling pillar during storage: 
        %   vs = 0.00989999999999991, cp = 0.01
        % Struck by falling pillar during transport: 
        %   vs = 0.00989999999999991, cp = 0.01
        % Struck by falling wood plate: 
        %   vs = 0.00989999999999991, cp = 0.01
        % Bodily reaction when positioning pillars in/out: 
        %   vs = 0.00989999999999991, cp = 0.01
        % Bodily reaction when transporting pillars: 
        %   vs = 0.00989999999999991, cp = 0.01
        % Contact with electrical command cable: 
        %   vs = 0.00989999999999991, cp = 0.01
        % Contact with electrical current from command: 
        %   vs = 0.00989999999999991, cp = 0.01
      
      \noindent {\bf Controlling Obstacles.} Based on the generated FMECA
      tables, IBA engineers identified 32 mitigations; these were introduced as
      countermeasures goals. The techniques presented in
      \in{Chapter}[chap:controlling_obstacle] could however not be applied to
      select most appropriate countermeasures. The current practice at IBA
      intertwines assessment and control; it assesses the {\it risk after} as
      the severity of the risk once all mitigations are integrated. The
      mitigations are elicited as long as the {\it risk after} does not meet
      the target severity. Our approach separates the countermeasures
      identification from assessment and applies assessment to each individual
      countermeasure.
      
      \noindent {\bf Discussion.} By separating countermeasure identification
      from assessment, our approach may improve the cost-effectiveness of the
      selected mitigations. Such separation is discussed in the context of
      conflict resolutions in \cite[Eas94]. It enables the selection of most
      appropriate countermeasures at a system level but requires some extra
      assessment from domain experts. It appeared that {\it (a)} some of the
      selected mitigations might be redundant and {\it (b)} that more global
      mitigations resolving multiple risks could be favored over more local
      mitigations.
        
      Overall the goal/obstacle approach appeared to be helpful as it provides
      a complementary top-down approach to the current bottom-up approach in
      use at IBA. The goal/obstacle models provide a more global view of the
      system under scrutiny compared to the more detailed approach proposed by
      FMECA. Risks not accounted for in the risk analysis previously performed
      were found during the goal/obstacle analysis together with extra
      mitigations. How top-down and bottom-up risk analysis approaches could be
      integrated remains, to us, an open issue.
      
      In addition, the anchoring of probabilistic obstacles on real-world
      phenomena was felt helpful for estimating obstacle satisfaction rates,
      making the mapping to available statistics easier. IBA currently works on
      improving the quantitative aspects of their risk analysis to enable the
      reuse of such available statistics.
  
      Our techniques revealed that some obstacles were considered as likely and
      critical by our techniques while being considered as \quote{{\it As low
      as reasonably practicable} (II)}; those were risks causing minor
      injuries. Differentiating between minor and major injuries turned to be
      challenging. For example a minor injury, while moving a pillar of 20kg
      and a major injury resulting from being crushed by the 110 tons yoke both
      result in the non-satisfaction of the high-level goal \goal{Avoid [Injury
      and Illness Caused By YLS]}. There is, however, a major distinction, not
      accounted for in our approach, between the two.
  
    \stopsubsection

    \startsubsection[reference=sec:bads,title={The Brussels ambulance dispatch system}]
  
      The techniques from \in{Chapter}[chap:assessing]-\in{}[runtime] were
      applied to a benchmark commonly used for evaluating obstacle analysis
      techniques \cite[Fin96,Lam00,Let02,Alr12,Alr16] and for evaluating other
      techniques software engineering techniques, in particular self-adaptation
      \cite[Sou12]. The aim is evaluate the applicability of our {\it obstacle
      assessment} and {\it control} techniques to a system of significant size.
      
      The model used for the evaluation was inspired by the one used for
      evaluating obstacle analysis techniques in \cite[Lam00,Let02]. In
      addition, the model was adapted and enriched based on the author's
      experience as a volunteer Emergency Medical Technician (EMT) in this type
      of system. When appropriate, the system was tailored to match the
      regulations and practice applicable in the Brussels area (Belgium)
      \cite[Pub13].
      
      \noindentation The following description briefly explains the Ambulance
      Dispatch System (ADS).
    
      {\vskip.8\baselineskip}
      \framed[width=local,align={width},offset=10pt,after={\vskip-\baselineskip}
       ] {\it A "112" call connects the caller to a dispatch operator who
      inputs information about the incident into the dispatch software. The
      system then has to allocate and mobilize an appropriate ambulance, and
      transmit all relevant details to the selected vehicle. The system has
      also to track the actual availability and position of the ambulances.}
      
      \noindent For a more detailed system description, see \cite[Let02].
    
      \noindent {\bf Goal and obstacle model.} The goal model contains 56 goals
      with 3 root goals. There are 27 refinements leading to 34 leaf goals. The
      obstacle model includes 84 obstacles. The root obstacles obstruct 17 leaf
      goals. There are 69 refinements for these root obstacles, leading to 54
      leaf obstacles. \in{Section}[app:bas] details the complete specification
      for the goals and obstacles.
    
      The primary root goal is \goal{Achieve [Incident Resolved]}, with a
      $95\%$ required satisfaction rate.
    
      \startkaosspec
        \GoalName {Achieve [Incident Resolved]}
        \KaosAttribute {Def} {All incidents shall be
        resolved. An incident is resolved if the victim is treated on the
        incident scene or transported to an hospital.}
        \KaosAttribute {RSR} {$95\%$}
      \stopkaosspec
    
      A {\it milestone-driven} refinement decomposes the root goal
      \goal{Achieve [Incident Resolved]} into three subgoals: \goal{Achieve
      [Incident Reported]}, \goal{Achieve [Ambulance On Scene When Incident
      Reported]}, and \goal{Achieve [Incident Resolved When Ambulance On
      Scene]}. The \agent{Public} agent is responsible for the goal
      \goal{Achieve [Incident Reported]}; the \agent{Ambulance staff} agent in
      collaboration with the \agent{Automated Dispatch System} (\agent{ADS}) is
      responsible for the goal \goal{Achieve [Incident Resolved When Ambulance
      On Scene]}; the goal is therefore refined until assigned to these agents,
      not seen in \in{Figure}[fig:ads_goal_1]. Three subgoals refine the goal
      \goal{Achieve [Ambulance On Scene When Incident Reported]} as
      \in{Figure}[fig:ads_goal_1] shows. The latter goals are recursively
      refined until assigned to single agents.
    
      \placefigure[]
           [fig:ads_goal_1]
           {Goal model fragment for the Ambulance Dispatch System.}
        {\externalfigure[../images/chap9/ads_goal_1.pdf]}
    
      The two other root goals are \goal{Maintain [Accurate Ambulance Status
      Known]} and \goal{Maintain [Accurate Ambulance Location Known]}:
    
      \startkaosspec
        \GoalName {Maintain [Accurate Ambulance Status Known]}
        \KaosAttribute {Def} {The status of the ambulance shall be accurately
        known to the dispatching software. A status is accurate if the
        known status corresponds to the status of the ambulance
        in the real-world. The status shall be known to the dispatching software
        within 3 minutes when encoded by the staff.}
        \KaosAttribute {RSR} {$90\%$}
      \stopkaosspec
    
      \startkaosspec
        \GoalName {Maintain [Accurate Ambulance Location Known]}
        \KaosAttribute {Def} {The accurate location of the ambulance is known
        by the dispatching software. A location is accurate if the position of
        the ambulance in the real-world and the dispatching position do not differ by
        more than 250 meters.}
        \KaosAttribute {RSR} {$90\%$}
      \stopkaosspec
    
      \placefigure[bottom]
           [fig:ads_obstacle_1]
           {Obstacle model fragment for the Ambulance Dispatch System.}
        {\externalfigure[../images/chap9/ads_obstacle_1.pdf]}
    
      Negating the leaf goals generates obstacles that are later refined. For
      example, the leaf goal \goal{Achieve [Mobilized Ambulance On Scene]}
      generates the root obstacle \obstacle{Mobilized Ambulance Not On Scene}.
      This obstacle is refined until experts can estimate the satisfaction
      rates for the leaf obstacles. \in{Figure}[fig:ads_obstacle_1] shows an
      obstacle tree for the goal \goal{Achieve [Mobilized Ambulance On Scene]}.
    
      The available resolution tactics \cite[Lam00,Dar95] produce
      countermeasure goal candidates. For example, here are five countermeasure
      goals for two leaf obstacles:
  
      \startkaosspec
        \ObstacleName {Mobilization Taken By Other Ambulance}
        \KaosAttribute {Resolvedby} {Achieve [Mobilization By Other Ambulance Known]}
        \KaosAttribute {Resolvedby} {Avoid [Mobilization Without Order]}
      \stopkaosspec
  
      \startkaosspec
        \ObstacleName {Displayed Mobilization Order Ignored}
        \KaosAttribute {Resolvedby} {Achieve [Alarm When Mobilization Order Displayed]}
        \KaosAttribute {Resolvedby} {Achieve [Failed Mobilization Recovered]}
        \KaosAttribute {Resolvedby} {Achieve [Late Mobilization When Crew Not Responsive]}
      \stopkaosspec
    
      \subsubsubject{Assessing likelihood and criticality of obstacles}

        In order to evaluate the techniques presented in
        \in{Chapter}[chap:assessing], the satisfaction rate of 34 leaf
        obstacles was estimated based upon the author's experience as a
        volunteer Emergency Medical Technician (EMT) exposed to the Brussels
        ambulance system.

        Our global impact analysis using the techniques in
        \in{Section}[sec:indentifying_most_critical_obstacles] revealed that
        the high-level goal \goal{Achieve [Incident Resolved]} in \in{Figure}[fig:ads_goal_1] has a computed
        satisfaction rate of 6.20\%; this is far from the required satisfaction
        rate of 95\% to be achieved by the system-to-be while agreeing with the
        observed rate of ideal ambulance interventions.
      
        Local impact analysis using the techniques in
        \in{Section}[sec:indentifying_most_critical_obstacles] revealed that 17
        obstacles are sufficient alone to prevent the required satisfaction
        rate to be achieved. \in{Table}[tab:ads_sv] presents the likely and
        critical obstacles to be resolved with highest priority. There are 142
        pairs of obstacle, not containing one of the 17 obstacles previously
        identified, that cause the estimated satisfaction rate to drop below
        the required satisfaction rate.

        \placetable[top][tab:ads_sv]
          {Critical obstacles for the Ambulance Despatch System.}
          {\setupTABLE[c][each][align={left,lohi},frame=off,offset=0pt]
          \setupTABLE[r][1][style=bold,bottomframe=on,boffset=4pt]
          \setupTABLE[c][1,3][align={right,lohi},loffset=4pt,roffset=4pt]
          \setupTABLE[c][2,3][align={middle,lohi}]
          \setupTABLE[r][2][toffset=4pt]
          \setupTABLE[c][1][width=8cm]
          \setupTABLE[c][2][width=2cm]
          \setupTABLE[c][3][width=2cm]
          \switchtobodyfont[small]
          \bTABLE
            \bTR \bTD Obstacle \eTD \bTD Probability \eTD \bTD Violation Severity \eTD \eTR
            \bTR \bTD Out Of Paper \eTD \bTD 40.00\% \eTD \bTD 35.00\% \eTD \eTR
            \bTR \bTD Allocated Ambulance Not At Station \eTD \bTD 33.33\% \eTD \bTD 28.33\% \eTD \eTR
            \bTR \bTD Patient Not Transportable \eTD \bTD 13.33\% \eTD \bTD 8.33\% \eTD \eTR
            \bTR \bTD Patient Cannot Reach Ambulance \eTD \bTD 13.33\% \eTD \bTD 8.33\% \eTD \eTR
            \bTR \bTD Special Unit Required \eTD \bTD 13.33\% \eTD \bTD 8.33\% \eTD \eTR
            \bTR \bTD Crew Not In Ambulance \eTD \bTD 13.33\% \eTD \bTD 8.33\% \eTD \eTR
            \bTR \bTD Paper Jam \eTD \bTD 10.00\% \eTD \bTD 5.00\% \eTD \eTR
            \bTR \bTD GPS Black Spot \eTD \bTD 10.00\% \eTD \bTD 5.00\% \eTD \eTR
            \bTR \bTD Ambulance In Traffic Deviation \eTD \bTD 10.00\% \eTD \bTD 5.00\% \eTD \eTR
            \bTR \bTD Ambulance Stuck In Traffic Jam Toward Incident \eTD \bTD 7.00\% \eTD \bTD 2.00\% \eTD \eTR
            \bTR \bTD Printer Off \eTD \bTD 6.67\% \eTD \bTD 1.67\% \eTD \eTR
            \bTR \bTD Destination Confused \eTD \bTD 6.67\% \eTD \bTD 1.67\% \eTD \eTR
            \bTR \bTD Hazardous Environment \eTD \bTD 6.67\% \eTD \bTD 1.67\% \eTD \eTR
            \bTR \bTD Unreachable Patient \eTD \bTD 6.67\% \eTD \bTD 1.67\% \eTD \eTR
            \bTR \bTD Ressource Out Of Order \eTD \bTD 6.67\% \eTD \bTD 1.67\% \eTD \eTR
            \bTR \bTD Insufficient Capacity \eTD \bTD 6.67\% \eTD \bTD 1.67\% \eTD \eTR
            \bTR \bTD Crew Distracted \eTD \bTD 6.67\% \eTD \bTD 1.67\% \eTD \eTR
          \eTABLE
          }

      \subsubsubject{Controlling likely and critical obstacles} 

        The number of obstacles and countermeasure goals called for
        countermeasure integration according to the techniques presented in
        \in{Chapter}[chap:controlling_obstacle]. As a result, the
        countermeasure goals appeared to focus on a small number of important
        goals; e.g., the goal \goal{Achieve [Incident Resolved When Ambulance
        On Scene]} has 8 {\it Except} constructs. The overall integration
        produced 28 {\it Except} constructs distributed among 8 goals only.
   
        The techniques presented in \in{Chapter}[chap:controlling_obstacle]
        helped significantly for the following reasons.
  
        \noindent {\bf Model simplification by separation of concerns.} The
        goals referring to normal situations are systematically distinguished
        from those handling obstacle occurrences. Emerging assumptions were
        incrementally down-propagated to the obstructed descendants of the
        corresponding anchor goals. This produced 62 {\it Provided} annotations
        distributed over 13 goals. Without these annotations, details related
        to exceptional cases would have cluttered the formal specification of
        those goals.

        For example, the goal \goal{Achieve [Allocated Ambulance Mobilized When
        Mobilization Order Printed And Phone Contact]} is defined as follows
        after integration in the model:
    
        \startkaosspec
          \GoalName {Achieve [Allocated Ambulance Mobilized \blank[none]\hskip4em When Mobilization Order Printed And Phone Contact]}
          \KaosAttribute {ProvidedNot} {Mobilization Taken By Other Ambulance}
          \KaosAttribute {ProvidedNot} {Allocated Ambulance Not At Station}
          \KaosAttribute {ProvidedNot} {Allocated Ambulance Not Available}
          \KaosAttribute {ProvidedNot} {Printed Mobilization Order Ignored}
        \stopkaosspec
   
        The full equivalent specification of this goal without {\it
        ProvidedNot} annotations would have completely hidden the ideal case.
        It would then appear hard to distinguish the part of the goal
        antecedent related to the ideal case from the parts related to
        exceptional cases. An example is provided here.
   
        The {\it Detach-Except} operator was applied to the {\it case-driven}
        refinement of the goal \goal{Achieve [Allocated Ambulance
        Mobilization]}. Allocating an ambulance when not at its station was
        estimated fairly rare\emdash{}5\% of cases according to typical figures in
        the domain. The parent goal of these two goals was therefore modified
        accordingly:
   
        \startkaosspec
          \GoalName {Achieve [Allocated Ambulance Mobilization]}
          \KaosAttribute {Except} {AllocatedAmbulanceNotAtStation \blank[none]\hskip2em{\bf then} Achieve [Allocated Ambulance Mobilization At Station]}
        \stopkaosspec
      
        Such refactoring reduces model complexity by hiding the part of the
        model handling the mobilization of an ambulance when on the road. The
        resulting ideal goal model contains fewer refinements and fewer goals,
        making it easier to understand and separate typical behaviors from
        exceptional ones.
      
        \noindent {\bf Compositionality.} Without the techniques in
        \in{Section}[sec:exception_handling], the integration of so many
        exceptions for only 8 goals would have resulted in large, complicated
        refinements with a combinatorial blow-up of exceptional cases. To
        illustrate this important point, consider the goal \goal{Achieve
        [Ambulance Mobilized When Allocated]}. The original, ideal
        specification is:
      
        \startformula\startalign[n=1,align=left] \NC \forall amb: Ambulance, inc: Incident \cdot \NR\noalign{\vskip-.5em} \NC \hskip.5cm Allocated (amb, inc) \Rightarrow \ltlF_{\leq 1 min} Mobilized (amb, inc) \NR \stopalign\stopformula
      
        After obstacle analysis, this goal is ensured through 5
        countermeasure goals (see \in{Figure}[fig:ads_exception_1]). A
        brute-force integration of only three of those (in red,
        see \in{Figure}[fig:ads_exception_1]) would have resulted in the
        following formal specification for the final version of the goal
        \goal{Achieve [Ambulance Mobilized When Allocated]}:

        \startformula\startalign[n=1,align=left]
          \NC \forall amb: Ambulance, inc: Incident \cdot \NR\noalign{\vskip-.5em}
          \NC \hskip.5cm Allocated (amb, inc) \NR\noalign{\vskip-.5em}
          \NC \hskip1cm \Rightarrow \ltlF_{\leq 1 min} Mobilized (amb, inc)\NR\noalign{\vskip-.5em}
          \NC \hskip1.5cm \vee [\ltlG_{> 3 min} \neg AmbAvailable (amb, inc) \NR\noalign{\vskip-.5em}
          \NC \hskip2.5cm \rightarrow\ltlF_{\leq 6 min} \exists amb': Ambulance \cdot Mobilized (amb', inc)]\NR\noalign{\vskip-.5em}
          \NC \hskip1.5cm \vee [\ltlG_{> 3 min} DisplayedMobilizationIgnored (amb, inc) \NR\noalign{\vskip-.5em}
          \NC \hskip2.5cm  \rightarrow\ltlF_{\leq 6 min} \exists amb': Ambulance \cdot Mobilized (amb', inc) ] \NR\noalign{\vskip-.5em}
          \NC \hskip1.5cm \vee  [\ltlG_{> 3 min} PrintedMobilizationIgnored (amb, inc) \NR\noalign{\vskip-.5em}
          \NC \hskip2.5cm    \rightarrow \ltlF_{\leq 6 min} \exists amb': Ambulance \cdot Mobilized (amb', inc) ]\NR\noalign{\vskip-.5em}
        \stopalign\stopformula

        In addition to this complex specification, the goal refinement
        structure would have been heavily modified as follows:
      
        \startkaosspec
          Achieve [Ambulance Mobilized When Allocated] \blank[none]
          \hskip.5cm \leftarrow~ Achieve [Other Ambulance Mobilized \blank[none]
          \hskip2.3cm When Allocated Ambulance Not Available] \blank[none]
          \hskip.5cm \leftarrow~ Achieve [Ambulance Mobilized When Allocated And Available] \blank[none]
          \hskip1cm  \leftarrow~ Achieve [Late Mobilization When Crew Not Responsive] \blank[none]
          \hskip1cm  \leftarrow~ Achieve[Ambulance Mobilized When Allocated And Available \blank[none]
          \hskip2.8cm        And Displayed Mobilization Order Not Ignored] \blank[none]
          \hskip1.5cm  \leftarrow~ Achieve [Other Ambulance Mobilized After Timeout] \blank[none]
          \hskip1.5cm  \leftarrow~ Achieve [Ambulance Mobilized When Allocated And Available  \blank[none]
          \hskip3.4cm            And Displayed Mobilization Order Not Ignored \blank[none]
          \hskip3.4cm            And Printed Mobilization Order Not Ignored]
        \stopkaosspec

        With such brute-force integration, each countermeasure goal must be
        refined by taking other countermeasures into account. This lead to a
        combinatorial blow-up of cases. Thanks to our technique, the original
        specification of this goal and its refinement structure are preserved.
        The {\it Except} and {\it Provided} constructs encapsulate the
        modifications for a more robust system.
    
        \placefigure[]
             [fig:ads_exception_1]
             {Exception Diagram for \goal{Achieve [Ambulance Mobilized When Allocated]}.}
          {\externalfigure[../images/chap9/ads_exception_1.pdf]}

        \noindent {\bf No premature decision and freedom of choice.} The
        exceptions separate the specification and documentation of unusual
        behaviors from the usual ones. This allowed delaying the decision of
        how and when the handling of exceptional cases should occur.

        \noindent {\bf Other benefits.} The {\it Replaces} annotation was felt
        useful for documenting the replacing countermeasure goals\emdash{}e.g.,
        \goal{Achieve [Ambulance On Scene Or Cancelled When Incident Reported]}
        replacing \goal{Achieve [Ambulance On Scene When Incident Reported]} to
        resolve the obstacle \obstacle{Mobilization Cancelled}. Without this
        annotation, we would have lost the previous version of the goal.

        Exception diagrams significantly helped understand the model where all
        countermeasures are integrated; they document exceptions one single
        goal at a time, as shown in \in{Figure}[fig:ads_exception_1]. For the
        ADS, 14 exception diagrams document exceptional cases and
        countermeasure goals.
    
      \subsubsubject{Handling uncertainty in satisfaction rates}
    
        To evaluate the techniques in
        \in{Section}[sec:assessment_k_uncertainty] for obstacle assessment in
        presence of knowledge uncertainty, we asked five experienced EMTs
        involved in the system to estimate missing or unavailable data. The
        satisfaction rates for the leaf obstacles in the Brussels area are not
        publicly available although they are partially recorded.
        \in{Table}[tab:ads_estimates] outlines some of the collected data. The
        experts provided estimates based on a custom number of interventions;
        all estimates were then converted into percentages (which explains
        decimal values in \in{Table}[tab:ads_estimates]).
  
        \placetable[here][tab:ads_estimates]
          {Expert estimates for the Ambulance Dispatch System.}
          {\setupTABLE[c][each][align={left,lohi},frame=off,offset=0pt]
          \setupTABLE[r][1][style=bold,bottomframe=on,boffset=4pt]
          \setupTABLE[c][1,2][align={right,lohi}]
          \setupTABLE[c][3,4,5,6,7][alignmentcharacter={text->.},aligncharacter=yes,align=middle]
          \setupTABLE[3,4,5,6,7][1][align={left,lohi},aligncharacter=no]
          \setupTABLE[c][1][roffset=4pt]
          \setupTABLE[c][2][roffset=4pt]
          \setupTABLE[c][3,4,5,6,7][roffset=4pt,loffset=4pt]
          \setupTABLE[r][2][toffset=4pt]
          \switchtobodyfont[small]
          \bTABLE
\bTR	\bTD		\eTD	\bTD		\eTD	\bTD	Expert 1	\eTD	\bTD	Expert 2	\eTD	\bTD	Expert 3	\eTD	\bTD	Expert 4	\eTD	\bTD	Expert 5	\eTD	\eTR
\bTR	\bTD	Ambulance In Unfamiliar Area	\eTD	\bTD	Min	\eTD	\bTD	6.67\%	\eTD	\bTD	15.\%	\eTD	\bTD	15.\%	\eTD	\bTD	20.\%	\eTD	\bTD	20.\%	\eTD	\eTR
\bTR	\bTD		\eTD	\bTD	Mode	\eTD	\bTD	13.33\%	\eTD	\bTD	30.\%	\eTD	\bTD	18.\%	\eTD	\bTD	20.\%	\eTD	\bTD	24.\%	\eTD	\eTR
\bTR	\bTD		\eTD	\bTD	Max	\eTD	\bTD	20.\%	\eTD	\bTD	50.\%	\eTD	\bTD	25.\%	\eTD	\bTD	30.\%	\eTD	\bTD	30.\%	\eTD	\eTR
\bTR	\bTD	Ambulance In Traffic Deviation	\eTD	\bTD	Min	\eTD	\bTD	6.67\%	\eTD	\bTD	15.\%	\eTD	\bTD	30.\%	\eTD	\bTD	40.\%	\eTD	\bTD	20.\%	\eTD	\eTR
\bTR	\bTD		\eTD	\bTD	Mode	\eTD	\bTD	20.\%	\eTD	\bTD	50.\%	\eTD	\bTD	50.\%	\eTD	\bTD	50.\%	\eTD	\bTD	30.\%	\eTD	\eTR
\bTR	\bTD		\eTD	\bTD	Max	\eTD	\bTD	33.33\%	\eTD	\bTD	75.\%	\eTD	\bTD	65.\%	\eTD	\bTD	70.\%	\eTD	\bTD	40.\%	\eTD	\eTR
\bTR	\bTD	Printer Off	\eTD	\bTD	Min	\eTD	\bTD	0.\%	\eTD	\bTD	0.\%	\eTD	\bTD	0.\%	\eTD	\bTD	1.\%	\eTD	\bTD	0.\%	\eTD	\eTR
\bTR	\bTD		\eTD	\bTD	Mode	\eTD	\bTD	6.67\%	\eTD	\bTD	0.\%	\eTD	\bTD	0.\%	\eTD	\bTD	1.\%	\eTD	\bTD	0.\%	\eTD	\eTR
\bTR	\bTD		\eTD	\bTD	Max	\eTD	\bTD	6.67\%	\eTD	\bTD	0.\%	\eTD	\bTD	0.\%	\eTD	\bTD	1.\%	\eTD	\bTD	0.\%	\eTD	\eTR
\bTR	\bTD	MDT Turned Off	\eTD	\bTD	Min	\eTD	\bTD	53.33\%	\eTD	\bTD	0.\%	\eTD	\bTD	0.\%	\eTD	\bTD	1.\%	\eTD	\bTD	0.\%	\eTD	\eTR
\bTR	\bTD		\eTD	\bTD	Mode	\eTD	\bTD	66.67\%	\eTD	\bTD	0.\%	\eTD	\bTD	0.\%	\eTD	\bTD	1.\%	\eTD	\bTD	0.\%	\eTD	\eTR
\bTR	\bTD		\eTD	\bTD	Max	\eTD	\bTD	80.\%	\eTD	\bTD	0.\%	\eTD	\bTD	0.\%	\eTD	\bTD	1.\%	\eTD	\bTD	0.\%	\eTD	\eTR
\bTR	\bTD	AVLS Out of Service	\eTD	\bTD	Min	\eTD	\bTD	0.\%	\eTD	\bTD	0.\%	\eTD	\bTD	0.\%	\eTD	\bTD	0.\%	\eTD	\bTD	0.\%	\eTD	\eTR
\bTR	\bTD		\eTD	\bTD	Mode	\eTD	\bTD	0.\%	\eTD	\bTD	0.\%	\eTD	\bTD	0.\%	\eTD	\bTD	0.\%	\eTD	\bTD	0.\%	\eTD	\eTR
\bTR	\bTD		\eTD	\bTD	Max	\eTD	\bTD	0.\%	\eTD	\bTD	0.\%	\eTD	\bTD	0.\%	\eTD	\bTD	0.\%	\eTD	\bTD	0.\%	\eTD	\eTR
\bTR	\bTD	Forget To Encode Leaving Status	\eTD	\bTD	Min	\eTD	\bTD	13.33\%	\eTD	\bTD	0.\%	\eTD	\bTD	6.\%	\eTD	\bTD	20.\%	\eTD	\bTD	10.\%	\eTD	\eTR
\bTR	\bTD		\eTD	\bTD	Mode	\eTD	\bTD	20.\%	\eTD	\bTD	0.\%	\eTD	\bTD	10.\%	\eTD	\bTD	25.\%	\eTD	\bTD	12.\%	\eTD	\eTR
\bTR	\bTD		\eTD	\bTD	Max	\eTD	\bTD	26.67\%	\eTD	\bTD	10.\%	\eTD	\bTD	18.\%	\eTD	\bTD	40.\%	\eTD	\bTD	16.\%	\eTD	\eTR
\bTR	\bTD	Other Status Than Leaving Encoded	\eTD	\bTD	Min	\eTD	\bTD	0.\%	\eTD	\bTD	0.\%	\eTD	\bTD	0.\%	\eTD	\bTD	10.\%	\eTD	\bTD	0.\%	\eTD	\eTR
\bTR	\bTD		\eTD	\bTD	Mode	\eTD	\bTD	0.\%	\eTD	\bTD	0.\%	\eTD	\bTD	0.\%	\eTD	\bTD	10.\%	\eTD	\bTD	0.\%	\eTD	\eTR
\bTR	\bTD		\eTD	\bTD	Max	\eTD	\bTD	6.67\%	\eTD	\bTD	5.\%	\eTD	\bTD	0.\%	\eTD	\bTD	10.\%	\eTD	\bTD	0.\%	\eTD	\eTR
\bTR	\bTD	Patient Not Transportable	\eTD	\bTD	Min	\eTD	\bTD	0.\%	\eTD	\bTD	0.\%	\eTD	\bTD	10.\%	\eTD	\bTD	5.\%	\eTD	\bTD	2.\%	\eTD	\eTR
\bTR	\bTD		\eTD	\bTD	Mode	\eTD	\bTD	13.33\%	\eTD	\bTD	0.\%	\eTD	\bTD	15.\%	\eTD	\bTD	5.\%	\eTD	\bTD	2.\%	\eTD	\eTR
\bTR	\bTD		\eTD	\bTD	Max	\eTD	\bTD	20.\%	\eTD	\bTD	0.\%	\eTD	\bTD	20.\%	\eTD	\bTD	5.\%	\eTD	\bTD	4.\%	\eTD	\eTR
\eTABLE
          }
      
        As \in{Table}[tab:ads_estimates] shows, experts may disagree strongly
        on estimated satisfaction rates such as the one for \goal{MDT Turned
        Off}. Surprisingly, two leaf obstacles, \obstacle{Mobilization Taken By
        Other Ambulance} and \obstacle{AVLS Out of Service}, were estimated by
        all experts to have a satisfaction rate of 0\%. This calls for further
        evaluation before removing the obstacles from the model.

        For calibration, statistical data were obtained about the following
        obstacles: \obstacle{Allocated Ambulance Not At Station} (50\%),
        \obstacle{Mobilization Cancelled} (13\%), and \obstacle{MDT Turned Off}
        (33.3\%). These leaf obstacles were used as calibration variables.

        The results produced by the techniques presented in
        \in{Chapter}[chap:knowledge-uncertainty] proved helpful in the
        following respects.

        \noindent {\bf Managing knowledge uncertainty.} Based on the
        calibration and on collected data, the violation uncertainty obtained for
        the top goal \goal{Achieve [Incident Resolved]} was 100\%, with an
        uncertainty spread of $0.9258$. \in{Figure}[fig:usr_ads] shows the
        satisfaction uncertainty for this goal. This might seem low; the reason
        is that the model only captures the ideal case without taking any
        countermeasure to obstacles into account. The rate for ideal ambulance
        intervention was experienced to be roughly similar to the curve obtained
        with our technique.
      
        \placefigure[here]
             [fig:usr_ads]
             {Satisfaction Uncerainty for \goal{Achieve [Incident Resolved]}.}
          {\externalfigure[../images/chap9/uncertainty_sat_rate_achieve_incident_resolved.pdf]}

        Violation diagrams helped identify most likely and critical
        obstacles together with obstacles requiring further elicitation. The
        gray dots in \in{Figure}[fig:violation_diagram_ads] show the violation
        uncertainty and uncertainty spread for the top goal, taking all experts
        into account.
      
        Seven obstacles were estimated to cause the top goal not to meet its
        RSR with more than 90\% of certainty, namely, \obstacle{Ambulance In
        Traffic Deviation}, \obstacle{Allocated Ambulance Not At Station},
        \obstacle{GPS Black Spot}, \obstacle{Forget To Encode Leaving Status},
        \obstacle{Forget To Encode AvailableRadio Status}, \obstacle{Forget To
        Encode AvailableStation Status}, \obstacle{Wrong Info About Patient}.
        Adequate countermeasures to these critical obstacles should therefore be
        elaborated and integrated.
      
        Among all obstacles, 30 have an uncertainty spread higher than $0.10$.
        These obstacles might, therefore, be further refined; or more experts
        should be asked to reduce the uncertainty margins further. The
        uncertainty spread for the 24 other critical obstacles was low. This
        indicates that experts roughly agreed on their satisfaction rate. Over
        the 54 leaf obstacles, 20 are, with certainty, not causing the
        satisfaction rate of the top goal to fall below its RSR.

        \noindent {\bf Capturing uncertainty about risk estimates.} Emergency
        Medical Technicians were asked to estimate a lower bound, the most
        probable value, and an upper bound for the satisfaction rate of all
        leaf obstacles. To mitigate the common difficulty of estimating strict
        and accurate lower and upper bounds \cite[Vos08], the ones collected
        were used as 10th- and 90th- quantiles, with a 10\% overshoot being
        integrated. Eliciting probability distributions would have been
        impossible in practice as the required statistical background is far
        too important for the considered personnel.

        \noindent {\bf Integrating estimates from multiple experts.} The
        uncertainty spread for the goal \goal{Achieve [Incident Resolved]}
        taking only {\it Expert 1} into account is $0.9487$ which is very high.
        The uncertainty spread caused by each obstacle is then high on average,
        as the orange triangles in \in{Figure}[fig:violation_diagram_ads]
        shows. Using more than one expert helped us reduce the general
        uncertainty spread. However, in some cases, it increased the
        uncertainty spread as experts may disagree.
      
        \placefigure[here]
             [fig:violation_diagram_ads]
             {Violation Diagram for the Ambulance Dispatch System.}
          {\externalfigure[../images/chap9/violation_diagram_ads.pdf]}
          
      \subsubsubject{Handling obstacles at system runtime}

        To evaluate the risk-driven adaptation techniques in
        \in{Section}[sec:obstaclebased_adaptation], we simulated the operation
        of the ambulance dispatching system in Brussels, considering 15
        ambulances and 10 simultaneous incidents. Over a 4-hour simulation, a
        total of 100 incidents was reached. In comparison, the real Brussels
        ambulance dispatching system handles about 25 ambulances and 300
        incidents per day. The simulation was performed on a MacBook Pro 3Ghz
        equipped with 16Gb of RAM.
      
        \noindent {\bf Dispatching software implementation and simulator.} The
        ambulance dispatch software, developed in C\#, implements the
        identified software requirements. The software is freely available to
        replicate our experiments \cite[Cai17c]. Among its components, an
        alternative {\it OnRoadAllocator}, which first allocates ambulances
        that are on the road, can replace the default ambulance allocator.
        Extra components include {\it TrafficJamAllocator} or {\it
        StatusDetector}. The former re-allocates incidents for which the
        allocated ambulance is in a traffic jam; the latter automates the
        status reporting. \in{Figure}[fig:ads_visualization] shows a screenshot
        of the web application to visualize incidents and ambulances; dots
        represent ambulances and incidents together with their status.
          
        \placefigure[]
             [fig:ads_visualization]
             {ADS Visualizaton Web Application.}
          {\externalfigure[../images/chap9/ads_visualization.png][width=10cm]}

        The environment simulator simulates the behavior of the ambulance staff,
        the dispatchers, and other environment agents. For example, the simulator \quote{presses}
        buttons on mobile data terminals (MDT), \quote{drives} the ambulance to the
        incident location, and so forth. The simulator also generates incidents;
        the user can manually encode incidents using the \tool{ADS Console} tool.
        The simulator runs in a separate process (and might run on a different
        machine) and communicates with the dispatching software through RabbitMQ
        queues \cite[Vid12].
    
        \in{Figure}[fig:ads_architecture] shows the global architecture with
        communication queues. The {\it ADS Monitoring Client} takes snapshots
        of the software system and generates the truth value for the
        corresponding predicates. \tool{Utils.Monitor}, presented in
        \in{Chapter}[chap:tool_support], consumes these predicates and feeds a
        queue with the activation/deactivation procedures that need to be
        applied to guarantee the RSR of the high-level goal.
      
        \placefigure[]
             [fig:ads_architecture]
             {ADS Architecture.}
          {\externalfigure[../images/chap9/architecture_ads.pdf]}

        \noindent {\bf Runtime monitoring and adaptation.} At RE time, our
        monitoring tool built 109 monitors for all probabilistic leaf
        obstacles. At runtime, every second, 351 predicates were monitored, the
        monitors were updated, and the satisfaction rate of the high-level
        goals \goal{Achieve [Ambulance On Scene When Incident Reported]} and
        \goal{Avoid [Ambulance On Road Mobilized]} was computed. Every 5
        minutes, the tool compared the monitored satisfaction rate of these
        goals with their respective RSR. When required, it computed most
        appropriate countermeasures. Later on, the tool called the
        corresponding activation/deactivation procedure for reconfiguring the
        ADS. The optimization process took about 2 minutes. During simulation,
        however, one elapsed second corresponds to ten seconds in reality. This
        explains why \in{Figure}[fig:runtime_scenarios] exhibits a substantial
        delay between the time at which adaptation is found necessary and the
        time at which it is deployed.

        \placefigure[page]
             [fig:runtime_scenarios]
             {Simulation Scenarios.}
          {\startcombination[1*3]
            {\externalfigure[../images/chap9/runtime_scenario_1.pdf]}{\tfx (a) Rush Hour}
            {\externalfigure[../images/chap9/runtime_scenario_3.pdf]}{\tfx (b) Night Mobilization}
            {\externalfigure[../images/chap9/runtime_scenario_2.pdf]}{\tfx (c) Status Forgetting}
          \stopcombination}

        \noindent {\bf Simulation scenarios.} We ran three simulations over 4
        hours to cover the following three scenarios.
      
        \startitemize[a]
      
          \item {\bf Rush Hour.} During rush hour, the obstacle
          \obstacle{Ambulance Stuck In Traffic Jam Toward Incident} has an
          increased satisfaction rate. This increase caused the countermeasures
          \goal{Achieve [Ambulance Stuck In Traffic Jam ReAllocated]} and
          \goal{Achieve [Patient Transported At Hospital When Required And
          Traffic Jam]} to be selected, integrated and deployed in the running
          system. \in{Figure}[fig:runtime_scenarios]{(a)} shows the
          satisfaction rate of the root goal increasing again beyond its RSR
          threshold, as the two obstacles no longer obstruct it. The deployment
          of the first countermeasure changes the software. The second
          countermeasure does not alter it; the goal specification is only
          relaxed to allow more time between the incident scene and the
          hospital.

          \item {\bf On-Road Mobilization.} At night, the RSR for the goal
          \goal{Avoid [Ambulance On Road Mobilized]} is $.5$. During the day,
          however, ambulance staff prefer to not intervene in multiple
          incidents without going back to their station. The RSR is increased
          to $.8$, as \in{Figure}[fig:runtime_scenarios]{(b)} shows. As a
          result, the countermeasure \goal{Achieve [Ambulance Allocated At
          Station When Incident Reported]} was deployed during the day. The
          adaptation replaced the {\it DefaultAllocator} component. The new
          allocation strategy reduced the satisfaction rates of the obstacle
          \obstacle{Ambulance Mobilized On Road} and guaranteed the goal's RSRs.

          \item {\bf Forgetting Ambulance Status.} Late at night and early in
          the morning, ambulance staff tends to forget to push buttons. This
          results in an increase in the satisfaction rate of obstacles such as
          \obstacle{Accurate Leaving Status Not Encoded On MDT Or Innaccurate}.
          During that period, the countermeasure goals such as \goal {Achieve
          [Automated Leaving Status Detection]} and \goal {Achieve [Automated
          OnScene Status Detection]} were selected, integrated and deployed.
          The integration caused the computed satisfaction rate to increase
          again beyond their RSR threshold.
          \in{Figure}[fig:runtime_scenarios]{(c)} shows the satisfaction rate
          of \goal{Achieve [Ambulance On Scene When Incident Reported]}.
      
        \stopitemize
      
        \noindent Our techniques were felt to help significantly for the
        following reasons.

        \noindent {\bf Precise semantics regarding behaviors.} Thanks to formal
        specifications being anchored on real-world phenomena, the mapping
        between predicates and the running software system was
        straightforward. For example, the evaluation of the predicate {\ss
        ambulanceA9OnScene} triggered a simple database query.

        All monitored items had a clear, precise meaning. Surprising results
        were easily understandable. In particular, our technique identified
        inaccurate specifications with missing conditions or unrealistic time
        constraints.

        \noindent {\bf Traceability of monitored indicators and deployed
        countermeasures.} The monitored satisfaction rates of high-level goals
        significantly helped understand whether an increase in the
        satisfaction rate of an obstacle is critical. Comparing their monitored
        satisfaction rate with their RSR provided a traceable criterion for
        system adaptation.

        For example, the activation of the {\it TrafficJamAllocator} software
        component in the ADS is directly traceable to its countermeasure goal
        \goal{Achieve [Ambulance Stuck In Traffic Jam ReAllocated]}. The latter
        in turn is traceable to high-level goals such as \quote{{\it an
        ambulance shall be on scene within 12 minutes}}.

        \noindent {\bf Model-based adaptation.} The goal/obstacle model drives
        the selection of most appropriate countermeasure. As the results
        showed, their dynamic selection ensured that the high-level
        probabilistic goals remained satisfied. Without our technique, the
        monitoring and adaptation of the ambulance dispatching software would
        have required dedicated, application-specific code to be written.

        \noindent {\bf No explicit behavior modeling.} The ambulance
        dispatching system exhibits complex states and parallelism among
        processes. Building a complete, consistent, and adequate state machine
        model for this system appears quite hard.

        \noindent {\bf Other benefits.} The formalization effort was kept
        minimal as only leaf obstacles needed to be formalized. In addition, a
        change in the formal specification of a leaf obstacle did not require
        source code modification.

        This validation case study also highlighted areas for improvement. In
        particular, the satisfaction rates estimated by experts (as shown in
        \in{Section}[sec:selection_k_uncertainty]) might be used to improve the
        accuracy of the monitored satisfaction rates in case only a few data
        are available. The monitored satisfaction rates should also be filtered
        to smooth out the noise caused by a low number of observations. The
        technique in \cite[Fil15] might be used to improve the quality of
        monitoring.

        As in many monitoring-based self-adapting systems, the monitoring task
        may impact on the performance of the monitored system. Our preliminary
        experience suggests that most of the impact may be transferred to a
        separate computer to reduce the footprint on the monitored system.

      \stopsubsection
    
      \startsubsection[title={Discussion}]
      
      In addition to the discussion regarding the specific objectives of the
      techniques as presented in the previous subsection, the following briefly
      discusses the applicability of our techniques:
      
      \startitemize
      
        \item Goal and obstacles models have been applied to a wide variety of
        application domains; see for example,
        \cite[Lut07,Pon07,Dar07,Lam09,Let02,Cai13b]. We see no critical
        difficulty in building goal and obstacle models.
        
        \item The other inputs required by our approach are the estimated
        satisfaction rates for leaf obstacles and estimated probability
        distributions to be elicited from domain experts. Tools and
        methodologies are available for this as discussed in
        \cite[Otw92,Coo91,Vos08]. Other approaches also rely on estimated
        values, such as \cite[Lam00,Let04,Hea11,Let14], providing additional
        examples of the applicability of eliciting the required inputs.
        
        \item The Car Pooling System and Ambulance Despatching System showed
        how the proposed techniques for {\it obstacle assessment} and {\it
        control} are applicable to those type of software systems. The C230-YLS
        system showed how the proposed techniques are applicable for assessing
        the human safety risks. The applicability of our techniques to other
        application domains, such as the avionics industry, and to other type
        of risk analysis, such as security risk analysis, should be further
        evaluated.
      
        \item We played the role of the analysts in applying the techniques.
        Future studies should demonstrate the applicability of the techniques
        by people with a different expertise. In particular, the cost for
        training risk analysts, building the goal and obstacle models and
        applying the proposed technique should be compared to the cost of other
        risk assessment and control techniques. Whether our techniques are
        cost-effectively applicable by risk analysts should be assessed.
        
        \item The C230-YLS case study suggests that our techniques may
        integrate with existing practice such as FMECA. However, how the
        proposed techniques may be integrated, at a larger scale, within the
        existing practices in industry remains an open question. For example,
        the extend to which the risk analysis process currently in use at IBA
        need be modified should be carrefuly evaluated.
        
      \stopitemize
      
      \stopsubsection
    
    \stopsection

    \startsection[reference=sec:utility,title={Utility}]
    
      This section briefly discuss the extent to which our proposed techniques solve a real problem. The
      problems the thesis focuses on are: 
      
      \startitemize
        
        \item the requirements completeness problem addressed by the techniques
        presented in \in{Chapter}[chap:proba-framework], \in{}[chap:assessing]
        and \in{}[chap:controlling_obstacle];

        \item the handling of knowledge uncertainty addressed by the techniques
        presented in \in{Chapter}[chap:knowledge-uncertainty];
        
        \item the risk-driven runtime adaptation addressed in
        \in{Chapter}[runtime].
      
      \stopitemize
      
      \noindent {\bf Adressing the completeness problem. } This problem is
      recognized as a major challenge in software engineering. As discussed in
      the introduction of \in{Chapter}[chap:proba-framework], such
      incompleteness arises from our tendency to conceive over-ideal systems;
      integrating risk analysis into the requirement process helps producing
      more complete and accurate requirements \cite[Lam09]. Moreover,
      increasingly many software systems are built according to models
      \cite[Sta06,Bra12] and are driven, at runtime, by models \cite[Ben07].
      These models shall thereby be as complete as possible to provide robust
      software systems. Our techniques complement the obstacle analysis process
      to support quantitative reasoning on the satisfaction rate of obstacles
      and goals. Our techniques felt significantly helpful to identify missing
      risks (as seen in the IBA C230-YLS case-study) or inaccurate
      specifications (such as in the ADS case-study). Besides, our techniques
      help to focus on likely and critical risks and most appropriate
      countermeasure selection; these problems were confirmed by IBA engineers
      to be major challenges, as seen in the case-study.
      
      \noindent {\bf Adressing uncertain estimates.} As discussed in the
      introduction of \in{Chapter}[chap:knowledge-uncertainty], knowledge
      uncertainty impedes the application of standard risk prioritization
      techniques. The problem of managing uncertainties is recognized to be a
      major issue \cite[Coo91,Men03,Che09b,Let14]. Our techniques were felt
      significantly helpful identifying and controlling uncertainty margins, as
      seen in the Ambulance Despatching System case study in
      \in{Section}[sec:bads].
      
      \noindent {\bf Addressing risk-driven runtime adaptation.} Today's
      software systems should be flexible, robust, and efficient in adapting to
      changing operational contexts and environments. The problem of software
      adaptation at runtime is recognized to be a challenge for modern software
      engineering practice \cite[Che09,DeL13,Gie13]. Our techniques can be used
      to drive software adaptation by high-level goal satisfaction, relying on
      identified obstacles to highlight most appropriate adaptations to deploy.
    
    \stopsection

    \startsection[reference=sec:usability,title={Usability}]
    
      This section briefly discusses the usability of the techniques proposed
      in this thesis. Criteria proposed in \cite[Nie94] for evaluating the
      usability of a design are adapted here for our techniques:
      
      \startitemize
      
        \item {\it Learnability}: How easy is it for users to apply the
        techniques?
        
        \item {\it Errors}: How many errors do users make, how severe are these
        errors, and how easily can they recover from the errors?
        
        \item {\it Efficiency}: Once users have learned the techniques, how
        quickly can they perform tasks?
        
        \item {\it Memorability}: When users return to the techniques after a
        period of not using it, how easily can they reestablish proficiency?
      
      \stopitemize
      
      \noindent {\bf Learnability.} The goal/obstacle languages are used in
      industrial settings, as seen in
      \cite[Lut07,Pon07,Dar07,Lam09,Let02,Cai13b]. Other techniques such as
      CORAS \cite[Lun10] or Fault Tree Analysis \cite[Rui15,Ves81] require a
      similar level of expertise in model building.
      
      For obstacle assessment, as seen in the ADS or the IBA C230-YSL
      case-studies \in{Sections}[sec:bads] and \in{}[sec:iba], domain experts
      were, in practice, able to estimate the satisfaction rates for leaf
      obstacles.
      
      For obstacle control, applying our techniques presents no specific
      difficulty once the goal and obstacle models are built and leaf obstacles
      are estimated. How easily risk analysts can select most appropriate
      countermeasures remains an open question and should be further evaluated.
      
      For uncertain estimates, as seen in the ADS in \in{Section}[sec:bads],
      domain experts were able to estimates the satisfaction rates for leaf
      obstacles together with uncertainty margins. The extends to which the
      metrics {\it violation uncertainty} and {\it uncertainty spread} are easy
      to interpret by risk analysts also remains an open question to us.
      
      For runtime adaptation, our techniques rely on formal specifications of
      probabilistic obstacles. While formal specification remains a challenge
      \cite[Fra91], elicitations techniques and specification patterns
      \cite[Dwy98,Dar95] may be used to ease the production of formal
      specifications \cite[Lam09]. Future work should more precisely evaluate
      the impact of formal specifications on the usability of our monitoring
      techniques.
      
      \noindent {\bf Errors.} The level of feedback in case of errors provided
      by the tools implementing the techniques should be improved for better
      usability by users less familiar with the internal details of our
      techniques. In addition, the interpretation of the results should be
      performed with care. With proper training, we see no major impediment
      here. This should however be rigourously tested.
      
      \noindent {\bf Efficiency and memorability.} Future work should consider
      case studies where the role of the risk analysts is not played by
      ourselves in order to better evaluate the efficency and memorability of
      the proposed techniques.
    
    \stopsection

  \stopsection
  
  \startsection[title={Summary}]
  
    The chapter evaluated the correctness, performance and scalability,
    applicability, utility and usability of the techniques for {\it obstacle
    assessment} and {\it obstacle control}. The proposed techniques are shown
    to be correct with respect to their specification. The complexities and
    performance of those appears to be reasonable in practice. Three
    case-studies suggest that our techniques may be applicable to
    realistically-sized problems. Two case-studies, the CPS and the ADS are
    realistic case-studies showing the applicability of our techniques to this
    kind of software systems; the IBA C230-YLS is a real industrial case-study
    showing the applicability of our techniques for assessing human safety
    risks. Last, our techniques are shown to be useful and usable in realistic
    situations. 
  
  \stopsection

\stopcomponent
