% !TEX root = thesis.tex

\startcomponent chap-11
\environment common
\product thesis

\chapter[chap:conlusion]{Conclusion}

   This chapter concludes the thesis. It summarizes the contributions,
   discusses the strenghts and limitations, and presents open issues and future
   research directions.
  
  \section {Summary of contributions}
  
    Obstacles preventing the satisfaction of high-level objectives shall be
    identified, assessed and controlled to produce more adequate, complete,
    consistent, and precise requirement goal models. The techniques presented
    in the thesis are intended to fill the gaps in the {\it obstacle
    assessment} and {\it control} steps by proposing a quantitative extension to the KAOS
    framework for probabilistic goals and obstacles.

    For the {\it obstacle assessment} step, the thesis proposed a simple yet
    effective quantitative model-based technique. The KAOS framework is
    extended with a probabilistic layer allowing behavioral goals to be
    specified, as a first-class citizen, in terms of their {\it required} and
    {\it estimated} satisfaction rate. The thesis introduces probabilistic
    goals and probabilistic obstacles. The satisfaction rate of a probabilistic
    goal and the satisfaction rate of a probabilistic obstacle is precisely
    defined in terms of state probabilities; the latter caputre the probability
    that the behaviors rooted in a system state satisfies an assertion. The
    refinement structure and satisfaction arguments for non-probabilistic goals
    and obstacles is generalized.
    
    Obstacles are assessed agains their lilkelihood and criticality. The
    satisfaction rate of non-leaf obstacles is determined by up-propagation in
    the obstacle trees. The criticality of obstacles is systematically
    determined by propagating the satisfaction rate of obstacles along the goal
    refinement trees. The thesis proposed two propagation algorithms for
    computing the satisfaction rate of high-level goals from the estimated
    satisfaction rate of leaf obstacles. The BDD-based propagation procedure is
    efficient when repeated propagations are required whereas the Pattern-based
    propagation procedure is finer-grained at an increased specification cost.
    Most likely and critical obstacles and obstacle combinations are then highlighted to
    guide the exploration of countermeasures in the next {\it control} step.
    
    Regarding the {\it obstacle control} step, the thesis proposed appropriate
    countermeasures selection guaranteeing the satisfaction of the {\it
    required} satisfaction rate of the high-level goals while minimizing the
    cost of resolution. An integration mechanism was introduced as a model
    transformation guaranteeing progress towards more complete goal models,
    minimal changes to the ideal model and preservation of the correctness of
    the refinements. Anchor goals were introduced to define where the
    countermeasures goals shall be integrated. Countermeasures then document
    the ideal goal model with dedicated exception constructs. Refactoring
    operators allows the analyst to attach and detach those exceptions.
    
    The techniques for {\it obstacle assessment} and {\it obstacle control}
    were extended to cope with knowledge uncertainty about the satisfaction
    rate of probabilistic system goals and their obstructing obstacles. The
    uncertainty margins of satisfaction rate estimates are modeled as
    probability distributions. The impact of such uncertainty on the
    criticality of obstacles is quantified via two metrics: {\it goal violation
    uncertainty} and {\it uncertainty spread}. The satisfaction rate of
    non-leaf obstacles and the satisfaction rate of goals, together with their
    uncertainty margins, are computed by repeated propagation through the
    goal/obstacle models. As a result, the critical obstacles with problematic
    uncertainty margins are highlighted. Problematic uncertainties may then be
    reduced by combining multiple sources of estimates. Countermeasures may be
    selected to guarantee the {\it required} satisfaction rate of high-level
    goals up to an uncertainty threshold.
  
    Lastly, the thesis proposed an obstacle-driven runtime adaptation approach
    aimed at increasing the actual satisfaction rate of probabilistic system
    goals. Leaf obstacles together with their criticality are monitored at
    runtime to let the system dynamically switch to more appropriate
    countermeasures goals. The proposed monitoring technique extends the
    non-probabilistic $LTL_3$ approach \cite[Bau11] to support monitoring of
    probabilistic assertions. Most appropriate countermeasures, selected
    on-the-fly, increase the satisfaction rate of the high-level goals under
    the current conditions.
    
    The proposed techniques are supported by a set of tools, together with a
    specification language enabling their application on real case-studies. The
    tools and techniques were evaluated on three realistic case-studies: a car
    pooling system, an industrial yoke lifting system and an ambulance
    dispatching system.
    
  \section {Strenghts and limitations}
    
    The main strenghts of our proposed techniques for obstacle assesment and
    control are described below.
  
    \startitemize
  
      \item {\it Probabilistc goals enables the modeling of properties that
      hold in X\% of cases.} In a non-probabilistic framework, requirements
      such as \quote{\it an ambulance shall be on the incident scene within 14
      minutes in 95\% of cases?} cannot be expressed. Probabilistic goals
      enables such requirements to be included, as first-class citizen,
      together with non-probabilistic ones in the same RE
      framework.
    
      \item {\it Probabilistic goals and obstacles have a precise semantic
      anchored on real-world phenomena.} The precise semantic in terms of
      behaviors refering to domain-specific predicates enables expert to
      estimate and interpret satisfaction rate of probabilistic obstacle and
      goals. The precise semantic also enables the monitoring of satisfaction
      rates at runtime, as seen in \in{Chapter}[runtime].
    
      \item {\it Probabilistic obstacle refinements provide a structure for the
      assessment of the likelihood of occurence.} The satisfaction rate of
      non-leaf obstacles is systematically determined from the estimated
      satisfaction rates for leaf obstacles by up-propagation through the
      obstacle refinement trees. Only leaf obstacles need to be estimated by
      domain experts.
    
      \item {\it Probabilistic goal refinements provide a structure for the
      assessment of obstacle criticality.} The consequences of an obstacle are
      the goals whose satisfaction is prevented by this obstacle. The severity
      of the consequences is systematically determined by up-propagation of the
      obstacle satisfaction rate throught the goal refinement trees. An
      obstacle is critical if the satisfaction rate of the goal fall below its
      RSR.
    
      \item {\it Control of probabilistic obstacles guarantee progress towards
      complete models.} Valid countermeasures, introduced in
      \in{Section}[sec:identifying_cm], guarantee that the satisfaction rate of
      a high-level increases when properly integrated in the ideal goal model.
      Integrating more and more countermeasures leads to high-level goals that
      are more and more satisfied.
    
      \item {\it Probabilistic goals and obstacles enables cost-effective
      selection of countermeasures.} Selecting most appropriate countermeasures
      amount to solve an optimization problem where the satisfaction rate of
      high-level goals is maximized while minimizing the resolution cost.
    
      \item {\it Integration of countermeasures to probabilistic obstacles
      preserve the normal behaviors.} The two integration schemas introduced in
      \in{Section}[sec:integration_schema] guarantee that goals not affected by
      the resolved obstacle are not modified. The integration schema, picked
      according the resolution tactic of the countermeasure goal, either
      replace the anchor goals or refines the anchor goal, leaving the rest of
      the goal model intact. Normal behaviors not impacted by the
      obstacle are thereby preserved.
    
      \item {\it Integration of countermeasures to probabilistic obstacles
      preserve the correctness of the refinement structure.} Propagating
      required changes in the specification of goals though the obstacle/goal
      models guarantee that the refinement structure remains complete,
      consistent and minimal (as seen in \in{Section}[sec:propagation]).
    
      \item {\it Exception handling of probabilistic obstacles separates ideal
      from exceptional cases.} The dedicated constructs such as {\it Except},
      {\it Provided} and {\it RelaxedTo}, as presented in
      \in{Section}[sec:exception_handling], separate the specification of the
      ideal cases from the exceptional ones. The ideal goal is thereby not
      clutered with the conditions related to specific exceptional cases.
    
      \item {\it Exception handling of probabilistic obstacles enables
      incremental integration of countermeasures.} Refactoring operators allows
      the analyst to attach and detach exceptions, deferring the decision of
      what is exceptional from what is not to the later stages of the
      requirement process.
    
      \item {\it Exception handling of probabilistic obstacles mitigates the
      combinatorial explosion of countermeasures.} Integration without
      dedicated exception handling impacts both the anchor goal specification
      and its refinement structure. With such a brute-force integration, each
      countermeasure goal must be refined by taking other countermeasures into
      account. This might lead to a combinatorial blow-up of cases. Thanks to
      the {\it Except} and {\it Provided} constructs, the original
      specification of the anchor goal and its refinement structure are
      preserved.

      \item {\it Satisfaction uncertainty of probabilistic goals and obstacles
      allow to cope with knowledge uncertainty.} The satisfaction uncertainty
      of probabilistic goals and obstacles allows domain experts to represent
      their uncertainty margin about estimates. Probability distribtions are a
      natural abstraction to model uncertainty about estimates \cite[Vos08].
    
      \item {\it Probabilistic goals and obstacles enables the highlight of
      problematic uncertainty margins about estimates.} Two metrics are
      introduced for measuring problematic uncertainties of goal satisfaction
      rates. When applied to the probability distributions obtained by
      up-propagation to the top-level goals, the metrics allow critical leaf
      obstacles with most problematic uncertainty margins to be highlighted.
    
      \item {\it Probabilistic goals and obstacles support multiple source of
      estimates for uncertainty reductions.} Multiple source of estimates are
      generally recognized to produce more accurate estimates when combined
      \cite[Bed01]. Problematic uncertainty margins are reduced by combining
      multiple sources.
    
      \item {\it Satisfaction rate of probabilistic goals and obstacles may be
      monitored at runtime.} Monitors, built at RE time from the formal
      specification of leaf obstacles, computes the actual satisfaction rate of
      leaf obstacles at runtime. The latter are propagated through the
      obstacle/goal model up to the system’s high-level goals in order to
      determine the actual criticality of those leaf obstacles.
    
      \item {\it Risk-driven runtime adaptation ensures that only required
      adaptation are performed.} The risk-driven runtime adaptation selections,
      at runtime, most appropriate countermeasures. Those maximize the
      satisfaction rate of high-level goals while minimizing the resolution
      cost. It guarantee the required satisfaction rate of high-level goals
      remains satisfied when the obstacle satisfaction rates are changing.
    
      \item {\it The monitoring of probabilistic goals and obstacles do not
      requires explicit behavior models.} Building a consistent and complete
      behavior model for large distributed systems with many complex states and
      parallelism is often quite challenging. Our monitoring technique does not
      requires such model to be explicitely provided.
    
      \item {\it Probabilistic goals and obstacles support traceability
      managment.} Decision criteria for countermeasure selection, and therefore
      system adaptation, are traceable to system objectives. Exception handling
      constructs such as {\it Except}, {\it Provided}, and {\it RelaxedTo}
      document why goal revisions were required. Lastly, why such or such
      monitored information is required is thereby documented as the predicates
      to be monitored are infered from the formal specification of the leaf
      obstacles to be monitored.
    
    \stopitemize
    
    However, our techniques have limitations, paving the way for future
    research.
  
    \startitemize
    
      \item {\it No technique support probabilistic goal elicitation and
      refinement.} Unlike the approach presented in \cite[Let04], the thesis
      proposes no support for building probabilistic goal models. In
      \cite[Let04], domain-specific equations might guide the elicitation
      process; the equations terms might correspond to different subgoals.
  
      \item {\it No technique support probabilistic obstacle identification.}
      The thesis proposes no specific support for identifying probabilistic
      obstacles. Obstacles to probabilistic goals are currently identified
      using non-probabilistic regression calculus \cite[Lam09] or a
      learning-based approach on observed behavior \cite[Alr12].
    
      \item {\it No technique support indentification of countermeasures to
      probabilistic obstacles.} The thesis proposes no specific support for
      identifying countermeasure to probabilistic obstacles. Currently, the
      countermeasures are identified using a variety of resolution tactics
      \cite[Lam09] or a learning-based approach based on observed behaviors
      \cite[Alr16].
      
      \item {\it No technique support agents responsible for probabilistic
      goals.} Intuitively, some agents might be more reliable than others, and
      some instances of agents might exhibit different characteristics. A
      resolution tactic transfer the responsibilities from a less reliable
      agent to a more reliable one; such transfer cannot be formally expressed
      using the techniques presented in the thesis. How agent reliability can
      be precisely defined at the requirements level, how it impacts the
      obstacle {\it assessment} and {\it control} are not discussed.
    
      \item {\it The obstacle control technique overlooks domain-specific
      variables and soft goals contributions.} As seen in the IBA case-study
      (see \in{Section}[sec:iba]) and unlike the approach presented in
      \cite[Let04], the proposed technique do not leverage domain-specific
      knowledge. In particular, during the {\it obstacle control} step,
      contributions to soft goal are not accounted for in the trade-off
      analysis.
    
      \item {\it The obstacle assessment technique only consider a static view
      of the system.} The occurence of an obstacle might increase the
      satisfaction rate of an other obstacle, a specific selection of
      countermeasures might increase the satisfaction rate of a set of
      obstacles, satisfaction rates might increase or decrease overtime
      depending on whether a countemeasure is selected, and so forth. Unlike
      Dynamic Fault Trees \cite[Rui15], the proposed techniques do not
      currently support such dynamic models.
  
    \stopitemize
    
	\section {Open issues and perspectives}
	
  This section lists some open issues and discusses perspectives for future
  work.
  
  \noindent {\bf Obstacle and countermeasure identification.} 
  Detecting incomplete goal and obstacle models at runtime could improve the
  obstacle identification step. Monitoring the high-level goals and obstacles
  may highlight missing obstacles by the comparing the estimated satisfaction
  rate obtained by up-propagation with the monitored one. This appears to be a
  promising approach to highlight behavior traces exposing unknown obstacles
  or countermeasures. This could be integrated with learning obstacles and
  countermeasures from observed traces \cite[Alr12,Alr16] or from an
  executable specification \cite[Ram12].
  
  Related to identification, it is common for risk analysis to be performed
  once and not maintained as software evolve. The risk analysis quickly becomes
  obsolete, incomplete and inadequate. Previous research efforts shows that 
  missing requirements might be detected
  during testing \cite[Lut03]. Integrating testing and risk analysis could
  continuously check whether the software system covers the identified risks,
  and whether the risk analysis is adequate, precise, complete, and consistent
  with regard to the evolving software. Requirements-Driven
  \cite[Raj08,Wha06] and Risk-Driven testing \cite[Erd14,Red05,Klo11]
  approaches appears to be a promising direction.
  
  \noindent {\bf Dynamic obstacle models.} Techniques and model for Dynamic
  Fault Trees \cite[Rui15] may improve the expressiveness and accuracy of the
  probabilistic goal and obstacle models. How to model, elicit, and analyse
  probabilistic goals and obstacles with a high-level of interdepency remains
  an open question to us.
  
  \noindent {\bf Risk-driven adaptation at runtime.} 
  The proposed approach for runtime adaptation might be improved in the following
  directions.
  
  \startitemize
  
    \item The risk-driven adaptation technique relies on satisfaction rates
    monitoring. However, monitoring is not always possible or desirable, for
    exemple, in ultra-low power systems. On-the-fly adaptations are not
    necessarily desirable in very-critical safety applications where these
    adaptation may make the certification of such systems hard to be obtained.
    Explicit modeling of the evolution of the satisfaction rates over the time
    would enable optimal countermeasure selection to be computed at RE time for
    software systems where monitoring is not possible or desirable.
  
    \item Over time, \quote{real} satisfaction rates could abruptly change, the
    monitored satisfaction rate however then tends to slowly reflect this
    changes. Algorithms are available for detecting such abrupt changes such as
    \cite[Epi10,Liu13,Fil14] and the filtering approach proposed in
    \cite[Fil15] could improve the accuracy of the monitored satisfaction rates.
  
    \item The question of \quote{when} to adapt is still an open issue. If
    deploying an adaptation is cheaper than the penalty of a violation, it
    might be preferable to adapt before the violation occurs. However, some
    adaptations are costly and should not be performed unless the violation
    last enough for the penalty to exceed the cost. In addition, software
    adaptations exhibit latencies that should be taken into account at RE time
    \cite[Cam14,Cam16]. Prediction approaches, such as \cite[Pol07,Mor16],
    appears to be a promising direction for more cost-effective adaptation.
  
  \stopitemize
  
  \noindent {\bf Risk-driven design.} Our approach is focused on determining
  whether the required satisfaction rate of high-level goals are met, given the
  satisfaction rates of leaf obstacles. Another type of analysis would include
  the following reverse question: what are the satisfaction rates of low-level
  obstacles given a required degree of satisfaction? Parametric model checking
  might be used to questions on the low-level obstacles such as \quote{\it how
  many ambulances are required to guarantee a 5\% satisfaction rate?}.
  Combining these with our proposed quantitative framework might answer
  questions on high-level goals, such as \quote{\it how many ambulances are
  required to guarantee a 95\% required satisfaction rate for the ambulance
  timely on scene?}.
  
  \noindent {\bf Finer-grained obstacle assessment and control.} The
  satisfaction rate of an obstacle depends on the lowest state probability.
  However, the state corresponding to this state probability might not be
  representative of the satisfaction rate. For example, such state might be
  transient and never met once the system has reach stability. The presented
  {\it obstacle assessmnt} technique provides coarse-grained assessment
  ignoring the above situation. The following directions might improve the
  obstacle {\it assessment} and {\it control} step.
  
  \startitemize
  
    \item The satisfaction rate of a goal, an obstacle respectively, is defined
    as the highest, the lowest respectively, state probability that the
    specification is satisfied. Considering the highest, lowest and average
    state probability may provide finer-grained analysis of the system
    internals. In particular, this extra information could be used to make
    better decisions about the software adaptations to be performed at
    runtime\emdash{}for example, it might be worthless to trigger an adaptation if
    the state with the highest state probability is not visited often.
  
    \item Our proposed approach selects countermeasures in order to minimize a
    cost. However, design decisions often result from fine-grained tradeoff
    analysis with multiple competing criteria \cite[Che06]. Refined criteria
    taking soft goals or domain-specific variables into account could lead to
    better countermeasure selections.
    
    \item A drawback of Monte Carlo simulations is that the number of
    simulations required to obtain accurate estimations grows in the presence
    of rare events \cite[Jeg13,Rui17]. Specific propagation procedure could
    be developped to better support rare obstacles.    
  
  \stopitemize
  
  \noindent {\bf Security risk analysis.} How the proposed approach transpose
  to security risk analysis is an open question to us. In particular, an
  in-depth comparision with existing security-based approaches such as {\it
  attack trees} and {\it threat trees} \cite[Sch99,Hel02,Sch11], {\it UMLSec}
  \cite[Jur01,Jur02], {\it SecureUML} \cite[Lod02], {\it Abuse Case}
  \cite[McD99], or {\it Misuse Case} \cite[Sin05] could exhibit interesting
  technique transfer from one community to the other.
  
\stopcomponent
