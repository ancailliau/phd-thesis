% !TEX root = thesis.tex

\startcomponent chap-11
\environment common
\product thesis

\chapter[chap:conlusion]{Conclusion}

  Obstacles preventing the satisfaction of high-level objectives shall be
  identified, assessed and controlled to produce more adequate, precise and
  complete requirement goal models. The techniques presented in the thesis are
  intended to fill the gaps in the {\it obstacle assessment} and {\it
  resolution} steps by proposing a quantitative framework.
  
  For the {\it obstacle assessment} step, the thesis proposed a quantitative
  model\-based framework, anchored on an existing goal-oriented framework for
  requirements engineering. The framework is extended with a probabilistic
  layer allowing behavioral goals to be specified, as a first-class citizen, in
  terms of their required and estimated satisfaction rate. The formal
  characterization in terms of behaviors enables the estimates to be grounded
  on real-world phenomena, allowing experts to estimates and interpret the
  satisfaction rates, and allow the integration of other techniques such as
  runtime monitoring. The severity of obstacles is determined by propagating
  the satisfaction rate of fine-grained obstacles along the goal/obstacle
  refinement trees. The most critical obstacle combinations are then
  highlighted to guide the exploration of countermeasures in the next {\it
  resolution} step.
  
  Regarding the {\it obstacle resolution} step, the thesis proposed to select
  appropriate countermeasures guaranteeing the satisfaction of the required
  satisfaction rate of the high-level goals while minimizing the cost. An
  integration operator was introduced as a model transformation guaranteeing
  progress towards more complete goal models, minimal changes to the ideal
  model and preservation of the correctness of the refinements. Anchor goals
  were introduced to define where the countermeasures goals shall be
  integrated. Countermeasures then document the ideal goal model with dedicated
  constructs thereby separating the specification of the ideal cases from the
  exceptional ones. This enables a greater readability of the specifications
  and incremental integration. Refactoring operators allows the analyst to
  attach and detach exceptions, deferring the decision of what is exceptional
  from what is not to the later stages of the requirement process. Design
  decisions, as countermeasure goal integration, are documented and traceable
  to domain-specific risks and high-level objectives.
  
  The quantitative techniques presented in the thesis were extended to cope
  with knowledge uncertainty about the satisfaction rate of probabilistic
  system goals and their obstructing obstacles. The impact of uncertainty on
  high-level goals is quantified via two metrics: goal violation uncertainty
  and uncertainty spread. The satisfaction rate of high-level goals, together
  with their uncertainty margins, are computed by repeated propagation through
  the goal/obstacle model of sampled satisfaction rate of leaf obstacles. As a
  result, the critical obstacles with problematic uncertainty margins are
  highlighted. Problematic uncertainties might then be reduced by combining
  multiple sources of estimates. Countermeasures are then selected to guarantee
  the required satisfaction rate of high-level goals up to an uncertainty
  threshold.
  
  Lastly, the thesis proposed an obstacle-driven runtime adaptation approach
  aimed at increasing the actual satisfaction rate of probabilistic system
  goals. Leaf obstacles are monitored at runtime to let the system dynamically
  switch to more appropriate countermeasures goals that increase the
  satisfaction rate of the high-level goals under the current conditions
  thereby documenting the rationale for adaptation. The approach ensures that
  the required satisfaction rate of high-level goals remains satisfied when the
  obstacle satisfaction rates are changing, taking advantage of the
  goal/obstacle refinement structure. Our proposed technique does not depend on
  explicit and detailed behavior models, enabling the application of monitoring
  to more complex software systems.
  
  The proposed techniques proved to be effective for the {\it obstacle
  assessment} and {\it resolution} steps. The two extensions enabling to cope
  with uncertainty and to perform assessment and control at runtime suggest
  that the quantitative framework could be extended to support other type of
  analysis and reasoning. The proposed techniques are domain-agnostic and not
  support application specific soft goals or variables that may enable more
  finer-grained analysis at an increased specification cost. The uncertainty
  managment layer is based on probability distributions and the runtime
  adaptation technique requires formal specifications that might impede the
  application of the techniques by untrained experts.
  
  \section {Summary of Contributions}
    
    The contributions of the thesis are summarized below.
    
    \startitemize
    
      \item The thesis introduced probabilistic goals and obstacles with a
      formal semantic in terms of desirable/undesirable behaviors enabling the
      estimates to be grounded on real-world phenomena and the integration with
      other techniques such as runtime monitoring. The thesis also defined
      independence of probabilistic goals and obstacles together with
      structural criteria.
      
      \item The thesis proposed two propagation algorithms for computing the
      satisfaction rate of high-level goals from the estimated satisfaction
      rate of leaf obstacles. The BDD-based propagation procedure is efficient
      when repeated propagations are required whereas the Pattern-based
      propagation procedure is finer-grained at an increased specification
      cost. The thesis proposed an algorithm for highlighting the critical and
      likely risks.
      
      \item The thesis introduced valid countermeasures that guarantee progress
      towards more complete goal models. The thesis proposed an algorithm for
      selecting the most appropriate countermeasures and an integration
      operator for integrating these selected countermeasures. The integration
      operator guarantee progress towards more complete models,
      minimal changes to the ideal model, and the correctness of the refinement
      structure by propagating required changes. The thesis extended the goal
      specification language for supporting exception handling at a requirement
      level.
      
      \item The thesis extended the quantitative framework for {\it obstacle
      assessment} and {\it obstacle resolution} steps to cope with knowledge
      uncertainty by supporting multi-value estimates of satisfaction rate. The
      thesis proposed two metrics for measuring the uncertainty margins on
      high-level objectives: the goal violation uncertainty and the uncertainty
      spread. The thesis proposed an algorithm for identifying the critical
      obstacles with problematic uncertainty margins and an algorithm for
      selecting the most appropriate countermeasures given an uncertainty
      threshold.
      
      \item The thesis proposed an obstacle-driven adaptation mechanism to
      drive runtime adaptation using the satisfaction of high-level
      probabilistic goals. The proposed monitoring technique extends the
      non-probabilistic $LTL_3$ approach \cite[Bau11a] to support monitoring of
      probabilistic assertions. Countermeasure selection and integration is
      performed on-the-fly.
      
      \item The techniques are supported by a set of tools, together with a
      specification language enabling their application on real case-studies.
      
      \item The tools and techniques were evaluated on three realistic
      case-studies: a car pooling system, an industrial yoke lifting system and
      an ambulance dispatching system.
      
      \item The thesis provided a state-of-the-art review presenting an overview of
      the research on quantitative, risk-driven requirements engineering.
    
    \stopitemize
  
	\section {Open Issues and Perspectives}
	
  This section lists some open issues and discusses perspectives for future work.
  
  \noindent {\bf Obstacle and countermeasure identification} The obstacle
  assessment and resolution steps are as good as the initial obstacle
  identification step. It is not possible to assess and resolve obstacles that
  were not identified. In the same vein, it is not possible to select
  countermeasures that were not identified. Unknown unknown is a critical
  challenge in Requirements Engineering.
  
  Detecting incomplete goal and obstacle models at runtime could improve the
  obstacle identification step. Monitoring the high-level goals and obstacles
  may highlight missing obstacles by the comparing the estimated satisfaction
  rate obtained by up-propagation with the monitored one. This appears to be a
  promising approach to highlight behavior traces exposing unknown obstacles
  or countermeasures. This could be integrated with learning obstacles and
  countermeasures from observed traces \cite[Alr12a,Alr16a] or from an
  executable specification \cite[Ram12a].
  
  Related to identification, it is common for risk analysis to be performed
  once and not maintained as software evolve. The risk analysis quickly becomes
  obsolete, incomplete and inadequate. Missing requirements might be detected
  during testing \cite[Lut03a]. Integrating testing and risk analysis could
  continuously check whether the software system covers the identified risks,
  and whether the risk analysis is adequate, precise, complete, and consistent
  with regard to the evolving software. Requirements-Driven
  \cite[Raj08a,Wha06a] and Risk-Driven testing \cite[Erd14a,Red05a,Klo11a]
  approaches appears to be a promising direction.
  
  \noindent {\bf Dynamic risk analysis.} To date, our approach and most of the
  risk analysis approaches only consider a static view of the software system,
  where the dynamic of the system and its evolution over time are barely taken
  into account. Most of the analysis are performed on snapshots of the software
  system.
  
  The risk-driven adaptation technique presented an approach where satisfaction
  rates are monitored to switch to appropriate countermeasures. However,
  monitoring is not always possible or desirable, for exemple, in ultra-low
  power systems. On-the-fly adaptations are not necessarily desirable in
  very-critical safety applications where these on-the-fly adaptation makes the
  certification of theses systems hard. Explicit modeling of the evolution of
  the satisfaction rates over the time would enable optimal countermeasure
  selection to be computed at RE time for software systems where monitoring is
  not possible or desirable.
  
  \noindent {\bf Change detection.} Over time, \quote{real} satisfaction rates
  could abruptly change, the monitored satisfaction rate however then tends to
  slowly reflect this changes. Algorithms are available for detecting such
  abrupt changes such as \cite[Epi10a,Liu13a,Fil14a] and the filtering approach
  proposed in \cite[Fil15a] could improve the accuracy of the monitored
  satisfaction rates.
  
  \noindent {\bf When to adapt?} The question of \quote{when} to adapt is still
  an open issue. If the cost of an adaptation is cheaper than the penalty of a
  violation, it might be preferable to adapt before the violation occurs.
  However, some adaptations are costly and should not be performed unless the
  violation last. In addition, software adaptations exhibit latencies that
  should be taken into account at RE time \cite[Cam14a,Cam16a]. Prediction
  approaches, such as \cite[Pol07a,Mor16a], appears to be a promising direction
  for more cost-effective adaptation.
  
  \noindent {\bf Obstacle dependence.} Little support is available to model the
  impact of obstacles on other obstacles: the occurence of an obstacle might
  increase the satisfaction rate of others, the selection of a countermeasures
  might changes the probability of other obstacles, and so forth. These complex
  interactions are largely overlooked in today risk-driven requirements
  engineering. How dynamic fault trees model and deals with such complexity
  could improves the current requirements-based approaches.
  
  \noindent {\bf Agent reliability.} The impact of agents was not studied in
  this thesis. Intuitively, some agents might be more reliable than others, and
  some instances of agents might exhibit different characteristics. In
  addition, a resolution strategy transfer the responsibilities of an
  unrealiable to a more reliable agent. How agent reliability can be precisely
  defined at the requirements level, how it impacts the obstacle assessment and
  resolution are open issues.
  
  \noindent {\bf Risk-driven design.} Our approach is focused on determining
  whether the required satisfaction rate of high-level goals are met, given the
  satisfaction rates of leaf obstacles. Another type of analysis would include
  the following reverse question: what are the satisfaction rates of low-level
  obstacles given a required degree of satisfaction? Parametric model checking
  might be used to questions such as \quote{\it how many ambulances are
  required to guarantee a 5\% satisfaction rate?} on the low-level obstacles.
  Combining these with the proposed quantitative framework might answer
  questions on high-level goals, such as \quote{\it how many ambulances are
  required to guarantee a 95\% required satisfaction rate for the ambulance
  timely on scene?}.
  
  \noindent {\bf Finer-grained obstacle assessment and resolution.} Two
  interesting directions might improve the obstacle assessment and resolution
  step.
  
  \startitemize
  
    \item The satisfaction rate of a goal, an obstacle respectively, is defined
    as the highest, the lowest respectively, state probability that the
    specification is satisfied. Considering the highest, lowest and average
    state probability may provide finer-grained analysis of the system
    internals. In particular, this extra information could be used to make
    better decisions about the software adaptations to be performed at
    runtime\emdash{}for example, it might be worthless to trigger an adaptation if
    the state with the highest state probability is not visited often.
  
    \item The proposed approach selects countermeasures in order to minimize a
    cost. However, design decisions often result from fine-grained tradeoff
    analysis with multiple competing criteria \cite[Che06a]. Refined criteria
    taking soft goals or domain-specific variables into account could lead to
    better countermeasure selections.
    
    \item A drawback of Monte Carlo simulations is that the number of
    simulations required to obtain accurate estimations grows in the presence
    of rare events \cite[Jeg13a,Rui17a]. Specific propagation procedure could
    be developped to better support rare obstacles.    
  
  \stopitemize
  
  \noindent {\bf Security risk analysis.} How the proposed approach transpose
  to security risk analysis is an open question to us. In particular, an
  in-depth comparision with existing security-based approaches such as {\it
  attack trees} and {\it threat trees} \cite[Sch99a,Hel02a,Sch11a], {\it
  UMLSec} \cite[Jur01a,Jur02a], {\it SecureUML} \cite[Lod02a], {\it Abuse Case}
  \cite[McD99a], or {\it Misuse Case} \cite[Sin05a] could exhibit interesting
  transfer from one community to the other.
  
  While these limitations might limit the applicability of our techniques to
  specific cases, it also paves the way to more research in the area of
  risk-driven requirements engineering.
  
\stopcomponent